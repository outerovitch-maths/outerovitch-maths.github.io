%% Font size %%
\documentclass[11pt]{article}

%% Load the custom package
\usepackage{Mathdoc}

%% Numéro de séquence %% Titre de la séquence %%
\renewcommand{\centerhead}{Multiplications, Divisions de fractions}

%% Spacing commands %%
\renewcommand{\baselinestretch}{1} \setlength{\parindent}{0pt}

\begin{document}

\begin{exercice}[1]
\begin{multicols}{2}
\begin{enumerate}[itemsep=3em,label={}]
	\item  $A = \dfrac{4}{7}\times\dfrac{5}{9}$ 
	\item  $B = \dfrac{3}{7}\times\dfrac{1}{6}$ 
	\item  $C = \dfrac{7}{9}\times\dfrac{3}{5}$ 
	\item  $D = \dfrac{2}{9}\times\dfrac{5}{8}$ 
	\item  $E = \dfrac{1}{4}\times\dfrac{6}{7}$ 
	\item  $F = \dfrac{2}{5}\times\dfrac{3}{10}$ 
	\item  $G = \dfrac{7}{10}\times\dfrac{1}{6}$ 
	\item  $H = \dfrac{4}{9}\times\dfrac{2}{3}$ 
\end{enumerate}
\end{multicols}
\end{exercice}


\begin{exercice}[2]
\begin{multicols}{2}
\begin{enumerate}[itemsep=3em,label={}]
	\item  $A = \dfrac{3}{14}\times\dfrac{2}{18}$ 
	\item  $B = \dfrac{44}{10}\times\dfrac{2}{88}$ 
	\item  $C = \dfrac{10}{63}\times\dfrac{28}{14}$ 
	\item  $D = \dfrac{8}{25}\times\dfrac{10}{6}$ 
	\item  $E = \dfrac{5}{70}\times\dfrac{28}{45}$ 
	\item  $F = \dfrac{77}{27}\times\dfrac{3}{88}$ 
	\item  $G = \dfrac{25}{18}\times\dfrac{3}{50}$ 
	\item  $H = \dfrac{28}{55}\times\dfrac{44}{63}$ 
\end{enumerate}
\end{multicols}
\end{exercice}

\newpage

\begin{multicols}{2}
\begin{exercice}[1]
\begin{enumerate}[itemsep=2em]
\item Quel est l'inverse de $9$ ?
\item Quel est l'inverse de $13$ ?
\item Quel est l'inverse de $2$ ?
\item Quel est l'inverse de $5$ ?
\item Quel est l'inverse de $1$ ?
\end{enumerate}
\end{exercice}


\begin{exercice}[1]
\begin{enumerate}[itemsep=2em]
\item Quel est l'inverse de $\dfrac{5}{9}$ ?
\item Quel est l'inverse de $\dfrac{9}{4}$ ?
\item Quel est l'inverse de $\dfrac{16}{6}$ ?
\item Quel est l'inverse de $-\dfrac{9}{7}$ ?
\item Quel est l'inverse de $-\dfrac{12}{18}$ ?
\end{enumerate}
\end{exercice}
\end{multicols}

\begin{exercice}[1]
\begin{multicols}{2}
\begin{enumerate}[itemsep=2em]
	\item  $\dfrac{2}{7}\div\dfrac{1}{4}$ 
	\item  $\dfrac{1}{9}\div\dfrac{7}{8}$ 
	\item  $\dfrac{1}{2}\div\dfrac{6}{7}$ 
	\item  $\dfrac{7}{9}\div\dfrac{5}{6}$ 
	\item  $\dfrac{4}{5}\div\dfrac{1}{7}$ 
	\item  $\dfrac{3}{4}\div\dfrac{4}{5}$ 
	\item  $\dfrac{8}{9}\div\dfrac{3}{7}$ 
	\item  $\dfrac{1}{10}\div\dfrac{4}{9}$ 
\end{enumerate}
\end{multicols}
\end{exercice}



\begin{exercice}[1]
\begin{multicols}{2}
\begin{enumerate}[itemsep=3em,label={}]
	\item  $A = \dfrac{1}{10} \div \dfrac{7}{4}$ 
	\item  $B = \dfrac{1}{9} \div \dfrac{5}{4}$ 
	\item  $C = \dfrac{4}{5} \div 3$ 
	\item  $D = \dfrac{2}{9} \div 4$ 
	\item  $E = \dfrac{4}{5} \div \dfrac{5}{2}$ 
	\item  $F = \dfrac{3}{5} \div \dfrac{10}{7}$ 
	\item  $G = \dfrac{6}{7} \div \dfrac{9}{2}$ 
	\item  $H = \dfrac{3}{4} \div 9$ 
\end{enumerate}
\end{multicols}
\end{exercice}


\begin{exercice}[2]
\begin{multicols}{2}
\begin{enumerate}[itemsep=3em,label={}]
	\item  $A = \dfrac{20}{10} \div \dfrac{35}{2}$ 
	\item  $B = \dfrac{20}{55} \div \dfrac{45}{22}$ 
	\item  $C = \dfrac{40}{99} \div \dfrac{35}{11}$ 
	\item  $D = \dfrac{55}{24} \div \dfrac{33}{6}$ 
	\item  $E = \dfrac{11}{15} \div \dfrac{44}{3}$ 
	\item  $F = \dfrac{8}{45}\dfrac{10}{15}$ 
	\item  $G = \dfrac{5}{88} \div \dfrac{35}{11}$ 
	\item  $H = \dfrac{16}{45} \div \dfrac{18}{20}$ 
\end{enumerate}
\end{multicols}
\end{exercice}


\begin{exercice}[3]

Les nombres situés à l'extrémité des flèches sont les produits des nombres dont les flèches sont issues.
\begin{enumerate}[itemsep=2em]
	\item Calculer les produits à l'extrémité des flèches.\\\begin{tikzpicture}[baseline,scale = 0.6]

    \tikzset{
      point/.style={
        thick,
        draw,
        cross out,
        inner sep=0pt,
        minimum width=5pt,
        minimum height=5pt,
      },
    }
    \clip (-7.76,-7.76) rectangle (7.76,7.76);
    	
	\draw[color ={black}] (0,0)--(3,0);
	\draw[color ={black}] (3,0)--(2.66,-2.66);
	\draw[color ={black}] (2.66,-2.66)--(0,-3);
	\draw[color ={black}, densely dash dot dot ,-{Stealth[width=5mm]}] (2.66,-2.66)--(5.46,-0.59);
	\draw[color ={black}, densely dash dot dot ,-{Stealth[width=5mm]}] (2.66,2.66)--(5.46,0.59);
	\draw (1.77,-1.77) node[anchor = center] {\footnotesize \color{black}{$\dfrac{1}{5}$}};
	\draw [color={black}] (6.26,0) node[anchor = center,scale=1, rotate = 0] {a};
	\draw[color={black}] (7.26,0)--(6.26,-1)--(5.26,0)--(6.26,1)--cycle;
	
	\draw[color ={black}] (0,0)--(0,3);
	\draw[color ={black}] (0,3)--(2.66,2.66);
	\draw[color ={black}] (2.66,2.66)--(3,0);
	\draw[color ={black}, densely dash dot dot ,-{Stealth[width=5mm]}] (2.66,2.66)--(0.59,5.46);
	\draw[color ={black}, densely dash dot dot ,-{Stealth[width=5mm]}] (-2.66,2.66)--(-0.59,5.46);
	\draw (1.77,1.77) node[anchor = center] {\footnotesize \color{black}{$\dfrac{5}{2}$}};
	\draw [color={black}] (0,6.26) node[anchor = center,scale=1, rotate = 0] {b};
	\draw[color={black}] (1,6.26)--(0,5.26)--(-1,6.26)--(0,7.26)--cycle;
	
	\draw[color ={black}] (0,0)--(-3,0);
	\draw[color ={black}] (-3,0)--(-2.66,2.66);
	\draw[color ={black}] (-2.66,2.66)--(0,3);
	\draw[color ={black}, densely dash dot dot ,-{Stealth[width=5mm]}] (-2.66,2.66)--(-5.46,0.59);
	\draw[color ={black}, densely dash dot dot ,-{Stealth[width=5mm]}] (-2.66,-2.66)--(-5.46,-0.59);
	\draw (-1.77,1.77) node[anchor = center] {\footnotesize \color{black}{$3$}};
	\draw [color={black}] (-6.26,0) node[anchor = center,scale=1, rotate = 0] {c};
	\draw[color={black}] (-5.26,0)--(-6.26,-1)--(-7.26,0)--(-6.26,1)--cycle;
	
	\draw[color ={black}] (0,0)--(0,-3);
	\draw[color ={black}] (0,-3)--(-2.66,-2.66);
	\draw[color ={black}] (-2.66,-2.66)--(-3,0);
	\draw[color ={black}, densely dash dot dot ,-{Stealth[width=5mm]}] (-2.66,-2.66)--(-0.59,-5.46);
	\draw[color ={black}, densely dash dot dot ,-{Stealth[width=5mm]}] (2.66,-2.66)--(0.59,-5.46);
	\draw (-1.77,-1.77) node[anchor = center] {\footnotesize \color{black}{$\dfrac{9}{5}$}};
	\draw [color={black}] (0,-6.26) node[anchor = center,scale=1, rotate = 0] {d};
	\draw[color={black}] (1,-6.26)--(0,-7.26)--(-1,-6.26)--(0,-5.26)--cycle;

\end{tikzpicture}\\

	\item Calculer les produits à l'extrémité des flèches.\\\begin{tikzpicture}[baseline,scale = 0.6]

    \tikzset{
      point/.style={
        thick,
        draw,
        cross out,
        inner sep=0pt,
        minimum width=5pt,
        minimum height=5pt,
      },
    }
    \clip (-7.76,-7.76) rectangle (7.76,7.76);
    	
	\draw[color ={black}] (0,0)--(3,0);
	\draw[color ={black}] (3,0)--(2.66,-2.66);
	\draw[color ={black}] (2.66,-2.66)--(0,-3);
	\draw[color ={black}, densely dash dot dot ,-{Stealth[width=5mm]}] (2.66,-2.66)--(5.46,-0.59);
	\draw[color ={black}, densely dash dot dot ,-{Stealth[width=5mm]}] (2.66,2.66)--(5.46,0.59);
	\draw (1.77,-1.77) node[anchor = center] {\footnotesize \color{black}{$5$}};
	\draw [color={black}] (6.26,0) node[anchor = center,scale=1, rotate = 0] {a};
	\draw[color={black}] (7.26,0)--(6.26,-1)--(5.26,0)--(6.26,1)--cycle;
	
	\draw[color ={black}] (0,0)--(0,3);
	\draw[color ={black}] (0,3)--(2.66,2.66);
	\draw[color ={black}] (2.66,2.66)--(3,0);
	\draw[color ={black}, densely dash dot dot ,-{Stealth[width=5mm]}] (2.66,2.66)--(0.59,5.46);
	\draw[color ={black}, densely dash dot dot ,-{Stealth[width=5mm]}] (-2.66,2.66)--(-0.59,5.46);
	\draw (1.77,1.77) node[anchor = center] {\footnotesize \color{black}{$\dfrac{1}{5}$}};
	\draw [color={black}] (0,6.26) node[anchor = center,scale=1, rotate = 0] {b};
	\draw[color={black}] (1,6.26)--(0,5.26)--(-1,6.26)--(0,7.26)--cycle;
	
	\draw[color ={black}] (0,0)--(-3,0);
	\draw[color ={black}] (-3,0)--(-2.66,2.66);
	\draw[color ={black}] (-2.66,2.66)--(0,3);
	\draw[color ={black}, densely dash dot dot ,-{Stealth[width=5mm]}] (-2.66,2.66)--(-5.46,0.59);
	\draw[color ={black}, densely dash dot dot ,-{Stealth[width=5mm]}] (-2.66,-2.66)--(-5.46,-0.59);
	\draw (-1.77,1.77) node[anchor = center] {\footnotesize \color{black}{$2$}};
	\draw [color={black}] (-6.26,0) node[anchor = center,scale=1, rotate = 0] {c};
	\draw[color={black}] (-5.26,0)--(-6.26,-1)--(-7.26,0)--(-6.26,1)--cycle;
	
	\draw[color ={black}] (0,0)--(0,-3);
	\draw[color ={black}] (0,-3)--(-2.66,-2.66);
	\draw[color ={black}] (-2.66,-2.66)--(-3,0);
	\draw[color ={black}, densely dash dot dot ,-{Stealth[width=5mm]}] (-2.66,-2.66)--(-0.59,-5.46);
	\draw[color ={black}, densely dash dot dot ,-{Stealth[width=5mm]}] (2.66,-2.66)--(0.59,-5.46);
	\draw (-1.77,-1.77) node[anchor = center] {\footnotesize \color{black}{$\dfrac{1}{5}$}};
	\draw [color={black}] (0,-6.26) node[anchor = center,scale=1, rotate = 0] {d};
	\draw[color={black}] (1,-6.26)--(0,-7.26)--(-1,-6.26)--(0,-5.26)--cycle;

\end{tikzpicture}\\

\end{enumerate}
\end{exercice}

\end{document}
