\message{ !name(DS-1-S06-Etendue-.tex)}%% Font size %%
\documentclass[11pt]{article}

%% Load the custom package
\usepackage{Mathdoc}

%% Numéro de séquence %% Titre de la séquence %%
\renewcommand{\centerhead}{}

%% Spacing commands %%
\renewcommand{\baselinestretch}{1} \setlength{\parindent}{0pt}

\begin{document}

\message{ !name(DS-1-S06-Etendue-.tex) !offset(-3) }


\phantom{0}
\vspace{-1.5cm}

\begin{center}
\duree{30 minutes} 
\coefficient{0,5}
\calculatrice{1}
\end{center}

\entetedevoirs{10}


\begin{exercicedevoir}[1]

Valérie a obtenu ces notes ce trimestre-ci en mathématiques :\\$16$; $11$ ; $5$ ; $14$ ; $7$ ; $5$ ; $11$ et $12$.\\Calculer l'étendue de ces notes.
\end{exercicedevoir}


\begin{exercicedevoir}[1]

En septembre 2017, à Paris, on a relevé les températures suivantes : 

\medskip
$\def\arraystretch{1.5}\begin{array}{|c|c|c|c|c|c|c|c|c|c|c|c|c|c|c|c|c}\hline  \text{Jour}&1&2&3&4&5&6&7&8&9&10&11&12&13&14&15\\\hline \text{Température en}  ^\circ\text{C}&21&19&21&21&21&21&21&19&19&17&19&21&23&24&23\\\hline\end{array}$

\medskip
$\def\arraystretch{1.5}\begin{array}{|c|c|c|c|c|c|c|c|c|c|c|c|c|c|c|c|c}\hline  \text{Jour}&16&17&18&19&20&21&22&23&24&25&26&27&28&29&30\\\hline \text{Température en}  ^\circ\text{C}&25&26&25&25&24&25&23&23&23&24&22&20&22&22&20\\\hline\end{array}$

\medskip
Calculer l'étendue des températures.
\end{exercicedevoir}


\begin{exercicedevoir}[1]
\begin{multicols}{2}
Gaspard a obtenu ces notes ce trimestre-ci en mathématiques :\\$12$; $10$ ; $11$ ; $10$ ; $7$ ; $9$ ; $13$ ; $4$ ; $5$ ; $12$ ; $12$ et $16$.\\
Calculer l'étendue de ces notes. \\
Réponse : \\ \encart{1cm}

\end{multicols}
\end{exercicedevoir}

\end{document}

%%% Local Variables:
%%% mode: LaTeX
%%% TeX-master: t
%%% TeX-master: t
%%% End:


\message{ !name(DS-1-S06-Etendue-.tex) !offset(-69) }
