%% Font size %%
\documentclass[11pt]{article}

%% Load the custom package
\usepackage{Mathdoc}

%% Numéro de séquence %% Titre de la séquence %%
\renewcommand{\centerhead}{Chapitre 6 : Statistiques - Éval. 2 : Fréquences}

%% Spacing commands %%
\renewcommand{\baselinestretch}{1} \setlength{\parindent}{0pt}

\begin{document}

\phantom{0}
\vspace{-1.5cm}

\begin{center}
\duree{30 minutes} 
\total{10 points}
\coefficient{0,5}
\calculatrice{1}
\brouillon
\copieseparee{0}
\end{center}

\entetedevoirs{10}

\begin{exercicedevoir}[2]

On a réalisé $50$ lancers d'un dé à $9$ faces.\\Les résultats sont inscrits dans le tableau ci-dessous.

\medskip
$\def\arraystretch{1.5}\begin{array}{|c|c|c|c|c|c|c|c|c|c|c}\hline  \text{Scores}&1&2&3&4&5&6&7&8&9\\\hline \text{Nombre d'apparitions}&3&3&4&4&4&7&7&5&6\\\hline\end{array}$

\medskip
Déterminer la fréquence de la valeur $4$. \\ \\
\textbf{Fraction : }\fbox{\makebox[4cm][l]{\strut}} \\ \\
\textbf{Décimal : }\fbox{\makebox[4cm][l]{\strut}} \\ \\
\textbf{Pourcentage : }\fbox{\makebox[4cm][l]{\strut}}
\end{exercicedevoir}


\begin{exercicedevoir}[2]

Tania a obtenu ces notes ce trimestre-ci en mathématiques :\\$12$; $10$ ; $7$ ; $8$ ; $1$ ; $12$ ; $7$ et $10$.

\medskip
Calculer la fréquence de la note $7$. \\ \\
\textbf{Fraction : }\fbox{\makebox[4cm][l]{\strut}} \\ \\
\textbf{Décimal : }\fbox{\makebox[4cm][l]{\strut}} \\ \\
\textbf{Pourcentage : }\fbox{\makebox[4cm][l]{\strut}}
\end{exercicedevoir}

\newpage

\begin{exercicedevoir}[3]
En février 2010, à Rome, on a relevé les températures suivantes.

\medskip
$\def\arraystretch{1.5}\begin{array}{|c|c|c|c|c|c|c|c|c|c|c|c|c|c|c|c}\hline  \text{Jour}&1&2&3&4&5&6&7&8&9&10&11&12&13&14\\\hline \text{Température en}  ^\circ\text{C}&6&6&8&10&11&12&11&11&13&13&15&17&15&15\\\hline\end{array}$

\medskip
$\def\arraystretch{1.5}\begin{array}{|c|c|c|c|c|c|c|c|c|c|c|c|c|c|c|c}\hline  \text{Jour}&15&16&17&18&19&20&21&22&23&24&25&26&27&28\\\hline \text{Température en}  ^\circ\text{C}&8&16&15&17&15&14&16&15&16&16&16&14&8&14\\\hline\end{array}$

\medskip
Calculer la fréquence de la température $8^\circ\text{C}$.\\ \\
\textbf{Fraction : }\fbox{\makebox[4cm][l]{\strut}} \\ \\
\textbf{Décimal : }\fbox{\makebox[4cm][l]{\strut}} \\ \\
\textbf{Pourcentage : }\fbox{\makebox[4cm][l]{\strut}}
\end{exercicedevoir}


\begin{exercicedevoir}[3]

Dans un parking de supermarché comptant 100 voitures, on a noté leur couleur.\\On a consigné les résultats dans le tableau suivant :

\medskip
$\renewcommand{\arraystretch}{1.5}
\begin{array}{|c|c|c|c|c|c|c|c|c|c|}
\hline
\text{\textbf{Couleurs}} &  \text{Noir} &  \text{Blanc} &  \text{Gris} &  \text{Vert} &  \text{Rouge} &  \text{Marron} &  \text{Jaune} &  \text{Bleu} &  \text{\textbf{TOTAL}}\\
\hline
\text{\textbf{Effectifs}} & 16 & 4 & 3 & 28 & 19 & 9 &  & 16 & 100\\
\hline
\text{\textbf{Fréquences}} &  &  &  &  &  &  &  &  & \\
\hline
\end{array}
\renewcommand{\arraystretch}{1}$
\\
\begin{enumerate}
\item Déterminer l'effectif manquant. \textbf{Réponse :} \fbox{\makebox[4cm][l]{\strut}} \\ 
\item Déterminer les fréquences pour chaque couleur (en pourcentage, arrondir au dixième si besoin).\\ \\
\textbf{Détails des calculs :}\\ \\ \encart{6cm}
\end{enumerate}
\end{exercicedevoir}

\nonewpage
\end{document}

%%% Local Variables:
%%% mode: LaTeX
%%% TeX-master: t
%%% TeX-master: t
%%% End:

