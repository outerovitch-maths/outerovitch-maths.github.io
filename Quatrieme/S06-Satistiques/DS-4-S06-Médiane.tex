%% Font size %%
\documentclass[11pt]{article}

%% Load the custom package
\usepackage{Mathdoc}

%% Numéro de séquence %% Titre de la séquence %%
\renewcommand{\centerhead}{Chapitre 6 : Statistiques - Éval. 4 : Médianes}

%% Spacing commands %%
\renewcommand{\baselinestretch}{1} \setlength{\parindent}{0pt}

\begin{document}

\phantom{0}
\vspace{-1.5cm}

\begin{center}
\duree{30 minutes} 
\coefficient{0,5}
\calculatrice{1}
\brouillon
\copieseparee{0}
\end{center}

\entetedevoirs{15}

\begin{exercicedevoir}[1]

\begin{multicols}{2}
On donne la série statistique suivante : 
16 ; 8 ; 10 ; 11 ; 2 ; 4\\
Déterminer la médiane de cette série ? \\ \\
\textbf{Réponse : }\fbox{\makebox[4cm][l]{\strut}} \\ \\
\columnbreak \\
\textbf{Calculs :} \\ \encart{3cm}
\end{multicols}
\end{exercicedevoir}


\begin{exercicedevoir}[1]

\begin{multicols}{2}
Carine a obtenu ces notes ce trimestre-ci en mathématiques :\\$15$; $16$ ; $9$ ; $12$ ; $3$ ; $13$ ; $15$ ; $16$ et $12$.\\Déterminer une médiane de cette série. \\ \\
\textbf{Réponse : }\fbox{\makebox[4cm][l]{\strut}} \\ \\
\columnbreak \\
\textbf{Calculs :} \\ \encart{3cm}
\end{multicols}
\end{exercicedevoir}

\newpage

\begin{exercicedevoir}[1]
Voici les températures, en degré Celsius, relevées sur une période de 7 jours : 11 ; 13 ; 13 ; 12 ; 13 ; 14 ; 16.
\begin{enumerate}[itemsep=1em]
\item Calculer la température moyenne , au dixième près,  de cette série. \\ \\
\textbf{Réponse : }\fbox{\makebox[4cm][l]{\strut}} \\ \\
\textbf{Calculs :} \\ \encart{3cm}
\item Quelle est la température médiane de cette série ? \\ \\
\textbf{Réponse : }\fbox{\makebox[4cm][l]{\strut}} \\ \\
\textbf{Calculs :} \\ \encart{3cm}
\end{enumerate}
\end{exercicedevoir}


\begin{exercicedevoir}[1]

On a réalisé $500$ lancers d'un dé à $6$ faces.\\Les résultats sont inscrits dans le tableau ci-dessous :

\medskip
$\def\arraystretch{1.5}\begin{array}{|c|c|c|c|c|c|c|c}\hline  \text{Scores} &1&2&3&4&5&6\\\hline \text{Nombre d'apparitions} &80&78&84&86&93&79\\\hline\end{array}$

\medskip
Déterminer une médiane de cette série. \\ \\
\textbf{Réponse : }\fbox{\makebox[4cm][l]{\strut}} \\ \\
\textbf{Calculs :} \\ \encart{3cm}
\end{exercicedevoir}

\nonewpage

\end{document}

%%% Local Variables:
%%% mode: LaTeX
%%% TeX-master: t
%%% TeX-master: t
%%% End:

