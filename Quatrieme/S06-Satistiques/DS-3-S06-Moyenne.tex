%% Font size %%
\documentclass[11pt]{article}

%% Load the custom package
\usepackage{Mathdoc}

%% Numéro de séquence %% Titre de la séquence %%
\renewcommand{\centerhead}{Chapitre 6 : Statistiques - Éval. 3 : Moyennes}

%% Spacing commands %%
\renewcommand{\baselinestretch}{1} \setlength{\parindent}{0pt}

\begin{document}

\phantom{0}
\vspace{-1.5cm}

\begin{center}
\duree{30 minutes} 
\coefficient{0,5}
\calculatrice{1}
\brouillon
\copieseparee{0}
\end{center}

\entetedevoirs{15}

%Moyenne 3 valeurs
\begin{exercicedevoir}[2]
\begin{multicols}{2}
$168$\,\, ; \,\, $165$ \,\, ; \,\,$162$\\

Quelle est la moyenne de cette série (arrondie au dixième) ? \\ \\
\textbf{Réponse : }\fbox{\makebox[4cm][l]{\strut}} \\ \\
\columnbreak \\
\textbf{Calculs :} \\ \encart{3cm}
\end{multicols}
\end{exercicedevoir}

%Moyenne 4 valeurs
\begin{exercicedevoir}[2]
\begin{multicols}{2}
$165$\,\, ; \,\, $171$ \,\, ; \,\,$159$ \,\, ; \,\,$164$\\
Quelle est la moyenne de cette série (arrondie au dixième) ?\\ \\
\textbf{Réponse : }\fbox{\makebox[4cm][l]{\strut}} \\ \\
\columnbreak \\
\textbf{Calculs :} \\ \encart{3cm}
\end{multicols}
\end{exercicedevoir}

%Moyenne Pointures Tableau
\begin{exercicedevoir}[3]

Pour passer une commande de chaussures de foot, Cyril a noté les pointures des membres de son club dans un tableau :
\begin{multicols}{2}
\medskip
$\def\arraystretch{1.5}\begin{array}{|c|c|c|c|c|c|c}\hline  \text{Pointure} &34&35&36&38&39\\\hline \text{Effectif} &4&12&9&10&6\\\hline\end{array}$

\medskip
Calculser la pointure moyenne des membres de ce club (arrondie au l'unité). \\ \\
\textbf{Réponse : }\fbox{\makebox[4cm][l]{\strut}} \\ \\
\columnbreak \\
\textbf{Calculs :} \\ \encart{3cm}
\end{multicols}
\end{exercicedevoir}

%Moyenne Notes
\begin{exercicedevoir}[3]
\begin{multicols}{2}
Abdou a obtenu ces notes ce trimestre-ci en mathématiques :\\$14$; $8$ ; $14$ ; $12$ ; $14$ ; $7$ ; $10$ ; $6$ ; $9$ et $14$.\\Calculser la moyenne des notes (arrondie à l'unité). \\ \\
\textbf{Réponse : }\fbox{\makebox[4cm][l]{\strut}} \\ \\
\columnbreak \\
\textbf{Calculs :} \\ \encart{3cm}
\end{multicols}
\end{exercicedevoir}

%Moyenne Temperatures Tableau
\begin{exercicedevoir}[5]

En octobre 2008, à Berlin, on a relevé les températures suivantes : 

\medskip
$\def\arraystretch{1.5}\begin{array}{|c|c|c|c|c|c|c|c|c|c|c|c|c|c|c|c|c|c}\hline  \text{Jour}&1&2&3&4&5&6&7&8&9&10&11&12&13&14&15&16\\\hline \text{Température en}  ^\circ\text{C}&16&17&19&20&18&18&16&14&16&18&19&21&19&18&17&17\\\hline\end{array}$

\medskip
$\def\arraystretch{1.5}\begin{array}{|c|c|c|c|c|c|c|c|c|c|c|c|c|c|c|c|c}\hline  \text{Jour}&17&18&19&20&21&22&23&24&25&26&27&28&29&30&31\\\hline \text{Température en}  ^\circ\text{C}&19&20&18&20&21&19&19&19&20&19&21&21&23&21&20\\\hline\end{array}$

\medskip
Calculser la moyenne des températures (arrondie au dixième).\\ \\
\textbf{Réponse : }\fbox{\makebox[4cm][l]{\strut}} \\ \\

\textbf{Calculs :} \\ \encart{3cm}
\end{exercicedevoir}


\end{document}

%%% Local Variables:
%%% mode: LaTeX
%%% TeX-master: t
%%% TeX-master: t
%%% End:

