%% Font size %%
\documentclass[11pt]{article}

%% Load the custom package
\usepackage{Mathdoc}

%% Numéro de séquence %% Titre de la séquence %%
\renewcommand{\centerhead}{}

%% Spacing commands %%
\renewcommand{\baselinestretch}{1} \setlength{\parindent}{0pt}

\begin{document}

\pagestyle{empty}

\phantom{0}
\vfill

\begin{exercice}[1][Produits de puissances]
Écrire sous la forme $a^n$. Comme dans l'exemple
\begin{multicols}{2}
\begin{enumerate}
\item  $A=7^{3}\times 7^{2}=7^{3+2}=7^{5}$ 
\item  $B=2^{5}\times 2^{3}=$
\item  $C=2^{3}\times 2^{2}=$
\item  $D=6^{4}\times 6^{3}=$
\item  $E=7^{4}\times 7^{2}=$
\item  $F=4^{3}\times 4^{2}=$
\item  $G=2^{4}\times 2^{3}=$
\item  $H=8^{4}\times 8^{2}=$
\item  $I=3^{3}\times 3^{2}=$
\item  $J=7^{5}\times 7^{2}=$
\end{enumerate}
\end{multicols}
\end{exercice}
\vfill
\vfill

\vfill

\setcounter{exercice}{0}
\begin{exercice}[1][Produits de puissances]
Écrire sous la forme $a^n$. Comme dans l'exemple
\begin{multicols}{2}
\begin{enumerate}
\item  $A=7^{3}\times 7^{2}=7^{3+2}=7^{5}$ 
\item  $B=2^{5}\times 2^{3}=$
\item  $C=2^{3}\times 2^{2}=$
\item  $D=6^{4}\times 6^{3}=$
\item  $E=7^{4}\times 7^{2}=$
\item  $F=4^{3}\times 4^{2}=$
\item  $G=2^{4}\times 2^{3}=$
\item  $H=8^{4}\times 8^{2}=$
\item  $I=3^{3}\times 3^{2}=$
\item  $J=7^{5}\times 7^{2}=$
\end{enumerate}
\end{multicols}
\end{exercice}
\vfill

\end{document}
