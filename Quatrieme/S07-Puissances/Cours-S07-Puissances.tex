%% Font size %%
\documentclass[11pt]{article}

%% Load the custom package
\usepackage{Mathdoc}

%% Numéro de séquence %% Titre de la séquence %%
\renewcommand{\centerhead}{Chap. 7 - Les Puissances}

%% Spacing commands %%
\renewcommand{\baselinestretch}{1} \setlength{\parindent}{0pt}


\begin{document}

\section{Comprendre la notation de puissance}

\subsection{Puissances d'exposant positif}

\begin{definition}
Soit $a$ un nombre relatif et $n$ un entier naturel.  
La puissance de $a$ d'exposant $n$ est définie par :
\[
a^n = a \times a \times a \times \dots \times a 
\quad \text{(}n\text{ facteurs).}
\]
\end{definition}

\begin{exemple}
$7^3 = 7 \times 7 \times 7 = 343$.

$( -5 )^4 = (-5) \times (-5) \times (-5) \times (-5) = 625$.
\end{exemple}

\begin{remarque}
$a^0 = 1$ par convention.

$0^0$ n'existe pas.

$a^2$ se lit « $a$ au carré » et $a^3$ se lit « $a$ au cube ».
\end{remarque}

\subsection{Puissances d'exposant négatif}

\begin{definition}
Pour tout nombre relatif $a$ et tout entier $n$, on définit :
\[
a^{-n} = \frac{1}{a^n}.
\]
\end{definition}

\begin{exemple}
$3^{-4} = \dfrac{1}{3^4} = \dfrac{1}{3 \times 3 \times 3 \times 3} = \dfrac{1}{81}$.

$6^{-5} = \dfrac{1}{6^5} = \dfrac{1}{7776}$.
\end{exemple}

\begin{remarque}
Une puissance d'exposant négatif est toujours l'inverse d'une puissance d'exposant positif.

L'inverse de $a$ se note en général $\dfrac{1}{a}$, et peut maintenant s'écrire $a^{-1}$.
\end{remarque}

\section{Propriétés générales}

\begin{propriete}
Soient $a$ et $b$ deux nombres relatifs et $m,n$ deux entiers :

\[
a^m \times a^n = a^{m+n}
\]

\[
\frac{a^m}{a^n} = a^{m-n} \quad (a \neq 0)
\]

\[
(a \times b)^n = a^n \times b^n
\]

\[
(a^m)^n = a^{m \times n}
\]
\end{propriete}

\begin{exemple}
$6^2 \times 6^5 = 6^{2+5} = 6^7$.

$\dfrac{3^4}{3^8} = 3^{4-8} = 3^{-4}$.

$(4 \times 7)^2 = 4^2 \times 7^2$.

$(11^2)^4 = 11^{2 \times 4} = 11^8$.
\end{exemple}

\section{Cas particulier : les puissances de 10}

\subsection{Principe de base}

\begin{exemple}
$10^4 = 10 \times 10 \times 10 \times 10 = 10\,000$.
\end{exemple}

La particularité est que $10^n$ s'écrit comme un $1$ suivi de $n$ zéros.

\begin{exemple}
$10^6 = 1\,000\,000$.

$10^9 = 1\,000\,000\,000$.
\end{exemple}

Les puissances négatives :

\begin{exemple}
$10^{-4} = \dfrac{1}{10^4} = \dfrac{1}{10\,000} = 0,0001$.

$10^{-9} = 0,000\,000\,001$.
\end{exemple}

\subsection{Définition}

\begin{definition}
L'écriture scientifique d'un nombre décimal non nul est une écriture de la forme  
$a \times 10^n$ où :

- $a$ est un nombre décimal compris entre $1$ et $10$ (exclu),
- $n$ est un entier relatif.
\end{definition}

\begin{exemple}
$150\,000\,000 = 1,5 \times 10^8$.

$0,00651 = 6,51 \times 10^{-3}$.
\end{exemple}

\subsection{Les notations avec préfixes}

\begin{center}
\begin{tabular}{c|c|c}
Préfixe & Symbole & $10^n$ \\
\hline
giga & G & $10^9$ \\
méga & M & $10^6$ \\
kilo & k & $10^3$ \\
unité &  & $10^0 = 1$ \\
milli & m & $10^{-3}$ \\
micro & $\mu$ & $10^{-6}$ \\
nano & n & $10^{-9}$
\end{tabular}
\end{center}

\begin{exemple}
$1\text{ km} = 10^3 \text{ m}$.

$1\ \mu\text{m} = 10^{-6}\text{ m}$.
\end{exemple}

\subsection{Application}

\begin{exemple}
Grain de sable :  
$0,000232 = 2,32 \times 10^{-4}$.

Fil de toile d'araignée :  
$6690\text{ nm} = 6,69 \times 10^{-6}\text{ m}$.

Particule de fumée :  
$0,27\ \mu\text{m} = 2,7 \times 10^{-7}\text{ m}$.
\end{exemple}

Comparaison :

\[
2,7 \times 10^{-7} < 6,69 \times 10^{-6} < 2,32 \times 10^{-4}
\]

\end{document}
