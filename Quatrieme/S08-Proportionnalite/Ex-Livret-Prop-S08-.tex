%% Font size %%
\documentclass[11pt]{article}

%% Load the custom package
\usepackage{Mathdoc}

\pagestyle{sectionhead}

\makeatletter
\renewcommand{\sectionmark}[1]{%
\markright{#1}%
}
\makeatother

%% Numéro de séquence %% Titre de la séquence %%
\renewcommand{\centerhead}{Propotionnalité : }

%% Spacing commands %%
\renewcommand{\baselinestretch}{1} \setlength{\parindent}{0pt}

%%\pagestyle{empty}

\begin{document}

\section{Leçon}

\subsection{Multiplications à trous et division}

\begin{propriete}
Compléter une multiplication à trou revient à calculer le résultat d'une division.
\end{propriete}

\begin{exemple}
Si je dois calculer $45 \div 5$, il est plus facile de chercher : $5 \times ? = 45$.  
Ici, la réponse est $9$.
\end{exemple}

\begin{exemple}
$60 \div 10 = 6$ car $6 \times 10 = 60$.
\end{exemple}

\subsection{Situations de proportionnalité}

\begin{definition}
Un tableau est un tableau de proportionnalité lorsque l'on passe d'une ligne à l'autre
en multipliant toujours par le même nombre.
\end{definition}

\begin{vocabulaire}
\begin{enumerate}
\item Ce nombre est appelé le coefficient de proportionnalité.

\item On dira que les deux grandeurs, correspondant à chaque ligne, sont proportionnelles.
\end{enumerate}
\end{vocabulaire}

\begin{exemple}
Une station-service vend du sans-plomb 98 à 2 € le litre.\\
La quantité d’essence et le prix sont donc proportionnels.\\
On a le tableau de proportionnalité :


\medskip

\begin{center}
\begin{tabular}{|l|c|c|c|c|}
\hline
Quantité (en L) & 1 & 5 & 8 & 10 \\ \hline
Prix (en €)     & 2 & 10 & 16 & 20 \\ \hline
\end{tabular}
\end{center}

\end{exemple}


\textbf{Vérifier si un tableau est de proportionnalité :}

Dire si les tableaux suivants sont bien des tableaux de proportionnalité.

\begin{multicols}{2}


\begin{center}
\begin{tabular}{|l|c|c|c|c|}
\hline
Quantité A & 7 & 2 & 4 & 3 \\ \hline
Quantité B & 49 & 14 & 28 & 21 \\ \hline
\end{tabular}
\end{center}


Calculons :

\[
49 \div 7 = 7 \qquad
14 \div 2 = 7
\]
\[
28 \div 4 = 7 \qquad
21 \div 3 = 7
\]

Les coefficients sont égaux : les quantités sont proportionnelles.

\columnbreak

\begin{center}
\begin{tabular}{|l|c|c|c|c|}
\hline
Quantité A & 6 & 4 & 10 & 7 \\ \hline
Quantité B & 18 & 8 & 20 & 21 \\ \hline
\end{tabular}
\end{center}

Calculons :

\[
18 \div 6 = 3 \qquad
8 \div 4 = 2
\]
\[
20 \div 10 = 2 \qquad
21 \div 7 = 3
\]

Les coefficients ne sont pas égaux : les quantités ne sont pas proportionnelles.

\end{multicols}



\newpage
\subsection{Quatrième proportionnelle — coefficient de proportionnalité}

On suppose que les deux quantités sont proportionnelles.\\
Pour compléter le tableau suivant, on calcule le coefficient de proportionnalité.

\begin{multicols}{2}

\begin{tabular}{|l|c|c|}
\hline
Quantité A & 6 & 4 \\ \hline
Quantité B & $x$ & 36 \\ \hline
\end{tabular}

\columnbreak

\[
36 \div 4 = 9
\]
Donc :
\[
6 \times 9 = x
\]
c.a.d. $x = 54$.

\end{multicols}

On a donc :

\begin{center}
\begin{tabular}{|l|c|c|}
\hline
Quantité A & 6 & 4 \\ \hline
Quantité B & 54 & 36 \\ \hline
\end{tabular}
\end{center}

\subsection{Quatrième proportionnelle — passage à l'unité}

On suppose que les deux quantités sont proportionnelles.\\
Pour compléter le tableau suivant, on passe par le calcul d'une unité de la quantité A.

\begin{multicols}{2}

\begin{tabular}{|l|c|c|c|}
\hline
Quantité A & 7 & 2 & 1 \\ \hline
Quantité B & 35 & $x$ & \phantom{000} \\ \hline
\end{tabular}

\columnbreak

\[
35 \div 7 = 5
\]
Donc 1 unité correspond à 5.

\[
2 \times 5 = x \quad \text{c.a.d. } x = 10
\]

\end{multicols}

On a donc :

\begin{center}
\begin{tabular}{|l|c|c|c|}
\hline
Quantité A & 7 & 2 & 1 \\ \hline
Quantité B & 35 & 10 & 5 \\ \hline
\end{tabular}
\end{center}

\subsection{Quatrième proportionnelle — linéarité}

On suppose que les deux quantités sont proportionnelles.\\
Pour compléter le tableau suivant, on utilise les propriétés de linéarité du tableau de proportionnalité.

\begin{multicols}{2}

\begin{center}
\begin{tabular}{|l|c|c|c|c|}
\hline
Quantité A & 7 & 2 & 9 & 14 \\ \hline
Quantité B & 63 & 18 & $x$ & $y$ \\ \hline
\end{tabular}
\end{center}

\columnbreak

\[
7 + 2 = 9 \quad \text{et} \quad 7 \times 2 = 14
\]
\[
63 + 18 = 81 \quad \text{et} \quad 63 \times 2 = 126
\]

Donc :
\[
x = 81 \quad \text{et} \quad y = 126
\]

\end{multicols}

On a donc :

\begin{center}
\begin{tabular}{|l|c|c|c|c|}
\hline
Quantité A & 7 & 2 & 9 & 14 \\ \hline
Quantité B & 63 & 18 & 81 & 126 \\ \hline
\end{tabular}
\end{center}

\newpage

\subsection{Quatrième proportionnelle — produit en croix}

On suppose que les deux quantités sont proportionnelles.\\
Pour compléter les tableaux suivants, on utilise l’égalité des produits en croix.

\begin{multicols}{2}

\begin{tabular}{|l|c|c|}
\hline
Quantité A & 4 & 8 \\ \hline
Quantité B & 12 & ? \\ \hline
\end{tabular}

\vspace{0.75cm}

L’égalité donne :
\[
\frac{12 \times 8}{4}
\]
Donc $ ? = 24$.

\vspace{1cm}

\begin{tabular}{|l|c|c|}
\hline
Quantité A & 6 & ? \\ \hline
Quantité B & 36 & 30 \\ \hline
\end{tabular}

\vspace{0.75cm}

L’égalité donne :
\[
\frac{30 \times 6}{36}
\]
Donc $ ? = 5$.

\columnbreak

\begin{tabular}{|l|c|c|}
\hline
Quantité A & 2 & 6 \\ \hline
Quantité B & ? & 54 \\ \hline
\end{tabular}

\vspace{0.75cm}

L’égalité donne :
\[
\frac{2 \times 54}{6}
\]
Donc $ ? = 18$.

\vspace{1cm}

\begin{tabular}{|l|c|c|}
\hline
Quantité A & ? & 8 \\ \hline
Quantité B & 14 & 56 \\ \hline
\end{tabular}

\vspace{0.75cm}

L’égalité donne :
\[
\frac{14 \times 8}{56}
\]
Donc $ ? = 2$.

\end{multicols}

\subsection{Résolution de problèmes}

\subsubsection*{Énoncé 1 :}


Léa lit sur sa recette de mousse au chocolat que pour 9 personnes il faut 270 g de chocolat.\\
Elle veut adapter sa recette pour 11 personnes.\\
Quelle masse de chocolat doit-elle prévoir ?

\subsubsection*{Correction}

\textbf{1. Tableau}

\begin{center}
\begin{tabular}{|l|c|c|}
\hline
Nombre de personnes & 9 & 11 \\ \hline
Masse de chocolat (en g) & 270 & $x$ \\ \hline
\end{tabular}
\end{center}

\medskip

\textbf{2. Calculs}

Les quantités sont proportionnelles.

\[
270 \div 9 = 30
\]

Donc, pour une personne, il faut 30 g de chocolat.

\[
30 \times 11 = 330
\]

\medskip

\textbf{3. Tableau complété}

\begin{center}
\begin{tabular}{|l|c|c|}
\hline
Nombre de personnes & 9 & 11 \\ \hline
Masse de chocolat (en g) & 270 & 330 \\ \hline
\end{tabular}
\end{center}

\medskip

\textbf{4. Phrase-réponse}

Léa doit prévoir \textbf{330 g de chocolat} pour 11 personnes.

\subsubsection*{Énoncé 2  :}

Elsa a repéré, à l'épicerie, des melons qui l'intéressent.\\
Elle lit que 9 melons coûtent 27 €. Elle veut en acheter 11.\\
Combien va-t-elle dépenser ?

\subsubsection*{Correction}

\textbf{1. Tableau}

\begin{center}
\begin{tabular}{|l|c|c|}
\hline
Nombre de melons & 9 & 11 \\ \hline
Prix (en €) & 27 & $x$ \\ \hline
\end{tabular}
\end{center}

\medskip

\textbf{2. Calculs}

Les quantités sont proportionnelles.

\[
27 \div 9 = 3
\]

Donc, un melon coûte 3 €.

\[
3 \times 11 = 33
\]

\medskip

\textbf{3. Tableau complété}

\begin{center}
\begin{tabular}{|l|c|c|}
\hline
Nombre de melons & 9 & 11 \\ \hline
Prix (en €) & 27 & 33 \\ \hline
\end{tabular}
\end{center}

\medskip

\textbf{4. Phrase-réponse}

Elsa va dépenser \textbf{33 €} pour acheter 11 melons.


\newpage

\section{Multiplications et divisions}

\begin{exercice}[3][Calculer les multiplications suivantes]
\begin{multicols}{4}
\begin{enumerate}
\item $8 \times 6 =$ \ldots\ldots\ldots
\item $7 \times 8 =$ \ldots\ldots\ldots
\item $9 \times 6 =$ \ldots\ldots\ldots
\item $4 \times 7 =$ \ldots\ldots\ldots
\item $8 \times 5 =$ \ldots\ldots\ldots
\item $9 \times 4 =$ \ldots\ldots\ldots
\item $6 \times 6 =$ \ldots\ldots\ldots
\item $5 \times 9 =$ \ldots\ldots\ldots
\item $5 \times 5 =$ \ldots\ldots\ldots
\item $6 \times 8 =$ \ldots\ldots\ldots
\item $6 \times 5 =$ \ldots\ldots\ldots
\item $4 \times 4 =$ \ldots\ldots\ldots
\end{enumerate}
\end{multicols}
\end{exercice}

\begin{exercice}[3][Calculer les multiplications suivantes]
\begin{multicols}{4}
\begin{enumerate}
\item $11 \times 9 =$ \ldots\ldots\ldots
\item $8 \times 9 =$ \ldots\ldots\ldots
\item $11 \times 10 =$ \ldots\ldots\ldots
\item $10 \times 10 =$ \ldots\ldots\ldots
\item $8 \times 10 =$ \ldots\ldots\ldots
\item $9 \times 8 =$ \ldots\ldots\ldots
\item $10 \times 8 =$ \ldots\ldots\ldots
\item $8 \times 11 =$ \ldots\ldots\ldots
\item $8 \times 8 =$ \ldots\ldots\ldots
\item $11 \times 8 =$ \ldots\ldots\ldots
\item $10 \times 9 =$ \ldots\ldots\ldots
\item $9 \times 10 =$ \ldots\ldots\ldots
\end{enumerate}
\end{multicols}
\end{exercice}

\begin{exercice}[2][Compléter les multiplications suivantes]
\begin{multicols}{4}
\begin{enumerate}
\item $ \ldots\ldots \times 8 =40$
\item $ \ldots\ldots \times 8 =48$
\item $6\times  \ldots\ldots =12$
\item $ \ldots\ldots \times 5 =20$
\item $6\times  \ldots\ldots =42$
\item $ \ldots\ldots \times 5 =30$
\item $4\times  \ldots\ldots =16$
\item $4\times  \ldots\ldots =36$
\item $ \ldots\ldots \times 2 =8$
\item $ \ldots\ldots \times 3 =12$
\end{enumerate}
\end{multicols}
\end{exercice}

\begin{exercice}[3][Compléter les multiplications suivantes]
\begin{multicols}{4}
\begin{enumerate}
\item $8\times  \ldots\ldots =64$
\item $7\times  \ldots\ldots =56$
\item $ \ldots\ldots \times 2 =16$
\item $9\times  \ldots\ldots =54$
\item $ \ldots\ldots \times 6 =42$
\item $9\times  \ldots\ldots =72$
\item $8\times  \ldots\ldots =40$
\item $7\times  \ldots\ldots =28$
\item $ \ldots\ldots \times 7 =56$
\item $9\times  \ldots\ldots =18$
\end{enumerate}
\end{multicols}
\end{exercice}

\begin{exercice}[3][Calculer les divisions suivantes]
\begin{multicols}{4}
\begin{enumerate}
\item $8\times  \ldots\ldots =64$
\item $7\times  \ldots\ldots =56$
\item $ \ldots\ldots \times 2 =16$
\item $9\times  \ldots\ldots =54$
\item $ \ldots\ldots \times 6 =42$
\item $9\times  \ldots\ldots =72$
\item $8\times  \ldots\ldots =40$
\item $7\times  \ldots\ldots =28$
\item $ \ldots\ldots \times 7 =56$
\item $9\times  \ldots\ldots =18$
\end{enumerate}
\end{multicols}
\end{exercice}

\newpage

\section{Reconnaitre une situation de proportionnalité}


\end{document}
