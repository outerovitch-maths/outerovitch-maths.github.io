%% Font size %%
\documentclass[11pt]{article}

%% Load the custom package
\usepackage{Mathdoc}

%% Numéro de séquence %% Titre de la séquence %%
\renewcommand{\centerhead}{}

%% Spacing commands %%
\renewcommand{\baselinestretch}{1} \setlength{\parindent}{0pt}

\begin{document}

u
\begin{exercice}[1]

\medskip

Voici les tailles, en cm, de $29$ jeunes plants de blé $10$ jours après la mise en germination.

\medskip

\begin{tabularx}{\linewidth}{|l|*{9}{>{\centering \arraybackslash}X|}}\hline
Taille (en cm) &0 &10 &15 &17 &18 &19 &20 &21 &22\\ \hline
Effectif &1 &4 &6 &2 &3 &3 &4 &4 &2\\ \hline
\end{tabularx}

\medskip

\begin{enumerate}
\item Calculer la taille moyenne d'un jeune plant de blé.
\item 
\begin{enumerate}
\item Déterminer la médiane de cette série.
\item Interpréter ce résultat.
\end{enumerate}
\end{enumerate}



\end{exercice}


\begin{exercice}[1]

\medskip

Une entreprise de fabrication de bonbons souhaite vérifier la qualité de sa nouvelle
machine de conditionnement. Cette machine est configurée pour emballer environ $60$
bonbons par paquet. Pour vérifier sa bonne configuration, on a étudié $500$~paquets à
la sortie de cette machine.

\medskip

\textbf{Document 1 : Résultats de l'étude}

\begin{center}
\begin{tabularx}{\linewidth}{|m{2cm}|*{9}{>{\centering \arraybackslash}X|}}\hline
Nombre de bonbons	&56 &57 &58 &59 &60 	&61 &62 &63 &64\\ \hline
Effectifs 			&4 	&36 &53 &79 &145 	&82 &56 &38 &7\\ \hline
\end{tabularx}
\end{center}

\textbf{Document 2 : Critères de qualité}

\medskip

Pour être validée par l'entreprise, la machine doit respecter trois critères de qualité:

\setlength\parindent{8mm}
\begin{itemize}
\item[$\bullet~~$] Le nombre moyen de bonbons dans un paquet doit être compris entre 59,9 et
60,1.
\item[$\bullet~~$] L'étendue de la série doit être inférieure ou égale à $10$.
\item[$\bullet~~$] L'écart interquartile (c'est-à-dire la différence entre le troisième quartile et le premier quartile) doit être inférieur ou égal à 3.
\end{itemize}
\setlength\parindent{0mm} 

La nouvelle machine respecte-t-elle les critères de qualité ?

\emph{Il est rappelé que, pour l'ensemble du sujet, les réponses doivent être justifiées.}

\bigskip


\end{exercice}


\begin{exercice}[1]

\medskip

Les jeux Olympiques (JO) d'été ont généralement lieu tous les 4 ans.

Dans cet exercice, on s'intéresse aux coûts d'organisation des dernières éditions des JO d'été.

On rappelle que le coût est l'ensemble des dépenses entraînées par l'organisation des JO.

On précise que :

\begin{itemize}
\item le \textbf{coût prévisionnel} désigne les dépenses prévues par les organisateurs avant l'édition des JO ;
\item le \textbf{coût réel} désigne les dépenses réelles qui ont été nécessaires pour l'organisation des JO.
\end{itemize}

Le graphique ci-dessous compare ces deux coûts pour les dernières éditions des JO d'été.

\medskip

\begin{tabularx}{\linewidth}{|>{\centering \arraybackslash}X|}\hline
\textbf{Comparaison entre le coût prévisionnel et le coût réel de chaque édition des JO depuis 1992, en milliard d'euros}\\

\psset{xunit=0.9cm,hatchsep=2pt,yunit=1.25cm}
\begin{pspicture}(0,-1)(15,4)
%\psgrid
\psline(0,1.4)(15,1.4)
\psframe*[linecolor=gray](0.5,1.4)(0.7,1.575)\uput[u](0.6,1.575){\footnotesize 3,5}
\psframe*[linecolor=gray](2.3,1.4)(2.5,1.49)\uput[u](2.4,1.49){\footnotesize 1,8}
\psframe*[linecolor=gray](4.1,1.4)(4.3,1.55)\uput[u](4.2,1.55){\footnotesize 3}
\psframe*[linecolor=gray](5.9,1.4)(6.1,1.65)\uput[u](6.,1.65){\footnotesize 5,3}
\psframe*[linecolor=gray](7.7,1.4)(7.9,1.53)\uput[u](7.8,1.53){\footnotesize 2,6}
\psframe*[linecolor=gray](9.5,1.4)(9.7,1.64)\uput[u](9.6,1.64){\footnotesize 4,8}
\psframe*[linecolor=gray](11.3,1.4)(11.5,1.85)\uput[u](11.4,1.85){\footnotesize 9}
\psframe*[linecolor=gray](13.1,1.4)(13.3,2.05)\uput[u](13.2,2.05){\footnotesize 13}
\psframe[fillstyle=hlines](1,1.4)(1.2,1.865)\uput[u](1.1,1.865){\footnotesize 9,3}
\psframe[fillstyle=hlines](2.8,1.4)(3,1.55)\uput[u](2.9,1.55){\footnotesize 2,3}
\psframe[fillstyle=hlines](4.6,1.4)(4.8,1.65)\uput[u](4.7,1.65){\footnotesize 5,5}
\psframe[fillstyle=hlines](6.4,1.4)(6.6,1.9)\uput[u](6.5,1.9){\footnotesize 10}
\psframe[fillstyle=hlines](8.2,1.4)(8.4,2.95)\uput[u](8.3,2.95){\footnotesize 31}
\psframe[fillstyle=hlines](10,1.4)(10.2,1.9)\uput[u](10.1,1.9){\footnotesize 11}
\psframe[fillstyle=hlines](11.8,1.4)(12,2.2)\uput[u](11.9,2.2){\footnotesize 16,5}
\psframe[fillstyle=hlines](13.6,1.4)(13.8,2)\uput[u](13.7,2){\footnotesize 12,1}
\psframe*[linecolor=gray](5,3.4)(5.3,3.7)\psframe[fillstyle=hlines](10,3.4)(10.3,3.7)
\rput(7.2,3.5){Coût prévisionnel}\rput(11.8,3.5){Coût réel}
\rput(0.8,1){Barcelone}\rput(0.8,0.5){1992}
\rput(2.6,1){Atlanta}\rput(2.6,0.5){1996}
\rput(4.4,1){Sydney}\rput(4.4,0.5){2000}
\rput(6.2,1){Athènes}\rput(6.2,0.5){2004}
\rput(8,1){Pékin}\rput(8,0.5){2008}
\rput(9.8,1){Londres}\rput(9.8,0.5){2012}
\rput(11.6,1){Rio de}\rput(11.6,0.5){Janeiro}\rput(11.6,0){2016}
\rput(13.4,1){Tokyo}\rput(13.4,0.5){2021}
\rput(7.5,-0.5){\emph{La crise sanitaire de la Covid-$19$ a décalé à $2021$ les Jeux Olympiques de Tokyo prévus en $2020$.}}
\end{pspicture}\\ \hline
\end{tabularx}

\medskip

\begin{enumerate}
\item Entre 1992 et 2021, combien d'éditions ont eu un coût réel supérieur ou égal à $10$ milliards d'euros ?
\item Calculer le pourcentage d'augmentation entre le coût prévisionnel et le coût réel lors de l'édition des JO de Rio de Janeiro 2016, arrondi à l'unité.
\item Montrer que le coût réel moyen entre 1992 et 2021 est $12,2$~milliards d'euros, arrondi au dixième de milliard.
\item \textbf{Questions de journalistes}
\begin{enumerate}
\item Un journaliste mentionne que le coût réel moyen des JO sur la période 1992 à 2021 est de $12,2$~milliards d'euros. Il poursuit en affirmant: \og Cela signifie que la moitié des éditions entre 1992 et 2021 ont un coût réel supérieur à 12,2 milliards d'euros. \fg

Que penser de cette affirmation ?
\item Le coût prévisionnel moyen entre 1992 et 2024 est de l'ordre de $5,5$ milliards d'euros.

Une journaliste cherche à connaître le coût prévisionnel des JO de Paris 2024 pour préparer son intervention télévisée.

Calculer le coût prévisionnel des JO de Paris 2024 qu'elle devrait annoncer.
\end{enumerate}
\end{enumerate}

\bigskip


\end{exercice}


\begin{exercice}[1]

\medskip

\textbf{Les deux parties sont indépendantes}

\medskip

\textbf{Partie A : évolution du nombre de visiteurs sur un site touristique}

\medskip

\begin{enumerate}
\item Le diagramme ci-dessous représente le nombre de visiteurs par an de 2010 à 2021 sur ce site.

\begin{center}
\psset{xunit=0.95cm,yunit=0.000018cm}
\begin{pspicture}(-1.5,-70000)(12.5,450000)
\psaxes[linewidth=0pt,Dx=50,Dy=500000,labelFontSize=\scriptstyle](0,0)(0,0)(12.5,450000)
\psframe[fillstyle=solid,fillcolor=purple](0.7,0)(1.3,300000)\psframe[fillstyle=solid,fillcolor=purple](1.7,0)(2.3,310000)
\psframe[fillstyle=solid,fillcolor=purple](2.7,0)(3.3,320000)\psframe[fillstyle=solid,fillcolor=purple](3.7,0)(4.3,320000)
\psframe[fillstyle=solid,fillcolor=purple](4.7,0)(5.3,300000)\psframe[fillstyle=solid,fillcolor=purple](5.7,0)(6.3,320000)
\psframe[fillstyle=solid,fillcolor=purple](6.7,0)(7.3,330000)\psframe[fillstyle=solid,fillcolor=purple](7.7,0)(8.3,350000)
\psframe[fillstyle=solid,fillcolor=purple](8.7,0)(9.3,360000)\psframe[fillstyle=solid,fillcolor=purple](9.7,0)(10.3,400000)
\psframe[fillstyle=solid,fillcolor=purple](10.7,0)(11.3,187216)\psframe[fillstyle=solid,fillcolor=purple](11.7,0)(12.3,219042)
\uput[d](6.25,-50000){Année}\rput{90}(-1.5,225000){Nombre de visiteurs}
\rput(6.25,475000){Nombre de visiteurs sur le site touristique par année
}
\multido{\n=1+1,\na=2010+1}{12}{\uput[d](\n,0){\small \na}}
\multido{\n=0+50000}{10}{\psline[linewidth=0.6pt](0,\n)(12.5,\n)\uput[l](0,\n){\footnotesize \np{\n}}}
\end{pspicture}
\end{center}

\begin{enumerate}
\item Quel a été le nombre de visiteurs en 2010 ? Aucune justification n'est attendue.
\item En quelle année le nombre de visiteurs a-t-il été le plus élevé ? Aucune justification n'est attendue.
\end{enumerate}
\item Le tableau ci-dessous indique le nombre de visiteurs sur le site touristique de cette ville en 2020 et en 2021 :
\begin{center}
\begin{tabularx}{0.75\linewidth}{|l|*{2}{>{\centering \arraybackslash}X|}}\hline
Année 				&2020 		&2021\\ \hline
Nombre de visiteurs &\np{187216}&\np{219042}\\ \hline
\end{tabularx}
\end{center}

Le maire de cette ville avait pour objectif que le nombre de visiteurs progresse d'au moins 15\,\% entre 2020 et 2021.

L'objectif a-t-il été atteint ?
\end{enumerate}

\medskip

\textbf{Partie B :  étude des prix des hôtels de cette ville}

\medskip

Sur une période donnée, on relève les prix facturés pour une nuit par les hôtels de cette ville.

\begin{center}
\begin{tabularx}{\linewidth}{|m{4.5cm}|*{8}{>{\centering \arraybackslash}X|}}\hline
Prix facturés pour une nuit (en euro)&60 &80 &85 &90 &110 &120 &350 &500\\ \hline
Effectif &\np{1200} &\np{1350} &\np{1000} &\np{1100} &\np{1200} &\np{1300} &900 &300\\ \hline
\end{tabularx}
\end{center}

\begin{enumerate}[resume]
\item Déterminer l'étendue des prix facturés.
\item Quelle est la moyenne des prix facturés pour une nuit ? Arrondir à l'euro près.
\item L'association des hôteliers de cette ville cherche à attirer des touristes et annonce : \og Dans les hôtels de notre ville, au moins la moitié des nuits est facturée à moins de $100$~\euro{} \fg. Est-ce vrai ?
\end{enumerate}

\bigskip


\end{exercice}


\end{document}
