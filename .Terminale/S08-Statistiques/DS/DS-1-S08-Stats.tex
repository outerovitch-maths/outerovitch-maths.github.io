%% Font size %%
\documentclass[11pt]{article}

%% Load the custom package
\usepackage{Mathdoc}

%% Numéro de séquence %% Titre de la séquence %%
\renewcommand{\centerhead}{Statistiques à deux variables}

%% Spacing commands %%
\renewcommand{\baselinestretch}{1} \setlength{\parindent}{0pt}


\begin{document}

\entetedevoirs{5}

\begin{exercicedevoir}
Un professeur souhaite analyser la relation entre le temps de révision des élèves avant un contrôle de mathématiques et la note obtenue. Pour cela, il a relevé les données de 8 élèves, en notant le temps passé à réviser (en heures) et la note obtenue sur 20.

\begin{center}
\begin{tabular}{|c|c|c|c|c|c|c|c|c|}
\hline
Élève & 1 & 2 & 3 & 4 & 5 & 6 & 7 & 8 \\
\hline
Temps de révision (en h) & 2 & 4 & 1 & 3 & 5 & 6 & 3 & 4 \\
\hline
Note obtenue (/20) & 15 & 19 & 10 & 18 & 18 & 20 & 17 & 20 \\
\hline
\end{tabular}
\end{center}


\begin{center}
\begin{tikzpicture}[scale=1.4]
% Tracer les axes en gris
\draw[gray!80, thin] (0, 0) grid (7, 7);  
% Grille de la zone d'intérêt

% Repère
\draw[->] (0,0) -- (7,0) node[right] {Temps};
\draw[->] (0,0) -- (0, 7) node[above] {Note};

% Graduations
\foreach \x in {1,2,3,4,5,6}
\draw (\x,0.1) -- (\x,-0.1) node[below] { };

\foreach \y in {5,10,15,20,25,30}
\draw (0.1,\y/5) -- (-0.1,\y/5) node[left] { };

\end{tikzpicture}
\end{center}


\begin{enumerate}
\item Compléter les axes du repère orthogonal ci-dessus.
\item Placer les points du nuage de points sur le graphique.
\item Calculer les coordonnées du point moyen $M(x_M;y_M)$ et le
placer sur le graphique. \\ \dtf \\ \dtf \\ \dtf \\ \dtf \\ \dtf \\ \dtf
\end{enumerate}
\end{exercicedevoir}

\nonewpage

\end{document}
