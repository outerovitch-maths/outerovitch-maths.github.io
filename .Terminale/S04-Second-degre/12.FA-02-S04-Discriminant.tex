%% Font size %%
\documentclass[11pt]{article}

%% Load the custom package
\usepackage{Mathdoc}

%% Numéro de séquence %% Titre de la séquence %%
\renewcommand{\centerhead}{Chap. 4 : Polynôme du second degré - Discriminant}

%% Spacing commands %%
\renewcommand{\baselinestretch}{1}
\setlength{\parindent}{0pt}

\begin{document}

\section{Calcul de discriminant}

\begin{exercice}[0][Exercice Corrigé.]
\textbf{Calculer le discriminant de cette expression :}

 $A(x) = 5x^2-4x-1$

Correction : \\
\textit{
$\Delta_A = b^2 - 4 \times a \times c$ \\
$\Delta_A = \left(-4\right)^2-4\times5\times\left(-1\right)$ \\
$\Delta_A = 36$}
\end{exercice}

\begin{exercice}[1][Calculer le discriminant de chacune de ces expressions.]
  \begin{multicols}{2}
    \begin{enumerate}
    \item $A(x) = 5x^2+4x+2$ \\ \dtf \\ \dtf
    \item $B(x) = 2x^2+2x+4$ \\ \dtf \\ \dtf
    \item $C(x) = 5x^2+5x+4$ \\ \dtf \\ \dtf
    \item $D(x) = 5x^2+5x+5$ \\ \dtf \\ \dtf
    \end{enumerate}
  \end{multicols}
\end{exercice}

\begin{exercice}[2][Calculer le discriminant de chacune de ces expressions.]
  \begin{multicols}{2}
    \begin{enumerate}
    \item $A(x) = -2x^2+2x+1$ \\ \dtf \\ \dtf
    \item $B(x) = -3x^2+5$ \\ \dtf \\ \dtf
    \item $C(x) = -3x^2-3x-5$ \\ \dtf \\ \dtf
    \item $D(x) = -3x^2+x-1$ \\ \dtf \\ \dtf
    \end{enumerate}
  \end{multicols}
\end{exercice}

\begin{exercice}[3][Calculer le discriminant de chacune de ces expressions.]
  \begin{multicols}{2}
    \begin{enumerate}
    \item $A(x) = -x^2+\dfrac{1}{2}x$ \\ \dtf \\ \dtf \\ \dtf
    \item $B(x) = -\dfrac{6}{2}x^2-\dfrac{9}{2}x+\dfrac{2}{3}$ \\ \dtf
      \\ \dtf \\ \dtf
    \item $C(x) = -\dfrac{5}{3}x-\dfrac{6}{5}$ \\ \dtf \\ \dtf
    \item $D(x) = -\dfrac{7}{5}x^2+\dfrac{1}{3}x+\dfrac{7}{5}$ \\ \dtf
      \\ \dtf \\ \dtf
    \end{enumerate}
  \end{multicols}
\end{exercice}

\newpage

\section{Déterminer le nombre de solutions}

\begin{exercice}[0][Exercice Corrigé.]
\textbf{Calculer le discriminant et déterminer le nombre de solutions de cette
équation dans $\mathbb{R}$.} \\
$2x^2-1=0$

Correction :\\
\begin{minipage}[t]{\linewidth}$\Delta = 0^2-4\times2\times(-1)=8$\\$\Delta>0$ \textit{donc l'équation admet \underline{deux solutions}}.\end{minipage}
\end{exercice}

\begin{exercice}[1][Nombre de solutions.]
Calculer le discriminant et déterminer le nombre de solutions de cette
équation dans $\mathbb{R}$.
\begin{multicols}{2}
  \begin{enumerate}
  \item $x^2-x-2=0$ \\ \dtf \\ \dtf
  \item $-7x-3x^2-9=0$ \\ \dtf \\ \dtf
  \item $-x-6+x^2=0$ \\ \dtf \\ \dtf
  \item $-10x^2-x-9=0$ \\ \dtf \\ \dtf
  \end{enumerate}
\end{multicols}
\end{exercice}

\begin{exercice}[2][Nombre de solutions.]
Calculer le discriminant et déterminer le nombre de solutions de cette
équation dans $\mathbb{R}$.
\begin{multicols}{2}
  \begin{enumerate}
  \item $-16+8x-x^2=0$ \\ \dtf \\ \dtf
  \item $x^2+13x+40=0$ \\ \dtf \\ \dtf
  \item $-2-9x^2+x=0$ \\ \dtf \\ \dtf
  \item $7x^2-5x+6=0$ \\ \dtf \\ \dtf
  \end{enumerate}
\end{multicols}
\end{exercice}

\begin{exercice}[3][Nombre de solutions.]
Calculer le discriminant et déterminer le nombre de solutions de cette
équation dans $\mathbb{R}$.
\begin{multicols}{2}
  \begin{enumerate}
  \item $-2x^2-\dfrac{32}{25}+\dfrac{16}{5}x=0$ \\ \dtf \\ \dtf \\ \dtf 
  \item $-5x^2+8x+6=0$ \\ \dtf \\ \dtf \\ \dtf 
  \item $-2x^2-10=0$ \\ \dtf \\ \dtf \\ \dtf 
  \item $-10x^2+2x-8=0$ \\ \dtf \\ \dtf \\ \dtf 
  \end{enumerate}
\end{multicols}
\end{exercice}

\newpage

\section{Résolution d'équation}

\begin{exercice}
\textbf{Résoudre l'équation suivante.}\\
 $x^2-2x-3=0$ \\
Correction :\\
\textit{On calcule le discriminant :} \\
\[\Delta=\left(-2\right)^2-4\times1\times\left(-3\right)=16.\] \\
\textit{On a $\Delta=16$, donc l'équation a \underline{deux solutions}. \\
Les solutions sont données par la formule :
}\[x_{1,2}=\dfrac{-b \pm \sqrt{\Delta}}{2\times a}\]
\textit{Donc :}
\[x_{1}=\dfrac{-\left(-2\right)+\sqrt{16}}{2\times1} \quad
x_{2}=\dfrac{-\left(-2\right)-\sqrt{16}}{2\times1}\]\\
\textit{En calculant, on obtient} ${S = \left\{-1;3\right\}}$.
\end{exercice}


\begin{exercice}[1][Résoudre les équations suivantes.]
\begin{multicols}{2}
  $x^2+2x+1=0$  \\ \\ \encart{3cm}
  $x^2-4x=0$ \\ \\ \encart{3cm}
\end{multicols}
\end{exercice}

\begin{exercice}[2][Résoudre les équations suivantes.]
\begin{multicols}{2}
$15x+56+x^2=0$ \\ \\ \encart{3cm}
$-7x^2-2-7x=0$ \\ \\ \encart{3cm}
\end{multicols}
\end{exercice}

\newpage

\begin{exercice}[2][Résoudre les équations suivantes.]
\begin{multicols}{2}
$40x-200=2x^2$ \\ \\ \encart{3cm}
$-x=-x^2+42$ \\ \\ \encart{3cm}
\end{multicols}
\end{exercice}

\begin{exercice}[3][Résoudre les équations suivantes.]
\begin{multicols}{2}
$-14x-\dfrac{49}{2}-2x^2=0$ \\ \\ \encart{3cm}
$\dfrac{3}{2}x^2-\dfrac{63}{100}+\dfrac{33}{20}x=0$ \\ \\ \encart{3cm}
\end{multicols}
\end{exercice}

\begin{exercice}[3][Résoudre les équations suivantes.]
\begin{multicols}{2}
$3x^2-\dfrac{60}{3}x=-\dfrac{300}{9}$ \\ \\ \encart{3cm}
$-3x^2=\dfrac{141}{6}x-\dfrac{189}{6}$ \\ \\ \encart{3cm}
\end{multicols}
\end{exercice}

\begin{exercice}[4][Résoudre les équations suivantes.]
\begin{multicols}{2}
 $-\dfrac{69}{3}x+\dfrac{297}{9}+\dfrac{10}{3}x^2=-\dfrac{1}{3}-3x+\dfrac{1}{3}x^2$ \\ \\ \encart{3cm}
  $\dfrac{2}{3}x^2+\dfrac{8}{4}x=4x-\dfrac{4}{3}x^2-\dfrac{1}{2}$ \\ \\ \encart{3cm}
\end{multicols}
\end{exercice}

\end{document}
