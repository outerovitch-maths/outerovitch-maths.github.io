%% Font size %%
\documentclass[11pt]{article}

%% Load the custom package
\usepackage{Mathdoc}

%% Numéro de séquence %% Titre de la séquence %%
\renewcommand{\centerhead}{Fiche d'exercice suites géométrique}

%% Spacing commands %%
\renewcommand{\baselinestretch}{1}
\setlength{\parindent}{0pt}

\begin{document}

\section{Pourcentages d'évolutions}

\begin{multicols}{2}{2}
  \begin{exercice}
    \begin{enumerate}
    \item Dans une entreprise de $450$ salariés, il y a $24\,\%$ de
      cadres. \\Combien y a-t-il de cadres dans cette entreprise ?
    \item $1\,380$ personnes assistent à un concert. $60~\%$ ont moins
      de $18$ ans. \\Calculer le nombre de personnes mineures dans le
      public.
    \item Une réserve de protection d'oiseaux contient $2\,140$
      individus d'oiseaux. On dénombre $30~\%$ de pipits
      farlouse.\\Quel est le nombre de pipits farlouse ?
    \item Le cadeau commun que nous souhaitons faire à Nora coûte $40$
      €. Je participe à hauteur de $10~\%$ du prix total. \\Combien
      ai-je donné pour le cadeau de Nora ?
    \end{enumerate}
  \end{exercice}

  \begin{exercice}
    \begin{enumerate}
    \item Il y a 6 ans, la population d'une ville était de $82\,000$
      habitants. Depuis, elle a augmenté de $10~\%$. Calculer le
      nombre d'habitants actuel de cette ville.
    \item Un lycée avait $900$ élèves en 2023. Depuis, le nombre
      d'élèves a augmenté de $9~\%$. Calculer le nombre d'élèves dans
      ce lycée cette année.
    \item Un article coûtait $99$ €~et son prix a augmenté de
      $20~\%$. Calculer son nouveau prix.
    \item Le prix de mon ordinateur était de $750$ €~l'année dernière
      et il a augmenté de $12~\%$. Calculer son nouveau prix.
    \end{enumerate}
  \end{exercice}
\end{multicols}
\begin{exercice}
\begin{enumerate}
	\item Après une augmentation de $9~\%$ le prix de ma taxe d'habitation est maintenant $1\,315{,}63$ €. Calculer son prix avant l'augmentation.
	\item Depuis 2023 le nombre d'élèves d'un lycée a augmenté de $15~\%$. Il y a maintenant $483$ élèves. Calculer le nombre d'élèves en 2023 dans cet établissement.
	\item En 13 ans, la population d'une ville a diminué de $19~\%$ et est maintenant $129\,600$ habitants. Calculer sa population d'il y a 13 ans.
	\item Après une augmentation de $10~\%$ un article coûte maintenant $3{,}63$ €. Calculer son prix avant l'augmentation.
\end{enumerate}
\end{exercice}

\newpage

\section{Suites géométriques }

\begin{multicols}{2}
  \begin{exercice}
    \begin{enumerate}
    \item Soit $(w_n)$ une suite géométrique de raison $q$ définie
      pour tout $n\in \mathbb{N}$, telle que $w_0=7$ et $q=-3$.
      Donner l'expression de $w_n$ en fonction de $n$.
    \item Soit $(u_n)$ une suite géométrique de raison $q$ définie
      pour tout $n\in \mathbb{N}$, telle que $u_0=2$ et $q=-15$.
      Donner l'expression de $u_n$ en fonction de $n$.
    \item Soit $(w_n)$ une suite arithmétique de raison $r$ définie
      pour tout $n\in\mathbb{N}$, telle que $w_0=9$ et $r=2$.  Donner
      l'expression de $w_n$ en fonction de $n$.
    \item Soit $(v_n)$ une suite arithmétique de raison $r$ définie
      pour tout $n\in\mathbb{N}$, telle que $v_0=9$ et $r=14$.  Donner
      l'expression de $v_n$ en fonction de $n$.
    \end{enumerate}
  \end{exercice}

  \begin{exercice}
    \begin{enumerate}
    \item $(v_n)$ est une suite géométrique de raison $q=1{,}5$ et de premier terme $v_0=-10$.\\
      Calculer $v_{5}$.
    \item $(w_n)$ est une suite géométrique de raison $q=-1{,}6$ et de premier terme $w_0=-9$.\\
      Calculer $w_{10}$.
    \item $(w_n)$ est une suite géométrique de raison $q=-0{,}9$ et de premier terme $w_0=4$.\\
      Calculer $w_{6}$.
    \item $(v_n)$ est une suite géométrique de raison $q=1{,}4$ et de premier terme $v_0=-8$.\\
      Calculer $v_{7}$.
    \end{enumerate}
  \end{exercice}
\end{multicols}

\begin{exercice}
\begin{enumerate}
	\item Soit $w$ la suite géométrique de premier terme $w_1 = 9$ et de raison $0{,}4$.\\Calculer $\displaystyle S = w_1 + w_2 + ... + w_{12} =\sum_{k=1}^{12}w_k$ et donner un arrondi au millième près.
	\item Soit $w$ la suite géométrique de premier terme $w_1 = 2$ et de raison $0{,}4$.\\Calculer $\displaystyle S = w_1 + w_2 + ... + w_{10} =\sum_{k=1}^{10}w_k$ et donner un arrondi au millième près.
	\item Soit $v$ la suite géométrique de premier terme $v_0 = 10$ et de raison $1{,}9$.\\Calculer $\displaystyle S = v_0 + v_1 + ... + v_{12} =\sum_{k=0}^{12}v_k$ et donner un arrondi au millième près.
	\item Soit $w$ la suite géométrique de premier terme $w_0 = 10$ et de raison $0{,}3$.\\Calculer $\displaystyle S = w_0 + w_1 + ... + w_{12} =\sum_{k=0}^{12}w_k$ et donner un arrondi au millième près.
\end{enumerate}
\end{exercice}

\newpage

\section{Problèmes}

\begin{exercice}
Un entrepreneur décide d'investir une somme de 5 000 € dans un compte bancaire offrant un taux d'intérêt annuel de 4 \%.

\begin{enumerate}
    \item \textbf{Calcul du premier terme de la suite :}
    \begin{itemize}
        \item Au début de la première année, l'entrepreneur dépose 5 000 € sur son compte.
        \item À la fin de chaque année, les intérêts sont ajoutés au capital de départ.
    \end{itemize}

    \item \textbf{Modélisation de la suite :} Soit $u_n$ le capital au bout de $n$ années. Exprimez $u_{n+1}$ en fonction de $u_n$ et du taux d'intérêt. 
    \item \textbf{Terme explicite :} En déduire l'expression explicite de $u_n$ en fonction de $n$.
    \item \textbf{Évolution du capital :} Calculez le capital après 10
      ans.
    \item \textbf{Comparaison avec un autre taux d’intérêt :} \\
    Supposons qu'une autre banque offre un taux d’intérêt annuel de 5 \%.
    Calculez la valeur de l’investissement après 10 ans si le taux était de 5 \% au lieu de 4 \%, et comparez les deux évolutions.
\end{enumerate}
\end{exercice}

\begin{exercice}
Un entrepreneur souhaite financer un projet en mettant de côté une somme d'argent chaque mois. Il décide de commencer avec une somme de 200 € le premier mois et d'augmenter chaque mois le montant mis de côté de 5\,\% par rapport au mois précédent.

\begin{enumerate}
    \item \textbf{Détermination des termes de la suite :} \\
    Soit \( u_n \) la somme mise de côté le \( n \)-ième mois. Exprimez \( u_n \) en fonction de \( n \) et de \( u_1 \).
    
    \item \textbf{Expression de la somme des \( n \) premiers termes :} \\
    L'entrepreneur souhaite savoir combien il aura mis de côté au bout de \( n \) mois. Exprimez la somme \( S_n \) des \( n \) premiers termes de cette suite en fonction de \( n \), \( u_1 \) et du taux d’évolution.
    
    \item \textbf{Application numérique :} \\
    Calculez le montant total mis de côté au bout de 12 mois.
\end{enumerate}
\end{exercice}

\end{document}
