%% Font size %%
\documentclass[11pt]{article}

%% Load the custom package
\usepackage{Mathdoc}

%% Numéro de séquence %% Titre de la séquence %%
\renewcommand{\centerhead}{Proportionality - Evaluation 1}

%% Spacing commands %%
\renewcommand{\baselinestretch}{1} \setlength{\parindent}{0pt}

\begin{document}

\phantom{0}
\vspace{-0.5cm}

\textbf{First Name}: \dotfill \\
\textbf{Last Name}: \dotfill \\
\textbf{Class}: \dotfill

\begin{center}
\textit{Version A} \\
\textit{Duration: 30 minutes} \\
\textit{Total: 10 points} \\
\textit{The use of a calculator is permitted.} \\
\textit{The use of rough work paper is strongly recommended.} \\
\textit{Answers must be written on a separate answer sheet.} \\
\end{center}

\textbf{Exercise 1:} \hfill \textbf{\ldots ~/ 10 points} \\

\textbf{Context:}  
A walker uses his watch to find that he walks 2 km in 30 minutes and 20 km in 5 hours.

\medskip

\textbf{Questions:}

\begin{enumerate}
    \item Is this situation a \textbf{proportional relationship} between distance and time? \\
    Justify your answer. \hfill (2 points)

    \vspace{2em}

    \item Draw and then fill in a \textbf{table of proportionality} for this situation. \hfill (2 points)

    \vspace{2em}

    \item Using the table, find the time needed to walk 1 km at the same speed. \\
    Show your calculations. \hfill (1 point)

    \vspace{2em}

    \item Using the table, find the distance walked in 2 hours at the same speed. \\
    Show your calculations. \hfill (1 point)

    \vspace{2em}

    \item Find the time needed to walk 20 km at the same speed. \\ 
    You will have to use the \textbf{cross product method}. \\
    Show your calculations. \hfill (3 points)

    \vspace{2em}
\end{enumerate}

\textbf{Reminders:}  \\
Use the \textbf{table of proportionality}.  \\
Check the \textbf{constant of proportionality}.  \\
1 hour \textbf{=} 60 minutes. \\
Write a \textbf{full sentence} for each answer.

\newpage

\textbf{First Name}: \dotfill \\
\textbf{Last Name}: \dotfill \\
\textbf{Class}: \dotfill

\begin{center}
\textit{Version B} \\
\textit{Duration: 30 minutes} \\
\textit{Total: 10 points} \\
\textit{The use of a calculator is permitted.} \\
\textit{The use of rough work paper is strongly recommended.} \\
\textit{Answers must be written on a separate answer sheet.} \\
\end{center}

\textbf{Exercise 1:} \hfill \textbf{\ldots ~/ 10 points} \\

\textbf{Context:}  
A walker uses his watch to find that he walks 2 km in 30 minutes and 20 km in 5 hours.

\medskip

\textbf{Questions:}

\begin{enumerate}
    \item Is this situation a \textbf{proportional relationship} between distance and time? \\
    Justify your answer. \hfill (3 points)

    \vspace{2em}

    \item Find the time needed to walk 1 km at the same speed. \\
    Show your calculations. \hfill (2 points)

    \vspace{2em}

    \item Find the distance walked in 2 hours at the same speed. \\
    Show your calculations. \hfill (2 points)

    \vspace{2em}

    \item Find the time needed to walk 20 km at the same speed. \\ 
    You will have to use the \textbf{cross product method}. \\
    Show your calculations. \hfill (3 points)

    \vspace{2em}
\end{enumerate}

\end{document}
