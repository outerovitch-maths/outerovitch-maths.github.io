%% Font size %%
\documentclass[11pt]{article}

%% Load the custom package
\usepackage{Mathdoc}

%% Numéro de séquence %% Titre de la séquence %%
\renewcommand{\centerhead}{Proportionnality}

%% Spacing commands %%
\renewcommand{\baselinestretch}{1} \setlength{\parindent}{0pt}


\begin{document}

\section{Generalities, definitions}

\textbf{Definition 1 :} \\
Two quantities are said to be in \textbf{a proportional relationship} (or \textbf{directly proportional}) if one is obtained by multiplying the other by a constant number. \\

\textbf{Definition 2 :} \\
This number is called the \textbf{constant of proportionality}. \\

\textbf{Example 1 :} \\
If a car travels at a constant speed of 60 km/h, the distance traveled is proportional to the time : in 1 hour the car travels 60 km, in 2 hours 120 km, in 3 hours 180 km etc. \\

Here, the constant of proportionality is 60. \\

\textbf{Key Vocab :} \\
\textbf{Proportional relationship :} Relation proportionnelle. \\
\textbf{Constant of proportionality :} Coefficient de proportionnalité.

\section{Table of Proportionality}

\textbf{Definition 1 :} \\
A table represents a proportional situation if the ratio between the corresponding values in each column is equal. \\

\textbf{Property 1 :} \\
The given ratio is the \textbf{constant of proportionality}. \\

\textbf{Example 1 :} \\


\begin{center}
\begin{tabular}{|c|c|c|c|}
\hline
Quantity (kg) & $1$ & $3$ & $5$ \\
\hline
Price (£) & $3$ & $9$ & $15$ \\
\hline
\end{tabular}
\end{center}

Here, the situation is proportional because:

\[
\frac{3}{1} = \frac{9}{3} = \frac{15}{5} = 3
\]

The constant of proportionality is 3. \\

\textbf{Key Vocab :} \\
\textbf{Table of proportionality :} Tableau de proportionnalité. \\
\textbf{Rows :} Lignes. \\
\textbf{Columns :} Colonnes. \\
\textbf{Constant of proportionality :} Coefficient de proportionnalité.

\newpage

\section{Cross-Multiplication}

\textbf{Property 1 :} \\
In a table of proportionality we have the following result :

\begin{center}
\begin{tabular}{|c|c|c|}
\hline
Quantity A & $a_1$ & $a_2$  \\
\hline
Quantity B & $b_1$ & $b_2$  \\
\hline
\end{tabular}
\end{center}

\[
a_1 \times b_2 = b_1\times a_2
\]


This is called the \textbf{cross product method} (or \textbf{cross-multiplication}). \\

\textbf{Method 1 : (Filling a table using the cross product method)} \\
Here is a partially filled table, fill it with the right numbers using the cross product propriety :

\begin{center}
\begin{tabular}{|c|c|c|c|}
\hline
Quantity A & $7$ & $a$ & $11$ \\
\hline
Quantity B & $84$ & $252$ & $b$ \\
\hline
\end{tabular}
\end{center}

\phantom{0}

\begin{multicols}{2}
Firstly we have :
\[
7 \times 252 = 84 \times a
\]

So :

\[
a = \dfrac{7 \times 252}{84} = 21
\]


Then we have :
\[
a \times b = 252 \times 11
\]

$a = 21$, so :

\[
b = \dfrac{252 \times 11}{21} = 132
\]
\end{multicols}

\textbf{Key Vocab :} \\
\textbf{Table of proportionality :} Tableau de proportionnalité. \\
\textbf{Cross product method :} Méthode du produit en croix.


\section{Solving a Proportionality Problem}

\textbf{Method 2:}
\begin{enumerate}
    \item Identify the two quantities ;
    \item create and fill a table of proportionality using the quantities identified ;
    \item check that the situation is proportional ;
    \item find the constant of proportionality and/or use cross-multiplication ;
    \item interpret the result with a well written english sentence.
\end{enumerate}


\end{document}
