\documentclass[11pt]{article}
\usepackage{Mathdoc}
\renewcommand{\centerhead}{Proportionality - Exercises}

\renewcommand{\baselinestretch}{1.5} \setlength{\parindent}{0pt}


\begin{document}

\begin{exercice}
Read the two following situations.\\
For each situation, decide if it represents a \textbf{proportional relationship}. \\
Explain why or why not, and identify the \textbf{constant of proportionality} if it exists. \\

\textbf{Situation A:} A bicycle rental company charges a fixed \$5 per hour. If Maria rents a bike for 2 hours, she pays \$10. If she rents it for 6 hours, she pays \$30. 

\vspace{1.5em}

\textbf{Situation B:} A taxi company charges a base fare of \$3 plus \$2 per kilometer. If John travels 1 km, he pays \$5. If he travels 3 km, he pays \$9. 

\vspace{1.5em}
\end{exercice}

\begin{exercice}
For each table below, calculate the ratio for each columns. \\
\textbf{Check if the ratios are equal.} \\
Then determine which table represents a \textbf{proportional relationship}. \\
Explain your answer. \\
\begin{multicols}{2}
  \textbf{Table 1:} \\

  \begin{tabular}{|c|c|c|c|}
    \hline
    \textbf{Quantity A} & 2 & 4 & 6 \\
    \hline
    \textbf{Quantity B} & 4 & 8 & 12 \\
    \hline
  \end{tabular}

  \vspace{0.7em}

  \textbf{Table 2:} \\

  \begin{tabular}{|c|c|c|c|}
    \hline
    \textbf{Quantity A} & 1 & 2 & 3 \\
    \hline
    \textbf{Quantity B} & 3 & 6 & 10 \\
    \hline
  \end{tabular}
\end{multicols}

  \vspace{0.7em}


\end{exercice}

\begin{exercice}
Use the \textbf{cross product method} to complete the following tables. Show your calculations.

\begin{multicols}{2}

(a)
\begin{tabular}{|c|c|c|}
\hline
\textbf{Quantity A} & 2 & 4 \\
\hline
\textbf{Quantity B} & 3 & ? \\
\hline
\end{tabular}

\vspace{1.7em}

(b)
\begin{tabular}{|c|c|c|}
\hline
\textbf{Quantity A} & 4 & ? \\
\hline
\textbf{Quantity B} & 12 & 6 \\
\hline
\end{tabular}


(c)
\begin{tabular}{|c|c|c|}
\hline
\textbf{Quantity A} & 5 & 15 \\
\hline
\textbf{Quantity B} & 10 & ? \\
\hline
\end{tabular}

\vspace{1.7em}

(d)
\begin{tabular}{|c|c|c|}
\hline
\textbf{Quantity A} & ? & 12 \\
\hline
\textbf{Quantity B} & 24 & 36 \\
\hline
\end{tabular}

\end{multicols}


\vspace{1.5em}
\end{exercice}

\begin{exercice}[0][Listening comprehension -- Shopping]

\textbf{Instructions:}  
Listen carefully to the dialogue. Then answer the questions.

\medskip

\textbf{Transcript (to be read by the teacher):}

\textit{
Shopkeeper: Good morning. Can I help you? \\
Customer: Yes, please. How much are the apples? \\
Shopkeeper: They are 2 euros per kilogram. \\
Customer: Okay. And how much are the oranges? \\
Shopkeeper: The oranges are 3 euros for 2 kilograms. \\
Customer: I would like 3 kilograms of apples and 4 kilograms of oranges, please. \\
Shopkeeper: No problem. So, 3 kilograms of apples cost 6 euros, and 4 kilograms of oranges cost 6 euros. The total is 12 euros. \\
Customer: Here is a 20-euro note. \\
Shopkeeper: Thank you. Your change is 8 euros. \\
Customer: Thank you. Have a nice day. \\
Shopkeeper: You’re welcome. Goodbye.
}

\medskip

\textbf{Key Words : }price, per, kilogram, cost, total, change, euros, buy, pay 

\medskip

\textbf{Part 1 -- Listening comprehension (vocabulary)}

\begin{enumerate}
    \item What fruits does the customer buy?
    \item What is the \textbf{price} of apples \textbf{per} kilogram?

    \item What is the \textbf{price} of oranges for 2 kilograms?
    \item What does the word \textbf{change} mean in French?
\end{enumerate}

\medskip

\textbf{Part 2 -- Maths (proportionality)}

\begin{enumerate}
    \setcounter{enumi}{4}
    \item Complete the \textbf{table of proportionality} for apples.

    \begin{center}
    \begin{tabular}{|c|c|c|}
    \hline
    \textbf{Quantity (kg)} & 1 & 3 \\
    \hline
    \textbf{Price (€)} & ? & ? \\
    \hline
    \end{tabular}
    \end{center}

    \item How much does the customer pay in total?  
    Write a \textbf{full sentence} in English.

    \item The customer gives a 20-euro note.  
    How much \textbf{change} does he get?  
    Write a \textbf{full sentence} in English.

    \item If the customer had 20 euros, how many kilograms of apples could he buy?  
    Show your calculations.
\end{enumerate}

\end{exercice}


\end{document}
