%% Font size %%
\documentclass[11pt]{article}

%% Load the custom package
\usepackage{Mathdoc}

%% Numéro de séquence %% Titre de la séquence %%
\renewcommand{\centerhead}{Proportionnality - Oral Help}

%% Spacing commands %%
\renewcommand{\baselinestretch}{1}
\setlength{\parindent}{0pt}

\begin{document}

\section*{Topic}
Proportionality – La proportionnalité

\section*{Learning objectives}

\subsection*{Maths}
\begin{itemize}
    \item Identify proportional situations
    \item Find the coefficient of proportionality
    \item Solve proportionality problems
\end{itemize}

\subsection*{Language}
\begin{itemize}
    \item Understand basic instructions in English
    \item Use mathematical vocabulary
    \item Explain a method orally
\end{itemize}

\section*{Presentation of the lesson}

I chose proportionality in 4th grade because it is an important mathematical concept and pupils can use it in real life situations like shopping.  

The mathematical objective is to identify proportional situations and calculate the coefficient of proportionality.  

The linguistic objective is to understand instructions in English and use simple mathematical vocabulary.  

The lesson starts with a concrete situation. Pupils work in pairs on a table of values.  

We finish with a short assessment.  

I help weaker pupils with visual support and simple English sentences.

\section*{Jury -- Question 1}

\textbf{Can you present your lesson and explain why you chose proportionality for 4th grade?}

I chose to work on proportionality, and especially on the coefficient of proportionality, because it is a key concept in the 4th grade curriculum and pupils use it in real-life situations, for example shopping.

\section*{Jury -- Question 2}

\textbf{What are the mathematical objectives and the language objectives of your lesson?}

The mathematical objective is to identify whether a situation is proportional, using data or a table of values.  
If the situation is proportional, pupils have to complete the table and answer the problem.  

The language objectives are to learn and use key vocabulary about proportionality, especially words linked to tables (row, column), the coefficient of proportionality, and shopping vocabulary.

\section*{Jury -- Question 3}

\textbf{How does your lesson start? Can you describe the first activity?}

The lesson starts with a short mental calculation activity about products and quotients.  

Then we introduce the notion of proportionality: pupils have to identify whether a situation is proportional or not.  

After that, they work on an activity: they are given several tables and they have to decide if the situations are proportional.

\section*{Jury -- Question 4}

\textbf{How do you help weaker pupils or pupils who have difficulties with English?}

I give all pupils a key vocabulary sheet.  

Pupils who have difficulties can use this memo during the lesson.  

If they still have difficulties, I help them with gestures and visual support, written on the board or displayed on the wall of the classroom.  

If they still do not understand the English vocabulary, I simplify my English and I can use a few words in French to avoid losing them.

\section*{Jury -- Question 5}

\textbf{How do you evaluate pupils during and after the lesson?}

In a DNL lesson, we aim to assess pupils both in mathematics and in English.  

For the mathematical assessment, there is no problem because it is my subject: here, for example, I propose a short final assessment of about ten minutes.  

For the English assessment, I give pupils a mark for oral participation, and in each assessment, there is a part dedicated to English vocabulary, which is included in the final mark.

\section*{Dernier coup de boost avant le jour J}

Apprends par cœur ces 5 phrases-clés (elles te sauveront la vie à l’oral) :

\begin{enumerate}
    \item I chose proportionality because it is a key concept in the 4th grade curriculum.
    \item The mathematical objective is to identify proportional situations and calculate the coefficient of proportionality.
    \item The language objective is to use basic mathematical vocabulary in English.
    \item I help weaker pupils with visual support and simplified English.
    \item I assess pupils both in mathematics and in English.
\end{enumerate}

\end{document}
