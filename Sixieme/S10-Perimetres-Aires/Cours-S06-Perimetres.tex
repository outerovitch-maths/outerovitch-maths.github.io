%% Font size %%
\documentclass[11pt]{article}

%% Load the custom package
\usepackage{Mathdoc}

%% Numéro de séquence %% Titre de la séquence %%
\renewcommand{\centerhead}{Chapitre 6 : Conversions et périmètres}

%% Spacing commands %%
\renewcommand{\baselinestretch}{1} \setlength{\parindent}{0pt}

\begin{document}

\renewcommand{\baselinestretch}{1.5}

\section{Unités de mesure internationales}

\begin{remarque}
Pour simplifier les échanges scientifiques entre les pays du monde
entiers, certaines unités ont été fixées \\ internationalement.
\end{remarque}

\begin{definition}
\begin{itemize}
 \item Pour mesurer des longueurs, on utilise une longueur de référence, appelée mètre, que l’on peut noter en abrégé « m ».
\item Pour mesurer des masses, on utilise une masse de référence, appelée gramme, que l’on peut noter en abrégé « g ».
\item Pour mesurer des contenances, on utilise un volume de référence, appelée litre, que l’on peut noter en abrégé « L ».
\item Pour mesurer un temps, on utilise un temps de référence, appelée seconde, que l’on peut noter en abrégé « s ».
\end{itemize}
\end{definition}
\begin{vocabulaire}
Mètres, Grammes, Litres et Secondes s’appellent des unités de mesure internationales.
\end{vocabulaire}
  

\section{Tableaux de conversions}

\begin{remarque}
Comme pour la numération, les conversions se font en base 10. On 
peut utiliser les tableaux suivants.
\end{remarque}

\textbf{Pour les longueurs :}
\begin{center}\input{.data/tab-longueur.txt}
\end{center}

\textbf{Pour les masses :}
\begin{center}\input{.data/tab-masse.txt}
\end{center}

\textbf{Pour les contenances :}
\begin{center}\input{.data/tab-contenance.txt}
\end{center}

\begin{exemple}
1\ km = 10\ hm = 100\ dam = 1\ 000\ m et 1\ m = 10\ dm = 100\ cm = 1\
000\ mm.

1\ kg = 10\ hg = 100\ dag = 1\ 000\ g et 1 g\ = 10\ dg = 100\ cg = 1\
000\ mg.

1\ kL = 10\ hL = 100\ daL = 1\ 000\ L et 1 L\ = 10\ dL = 100\ cL = 1\
000\ mL.
\end{exemple}

\begin{remarque}
Ce principe de base ne fonctionne pas pour le temps. En effet, les
secondes sont exprimées en base 60. 60 secondes donnent 1 minute, 60
minutes donnent 1 heure, 24 heures donnent 1 journée \dots
\end{remarque}

\section{Unités de durée}

\begin{propriete}
On rappelle les égalités suivantes :
\begin{itemize}
\item 60 secondes = 1 minute
\item 60 minutes = 1 heure
\item 24 heures = 1 journée
\item 365,25 jours = 1 an
\item 100 ans = 1 siècle
\item 1 000 ans = 1 millénnaire
\end{itemize}
\end{propriete}


\begin{exemple}
\phantom{0} \vspace{-.5cm}
\begin{itemize}
\item $4\text{ h} = 4 \times 60 \text{ min} = 240 \text{ min}$
\item $20 \text{ min} = 20 \times 60 \text{ sec} = 1~200 \text{ sec}$
\item $7 \text{ j} = 7 \times 24 \text{ h} = 168 \text{ h}$
\end{itemize}
\end{exemple}



\section{Périmètre d'une figure}

\begin{vocabulaire}
La mesure du contour d'une figure quelconque est appelée périmètre.
\end{vocabulaire}

\begin{center}\input{.data/peri.txt}
\end{center}

Le périmètre de cette figure est de $16,83$ cm. 

\begin{center}
\opmanyadd[voperator=bottom]{4}{1}{2}{3}{4}{2.83}
\end{center}

\section{Quadrilatères}
\begin{definition}
Un quadrilatère est un polygone qui possède quatre côtés.
\end{definition}

\begin{multicols}{2}

\begin{definition}
Un rectangle est un quadrilatère qui possède 4 angles droits.
\end{definition}

\begin{center}
\input{.data/rectangle.txt}
\end{center}

\begin{definition}
Un carré est un quadrilatère qui possède 4 côtés de même longueurs
\underline{et} 4 angles droits.
\end{definition}

\begin{center}
\input{.data/carre.txt}
\end{center}

\end{multicols}

\section{Périmètre des polygones}


\begin{definition}
Le périmètre $\mathcal{P}$ d'un polygone est égale à la somme des
longueurs de ses côtés
\end{definition}

\begin{exemple} {\large
  \begin{center}
  \begin{tabular}{|l|c|c|}
\hline
Périmètre d'un \underline{carré} de coté $c$ &  $\mathcal{P}_{carre}~=~c \times 4$ \\ \hline
Périmètre d'un \underline{rectangle} de cotés $L$ et $\ell$ &  $\mathcal{P}_{rectangle}~=(~L+\ell~) \times 2$ \\ \hline
Périmètre d'un \underline{losange} de coté $c$ &  $\mathcal{P}_{losange}~=c \times 4$ \\ \hline
Périmètre d'un \underline{triangle équilateral} de coté $c$ \phantom{00} &  $\mathcal{P}_{triangle}~=c \times 3$ \\ \hline
\end{tabular} 
\end{center} }
\end{exemple}


\end{document}
