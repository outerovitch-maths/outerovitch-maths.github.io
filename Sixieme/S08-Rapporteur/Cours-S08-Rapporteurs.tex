%% Font size %%
\documentclass[11pt]{article}

%% Load the custom package
\usepackage{Mathdoc}

%% Numéro de séquence %% Titre de la séquence %%
\renewcommand{\centerhead}{}

%% Spacing commands %%
\renewcommand{\baselinestretch}{1} \setlength{\parindent}{0pt}

\begin{document}

\section{Définitions, exemples}

\begin{definition}
Deux demi-droites de même origine définissent un angle.
\end{definition}

\begin{vocabulaire}
Cette origine est le \textbf{sommet} de l'angle et les deux demi-droites sont
appelées \textbf{côtés} de l'angle.
\end{vocabulaire}

\begin{notation}
Pour noter un angle, on utilise trois lettres :
\begin{itemize}
\item La lettre du milieu est le sommet de l'angle;
\item les deux autres lettres sont chacune sur un côtés de l'angle.
\end{itemize}
\end{notation}

\begin{exemple}
On a représenté l'angle $\widehat{AOB}$
\end{exemple}
\begin{center}
\begin{tikzpicture}[line cap=round,line join=round,>=triangle 45,x=1cm,y=1cm]
\clip(-4.5,-2) rectangle (9.5,5.5);
\draw [shift={(0,0)},line width=2.8pt,fill=black,fill opacity=0.6] (0,0) -- (-14.036243467926479:1.323915838904143) arc (-14.036243467926479:33.69006752597979:1.323915838904143) -- cycle;
\draw [line width=1pt] (0,0)-- (3,2);
\draw [line width=1pt] (0,0)-- (4,-1);
\draw [line width=1pt] (9,0)-- (9,3);
\draw [line width=1pt] (9,3)-- (4,3);
\draw [line width=1pt] (4,3)-- (4,0);
\draw [line width=1pt] (4,0)-- (9,0);
\draw [line width=1pt] (-4,2)-- (-4,5);
\draw [line width=1pt] (-4,5)-- (1,5);
\draw [line width=1pt] (1,5)-- (1,2);
\draw [line width=1pt] (1,2)-- (-4,2);
\draw [line width=1pt] (-0.7573394331848615,2)-- (-0.13320768055862275,0.4232778463907834);
\draw [line width=1pt] (4,1.1608880994945172)-- (1.7581006607330099,1.0474095990170196);
\draw [line width=1pt] (4,1.1608880994945172)-- (2.2498408294688343,-0.23868007306128536);
\begin{scriptsize}
\draw [color=black] (0,0)-- ++(-2.5pt,-2.5pt) -- ++(5pt,5pt) ++(-5pt,0) -- ++(5pt,-5pt);
\draw[color=black] (-0.8140786834236104,0.13012505349058154) node {$O$};
\draw [color=black] (1.4676923076923076,0.9784615384615384)-- ++(-2.5pt,-2.5pt) -- ++(5pt,5pt) ++(-5pt,0) -- ++(5pt,-5pt);
\draw[color=black] (1.474404409539265,1.5675193928722164) node {$A$};

\draw[color=black] (6.5,1.5675193928722164) node {\textbf{Côtés de l'angle}};

\draw[color=black] (-1.5,3.5) node {\textbf{Sommet de l'angle}};

\draw [color=black] (1.9694117647058824,-0.4923529411764706)-- ++(-2.5pt,-2.5pt) -- ++(5pt,5pt) ++(-5pt,0) -- ++(5pt,-5pt);
\draw[color=black] (2.0417969119267547,-1.118138451761891) node {$B$};
\end{scriptsize}
\end{tikzpicture}
\end{center}
Le point O est le sommet de l'angle et les demi-droites [OA) et [OB) sont les côtés de l'angle $\widehat{AOB}$.


\section{Angles particuliers}

\begin{center}
\begin{tabular}{|c|c|c|c|c|}
\hline 
\textit{Nom} & \textbf{Angle Plat} & \textbf{Angle droit}& \textbf{Angle obtus}& \textbf{Angle Aigu}\\
\hline
 & Formé de trois & Moitié & Plus grand & Plus petit \\
\textit{Propriété} & points alignés & d'un angle plat& qu'un angle droit & qu'un angle droit \\
\hline
\textit{Exemple} 
&
\begin{tikzpicture}[line cap=round,line join=round,>=triangle 45,x=1cm,y=1cm]
\clip(-1.5,-0.5) rectangle (2,1);
\draw [shift={(0,0)},line width=1pt,fill=black,fill opacity=0.10000000149011612] (0,0) -- (0:0.4491382054346098) arc (0:180:0.4491382054346098) -- cycle;
\draw [line width=1pt] (-1.5,0)-- (2,0);
\draw [line width=1pt] (2,0)-- (-1.5,0);
\begin{scriptsize}
\draw [color=black] (-1,0)-- ++(-2pt,-2pt) -- ++(4pt,4pt) ++(-4pt,0) -- ++(4pt,-4pt);
\draw[color=black] (-0.9201149059420427,0.19462655568832887) node {$A$};
\draw [color=black] (0,0)-- ++(-2pt,-2pt) -- ++(4pt,4pt) ++(-4pt,0) -- ++(4pt,-4pt);
\draw[color=black] (0.07796999502375684,0.19462655568832887) node {$B$};
\draw [color=black] (1.5,0)-- ++(-2pt,-2pt) -- ++(4pt,4pt) ++(-4pt,0) -- ++(4pt,-4pt);
\draw[color=black] (1.575097346472456,0.19462655568832887) node {$C$};
\end{scriptsize}
\end{tikzpicture}
&
\begin{tikzpicture}[line cap=round,line join=round,>=triangle 45,x=1cm,y=1cm]
\clip(-1.5,-0.5) rectangle (2,1.5);
\draw[line width=1pt,fill=black,fill opacity=0.10000000149011612] (0.31758867075276925,0) -- (0.3175886707527693,0.31758867075276925) -- (0,0.31758867075276925) -- (0,0) -- cycle; 
\draw [line width=1pt] (1.5,0)-- (0,0);
\draw [line width=1pt] (0,0)-- (0,1);
\begin{scriptsize}
\draw [color=black] (0,0)-- ++(-2pt,-2pt) -- ++(4pt,4pt) ++(-4pt,0) -- ++(4pt,-4pt);
\draw[color=black] (-0.23143632427564098,0.2345499517269608) node {};
\end{scriptsize}
\end{tikzpicture}
&
\begin{tikzpicture}[line cap=round,line join=round,>=triangle 45,x=1cm,y=1cm]
\clip(-1.5,-0.5) rectangle (2,1.5);
\draw [shift={(0,0)},line width=1pt,fill=black,fill opacity=0.10000000149011612] (0,0) -- (0:0.4491382054346098) arc (0:153.434948822922:0.4491382054346098) -- cycle;
\draw [line width=1pt] (1.5,0)-- (0,0);
\draw [line width=1pt] (0,0)-- (-1,0.5);
\begin{scriptsize}
\draw [color=black] (0,0)-- ++(-2pt,-2pt) -- ++(4pt,4pt) ++(-4pt,0) -- ++(4pt,-4pt);
\end{scriptsize}
\end{tikzpicture}
&
\begin{tikzpicture}[line cap=round,line join=round,>=triangle 45,x=1cm,y=1cm]
\clip(-1.5,-0.5) rectangle (2,1.5);
\draw [shift={(0,0)},line width=1pt,fill=black,fill opacity=0.10000000149011612] (0,0) -- (0:0.4491382054346098) arc (0:26.56505117707799:0.4491382054346098) -- cycle;
\draw [line width=1pt] (1.5,0)-- (0,0);
\draw [line width=1pt] (0,0)-- (1,0.5);
\begin{scriptsize}
\draw [color=black] (0,0)-- ++(-2pt,-2pt) -- ++(4pt,4pt) ++(-4pt,0) -- ++(4pt,-4pt);
\end{scriptsize}
\end{tikzpicture} \\
\hline
\end{tabular}
\end{center}

\section{Mesurer un angle}

\begin{definition}
L'unité de mesure des angles est le degré, noté "$\degre$" \\ Pour mesurer un
angle on peut utiliser un rapporteur
\end{definition}

\begin{multicols}{2}
\encart{5cm}
\columnbreak
\begin{itemize}[label=\textbullet]
\item On lit la mesure de l'angle $\widehat{FCE}$ sur les \textbf{graduations extérieures}. \\ $\widehat{FCE}=50$° 
\item On lit la mesure de l'angle $\widehat{DCE}$ sur les \textbf{graduations intérieures}. \\ $\widehat{DCE}=130$° 
\end{itemize}
\end{multicols}


\begin{multicols}{2}
Deux angles de même mesure sont codés de la même façon. $\widehat{IHJ}=\widehat{KML}$
\begin{tikzpicture}[line cap=round,line join=round,>=triangle 45,x=2cm,y=2cm]
\clip(-0.5,-1) rectangle (3.5,1);
\draw [shift={(0,0)},line width=1pt,fill=black,fill opacity=0.10000000149011612] (0,0) -- (0:0.47730452283397756) arc (0:26.56505117707799:0.47730452283397756) -- cycle;
\draw [shift={(3,-0.5)},line width=1pt,fill=black,fill opacity=0.10000000149011612] (0,0) -- (180:0.47730452283397756) arc (180:206.565051177078:0.47730452283397756) -- cycle;
\draw [shift={(0,0)},line width=1pt] (0:0.47730452283397756) arc (0:26.56505117707799:0.47730452283397756);
\draw[line width=1pt] (0.4343674562627567,0.1240527944935223) -- (0.48354113055665376,0.13809650707769497);
\draw[line width=1pt] (0.4439881600241419,0.08329883936890714) -- (0.49425097059291273,0.09272889665595326);
\draw [line width=1pt] (1.5,0)-- (0,0);
\draw [line width=1pt] (0,0)-- (1,0.5);
\draw [line width=1pt] (1.5,-0.5)-- (3,-0.5);
\draw [line width=1pt] (3,-0.5)-- (2,-1);
\draw [shift={(3,-0.5)},line width=1pt] (180:0.47730452283397756) arc (180:206.565051177078:0.47730452283397756);
\draw[line width=1pt] (2.565632543737243,-0.6240527944935229) -- (2.5164588694433463,-0.6380965070776952);
\draw[line width=1pt] (2.5560118399758576,-0.5832988393689074) -- (2.5057490294070868,-0.5927288966559535);
\begin{scriptsize}
\draw [color=black] (0,0)-- ++(-2.5pt,-2.5pt) -- ++(5pt,5pt) ++(-5pt,0) -- ++(5pt,-5pt);
\draw[color=black] (-0.1433995465088165,-0.021089601039946962) node {$H$};
\draw [color=black] (3,-0.5)-- ++(-2.5pt,-2.5pt) -- ++(5pt,5pt) ++(-5pt,0) -- ++(5pt,-5pt);
\draw[color=black] (3.0545407564788327,-0.3552027670237303) node {$M$};
\draw [color=black] (0.7100818465242826,0.3550409232621413)-- ++(-2.5pt,-2.5pt) -- ++(5pt,5pt) ++(-5pt,0) -- ++(5pt,-5pt);
\draw[color=black] (0.7634790468757408,0.5039453740774268) node {$I$};
\draw [color=black] (1.0225872164141856,0)-- ++(-2.5pt,-2.5pt) -- ++(5pt,5pt) ++(-5pt,0) -- ++(5pt,-5pt);
\draw[color=black] (1.0771363047380689,0.14937629997218738) node {$J$};
\draw [color=black] (2.065838530608451,-0.5)-- ++(-2.5pt,-2.5pt) -- ++(5pt,5pt) ++(-5pt,0) -- ++(5pt,-5pt);
\draw[color=black] (2.120387618932334,-0.3552027670237303) node {$K$};
\draw [color=black] (2.3801398052902196,-0.8099300973548902)-- ++(-2.5pt,-2.5pt) -- ++(5pt,5pt) ++(-5pt,0) -- ++(5pt,-5pt);
\draw[color=black] (2.434044876794662,-0.6620413888455722) node {$L$};
\end{scriptsize}
\end{tikzpicture}
\end{multicols}

\begin{remarque}
Il existe deux angles particuliers : \textbf{L'angle droit} mesure 90°
et \textbf{l'angle plat} mesure 180°.
\end{remarque}

\newpage
\section{Construire un angle}

\textbf{Énoncé :} Tracer un angle $\widehat{BAC}$ de mesure $137$°.

\begin{multicols}{2}
\makebox[\linewidth]{} \\ \makebox[\linewidth]{} \\ 
\noindent \textbf{Étape 1 :} Le sommet de l'angle est A, il faut donc commencer par tracer la demie-droite [AB) (par exemple), et placer le rapporteur afin d'avoir le "0" de la graduation sur la demie-droite [AB). \\
\begin{tikzpicture}[line cap=round,line join=round,>=triangle 45,x=1cm,y=1cm]
\clip(-3.5,-3.5) rectangle (3.5,3.5);
\draw [line width=1pt] (-3,-3)-- (-3,3);
\draw [line width=1pt] (-3,3)-- (3,3);
\draw [line width=1pt] (3,3)-- (3,-3);
\draw [line width=1pt] (3,-3)-- (-3,-3);
\draw [line width=1pt] (-2.5,-1)-- (0,-1);
\begin{scriptsize}
\draw [color=black] (0,-1)-- ++(-2.5pt,-2.5pt) -- ++(5pt,5pt) ++(-5pt,0) -- ++(5pt,-5pt);
\draw[color=black] (0.4352289909462427,-1.351713241466346) node {$A$};
\draw [color=black] (-2,-1)-- ++(-2.5pt,-2.5pt) -- ++(5pt,5pt) ++(-5pt,0) -- ++(5pt,-5pt);
\draw[color=black] (-1.9016326821732072,-1.6343981212791856) node {$B$};
\end{scriptsize}
\end{tikzpicture}
\end{multicols}

\begin{multicols}{2}
\makebox[\linewidth]{} \\ \makebox[\linewidth]{} \\ \makebox[\linewidth]{} \\ 
\noindent \textbf{Étape 2 :} Comme la graduation "0" est sur l’extérieur, on cherche la mesure $137$° sur la \textbf{ligne extérieure}, puis on fait une marque au crayon.  \\
\begin{tikzpicture}[line cap=round,line join=round,>=triangle 45,x=1cm,y=1cm]
\clip(-3.5,-3.5) rectangle (3.5,3.5);
\draw [line width=1pt] (-3,-3)-- (-3,3);
\draw [line width=1pt] (-3,3)-- (3,3);
\draw [line width=1pt] (3,3)-- (3,-3);
\draw [line width=1pt] (3,-3)-- (-3,-3);
\draw [line width=1pt] (-2.5,-1)-- (0,-1);
\draw [line width=1pt] (0,-1)-- (1.889297573418096,0.7617984894156513);
\begin{scriptsize}
\draw [color=black] (0,-1)-- ++(-2.5pt,-2.5pt) -- ++(5pt,5pt) ++(-5pt,0) -- ++(5pt,-5pt);
\draw[color=black] (0.4352289909462427,-1.351713241466346) node {$A$};
\draw [color=black] (-2,-1)-- ++(-2.5pt,-2.5pt) -- ++(5pt,5pt) ++(-5pt,0) -- ++(5pt,-5pt);
\draw[color=black] (-1.9016326821732072,-1.6343981212791856) node {$B$};
\end{scriptsize}
\end{tikzpicture}
\end{multicols}

\begin{multicols}{2}
\makebox[\linewidth]{} \\ \makebox[\linewidth]{} \\ \makebox[\linewidth]{} \\ 
\noindent \textbf{Étape 3 :} Il ne reste plus qu'à tracer la demi-droite d'origine A passant par la marque laissée au crayon. On place un point C sur cette demi-droite. \\
\begin{tikzpicture}[line cap=round,line join=round,>=triangle 45,x=1cm,y=1cm]
\clip(-3.5,-3.5) rectangle (3.5,3.5);
\draw [shift={(0,-1)},line width=1pt,fill=black,fill opacity=0.10000000149011612] (0,0) -- (43:0.9422829327094556) arc (43:180:0.9422829327094556) -- cycle;
\draw [line width=1pt] (-3,-3)-- (-3,3);
\draw [line width=1pt] (-3,3)-- (3,3);
\draw [line width=1pt] (3,3)-- (3,-3);
\draw [line width=1pt] (3,-3)-- (-3,-3);
\draw [line width=1pt] (-2.5,-1)-- (0,-1);
\draw [line width=1pt] (0,-1)-- (1.889297573418096,0.7617984894156513);
\begin{scriptsize}
\draw [color=black] (0,-1)-- ++(-2.5pt,-2.5pt) -- ++(5pt,5pt) ++(-5pt,0) -- ++(5pt,-5pt);
\draw[color=black] (0.4352289909462427,-1.351713241466346) node {$A$};
\draw [color=black] (-2,-1)-- ++(-2.5pt,-2.5pt) -- ++(5pt,5pt) ++(-5pt,0) -- ++(5pt,-5pt);
\draw[color=black] (-1.9016326821732072,-1.6343981212791856) node {$B$};
\draw [color=black] (1.462707403238341,0.3639967201249972)-- ++(-2.5pt,-2.5pt) -- ++(5pt,5pt) ++(-5pt,0) -- ++(5pt,-5pt);
\draw[color=black] (2.1124926111690736,0.3255503787565038) node {$C$};
\draw[color=black] (-0.6955105283051041,0.2878590614481251) node {$137$°};
\end{scriptsize}
\end{tikzpicture}
\end{multicols}


\end{document}
