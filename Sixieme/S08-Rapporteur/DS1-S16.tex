%%CLASS%%
\documentclass[french,12pt]{article}
%%PACKAGES%%
\usepackage[T1]{fontenc}
\usepackage[none]{hyphenat}
\usepackage[utf8]{inputenc}
\usepackage{multicol, caption}
\usepackage{tabularx}
\usepackage{graphicx}
\renewcommand{\arraystretch}{1.5}
%\setlength{\arrayrulewidth}{0,5mm}
%\usepackage[french]{babel}
\usepackage[left=2cm,right=2cm,top=2cm,bottom=2cm]{geometry}
\usepackage{mathtools, bm}
\usepackage{enumitem}
\usepackage{amsmath}
\usepackage{amssymb}
\usepackage{pstricks}
\usepackage{titlesec}
\usepackage{xlop}
\usepackage[pdftex=true,colorlinks=true,linkcolor=black,citecolor=black,filecolor=black,urlcolor=black,bookmarks=true,bookmarksopen=false,bookmarksnumbered=false,bookmarksopenlevel=0,pdfstartview=FitH,pdftitle={},pdfauthor={ColinOUTEROVITCH,bookmarks=false}]{hyperref}
\newenvironment{Figure}
  {\par\medskip\noindent\minipage{\linewidth}}
  {\endminipage\par\medskip}
%usepackage{biblatex}
%\addbibresource{biblio.bib}
\usepackage{fancyhdr}
\usepackage{pgfplots}
\pgfplotsset{compat=1.15}
\usepackage{mathrsfs}
\usetikzlibrary{arrows}

%%PAGE STYLE%%
\pagestyle{fancy}
\usepackage{tikz}
\fancyhf{}

%%NEW COMMAND%%



\newcommand*\circled[1]{\tikz[baseline=(char.base)]{
            \node[shape=circle,draw,inner sep=2pt] (char) {#1};}}
            
\renewcommand{\baselinestretch}{1}
\titleformat{\section}
{\normalfont\Large\bfseries}{\thesection}{1em}{-~~}

\newcommand{\mc}[1]{\begin{multicols}{2}#1\end{multicols}}

\newcommand{\enu}[1]{\begin{enumerate}[label=(\alph*)]#1\end{enumerate}}

\newcommand{\itmz}[1]{\begin{itemize}[label=\textbullet]#1\end{itemize}}

\newcommand{\cntr}[1]{\begin{center}#1\end{center}}

\newcommand{\dtf}{\makebox[\linewidth]{\dotfill}}

\newcommand{\exercice}{\section{}}

\newcommand{\mkb}[2]{\makebox[#1]{#2}}

\newcommand{\mybox}[1]{\begin{tabular}{|l|}
\hline
#1 \\
\hline
\end{tabular}}

\newcommand{\mkbdtf}[1]{\makebox[#1cm]{\dotfill}}

\newcommand{\mkblw}[1]{\makebox[\linewidth]{#1}}

\newcommand{\myrule}[1]{\rule[2mm]{#1cm}{.1pt}}

\newcommand{\myfigure}[1]{\begin{Figure}
\centering
\includegraphics[width=\linewidth]{images/#1}
\end{Figure}}

\newcommand{\entete}[1]{\lhead{Collège La Vallée - \textit{Mathématiques}}   \rhead{2021/2022}
\renewcommand{\headrulewidth}{0.5pt}
\cfoot{\circled{\thepage}}
\chead{\textbf{#1}}}

\newcommand{\titre}[2]{\cntr{
\begin{LARGE}
\myrule{#2} \textsc{#1} \myrule{#2}
\end{LARGE}}}

\newcommand{\colonnesep}[1]{\setlength{\columnseprule}{#1pt}}

\titleformat{\section}
{\normalfont\Large\bfseries}{Exercice~\thesection}{1em}{}

\setlength{\columnseprule}{1pt}
\setlength{\parindent}{0pt}

\newcommand{\myfig}[2]{\begin{Figure}
\centering
\includegraphics[width=#1]{images/#2}
\end{Figure}}


\begin{document}


\entete{14-15}{1}{Angles}

\colonnesep{0}


\hfill
\hfill

\nom

\hfill
\hfill

\appreciationnote{10}

\exercice \diff[1] \\
\textbf{Mesurer les angles suivants.}

\begin{multicols}{3}

\pgfmathsetmacro{\angle}{40}
\input{data/angle-1.tex}

\pgfmathsetmacro{\angle}{110}
\input{data/angle-1.tex}

\pgfmathsetmacro{\angle}{90}
\input{data/angle-1.tex}


\end{multicols}

\begin{multicols}{3}

\pgfmathsetmacro{\angle}{70}
\input{data/angle-1.tex}

\pgfmathsetmacro{\angle}{140}
\input{data/angle-1.tex}

\pgfmathsetmacro{\angle}{20}
\input{data/angle-1.tex}

\end{multicols}


\newpage
\exercice \diff[2] \\
\textbf{Mesurer les angles suivants.}

\begin{multicols}{3}

\pgfmathsetseed{7564}
\input{data/angle-3.tex}

\pgfmathsetseed{12345}
\input{data/angle-3.tex}

\pgfmathsetseed{98765}
\input{data/angle-3.tex}


\end{multicols}

\begin{multicols}{3}

\pgfmathsetseed{2345}
\input{data/angle-3.tex}

\pgfmathsetseed{98}
\input{data/angle-3.tex}

\pgfmathsetseed{9876}
\input{data/angle-3.tex}

\end{multicols}


\exercice \diff[1] \\
\textbf{Construire les angles demandés.}

\vfill

\begin{multicols}{3}

\pgfmathsetseed{54}
\input{data/angle-2.tex}

\pgfmathsetseed{1234}
\input{data/angle-2.tex}

\pgfmathsetseed{7654}
\input{data/angle-2.tex}

\end{multicols}

\newpage
\exercice \diff[2] \\
\textbf{Dans le cadre ci-dessous, construire les angles suivants.} \\
\textbf{$\widehat{A} = 35$ \degre ; $\widehat{B} = 75$ \degre ; $\widehat{C} = 115$ \degre ; $\widehat{D} = 122$ \degre ; $\widehat{E} = 47$ \degre ; $\widehat{F} = 92$ \degre ; $\widehat{G} = 154$ \degre ; $\widehat{H} = 78$ \degre} \\

\encart{7cm}

\mc{
\exercice \diff[2] \\
\textbf{Dans le cadre ci-dessous, reproduire la figure ci-contre (les côtés sont donnés en centimètres et les angles en degrés).}
\columnbreak
\myfig{0.7\textwidth}{fig.pdf}


}

\encart{7cm}

\newpage
\mc{
\exercice \diff[3] \\
\textbf{Mesurer les angles suivants puis les ranger dans le tableau ci-contre en indiquant leur mesure.}

\cntr{
\begin{tabular}{|c|c|c|}
\hline
Obtus&Droits&Aigus \\ \hline
\phantom{00000000000}&\phantom{00000000000}&\phantom{00000000000} \\ 
&& \\ 
&& \\ \hline
\end{tabular}} }

\begin{multicols}{3}

\pgfmathsetseed{654}
\newcommand{\point}{A}
\input{data/angle-4.tex}

\pgfmathsetseed{45}
\renewcommand{\point}{B}
\input{data/angle-4.tex}

\pgfmathsetseed{765}
\renewcommand{\point}{C}
\input{data/angle-4.tex}


\end{multicols}

\begin{multicols}{3}

\pgfmathsetseed{987}
\newcommand{\point}{D}
\input{data/angle-4.tex}

\pgfmathsetseed{999}
\renewcommand{\point}{E}
\input{data/angle-4.tex}

\pgfmathsetseed{345}
\renewcommand{\point}{F}
\input{data/angle-4.tex}

\end{multicols}

\begin{multicols}{3}

\pgfmathsetseed{444}
\newcommand{\point}{G}
\input{data/angle-4.tex}

\pgfmathsetseed{888}
\renewcommand{\point}{H}
\input{data/angle-4.tex}

\pgfmathsetseed{222}
\renewcommand{\point}{I}
\input{data/angle-4.tex}

\end{multicols}





\end{document}