%% Font size %%
\documentclass[11pt]{article}

%% Load the custom package
\usepackage{Mathdoc}

%% Numéro de séquence %% Titre de la séquence %%
\renewcommand{\centerhead}{Chapitre 6 : Droites - Tracer une perpendiculaire}
\setlength{\headheight}{14pt}

%% Spacing commands %%
\renewcommand{\baselinestretch}{1} \setlength{\parindent}{0pt}

\begin{document}

%\phantom{0}

\entetedevoirs{10}

\begin{exercicedevoir}[5]
\textbf {a.} Tracer les droites $(MN)$, $(MO)$ et $(NO)$.\\
\textbf {b.} Tracer la droite $(d_1)$ perpendiculaire à $(MN)$ passant par $O$.\\
\textbf {c.} Tracer la droite $(d_2)$ perpendiculaire à $(MO)$ passant par $N$.\\
\textbf {d.} Tracer la droite $(d_3)$ perpendiculaire à $(NO)$ passant par $M$.

\begin{center}
  \begin{tikzpicture}[baseline,scale = 0.5]


    \tikzset{
      point/.style={
        thick,
        draw,
        cross out,
        inner sep=0pt,
        minimum width=5pt,
        minimum height=5pt,
      },
    }
    \clip (-1,-3) rectangle (21,9);
    \draw[color ={black},line width = 0.625,opacity = 0.8] (-0.19,0.19)--(0.19,-0.19);\draw[color ={black},line width = 0.625,opacity = 0.8] (-0.19,-0.19)--(0.19,0.19);\draw[color ={black},line width = 0.625,opacity = 0.8] (19.81,-1.81)--(20.19,-2.19);\draw[color ={black},line width = 0.625,opacity = 0.8] (19.81,-2.19)--(20.19,-1.81);\draw[color ={black},line width = 0.625,opacity = 0.8] (3.81,8.19)--(4.19,7.81);\draw[color ={black},line width = 0.625,opacity = 0.8] (3.81,7.81)--(4.19,8.19);
    \draw [color={black}] (-0.5,0.5) node[anchor = center,scale=1, rotate = 0] {M};
    \draw [color={black}] (20.5,-1.5) node[anchor = center,scale=1, rotate = 0] {N};
    \draw [color={black}] (3.5,8.5) node[anchor = center,scale=1, rotate = 0] {O};
    
    

  \end{tikzpicture}
\end{center}
\end{exercicedevoir}

\vspace{-1.5cm}
\begin{exercicedevoir}[5]
\textbf {a.}  Tracer la droite perpendiculaire à $(AB)$ passant par $B$.\\
\textbf {b.}  Tracer la droite perpendiculaire à $(AB)$ passant par $C$ et nomme $M$, le point d'intersection de cette droite avec la droite $(AB)$.\\
\textbf {c.}  Tracer la droite parallèle à $(AB)$ passant par $D$ et nomme $N$, le point d'intersection de cette droite avec la droite $(BE)$.\\
\textbf {d.}  Tracer la droite parallèle à $(AB)$ passant par $E$ et nomme $O$, le point d'intersection de cette droite avec la droite $(CM)$.
\begin{center}
  \begin{tikzpicture}[baseline,scale = 0.5]

    \tikzset{
      point/.style={
        thick,
        draw,
        cross out,
        inner sep=0pt,
        minimum width=5pt,
        minimum height=5pt,
      },
    }
    \clip (-1,-8) rectangle (12,6);
    \draw[color ={black},line width = 0.625,opacity = 0.8] (-0.19,0.19)--(0.19,-0.19);\draw[color ={black},line width = 0.625,opacity = 0.8] (-0.19,-0.19)--(0.19,0.19);\draw[color ={black},line width = 0.625,opacity = 0.8] (9.81,-1.81)--(10.19,-2.19);\draw[color ={black},line width = 0.625,opacity = 0.8] (9.81,-2.19)--(10.19,-1.81);\draw[color ={black},line width = 0.625,opacity = 0.8] (1.81,3.19)--(2.19,2.81);\draw[color ={black},line width = 0.625,opacity = 0.8] (1.81,2.81)--(2.19,3.19);\draw[color ={black},line width = 0.625,opacity = 0.8] (7.81,-5.81)--(8.19,-6.19);\draw[color ={black},line width = 0.625,opacity = 0.8] (7.81,-6.19)--(8.19,-5.81);\draw[color ={black},line width = 0.625,opacity = 0.8] (10.81,3.19)--(11.19,2.81);\draw[color ={black},line width = 0.625,opacity = 0.8] (10.81,2.81)--(11.19,3.19);
    \draw [color={black}] (-0.5,0.5) node[anchor = center,scale=1, rotate = 0] {A};
    \draw [color={black}] (10.5,-1.5) node[anchor = center,scale=1, rotate = 0] {B};
    \draw [color={black}] (1.5,3.5) node[anchor = center,scale=1, rotate = 0] {C};
    \draw [color={black}] (8,-5.5) node[anchor = center,scale=1, rotate = 0] {D};
    \draw [color={black}] (10.5,3) node[anchor = center,scale=1, rotate = 0] {E};
    \draw[color={black}] (-49.03,9.81)--(59.03,-11.81);
    
    

  \end{tikzpicture}
\end{center}
\end{exercicedevoir}

\nonewpage
\end{document}

%%% Local Variables:
%%% mode: LaTeX
%%% TeX-master: t
%%% TeX-master: t
%%% End:

