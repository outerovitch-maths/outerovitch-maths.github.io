%% Font size %%
\documentclass[11pt]{article}

%% Load the custom package
\usepackage{Mathdoc}

%% Numéro de séquence %% Titre de la séquence %%
\renewcommand{\centerhead}{Proportionnalité - DS 3}

%% Spacing commands %%
\renewcommand{\baselinestretch}{1} \setlength{\parindent}{0pt}

\begin{document}

\phantom{0}
\vspace{-.5cm}

\entetedevoirs{20}

\begin{center}
\duree{55 minutes} 
\coefficient{1}
\calculatrice{1}
\end{center}

\renewcommand{\arraystretch}{2}

\begin{exercicedevoir}[6][Compléter un tableau]
Avec la méthode de votre choix, déterminer la valeur manquante dans les tableaux de
proportionnalité suivants. Justifier.
\begin{multicols}{2}
\begin{enumerate}
\item 
$\renewcommand{\arraystretch}{1.5}
\begin{array}{|c|c|c|c|} \hline
\text{Quantité A} & \phantom{000}11\phantom{000} & \phantom{000}\ldots\phantom{000} & \phantom{000}5\phantom{000} \\
\hline
\text{Quantité B} & 55 & 45 & 25 \\ 
\hline
\end{array}$ \\ \\ \\ \\ \\ \\

\item 
$\renewcommand{\arraystretch}{1.5}
\begin{array}{|c|c|c|c|} \hline
\text{Quantité A} & \phantom{000}12\phantom{000} & \phantom{000}\ldots\phantom{000} & \phantom{000}9\phantom{000} \\
\hline
\text{Quantité B} & 8 & 6 & 6 \\ 
\hline
\end{array}$ \\ \\ \\ \\ \\ \\

\item 
$\renewcommand{\arraystretch}{1.5}
\begin{array}{|c|c|c|c|}
\hline
\text{Quantité A} & \phantom{000}15\phantom{000} & \phantom{000}10\phantom{000} & \ldots \\ 
\hline
\text{Quantité B} & 3 & 2 & \phantom{000}5\phantom{000} \\ 
\hline
\end{array}$ \\ \\ \\ \\ \\ \\

\item 
$\renewcommand{\arraystretch}{1.5}
\begin{array}{|c|c|c|c|}
\hline
\text{Quantité A} & \phantom{000}18\phantom{000} & \phantom{000}12\phantom{000} & \phantom{000}6\phantom{000} \\ 
\hline
\text{Quantité B} & 9 & \ldots & 3 \\ 
\hline
\end{array}$ \\ \\ \\ \\ \\ \\

\end{enumerate}
\end{multicols}



\end{exercicedevoir}

\begin{exercicedevoir}[6][Problème à résoudre]
Ahmed a repéré, à la boutique, des bonbons qui l'intéressent. Il sait que 6 paquets coûtent 800FCFA.
\begin{multicols}{2}
\begin{enumerate}
\item Il veut en acheter 36.\\
Combien va-t-il dépenser ? \\ \dtf \\ \dtf 
\columnbreak
\item Anta veut elle aussi acheter ces bonbons. Elle dispose de 1~600FCFA.\\
Combien peut-elle en acheter\,?\\\dtf \\\dtf
\end{enumerate}

\end{multicols}
\begin{center}
\begin{tabular}{|c|c|c|c|}
\hline
\kern5cm  & \kern1cm & \kern1cm & \kern1cm  \\
\hline
\kern5cm  & \kern1cm & \kern1cm & \kern1cm  \\
\hline
\end{tabular}
\end{center}
\end{exercicedevoir}

\begin{exercicedevoir}[8][Problème à prise d'initiative]
Voici la recette du far breton pour 6 personnes :
\begin{itemize}
    \item Dans un grand saladier, mélanger 4 œufs avec 120g de sucre jusqu'à blanchiment ;
    \item Ajouter 180g de farine petit à petit en évitant les grumeaux ;
    \item Incorporer 1L lait progressivement puis laisser reposer 1h;
    \item Ajouter 16 pruneaux ;
    \item Verser dans un moule beurré et enfourner à 180°C pendant 45 minutes.
\end{itemize}
Marie souhaite adapter cette recette pour 12 personnes, calculer les
quantités nécessaires.
\end{exercicedevoir}

\encart{14cm}

\end{document}

%%% Local Variables:
%%% mode: LaTeX
%%% TeX-master: t
%%% TeX-master: t
%%% End:

