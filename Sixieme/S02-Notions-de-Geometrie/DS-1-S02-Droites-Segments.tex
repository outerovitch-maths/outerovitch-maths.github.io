%% Font size %%
\documentclass[11pt]{article}

%% Load the custom package
\usepackage{Mathdoc}

%% Numéro de séquence %% Titre de la séquence %%
\renewcommand{\centerhead}{}

%% Spacing commands %%
\renewcommand{\baselinestretch}{1} \setlength{\parindent}{0pt}

\begin{document}

\entetedevoirs{10}

\begin{exercicedevoir}[10]

\begin{enumerate}
\item Repasser en \underline{rouge} les \emph{droites}.
\item Repasser en \underline{vert} les \emph{demi-droites}.
\item Repasser en \underline{bleu} les \emph{segments}.


\bigskip

\begin{center}
\begin{tikzpicture}[scale=1.2, every node/.style={inner sep=5pt, font=\small}]
\clip (-3,-3) rectangle (8,5);
% Points (coordonnées choisies pour une belle disposition)
\coordinate (K) at (0,0); \coordinate (L) at (3,0.5); \coordinate (M) at (1.5,2.2); \coordinate (N) at (-1,1.7); \coordinate (O) at (4,-1.2); \coordinate (P) at (6,1.6); \coordinate (Q) at (2.7,3.2); \coordinate (R) at (-2,-0.6); \coordinate (S) at (0,3.8); \coordinate (T) at (5,-0.6); \coordinate (U) at (-1.8,2.9); \coordinate (V) at (7,0.2);

% Draw the "background" figure: points as filled circles + labels
\foreach \pt in {K,L,N,O,P,Q,T,U,V}{ \draw (\pt) +(-0.1,0.1) -- +(0.1,-0.1) (\pt) +(-0.1,-0.1) -- +(0.1,0.1); }
% Labels slightly offset
\node[above left] at (K) {K}; \node[above right] at (L) {L};
% \node[above] at (M) {M};
\node[left] at (N) {N}; \node[below left] at (O) {O}; \node[above right] at (P) {P}; \node[above right] at (Q) {Q}; \node[right] at (S) {S}; \node[below] at (T) {T}; \node[left] at (U) {U}; \node[right] at (V) {V};

% --- ELEMENTS à identifier (tracés en noir fin pour l'exercice) --- Droites (tracer en étendant la ligne dans les deux directions) droite (KL)
\draw[line width=0.6pt] ($(K)!-3!(L)$) -- ($(K)!4!(L)$);
% droite (NR)
\draw[line width=0.6pt] ($(N)!-3!(R)$) -- ($(N)!4!(R)$);
% droite (PV)
\draw[line width=0.6pt] ($(P)!-4!(V)$) -- ($(P)!4!(V)$);

% Demi-droites (rayons) [from first point through second, extend in one direction] demi-droite [MS)
\draw[line width=0.6pt] (M) -- ($(M)!4!(S)$);
% demi-droite [LO)
\draw[line width=0.6pt] (L) -- ($(L)!4!(O)$);
% demi-droite [QK)
\draw[line width=0.6pt] (Q) -- ($(Q)!4!(K)$);

% Segments
\draw[line width=0.6pt] (T) -- (O); % [TO]
\draw[line width=0.6pt] (U) -- (N); % [UN]
\draw[line width=0.6pt] (L) -- (M); % [LM]

% (optionnel) petites marques pour différencier visualement (non colorées)
\foreach \pt in {K,L,M,N,O,P,Q,R,S,T,U,V}{ \draw[black] (\pt) circle
  (0.0cm); % point already drawn
}


\end{tikzpicture}
\end{center}

\item Récapituler ces résultats dans le tableau.
\begin{center}
\renewcommand{\arraystretch}{1.8}
\begin{tabular}{|p{0.2\linewidth}|p{0.6\linewidth}|}
\hline
Droites &  \\
\hline
Demi-droites &  \\
\hline
Segments &  \\
\hline
\end{tabular}
\end{center}

\end{enumerate}
\end{exercicedevoir}
% ---------------------------
% SOLUTION (décommentez pour voir les couleurs)
% ---------------------------
% Pour afficher la solution, remplacez la ligne "%<solution>" par "" (décommenter tout le bloc).
%
% Exemple : enlevez les '%' devant \begin{center} et le bloc suivant.
%
% Solution : droites en rouge, demi-droites en vert, segments en bleu.
%
%\begin{center}
% \begin{tikzpicture}[scale=1, every node/.style={inner sep=1pt, font=\small}]
%  % mêmes points
%  \coordinate (K) at (0,0);
%  \coordinate (L) at (3,0.5);
%  \coordinate (M) at (1.5,2.2);
%  \coordinate (N) at (-1,1.7);
%  \coordinate (O) at (4,-1.2);
%  \coordinate (P) at (6,1.6);
%  \coordinate (Q) at (2.7,3.2);
%  \coordinate (R) at (-2,-0.6);
%  \coordinate (S) at (0,3.8);
%  \coordinate (T) at (5,-0.6);
%  \coordinate (U) at (-1.8,2.9);
%  \coordinate (V) at (7,0.2);
%
%  % Points
%  \foreach \pt in {K,L,M,N,O,P,Q,R,S,T,U,V}{
%    \fill (\pt) circle (1.8pt);
%  }
%  \node[above left]  at (K) {K};
%  \node[right]       at (L) {L};
%  \node[above]       at (M) {M};
%  \node[left]        at (N) {N};
%  \node[below right] at (O) {O};
%  \node[above right] at (P) {P};
%  \node[above right] at (Q) {Q};
%  \node[left]        at (R) {R};
%  \node[above]       at (S) {S};
%  \node[below]       at (T) {T};
%  \node[left]        at (U) {U};
%  \node[right]       at (V) {V};
%
%  % Solutions colorées
%  % Droites en rouge
%  \draw[red, line width=1pt] ($(K)!-3!(L)$) -- ($(K)!4!(L)$); % (KL)
%  \draw[red, line width=1pt] ($(N)!-3!(R)$) -- ($(N)!4!(R)$); % (NR)
%  \draw[red, line width=1pt] ($(P)!-4!(V)$) -- ($(P)!4!(V)$); % (PV)
%
%  % Demi-droites en vert
%  \draw[green!60!black, line width=1pt] (M) -- ($(M)!4!(S)$); % [MS)
%  \draw[green!60!black, line width=1pt] (L) -- ($(L)!4!(O)$); % [LO)
%  \draw[green!60!black, line width=1pt] (Q) -- ($(Q)!4!(K)$); % [QK)
%
%  % Segments en bleu
%  \draw[blue, line width=1pt] (T) -- (O); % [TO]
%  \draw[blue, line width=1pt] (U) -- (N); % [UN]
%  \draw[blue, line width=1pt] (L) -- (M); % [LM]
%
% \end{tikzpicture}
%\end{center}

\end{document}

%%% Local Variables:
%%% mode: LaTeX
%%% TeX-master: t
%%% TeX-master: t
%%% End:

