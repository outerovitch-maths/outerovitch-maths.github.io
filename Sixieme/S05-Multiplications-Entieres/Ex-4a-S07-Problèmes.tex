%% Font size %%
\documentclass[11pt]{article}

%% Load the custom package
\usepackage{Mathdoc}

%% Numéro de séquence %% Titre de la séquence %%
\renewcommand{\centerhead}{Résolution de problèmes}

%% Spacing commands %%
\renewcommand{\baselinestretch}{1} \setlength{\parindent}{0pt}

\begin{document}

\begin{multicols}{2}
\begin{exercice}[1]
J'ai 38 billes et je veux faire des paquets de 3 billes. \\
- Combien de billes devrais-je mettre par paquets ? \\
- Combien de billes me restera-t-il ?
\renewcommand{\columnseprule}{1pt}
\begin{multicols}{2}
\begin{center}
Calculs
\end{center}
\phantom{0} \\ \phantom{0} \\ \phantom{0} \\ \phantom{0} \\
\columnbreak
\begin{center}
Réponse
\end{center}
\end{multicols}
\end{exercice}

\begin{exercice}[1]
J'ai 72 jouets et je veux les ranger par 5.\\
- Combien de jouets devrais-je mettre par tas ?\\
- Combien de jouets me restera-t-il ?
\renewcommand{\columnseprule}{1pt}
\begin{multicols}{2}
\begin{center}
Calculs
\end{center}
\phantom{0} \\ \phantom{0} \\ \phantom{0} \\ \phantom{0} \\
\phantom{0} \\ 
\columnbreak
\begin{center}
Réponse
\end{center}
\end{multicols}
\end{exercice}
\end{multicols}

\begin{exercice}[1][Sur le cahiers résoudre les problèmes suivants]
\begin{multicols}{2}
\begin{enumerate}
\item Pour son goûter d’anniversaire Kamel a besoin de $720$ biscuits.\\
Les biscuits sont vendus par paquet de $8$.\\
Combien de paquets Kamel doit-il acheter ?
\item Les bouteilles sont vendues par pack de $3$ bouteilles.\\
Je dois acheter $18$ bouteilles.\\
Combien de packs dois-je acheter ?
\item Un fleuriste a $80$ roses.\\
Combien de bouquets de $10$ roses peut-il faire au maximum ?
\item Un immeuble de $9$ étages comporte $72$ appartements. \\
Il y a le même nombre d’appartements à chaque étage.\\
Combien d’appartements y a-t-il à chaque étage ?
\end{enumerate}
\end{multicols}
\end{exercice}

\begin{exercice}[1][Sur le cahiers résoudre les problèmes suivants]
\begin{multicols}{2}
\begin{enumerate}
\item Trois amis mangent au restaurant. L'addition s'élève à $87$ euros.
Les amis décident de partager la note en trois.\\
Quelle est la somme payée par chacun ?
\item Sept amis mangent au restaurant. L'addition s'élève à $175$
euros.
Les amis décident de partager la note en sept.\\
Quelle est la somme payée par chacun ?
\item Six amis mangent au restaurant. L'addition s'élève à $132$
euros.
Les amis décident de partager la note en six.\\
Quelle est la somme payée par chacun ?
\item Six amis mangent au restaurant. L'addition s'élève à $162$
euros.
Les amis décident de partager la note en six.\\
Quelle est la somme payée par chacun ?
\end{enumerate}
\end{multicols}
\end{exercice}

\begin{exercice}[1][Sur le cahiers résoudre les problèmes suivants]
\begin{multicols}{2}
\begin{enumerate}
\item Théo désire s’acheter des CD. Il dispose de 245 €. Un CD coûte 7 €.\\
Combien de CD peut-il acheter ? Combien lui restera-t-il d’argent ?
\item Pour l’achat d’un téléviseur LCD de 796 €, j’ai donné un premier
versement de 340€ et le reste en 6 mensualités. \\
Quel est le montant de chaque mensualité ?
\end{enumerate}
\end{multicols}
\end{exercice}


\end{document}

% Local Variables:
% gptel-model: deepseek-chat
% gptel--backend-name: "DeepSeek"
% gptel--bounds: ((498 . 532) (751 . 919) (1148 . 1149) (1862 . 1864) (2658 . 2659) (3122 . 3123))
% End:
