%% Font size %%
\documentclass[11pt]{article}

%% Load the custom package
\usepackage{Mathdoc}

%% Numéro de séquence %% Titre de la séquence %%
\renewcommand{\centerhead}{Chap. 7 : Multiplications, divisions}

%% Spacing commands %%
\renewcommand{\baselinestretch}{1} \setlength{\parindent}{0pt}

\begin{document}


\section{Multiplications}

\begin{rappel}
\begin{tabular}{ccccc}
$12,3$&$\times$&$3,2$&=&$39,36$ \\
facteur 1& $\times$ & facteur 2& = & produit \\
\end{tabular}
\end{rappel}

\begin{exercicedevoir}[0]
\textit{Calculer $4,5 \times 2,1$.}
\end{exercicedevoir}

\begin{multicols}{2}
\hfill\opmul[resultstyle=\color{black},displayshiftintermediary=all,displayintermediary=all,decimalsepsymbol={,}] {4,5}{2,1}\hfill \hfill \\ \\

\columnbreak

\circled{1} On pose correctement la multiplication. \\
\circled{2} On calcule $2 \times 4,5$. \\
\circled{3} On pose le \textbf{0} ou le \textbullet \\
\circled{4} On calcule $1 \times 4,5$. \\
\circled{5} On calcul la somme des deux résultats. \\
\circled{6} On place la virgule au bon endroit (Deux chiffres après la virgules).
\end{multicols}
\section{Multiplications par 10, 100 \ldots}

\begin{propriete}
Multiplier un nombre \textbf{entier} par 10, 100, 1~000 \ldots ~
revient à ajouter autant de zéros à droite de ce nombre.
\end{propriete}

\begin{propriete}
Multiplier un nombre \textbf{décimal} par 10, 100, 1~000 \ldots ~
revient à décaler la virgule vers la \underline{droite} du nombre de
rang correspondant au nombre de zéros.
\end{propriete}

\begin{exemple}
\phantom{} \vspace{-1cm}
\begin{multicols}{2}
\begin{enumerate}[label=\Alph*]
\item $= 12 \times 10 $ \\ $ =120$
\item $= 2,5 \times 10 $ \\ $ =25$
\item $= 657 \times 100 $ \\ $ =65~700$
\item $= 45,663 \times 100 $ \\ $ =4~566,3$
\item $= 345~645 \times 10 $ \\ $ =3~456~450$
\item $= 0,7 \times 1~000 $ \\ $ =700$
\end{enumerate}
\end{multicols}
\end{exemple}

\section{Multiplication par 0,1, 0,01, 0,001 \ldots}

\begin{propriete}
Multiplier un nombre \textbf{entier} ou \textbf{décimal} par 0,1,
0,01, 0,001 \ldots ~revient à décaler la virgule vers la
\underline{gauche} du nombre de rang correspondant au nombre de zéros.
\end{propriete}

\begin{exemple}
\phantom{} \vspace{-1cm}
\begin{multicols}{2}
\begin{enumerate}[label=\Alph*]
\item $= 12 \times 0,1 $ \\ $ =1,2$
\item $= 134,9 \times 0,01 $ \\ $ =1,349$
\item $= 180~342 \times 0,001 $ \\ $ =180,342$
\item $= 0,003~5 \times 0,01 $ \\ $ =0,000~035$
\end{enumerate}
\end{multicols}
\end{exemple}

\end{document}
