%% Font size %%
\documentclass[11pt]{article}

%% Load the custom package
\usepackage{Mathdoc}

%% Numéro de séquence %% Titre de la séquence %%
\renewcommand{\centerhead}{Chap. 4 : Le vocabulaire du cercle}

%% Spacing commands %%
\renewcommand{\baselinestretch}{1}
\setlength{\parindent}{0pt}

\begin{document}

\section{Vocabulaire}

\begin{definition}
  \textbf{Un cercle} est l'ensemble des points situés à la même distance d'un
  point donné.
\end{definition}

\begin{vocabulaire}
  \begin{itemize}
  \item \textbf{Le centre} est le point à partir duquel on mesure la
    distance vers chaque point du cercle.
  \item \textbf{Le rayon} est la distance fixe entre le centre et chaque point
    du cercle.
  \end{itemize}
\end{vocabulaire}

\begin{exemple}
  \begin{center}
    \input{.data/vocabulaire-cercle.tex}
  \end{center}
\end{exemple}

\begin{vocabulaire}
  \begin{enumerate}
  \item On appelle $\alpha$ le cercle de centre O ; 
  \item Les points A, B, C et D \textbf{appartiennent} au cercle
    $\alpha$ ci-dessus. On note $A \in \alpha$ ;
  \item Le segment [OA] est \underline{un} \textbf{rayon} du cercle
    $\alpha$ ;
  \item Le segment [DB] est \underline{un} \textbf{segment} du cercle
    $\alpha$ ;
  \item Le segment [CD] est \textbf{une corde}.
  \end{enumerate}
\end{vocabulaire}

\begin{remarque}
Il faut bien distinguer \textbf{le/un} rayon et \textbf{le/un}
diamètre. \\
Ce sera toujours \textbf{le} centre et \textbf{une} corde.            
\end{remarque}

\newpage

\section{Tracer un cercle}




\end{document}
