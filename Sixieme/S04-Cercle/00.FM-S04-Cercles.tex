%% Font size %%
\documentclass[11pt]{article}

%% Load the custom package
\usepackage{Mathdoc}

%% Numéro de séquence %% Titre de la séquence %%
\renewcommand{\centerhead}{Chap. 4 : Le vocabulaire du cercle}

%% Spacing commands %%
\renewcommand{\baselinestretch}{1}
\setlength{\parindent}{0pt}

\begin{document}

\section{Vocabulaire}

\begin{definition}
  \textbf{Un cercle} est l'ensemble des points situés à la même distance d'un
  point donné.
\end{definition}

\begin{vocabulaire}
  \begin{itemize}
  \item \textbf{Le centre} est le point à partir duquel on mesure la
    distance vers chaque point du cercle.
  \item \textbf{Le rayon} est la distance fixe entre le centre et chaque point
    du cercle.
  \end{itemize}
\end{vocabulaire}

\begin{exemple}
  \begin{center}
    \begin{tikzpicture}[scale=2]

\draw[white] (0,0) grid (4,4) ;

\draw (2,2) circle (1.5cm);

\coordinate[label=left:$O$] (O) at (2,2);

\coordinate[label=right:$A$] (A) at (2.9,0.8);
\coordinate[label=below:$D$] (D) at (1,0.88);
\coordinate[label=left:$B$] (B) at (0.63,2.6);
\coordinate[label=right:$C$] (C) at (3,3.12);

\coordinate[label=right:$\alpha$] (alph) at (1.8,3.7);

\tkzDrawPoint[shape=cross,minimum size= 5pt](O)
\tkzDrawPoint[shape=cross,minimum size= 5pt](A)
\tkzDrawPoint[shape=cross out,minimum size= 5pt](B)
\tkzDrawPoint[shape=cross,minimum size= 5pt](C)
\tkzDrawPoint[shape=cross,minimum size= 5pt](D)

\draw[black] (D) -- (C);
\draw[black] (O) -- (A);
\draw[black] (D) -- (B);

\begin{scriptsize}
\tkzLabelSegment[sloped](O,A){Rayon}
\tkzLabelSegment[sloped](O,C){Diamètre}
\tkzLabelSegment[sloped](B,D){Corde}
\end{scriptsize}

\end{tikzpicture}

%%% Local Variables:
%%% mode: LaTeX
%%% TeX-master: "../00.FM-S04-Cercles"
%%% End:

  \end{center}
\end{exemple}

\begin{vocabulaire}
  \begin{enumerate}
  \item On appelle $\alpha$ le cercle de centre O ; 
  \item Les points A, B, C et D \textbf{appartiennent} au cercle
    $\alpha$ ci-dessus. On note $A \in \alpha$ ;
  \item Le segment [OA] est \underline{un} \textbf{rayon} du cercle
    $\alpha$ ;
  \item Le segment [DB] est \underline{un} \textbf{segment} du cercle
    $\alpha$ ;
  \item Le segment [CD] est \textbf{une corde}.
  \end{enumerate}
\end{vocabulaire}

\begin{remarque}
Il faut bien distinguer \textbf{le/un} rayon et \textbf{le/un}
diamètre. \\
Ce sera toujours \textbf{le} centre et \textbf{une} corde.            
\end{remarque}

\newpage


\section{Tracer un cercle}

\section{Programme de construction}


\end{document}
