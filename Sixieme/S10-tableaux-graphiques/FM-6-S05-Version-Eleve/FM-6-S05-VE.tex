\input{~/Documents/Cours/.parametre/parametre-fm/parametre-fm.tex}

\begin{document}

\entete{Chapitre 5}

\renewcommand{\baselinestretch}{1.5}

\colonnesep{1}

\vfill

\titre{Gestion de données}{4.5}

\section{Tableau simple}

\vfill

\definition{
\dtf \\ \dtf \\ \dtf }

\vfill

\exemple{
Nombre d'habitants des quatres villes françaises les plus peuplées en 2013 :

\vfill

\cntr{
\begin{tabular}{|c|c|c|c|c|}
\hline
Ville&Paris&Marseille&Lyon&Toulouse \tabline
Nombre d'habitants&2~201~000&826~700&467~400&437~100 \tabline
\end{tabular}}

\vfill

La quatrième colonne indique que Lyon avait \dotfill habitants en 2013.}

\vfill

\exemple{
Altitude des trois plus hauts sommets africains :

\vfill

\cntr{
\begin{tabular}{|c|c|}
\hline
Nom du sommet&Altitude (en m) \tabline
Kilimandjaro&5~815 \tabline
Mont Kenya&5~199 \tabline
Mawenzi&5~149 \tabline
\end{tabular}}

\vfill

La troisième ligne indique que le \dotfill a une altitude de 5~199 m. }

\newpage
\section{Tableau à double entrée}

\vspace{1cm}

\definition{
\dtf \\ \dtf \\ \dtf}

\vspace{0.5cm}

\exemple{
Nombre de tee-shirts dans une boutique de vêtement :

\vspace{1cm}

\cntr{
\rowcolors{2}{gray!10}{gray!40}
\begin{tabular}{|c||c|c|c|c|c||c|}
\hline
\backslashbox{Taille}{Couleur}&Jaune&	Bleu&	Rouge&	Vert&	Orange&	Total \\ \hline \hline
S&	10&	12&	8&	5&	4&	39 \tabline 
M&	9&	12&	10&	11&	9&	51 \tabline
L&	12&	15&	17&	16&	12&	72 \tabline
XL&	8&	7&	5&	6&	8&	34 \tabline \hline
Total&	39&	46&	40&	38&	33&	196 \tabline
\end{tabular}}

\vspace{1cm}

On lit dans le tableau que l'on peut trouver \dotfill tee-shirts verts en taille M dans la boutique. \\
Et que la taille ayant le plus de quantité est le \ldots\ldots\ldots\ldots}

\newpage
\section{Graphique cartésien}

\vfill

\definition{
\dtf \\ \dtf \\ \dtf \\ \dtf }

\vfill

\remarque{
\dtf \\ \dtf}

\vfill

\exemple{
Ce graphique cartésien représente l'évolution de la population française en fonction de l'année :

\vfill

\cntr{\input{figures/graph1.txt}}

\vfill

On y lit que la population française atteint environ \dotfill millions d'habitants en 2010  \\ alors qu'elle était d'environ 55 millions d'habitants en \dotfill.}

\newpage

\section{Diagramme en bâtons}

\definition{
\dtf \\ \dtf \\ \dtf \\ \dtf }

\vfill

\exemple{
Voici le diagramme en bâtons représentant une série de notes obtenue par une classe :

\vfill

\cntr{\input{figures/graph2.txt}}

\vfill

On y lit que \dotfill élèves ont eu plus de 10.}

\newpage
\section{Diagramme circulaire}

\vspace{1cm}

\definition{
\dtf \\ \dtf \\ \dtf \\ \dtf }

\vspace{1cm}

\exemple{
Voici le diagramme circulaire représentant le moyen de transport principal utilisé par les élèves d'un collège pour venir en cours :

\vspace{1cm}

\cntr{\input{figures/graph3.txt}}

\vspace{1cm}

On y lit que \ldots\ldots\ldots\ldots\ldots\ldots des élèves prennent le bus.}


\end{document}
