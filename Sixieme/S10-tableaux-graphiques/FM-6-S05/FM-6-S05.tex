\input{~/Documents/Cours/.parametre/parametre-fm/parametre-fm.tex}

\begin{document}

\entete{Chapitre 5}

\renewcommand{\baselinestretch}{1.5}

\colonnesep{1}

\titre{Gestion de données}{4}

\section{Tableau simple}

\definition{
Un tableau simple est un tableau qui comprend uniquement deux lignes ou uniquement deux colonnes.}

\exemple{
Nombre d'habitants des quatres villes françaises les plus peuplées en 2013 :
\cntr{
\begin{tabular}{|c|c|c|c|c|}
\hline
Ville&Paris&Marseille&Lyon&Toulouse \tabline
Nombre d'habitants&2~201~000&826~700&467~400&437~100 \tabline
\end{tabular}}

La quatrième colonne indique que Lyon avait 467~400 habitants en 2013.}

\exemple{
Altitude des trois plus hauts sommets africains :

\cntr{
\begin{tabular}{|c|c|}
\hline
Nom du sommet&Altitude (en m) \tabline
Kilimandjaro&5~815 \tabline
Mont Kenya&5~199 \tabline
Mawenzi&5~149 \tabline
\end{tabular}}


La troisième ligne indique que le Mont Kenya a une altitude de 5~199 m. }

\newpage
\section{Tableau à double entrée}

\definition{
Un tableau à double entrée est un tableau qui comporte plus de deux lignes et plus de deux colonnes de données.}

\exemple{
Nombre de tee-shirts dans une boutique de vêtement :

\cntr{
\rowcolors{2}{gray!10}{gray!40}
\begin{tabular}{|c|c|c|c|c|c|c|c|}
\hline
\backslashbox{Taille}{Couleur}&Jaune&	Bleu&	Rouge&	Vert&	Orange&	Total \tabline
S&	10&	12&	8&	5&	4&	39 \tabline
M&	9&	12&	10&	11&	9&	51 \tabline
L&	12&	15&	17&	16&	12&	72 \tabline
XL&	8&	7&	5&	6&	8&	34 \tabline
Total&	39&	46&	40&	38&	33&	196 \tabline
\end{tabular}}

On lit dans le tableau que l'on peut trouver 11 tee-shirts verts en taille M dans la boutique.}

\newpage
\section{Graphique cartésien}

\definition{
Un graphique cartésien est une représentation qui permet de visualiser l'évolution d'une grandeur (en ordonnées) « en fonction » d'une autre (en abscisse).}

\remarque{
L'abscisse se lit sur l'axe horizontal et l'ordonnée se lit sur l'axe vertical.}

\exemple{
Ce graphique cartésien représente l'évolution de la population française en fonction de l'année :

\cntr{\input{figures/graph1.txt}}

On y lit que la population française atteint environ 65 millions d'habitants en 2010 alors qu'elle était d'environ 55 millions d'habitants en 1990.}

\newpage

\section{Diagramme en bâtons}

\definition{
Un diagramme en bâtons (ou en barres) est formé de bâtons (de barres) dont les hauteurs sont proportionnelles aux nombres qu'ils représentent.}

\exemple{
Voici le diagramme en bâtons représentant une série de notes obtenue par une classe :

\cntr{\input{figures/graph2.txt}}

On y lit que 12 élèves ont eu au moins la moyenne.}

\newpage
\section{Diagramme circulaire}

\definition{
Un diagramme circulaire est formé de secteurs circulaires dont la mesure des angles est proportionnelle aux nombres ou aux pourcentages qu'ils représentent.}

\exemple{
Voici le diagramme circulaire représentant le moyen de transport principal utilisé par les élèves d'un collège pour venir en cours :

\cntr{\input{figures/graph3.txt}}

On y lit que 45 \% des élèves prennent le bus.}


\end{document}
