\input{~/Documents/Cours/.parametre/parametre-fm/parametre-fm.tex}

\begin{document}

\entete{Chapitre 6}

\titre{Conversions}{5}

\renewcommand{\baselinestretch}{1.5}

\colonnesep{1}

\section{Unités de mesure internationales}

\textit{Pour simplifier les échanges scientifiques entre les pays du monde entiers, certaines unités ont été fixées internationalement.}

\definition{ \phm \vspace{-0.5cm}
\itmz{
	\item Pour mesurer des longueurs, on utilise une longueur de référence, appelée mètre, que l’on peut noter en abrégé « m ».
    \item Pour mesurer des masses, on utilise une masse de référence, appelée gramme, que l’on peut noter en abrégé « g ».
    \item Pour mesurer des contenances, on utilise un volume de référence, appelée litre, que l’on peut noter en abrégé « L ».
    \item Pour mesurer un temps, on utilise un temps de référence, appelée seconde, que l’on peut noter en abrégé « s ».
    } }

\vocabulaire{
Mètres, Grammes, Litres et Secondes s’appellent des unités de mesure internationales.}

\newpage

\section{Tableaux de conversions}

\textit{Comme pour la numération, les conversions se font en base 10. On donc peut utiliser les trois tableaux similaires suivants.} \\

\textbf{Pour les longueurs :}
\cntr{\input{data/tab-longueur.txt}}

\textbf{Pour les masses :}
\cntr{\input{data/tab-masse.txt}}

\textbf{Pour les contenances :}
\cntr{\input{data/tab-contenance.txt}}

\exemple{

1\ km = 10\ hm = 100\ dam = 1\ 000\ m et 1\ m = 10\ dm = 100\ cm = 1\ 000\ mm.

1\ kg = 10\ hg = 100\ dag = 1\ 000\ g et 1 g\ = 10\ dg = 100\ cg = 1\ 000\ mg.

1\ kL = 10\ hL = 100\ daL = 1\ 000\ L et 1 L\ = 10\ dL = 100\ cL = 1\ 000\ mL. }

\remarque{
Ce principe de base ne fonctionne pas pour le temps. En effet, les secondes sont exprimées en base 60. 60 secondes donnent 1 minute, 60 minutes donnent 1 heure, 24 heures donnent 1 journée \dots }

\newpage
\section{Périmètre d'une figure}

\vocabulaire{
La mesure du contour d'une figure quelconque est appelée périmètre.}

\colonnesep{0}
\exemple{}
\mc{
\cntr{\input{figures/peri.txt}}

\mc{

\phm

Le périmètre de cette figure est de $16,83$ cm. 

\columnbreak

\cntr{\opmanyadd[voperator=bottom]{4}{1}{2}{3}{4}{2.83}}
}
}

\section{Quadrilatères}
\definition{
Un quadrilatère est un polygone qui possède quatre côtés.}

\begin{multicols}{2}

\definition{
Un rectangle est un quadrilatère qui possède 4 angles droits.}
\cntr{\input{figures/rectangle.txt}}

\definition{
Un carré est un quadrilatère qui possède 4 côtés de même longueurs \underline{et} 4 angles droits.}
\cntr{\input{figures/carre.txt}}

\end{multicols}

\section{Périmètre des polygones}
\definition{
Le périmètre $\mathcal{P}$ d'un polygone est égale à la somme des longueurs de ses côtés}

\exemple{}
{\large
\begin{center}
\begin{tabular}{|l|c|c|}
\hline
Périmètre d'un \underline{carré} de coté $c$ &  $\mathcal{P}_{carre}~=~c \times 4$ \\ \hline
Périmètre d'un \underline{rectangle} de cotés $L$ et $\ell$ &  $\mathcal{P}_{rectangle}~=(~L+\ell~) \times 2$ \\ \hline
Périmètre d'un \underline{losange} de coté $c$ &  $\mathcal{P}_{losange}~=c \times 4$ \\ \hline
Périmètre d'un \underline{triangle équilateral} de coté $c$ \phantom{00} &  $\mathcal{P}_{triangle}~=c \times 3$ \\ \hline
\end{tabular} 
\end{center} }




\end{document}
