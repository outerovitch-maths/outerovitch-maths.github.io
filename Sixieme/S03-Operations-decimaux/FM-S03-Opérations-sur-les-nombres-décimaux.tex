%% Font size %%
\documentclass[11pt]{article}

%% Load the custom package
\usepackage{Mathdoc}

%% Numéro de séquence %% Titre de la séquence %%
\renewcommand{\centerhead}{Chap. 3 : Opérations sur les nombres décimaux}

%% Spacing commands %%
\renewcommand{\baselinestretch}{1}
\setlength{\parindent}{0pt}

\begin{document}

\section{Fractions décimales}

\begin{definition}
  Une fraction décimale est une fraction de dénominateur 10, 100, 1
  000\ldots
\end{definition}

\begin{exemple}
  $\dfrac{34}{100}$ est une fraction décimale. Elle se lit
  trente-quatre centièmes.
\end{exemple}

\section{Ecriture décimale}

\begin{definition}
  Quand on additionne un nombre entier et une ou plusieurs fractions
  décimales, on obtient un nombre appelé nombre décimal.
\end{definition}

\begin{exemple}
  $5+\dfrac{4}{10}+\dfrac{7}{100}$ est un nombre décimal. \\
  Sa partie entière est 5, c'est un nombre \textbf{entier}. \\
  Sa partie \textbf{décimale} est $\dfrac{4}{10}+\dfrac{7}{100}$
\end{exemple}

\begin{definition}
  Tout nombre décimal peut aussi s’écrire sous une autre forme. Elle
  utilise la notation à virgule et s'appelle écriture décimale.
\end{definition}

\begin{exemple}
  \textit{Considérons le nombre décimal :} \\
  $203+\dfrac{4}{10}+\dfrac{5}{100}$ \\
  $203+\dfrac{4}{10}+\dfrac{5}{100}$ s'écrit alors $203{,}452$
\end{exemple}

\begin{remarque}
  On place la virgule entre le chiffre des unités et celui des
  dixièmes.
\end{remarque}

\section{Addtition, soustraction de décimaux}

\textit{Énoncé : }\textbf{Poser et calculer les opérations suivantes : $45,08+1,7$ et $67,453-79,06$.} \\

\hfill \opadd{45,08}{1,7} \hfill \opsub{67,453}{79,06} \hfill \hfill

\begin{remarque}
  On aligne toujours les unités avec les unités mais aussi les
  virgules jusque dans le résultat.
\end{remarque}

\section{Ordre de grandeur}

\begin{definition}
L'\textbf{ordre de grandeur} d'un nombre est une approximation de ce nombre, obtenue en l'arrondissant à une unité choisie (par exemple, la dizaine, la centaine, ou le millier) selon le contexte.
\end{definition}

\textbf{MÉTHODE :} \\
Comment déterminer l'ordre de grandeur ?
\begin{enumerate}
    \item On choisit le niveau de précision souhaité : unité, dizaine, centaine, etc.
    \item On arrondit le nombre en conséquence.
\end{enumerate}

\begin{exemple}
L'ordre de grandeur 450 à la centaine, est 500. \\
Pour 320, l'ordre de grandeur à la centaine est 300. \\
\end{exemple}

\section{Additions et Soustractions}

\begin{exemple}
Donner l'ordre de grandeur de la somme de 475 et 150  à la centaine :
\[
475 \approx 500 \quad \text{et} \quad 150 \approx 200
\]
L'ordre de grandeur de la somme est donc environ \(500 + 200 = 700\).
\end{exemple}

\begin{remarque}
L'estimation avec des ordres de grandeur peut différer du résultat exact, mais elle donne une bonne idée de la valeur attendue.
\end{remarque}

\section{Problème d'Application}

\begin{exercice}
Estime la somme de 475, 150 et 85 en arrondissant chaque nombre à la centaine.
\end{exercice}

\begin{exercice}
Un élève doit acheter trois articles coûtant environ 9,50€, 12,30€, et 3,90€. Peut-il savoir, en arrondissant chaque prix à l'unité, s'il aura assez avec un billet de 20€ ?
\end{exercice}

\begin{exercice}
Un agriculteur doit répartir 875 kg de pommes dans des caisses de 98 kg chacune. En arrondissant les quantités à la centaine, combien de caisses pleines pourra-t-il préparer ?
\end{exercice}

\begin{exercice}
 Claire prend le bus plusieurs fois dans la semaine pour un total de 11,75€. Elle utilise une carte de transport avec un crédit de 15€. Peut-elle savoir rapidement, en arrondissant à l’euro, combien il lui restera sur sa carte après ses trajets ?
\end{exercice}

\end{document}
