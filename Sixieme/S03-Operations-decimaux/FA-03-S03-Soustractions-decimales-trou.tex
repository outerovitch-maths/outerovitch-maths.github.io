%% Font size %%
\documentclass[11pt]{article}

%% Load the custom package
\usepackage{Mathdoc}

%% Numéro de séquence %% Titre de la séquence %%
\renewcommand{\centerhead}{Chap.3 - Fiche 3 : Compléter une opération décimale}

%% Spacing commands %%
\renewcommand{\baselinestretch}{1}
\setlength{\parindent}{0pt}

\begin{document}

%% XLOP CONFIGURATION %% 
\opset{carrysub,lastcarry,columnwidth=2.5ex,offsetcarry=-0.4,decimalsepoffset=-3pt}

\begin{exercice}[2][Compléter les soustractions à trous suivantes.]
\opsub[operandstyle.2=\trou]{328.7}{124.3}\hfill\opsub[operandstyle.2=\trou]{645.9}{312.6}\hfill\opsub[operandstyle.2=\trou]{759.4}{258.2}\hfill\opsub[operandstyle.2=\trou]{804.3}{415.7}\hfill\opsub[operandstyle.2=\trou]{1658.2}{946.5}

\vspace{1cm}

\opsub[operandstyle.2=\trou]{524.7}{315.8}\hfill\opsub[operandstyle.2=\trou]{937.6}{481.2}\hfill\opsub[operandstyle.2=\trou]{1328.5}{759.3}\hfill\opsub[operandstyle.2=\trou]{2857.4}{1638.9}

\vspace{1cm}

\opsub[operandstyle.1.2=\trou,operandstyle.2.1=\trou,resultstyle.4=\trou,resultstyle.3=\trou]{4829.67}{1294.35}\hfill\opsub[operandstyle.2.2=\trou,operandstyle.2.1=\trou,resultstyle.4=\trou,resultstyle.3=\trou]{3927.51}{1648.22}\hfill\opsub[operandstyle.1=\trou]{7532.89}{4856.74}
\end{exercice}


\begin{exercice}[3][Compléter les soustractions à trous suivantes.]
\opsub[operandstyle.2=\trou]{652.8}{314.7}\hfill\opsub[operandstyle.2=\trou]{872.9}{456.3}\hfill\opsub[operandstyle.2=\trou]{1294.6}{587.2}\hfill\opsub[operandstyle.2=\trou]{947.5}{328.9}\hfill\opsub[operandstyle.2=\trou]{1638.4}{789.5}

\vspace{1cm}

\opsub[operandstyle.2=\trou]{845.3}{417.6}\hfill\opsub[operandstyle.2=\trou]{1293.7}{542.8}\hfill\opsub[operandstyle.2=\trou]{2738.4}{1459.6}\hfill\opsub[operandstyle.2=\trou]{3275.9}{1624.7}

\vspace{1cm}

\opsub[operandstyle.1.2=\trou,operandstyle.2.1=\trou,resultstyle.4=\trou,resultstyle.3=\trou]{5382.46}{1938.72}\hfill\opsub[operandstyle.2.2=\trou,operandstyle.2.1=\trou,resultstyle.4=\trou,resultstyle.3=\trou]{6821.59}{2743.16}\hfill\opsub[operandstyle.1=\trou]{7593.68}{3924.57}
\end{exercice}

\begin{exercice}[4][Compléter les soustractions à trous suivantes.]
\opsub[operandstyle.1.2=\trou,operandstyle.2.1=\trou,resultstyle.4=\trou,resultstyle.3=\trou]{4823.67}{2473.52}\hfill\opsub[operandstyle.2.2=\trou,operandstyle.2.1=\trou,resultstyle.4=\trou,resultstyle.3=\trou]{6294.81}{3197.65}\hfill\opsub[operandstyle.1=\trou]{7945.38}{4628.97}

\vspace{1cm}

\opsub[operandstyle.1.2=\trou,operandstyle.2.1=\trou,resultstyle.4=\trou,resultstyle.3=\trou]{7392.84}{3629.75}\hfill\opsub[operandstyle.2.2=\trou,operandstyle.2.1=\trou,resultstyle.4=\trou,resultstyle.3=\trou]{8917.36}{4728.59}\hfill\opsub[operandstyle.1=\trou]{6243.59}{3518.27}
\end{exercice}

\end{document}

