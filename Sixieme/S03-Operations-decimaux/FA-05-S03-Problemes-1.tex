%% Font size %%
\documentclass[11pt]{article}

%% Load the custom package
\usepackage{Mathdoc}

%% Numéro de séquence %% Titre de la séquence %%
\renewcommand{\centerhead}{Chap. 3 - Fiche 4 : Résolution de problèmes}

%% Spacing commands %%
\renewcommand{\baselinestretch}{1}
\setlength{\parindent}{0pt}

\begin{document}

\newcommand{\cntr}[1]{
\begin{center}
#1
\end{center}
}

\begin{exercice}[1][Lire les informations dans un problème.]
  \textbf{Dans chaque problème, cocher les informations qui servent à
  sa résolution.}  \enu{
\item Dans une classe de 33 élèves âgés de 8  à 10  ans, un professeur distribue à chaque enfant 6 livres pesant en moyenne 440 g chacun.\\
  Quelle est la masse moyenne des livres distribués à chaque enfant ?
  \cntr{$\square\;$ 10 ans\qquad$\square\;$ 440 g\qquad$\square\;$ 33
  élèves\qquad$\square\;$ 8 ans\qquad$\square\;$ 6 livres\qquad}
\item Joachim, un élève de 5ème, de 12 ans, mesure 1,43 m. Corinne a 4 ans de plus que Joachim et mesure 40 cm de plus.\\
  Quel est l'âge de Corinne ?  \cntr{$\square\;$ 12
  ans\qquad$\square\;$ 40 cm\qquad$\square\;$ 1,43 m\qquad$\square\;$
  5ème\qquad$\square\;$ 4 ans\qquad}
\item Karole décide de programmer la box de son cousin pour enregistrer un film prévu le mercredi 10 août et une émission prévue le lendemain. Le film doit commencer à 18 h 25 min et se terminer à 20 h 10 min. L'émission commence à 19 h 25 min et dure 45 minutes.\\
  Quelle est la durée prévue du film ?  \cntr{$\square\;$ 20 h 10
  min\qquad$\square\;$ 19 h 25 min\qquad$\square\;$ 18 h 25
  min\qquad$\square\;$ 45 min\qquad$\square\;$ mercredi 10 août\qquad}
\item David décide de programmer la box de son frère pour enregistrer un film prévu le jeudi 21 décembre et une émission prévue le lendemain. Le film doit commencer à 21 h 20 min et se terminer à 23 h 50 min. L'émission commence à 22 h 20 min et dure 54 minutes.\\
  Quelle est la durée prévue du film ?  \cntr{$\square\;$ 22 h 20
  min\qquad$\square\;$ 54 min\qquad$\square\;$ 21 h 20
  min\qquad$\square\;$ jeudi 21 décembre\qquad$\square\;$ 23 h 50
  min\qquad} 
\item Le grand-père de Arthur, âgé de 55 ans, se rend 4 fois par semaine à Paris en train. Une fois arrivé, il prend le métro à 7 h 40 min, après avoir acheté systématiquement le même journal, dans un kiosque de la gare, qui coûte 1,40 €. Son trajet en métro dure 55 minutes pour se rendre au travail.\\
  Combien le grand-père de Arthur dépense-t-il chaque semaine pour son
  journal ?  \cntr{$\square\;$ 7 h 40 min\qquad$\square\;$ 1,40
  €\qquad$\square\;$ 4 fois\qquad$\square\;$ 55 ans\qquad$\square\;$
  55 min\qquad}
\item La sœur de Dalila lui a acheté un superbe vélo de 18 vitesses, coûtant 356 €, avec des roues de 17 pouces. Pour la protéger, son frère lui a offert un casque et du matériel d'éclairage valant 22,06 €. La sœur de Dalila a décidé de payer le vélo en 12 fois.\\
  Quel est le montant total des cadeaux offerts à Dalila ?
  \cntr{$\square\;$ 12 fois\qquad$\square\;$ 17
  pouces\qquad$\square\;$ 356 €\qquad$\square\;$ 22,06
  €\qquad$\square\;$ 18 vitesses\qquad}
\item Un cargo mesurant 109 m transporte 67 gros conteneurs de 8 tonnes chacun du Havre à Auckland. Ce bateau transporte aussi 34 petits conteneurs pour une masse totale de 28 tonnes.\\
  Quelle est la masse de chacun des petits conteneurs, sachant qu'ils
  ont tous la même masse ?\cntr{$\square\;$ 67
  conteneurs\qquad$\square\;$ 28 tonnes\qquad$\square\;$ 8
  tonnes\qquad$\square\;$ 34 conteneurs\qquad$\square\;$ 109 m\qquad}}
\end{exercice}

\newpage

\begin{exercice}[2][Résoudre les problèmes suivants en détaillant les calculs.]
  \begin{enumerate}
\item Un livreur part de son entrepôt avec 25 colis. Au premier arrêt, le plus près, il depose 14 colis. 10 km plus loin, il livre le reste de ses colis. Ensuite, à 12 h 50 min, le livreur reprend la même route et retourne à l'entrepôt, à 29 km de là.\\
Quelle distance sépare l'entrepôt du premier arrêt ?
\item Un cargo mesurant 118 m transporte 68 gros conteneurs de 12 tonnes chacun du Havre à Auckland. Ce bateau transporte aussi 16 petits conteneurs pour une masse totale de 18,72 tonnes.\\
Quelle est la masse, en kg, de chacun des petits conteneurs, sachant qu'ils ont tous la même masse ?
\item Arthur décide de programmer la box de sa sœur pour enregistrer un film prévu le jeudi 25 mai et une émission prévue le lendemain. Le film doit commencer à 16\,h\,15 et se terminer à 18\,h\,40. L'émission commence à 17\,h\,55 et dure 49 minutes.\\
Quelle est la durée prévue du film ?
\item Le cousin de Fernando, âgé de 55 ans, se rend 3 fois par semaine à Bordeaux en train. Une fois arrivé, il prend le métro à 9\,h\,25, après avoir acheté systèmatiquement le même journal, dans un kiosque de la gare, qui coûte 1,50 €. Son trajet en métro dure 40 minutes pour se rendre au travail.\\
Combien le cousin de Fernando dépense-t-il chaque semaine pour son journal ?
\item La grand-mère de Nawel lui a acheté un superbe vélo de 21 vitesses, coûtant 582 €, avec des roues de 18 pouces. Pour la protéger, son voisin lui a offert un casque et du matériel d'éclairage valant 27,04 €. La grand-mère de Nawel a décidé de payer le vélo en 12 fois.\\
Quel est le montant total des cadeaux offerts à Nawel ?
\item Dans une classe de 25 élèves âgés de 9  à 11  ans, un professeur distribue à chaque enfant 7 livres pesant en moyenne 330 g chacun.\\
Quel est le nombre total de livres distribués ?
  \end{enumerate}
\end{exercice}

\begin{multicols}{2}
\begin{exercice}[1]
  Pierre joue à un jeu. Il a 34 points.\\
  Il tombe sur une case " perte de 7 points". \\
  Combien lui reste-t-il de points ?
\end{exercice}
\begin{exercice}[1]
  Gérard est né en 1947, quel âge a-t-il en 2018 ? \\
  Quel âge a sa femme qui est née en 1943 ?
\end{exercice}
\begin{exercice}[1]
  John a 234 photos sur son smartphone. Il décide de supprimer les 61 selfies. \\
  Combien de photos lui reste-t-il ?
\end{exercice}
\begin{exercice}[1]
  Dans un parking qui contient 543 places, 362 voitures se sont déjà garées. Un panneau lumineux indique le nombre de places restantes. \\
  Combien ce panneau indique-t-il ?
\end{exercice}
\end{multicols}

\newpage

\begin{multicols}{2}
  \begin{exercice}[2]
    Louna compte ses paires de boucles d’oreilles. Elle a :
    \begin{enu}
    \item 3 paires avec des chats
    \item 4 paires avec des fleurs
    \item 2 paires avec des papillons
    \item 2 paires avec des cœurs Combien a-t-elle de boucles
      d'oreilles en tout ?
    \end{enu}
  \end{exercice}
\begin{exercice}[2]
  \begin{enu}
  \item Penda achète $3$ kg de carottes à $4$€/kg et 3kg de boeuf à $11$€/kg. \\
    Quel est le prix total à payer ?
  \item Benjamin achète $3$ kg de courgettes à $2$€/kg et $4$ kg de veau à $25$€/kg.\\
  Quel est le prix total à payer ?
  \end{enu}
\end{exercice}
\begin{exercice}[3]
  Un cinéma décide de projeter toute la saga Harry Potter. \\
  Les durées des 8 films sont en minutes : $152$ ; $161$ ; $142$ ; $157$ ; $138$ ; $153$ ; $146$ et $130$. \\
  Quelle sera la durée totale de la projection?
\end{exercice}
\begin{exercice}[3]
  \begin{enumerate}
    \item Sophie a 125 billes. Elle en donne 47 à son ami. Combien de billes lui reste-t-il ?
    \item Un marathonien court un total de 42 km. Après avoir couru 17 km, il prend une pause. Ensuite, il court encore 19 km avant de faire une autre pause. Combien de kilomètres lui restent-ils à courir ?
    \item Lucas a 315 € sur son compte bancaire. Il achète un smartphone à 159 € et une tablette à 99 €. Combien d'argent lui reste-t-il après ses achats?
    %\item Léa a une collection de 480 cartes postales. Elle en reçoit 150 de son cousin et en donne 89 à une amie. Combien de cartes postales a-t-elle maintenant ?
  \end{enumerate}
\end{exercice}
\begin{exercice}[4]
  Léa repère des casquettes dans un magazine de publicité à $10$€
  l'unité.
  \begin{enu}
  \item Quel serait le prix de $10$ casquettes ?
  \item Quel serait le prix de $7$ casquettes?
  \item Si Léa achetait une casquette à $10$€ l'unité puis d'autres
    articles pour $36$€ quel serait le prix final ?
  \item Léa dispose d'un bon de réduction de $5$€. Si Léa achetait
    deux casquettes, quelle somme d'argent paierait Léa au final ?
  \item Si Léa achetait $4$ casquettes et son grand-père en achetait
    également $9$, quelle somme d'argent paieraient-ils à eux deux ?
  \item Si Léa achetait $11$ casquettes mais que sa sœur lui propose
    de lui en rembourser $3$, quelle somme d'argent Léa
    dépenserait-elle ?
  \end{enu}
\end{exercice}
\begin{exercice}[4]
    \begin{enu}
    \item Béatrice dit à Yasmine : « J'ai $65$ €, soit $11$ € de moins que toi. » \\
      Combien d'argent en euros possèdent en tout les deux filles ?
    \item Carine dit à Béatrice  : « J'ai $43$ €, soit $10$ € de plus que toi.  » \\
      Combien d'argent en euros possèdent en tout les deux filles ?
    \item Nawel dit à Magalie  : « J'ai $50$ €, soit $29$ € de plus que toi.  » \\
      Combien d'argent en euros possèdent en tout les deux filles ?
    \item Nawel dit à Farida : « J'ai $41$ €, soit $16$ € de moins que toi. » \\
     Combien d'argent en euros possèdent en tout les deux filles ?
    \end{enu}
  \end{exercice}
\end{multicols}

\end{document}
