%% Font size %%
\documentclass[11pt]{article}

%% Load the custom package
\usepackage{Mathdoc}

%% Numéro de séquence %% Titre de la séquence %%
\renewcommand{\centerhead}{Statistiques à deux variables}

%% Spacing commands %%
\renewcommand{\baselinestretch}{1} \setlength{\parindent}{0pt}


\begin{document}

\begin{center}
\entetedevoirs{20} \duree{55 min} \calculatrice{1} \copieseparee{0}
\soin
\end{center}

\begin{exercicedevoir}[20]
Sur le graphique ci-dessous est représenté l'évolution annuelle des ventes en ligne et des ventes en magasin d'une enseigne de prêt-à-porter. Ces données permettent d'étudier l'impact de la digitalisation sur le comportement d'achat des consommateurs.
\end{exercicedevoir}

\begin{center}
\begin{tikzpicture}[scale=1.2]
% Points avec croix
\foreach \x/\y in {2.5/1, 4/2, 5/2.5, 4.5/3, 5.5/3} 
    \draw (\x,\y) node[draw, cross out, minimum size=5pt] {};


% Tracer les axes en gris
\draw[gray!80, thin, step=0.5] (0, 0) grid (7, 7);  

% Repère
\draw[->] (0,0) -- (7,0) node[right] {En ligne (M€)};
\draw[->] (0,0) -- (0, 7) node[above] {En magasin (M€)};

% Graduations axe x (en ligne)
\foreach \x/\text in {1/5, 2/10, 3/15, 4/20, 5/25, 6/30}
    \draw (\x,0.1) -- (\x,-0.1) node[below] {\text};

% Graduations axe y (magasin)
\foreach \y/\text in {1/5, 2/10, 3/15, 4/20, 5/25, 6/30}
    \draw (0.1,\y) -- (-0.1,\y) node[left] {\text};


\end{tikzpicture}
\end{center}


\begin{enumerate}
\item Compléter les 5 premières colonnes du tableau ci-dessous ; \bareme{5}
\renewcommand{\arraystretch}{1.5}
\begin{center}
\begin{tabular}{|c|c|c|c|c|c|c|}
\hline
\phantom{0000} & \phantom{0000} & \phantom{0000} &\phantom{0000}  & \phantom{0000} & \phantom{0000} & Moyenne  \\
\hline
$x_i$ : Ventes en ligne (M€) &  &  &  &  &  &  \\
\hline
$y_i$ : Ventes en magasin (M€)  &  &  &  &  &  &  \\
\hline
$x_iy_i$  &  &  &  &  &  &  \\
\hline
$x_i^2$  &  &  &  &  &  &  \\
\hline
$y_i^2$  &  &  &  &  &  &  \\
\hline
\end{tabular}
\end{center}
\item Calculer $\overline{x_i}$ et l'ajouter dans le tableau ; \bareme{1}
\\ \dtf \\ \dtf \\ \dtf
\item Calculer $\overline{y_i}$ et l'ajouter dans le tableau ; \bareme{1}
\\ \dtf \\ \dtf \\ \dtf
\item Déterminer les coordonnées du point moyen
$M(\overline{x_i};\overline{y_i})$ et le placer sur le graphique ; \bareme{1}
\\ \dtf \\ \dtf
\item Calculer $\overline{x_iy_i}$ et l'ajouter dans le tableau ; \bareme{1}
\\ \dtf \\ \dtf \\ \dtf
\item Calculer $\overline{x_i^2}$ et l'ajouter dans le tableau ; \bareme{1}
\\ \dtf \\ \dtf \\ \dtf
\item Calculer $\overline{y_i^2}$ et l'ajouter dans le tableau ; \bareme{1}
\\ \dtf \\ \dtf \\ \dtf
\item Calculer la valeur de  $Cov(x_i,y_i)$  ; \bareme{2}
\\ \dtf \\ \dtf \\ \dtf
\item Calculer la valeur de  $Var(x_i)$  ; \bareme{2}
\\ \dtf \\ \dtf \\ \dtf
\item Calculer la valeur de  $Var(y_i)$  ; \bareme{2}
\\ \dtf \\ \dtf \\ \dtf
\item Déterminer   $(d_{x_i}) : ax_i+b=y_i$ la droite de regression
linéaire des $x_i$ en $y_i$ ; \bareme{2}
\\ \dtf \\ \dtf \\ \dtf
\item Déterminer   $(d_{y_i}) : a'x_i+b'=y_i$ la droite de regression
linéaire des $y_i$ en $x_i$. \bareme{2}
\\ \dtf \\ \dtf \\ \dtf

\end{enumerate}


\end{document}

% Local Variables:
% gptel-model: deepseek-chat
% gptel--backend-name: "DeepSeek"
% gptel--bounds: ((response (714 896) (897 1413)))
% End:
