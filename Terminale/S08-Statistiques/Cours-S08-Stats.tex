%% Font size %%
\documentclass[11pt]{article}

%% Load the custom package
\usepackage{Mathdoc}

%% Numéro de séquence %% Titre de la séquence %%
\renewcommand{\centerhead}{}

%% Spacing commands %%
\renewcommand{\baselinestretch}{1} \setlength{\parindent}{0pt}

\begin{document}

\section{Définition, exemple}

\begin{definition}
On appelle série statistique à deux variables (ou série statistique doubles) une série statistique à deux caractères étudiés simultanément.
\end{definition}

\begin{exemple}
On a relevé, pour un modèle de voiture, la consommation en carburant (en L/100 km) pour différentes vitesses (en km/h) sur le cinquième rapport :

\begin{center}
\begin{tabular}{|c|c|c|c|c|c|c|}
\hline
Vitesse $x_i$ (en km/h) & 60 & 70 & 90 & 110 & 130 & 150 \\
\hline
Consommation $y_i$ (en L/100 km) & 3 & 3.1 & 3.7 & 4.7 & 6 & 9 \\
\hline
\end{tabular}
\end{center}
\end{exemple}

\begin{definition}
Dans un repère orthogonal, l’ensemble des points $M_i$ de coordonnées $(x_i, y_i)$ s’appelle le nuage de points.
\end{definition}

\begin{exemple}
Les points de l'exemple précédent forment le nuage de points suivant
représenté dans le repère orthogonal ci-dessous.

\begin{center}
\begin{tikzpicture}
    % Axes
    \draw[->] (0,0) -- (8,0) node[right]{Vitesse (km/h)};
    \draw[->] (0,0) -- (0,6) node[above]{Consommation (L/100 km)};

    % Graduations de l'axe des x
    \foreach \x in {1,2,3,4,5,6,7} {
        \draw (\x,0.1) -- (\x,-0.1) node[below] {\pgfmathparse{50+20*\x}\pgfmathprintnumber{\pgfmathresult}};
    }

    % Graduations de l'axe des y
    \foreach \y in {1,2,3,4,5} {
        \draw (0.1,\y) -- (-0.1,\y) node[left] {\pgfmathparse{1.5*\y}\pgfmathprintnumber{\pgfmathresult}};
    }

    % Points représentés par des croix (x)
    \foreach \x/\y in {60/3, 70/3.1, 90/3.7, 110/4.7, 130/6, 150/9} {
        \draw[thick] (\x/20-0.1, \y/1.5-0.1) -- (\x/20+0.1, \y/1.5+0.1);
        \draw[thick] (\x/20-0.1, \y/1.5+0.1) -- (\x/20+0.1, \y/1.5-0.1);
    }
\end{tikzpicture}
\end{center}
\end{exemple}

\section{Point moyen}

\begin{definition}
Le point moyen d’un nuage de points $G$ de coordonnées $(\overline{x}, \overline{y})$ où :
\begin{itemize}
    \item $\overline{x}$ représente la moyenne des $x_i$ :
    \[
    \overline{x} = \frac{x_1 + x_2 + \cdots + x_n}{n} = \frac{1}{n} \sum_{i=1}^{n} x_i
    \]
    \item $\overline{y}$ représente la moyenne des $y_i$ :
    \[
    \overline{y} = \frac{y_1 + y_2 + \cdots + y_n}{n} = \frac{1}{n} \sum_{i=1}^{n} y_i
    \]
\end{itemize}
\end{definition}

\begin{exemple}
Le point moyen de l'exemple 1 est :
\[
\overline{x} = \frac{60 + 70 + 90 + 110 + 130 + 150}{6} = 101.66
\]
\[
\overline{y} = \frac{3 + 3.1 + 3.7 + 4.7 + 6 + 9}{6} = 4.75
\]
\end{exemple}
\section{Ajustement affine par la méthode des moindres carrés}

\begin{definition}
On appelle covariance de $x$ et de $y$ le nombre :
\[
\text{cov}(x, y) = \frac{1}{N} \sum_{i=1}^{n} (x_i - \overline{x})(y_i - \overline{y}) = \left( \frac{1}{N} \sum_{i=1}^{n} x_i y_i \right) - \overline{x} \, \overline{y}
\]
\end{definition}

\begin{definition}
La variance du caractère $x$ est :
\[
V(x) = \frac{1}{N} \sum_{i=1}^{n} (x_i - \overline{x})^2 = \left( \frac{1}{N} \sum_{i=1}^{n} x_i^2 \right) - \overline{x}^2 = \text{cov}(x, x)
\]
\end{definition}

\begin{definition}
La variance du caractère $y$ est :
\[
V(y) = \frac{1}{N} \sum_{i=1}^{n} (y_i - \overline{y})^2 = \left( \frac{1}{N} \sum_{i=1}^{n} y_i^2 \right) - \overline{y}^2 = \text{cov}(y, y)
\]
\end{definition}

\begin{definition}
L'écart type de $x$ est :
\[
\sigma(x) = \sqrt{V(x)}
\]
L'écart type de $y$ est :
\[
\sigma(y) = \sqrt{V(y)}
\]
\end{definition}

\begin{exemple}
Calculons dans l'exemple 1 : $\text{cov}(x, y)$, $\text{cov}(x, x)$, $\text{cov}(y, y)$, $\sigma(x)$, $\sigma(y)$.

On a :
\[
\overline{x} = 101.66, \quad \overline{y} = 4.91
\]

\begin{center}
\begin{tabular}{|c|c|c|c|c|c|}
\hline
$x_i$ & $y_i$ & $x_i y_i$ & $x_i^2$ & $y_i^2$ \\
\hline
60 & 3 & 180 & 3600 & 9 \\
70 & 3.1 & 217 & 4900 & 9.61 \\
90 & 3.7 & 333 & 8100 & 13.69 \\
110 & 4.7 & 517 & 12100 & 22.09 \\
130 & 6 & 780 & 16900 & 36 \\
150 & 9 & 1350 & 22500 & 81 \\
\hline
$\sum$ & & 3377 & 68100 & 171.39 \\
\hline
\end{tabular}
\end{center}

\[
\text{cov}(x, y) = \frac{1}{N} \sum_{i=1}^{n} x_i y_i - \overline{x} \, \overline{y} = \frac{3377}{6} - 101.66 \times 4.91 = 63.68
\]

\[
V(x) = \frac{1}{N} \sum_{i=1}^{n} x_i^2 - \overline{x}^2 = \frac{68100}{6} - (101.66)^2 = 1015.2444
\]

\[
V(y) = \frac{1}{N} \sum_{i=1}^{n} y_i^2 - \overline{y}^2 = \frac{171.39}{6} - (4.91)^2 = 4.4569
\]

\[
\sigma(x) = \sqrt{V(x)} = \sqrt{1015.2444} = 31.86
\]

\[
\sigma(y) = \sqrt{V(y)} = \sqrt{4.4569} = 2.11
\]
\end{exemple}

\section{Théorème : Ajustement affine par la méthode des moindres carrés}

\begin{theoreme}
Lors d’un ajustement affine par la méthode des moindres carrés :
\begin{enumerate}
    \item La droite de régression \( D \) de \( Y \) en \( X \) a pour équation :
    \[
    D(Y/X) : Y = aX + b
    \]
    où :
    \[
    a = \frac{\text{cov}(x, y)}{V(x)}, \quad b = \overline{Y} - a\overline{X}
    \]
    Cette droite passe par le point moyen du nuage \( G(\overline{X}, \overline{Y}) \).

    \item La droite de régression \( D \) de \( X \) en \( Y \) a pour équation :
    \[
    D(X/Y) : X = a'Y + b'
    \]
    où :
    \[
    a' = \frac{\text{cov}(x, y)}{V(y)}, \quad b' = \overline{X} - a'\overline{Y}
    \]
    Cette droite passe également par le point moyen du nuage \( G(\overline{X}, \overline{Y}) \).
\end{enumerate}
\end{theoreme}

\begin{exemple}
Calculons dans l'exemple 1 les droites de régression :
\begin{itemize}
    \item Droite de régression \( D(Y/X) \) :
    \[
    a = \frac{\text{cov}(x, y)}{V(x)} = \frac{63.68}{1015.2444} = 0.0627
    \]
    \[
    b = \overline{Y} - a\overline{X} = 4.91 - 0.0627 \times 101.66 = -1.46
    \]
    Donc :
    \[
    D(Y/X) : Y = 0.0627X - 1.46
    \]

    \item Droite de régression \( D(X/Y) \) :
    \[
    a' = \frac{\text{cov}(x, y)}{V(y)} = \frac{63.68}{4.4569} = 14.287
    \]
    \[
    b' = \overline{X} - a'\overline{Y} = 101.66 - 14.287 \times 4.91 = 31.51
    \]
    Donc :
    \[
    D(X/Y) : X = 14.287Y + 31.51
    \]
\end{itemize}
\end{exemple}

\section{Coefficient de corrélation linéaire}

\begin{definition}
Le coefficient de corrélation linéaire d’une série statistique à deux variables \( x \) et \( y \) est le nombre \( r \) défini par :
\[
r = \frac{\text{cov}(x, y)}{\sqrt{V(x)} \sqrt{V(y)}} = \frac{\text{cov}(x, y)}{\sigma(x) \sigma(y)}
\]
\end{definition}

\begin{remarque}
\begin{itemize}
    \item \(-1 \leq r \leq 1\).
    \item Si \( r = 1 \) ou \( r = -1 \), il y a une corrélation positive ou négative parfaite entre \( X \) et \( Y \), et les points \( (x_i, y_i) \) sont tous sur la droite de régression.
    \item Si \( r = 0 \), il n’y a pas de corrélation entre \( X \) et \( Y \), et les points \( (x_i, y_i) \) sont dispersés au hasard.
    \item Si \( 0 < r < 1 \), il y a une corrélation positive faible, moyenne ou forte entre \( X \) et \( Y \).
    \item Si \( -1 < r < 0 \), il y a une corrélation négative faible, moyenne ou forte entre \( X \) et \( Y \).
\end{itemize}
\end{remarque}

\begin{exemple}
Calculons dans l'exemple 1 le coefficient de corrélation linéaire :
\[
\text{cov}(x, y) = 63.68, \quad \sigma(x) = 31.86, \quad \sigma(y) = 2.11
\]
Donc :
\[
r = \frac{\text{cov}(x, y)}{\sigma(x) \sigma(y)} = \frac{63.68}{31.86 \times 2.11} = 0.947
\]
Il y a une corrélation positive forte entre \( X \) et \( Y \).
\end{exemple}

\section{Exercices d'application}

\begin{exercice}
Dans la série statistique suivante, \( X \) représente le nombre de
jours d’exposition au soleil d’une feuille et \( Y \) le nombre de
stomates aérifères au millimètre carré :

\begin{center}
\begin{tabular}{|c|c|c|c|c|c|c|c|}
\hline
\( X \) & 2 & 4 & 8 & 10 & 24 & 40 & 52 \\
\hline
\( Y \) & 6 & 11 & 15 & 20 & 39 & 62 & 85 \\
\hline
\end{tabular}
\end{center}

1. Tracer le nuage des points.

2. Calculer le coefficient de corrélation linéaire entre \( X \) et \( Y \). Conclusion ?

3. Déterminer l’équation de la droite de régression de \( Y \) en fonction de \( X \).

4. Si on expose au soleil une feuille 15 jours, quel est le nombre de stomates aérifères peut-on prévoir ?
\end{exercice}

\begin{exercice}
Dans la série statistique suivante, \( X \) représente le nombre
d’heures d’étude par semaine et \( Y \) la note obtenue à un examen
(sur 20) :

\begin{center}
\begin{tabular}{|c|c|c|c|c|c|c|c|}
\hline
\( X \) & 5 & 7 & 10 & 12 & 15 & 18 & 20 \\
\hline
\( Y \) & 8 & 10 & 12 & 14 & 16 & 18 & 20 \\
\hline
\end{tabular}
\end{center}

1. Tracer le nuage des points.

2. Calculer le coefficient de corrélation linéaire entre \( X \) et \( Y \). Conclusion ?

3. Déterminer l’équation de la droite de régression de \( Y \) en fonction de \( X \).

4. Si un étudiant étudie 14 heures par semaine, quelle note peut-on prévoir ?
\end{exercice}


\end{document}

% Local Variables:
% gptel-model: deepseek-chat
% gptel--backend-name: "DeepSeek"
% gptel--bounds: ((280 . 281) (310 . 652) (666 . 861) (873 . 1172) (1174 . 2022) (2047 . 2727) (2728 . 3645) (3647 . 4689) (4691 . 7527) (7529 . 7564) (7580 . 8263) (8277 . 8279) (8295 . 8917))
% End:
