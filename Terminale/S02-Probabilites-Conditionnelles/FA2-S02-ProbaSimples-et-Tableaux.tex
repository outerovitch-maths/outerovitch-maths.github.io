%% Font size %%
\documentclass[11pt]{article}

%% Load the custom package
\usepackage{Mathdoc}

%% Numéro de séquence %% Titre de la séquence %%
\renewcommand{\centerhead}{Chaitre 2 : Probabilités conditionnelles}

%% Spacing commands %%
\renewcommand{\baselinestretch}{1}
\setlength{\parindent}{0pt}

\begin{document}

\section{Probabilités Simples}

\begin{exercice}[1][Calculs de probabilités simples.]
\begin{enumerate}[itemsep=1em]
	\item Dans un tiroir de la commode il y a 22 t-shirts. 2 sont rouges, 7 sont verts, 6 sont bleus, 3 sont noirs et 4 sont blancs.\\ Magalie choisit au hasard l'un d'entre eux.\\ \textbf {a.}  Quelle est la probabilité que son choix tombe sur l'un des t-shirts verts ?\\\textbf {b.}  Quelle est la probabilité que son choix tombe sur l'un des t-shirts blancs ?\\\textbf {c.}  Quelle est la probabilité que son choix ne tombe pas sur l'un des t-shirts noirs ?\\\textbf {d.}  Quelle est la probabilité que son choix tombe sur l'un des t-shirts verts ou blancs ?
	\item Dans une urne il y a 19 jetons. 4 sont oranges, 4 sont cyans, 2 sont roses, 4 sont jaunes et 5 sont violets.\\ Yazid choisit au hasard l'un d'entre eux.\\ \textbf {a.}  Quelle est la probabilité que son choix tombe sur l'un des jetons cyans ?\\\textbf {b.}  Quelle est la probabilité que son choix tombe sur l'un des jetons jaunes ?\\\textbf {c.}  Quelle est la probabilité que son choix ne tombe pas sur l'un des jetons violets ?\\\textbf {d.}  Quelle est la probabilité que son choix tombe sur l'un des jetons cyans ou jaunes ?
	\item Dans un tas de jetons de poker il y a 21 jetons. 4 sont rouges, 4 sont verts, 4 sont bleus, 6 sont noirs et 3 sont jaunes.\\ Karim choisit au hasard l'un d'entre eux.\\ \textbf {a.}  Quelle est la probabilité que son choix tombe sur l'un des jetons rouges ?\\\textbf {b.}  Quelle est la probabilité que son choix tombe sur l'un des jetons bleus ?\\\textbf {c.}  Quelle est la probabilité que son choix ne tombe pas sur l'un des jetons verts ?\\\textbf {d.}  Quelle est la probabilité que son choix tombe sur l'un des jetons rouges ou bleus ?
\end{enumerate}
\end{exercice}

\section{Probabilités conditionnelles simples}

\begin{exercice}[1][Calculs simples de Probabilités conditionnelles]
  \begin{multicols}{2}
    Ce tableau est un tableau de probabilités avec deux événements $A$
    et
    $B$  d’une expérience aléatoire.\\
    \textbf{Déterminer $P_B(A)$.}
    \columnnbreak

    \phantom{00000000000} $\renewcommand{\arraystretch}{1}
    \begin{array}{|c|c|c|c|}
      \hline
      \cellcolor{lightgray}  & \cellcolor{lightgray} A & \cellcolor{lightgray} \overline{A} & \cellcolor{lightgray} \text{Total}\\
      \hline
      \cellcolor{lightgray} B & 0{,}25 & 0{,}35 & 0{,}6\\
      \hline
      \cellcolor{lightgray} \overline{B} & 0{,}11 & 0{,}29 & 0{,}4\\
      \hline
      \cellcolor{lightgray} \text{Total} & 0{,}36 & 0{,}64 & 1\\
      \hline
    \end{array}
    \renewcommand{\arraystretch}{1}$
  \end{multicols}
\end{exercice}

\begin{exercice}[1][Calculs simples de Probabilités conditionnelles]
  \begin{multicols}{2}
    Ce tableau est un tableau de probabilités avec deux événements $A$
    et
    $B$  d’une expérience aléatoire.\\
     \textbf{Déterminer $P_{\overline{B}}(\overline{A})$.}
     \columnnbreak

     \phantom{00000000000}
    $\renewcommand{\arraystretch}{1}
    \begin{array}{|c|c|c|c|}
      \hline
      \cellcolor{lightgray}  & \cellcolor{lightgray} A & \cellcolor{lightgray} \overline{A} & \cellcolor{lightgray} \text{Total}\\
      \hline
      \cellcolor{lightgray} B & 0{,}19 & 0{,}23 & 0{,}42\\
      \hline
      \cellcolor{lightgray} \overline{B} & 0{,}13 & 0{,}45 & 0{,}58\\
      \hline
      \cellcolor{lightgray} \text{Total} & 0{,}32 & 0{,}68 & 1\\
      \hline
    \end{array}
    \renewcommand{\arraystretch}{1}$
  \end{multicols}
\end{exercice}

\section{Probabilités conditionnelles dans un tableau}

\begin{exercice}[2][Probabilités conditionnelles avec tableau]
 Le personnel d’une entreprise est constitué de $140$ personnes qui se
 répartissent de la manière suivante :
 \begin{center}
   $\renewcommand{\arraystretch}{1}
   \begin{array}{|c|c|c|c|}
     \hline
     \cellcolor{lightgray}  & \cellcolor{lightgray} \text{Femmes} & \cellcolor{lightgray} \text{Hommes} & \cellcolor{lightgray} \text{Total}\\
     \hline
     \cellcolor{lightgray} \text{Cadres} & 12 & 37 & 49\\
     \hline
     \cellcolor{lightgray} \text{Employés} & 36 & 55 & 91\\
     \hline
     \cellcolor{lightgray} \text{Total} & 48 & 92 & 140\\
     \hline
   \end{array}
   \renewcommand{\arraystretch}{1}$
 \end{center}
               Au cours de la fête de fin d’année, le comité
               d’entreprise offre un séjour à la montagne à une personne choisie au hasard parmi les $140$ personnes de cette
               entreprise.\\
               On définit les évènements suivants : \\
               C : \og{} la personne choisie fait partie des cadres
               \fg{} ; \\
               F : « la personne choisie est une femme ».

\medskip
\textbf {a.}   Calculer la probabilité de l'événement : « la personne choisie est une femme qui fait partie des employés ».

\medskip
\textbf {b.}   Calculer la probabilité de l'événement $\overline{F}\cup C$.

\medskip
\textbf {c.}   On sait que la personne choisie  est une femme.\\
          Quelle est la probabilité qu'elle soit employée ?
\end{exercice}

\begin{exercice}[2][Probabilités conditionnelles dans une entreprise]
  Une entreprise réalise des ventes à deux types de clients : des \textbf{grandes entreprises} et des \textbf{PME}. À la fin de l'année, elle constate que certains clients sont en situation de paiement à jour, tandis que d'autres sont en retard de paiement. Le tableau ci-dessous présente la répartition des clients selon leur type et leur situation de paiement.

\[
\begin{array}{|c|c|c|c|}
\hline
                          & \textbf{Paiement à jour} & \textbf{Paiement en retard} & \textbf{Total} \\
\hline
\textbf{Grandes entreprises}  & 600 & 150 & 750 \\ \hline
\textbf{PME}                  & 850 & 400 & 1250 \\
\hline
\textbf{Total}                & 1450 & 550 & 2000 \\
\hline
\end{array}
\]

\begin{multicols}{2}
  \begin{enumerate}
  \item Quelle est la probabilité qu'un client choisi au hasard soit
    en situation de paiement à jour ?
  \item Quelle est la probabilité qu'un client choisi au hasard soit
    en situation de paiement en retard ?
  \item Quelle est la probabilité qu'un client choisi soit une grande
    entreprise ?
  \item Quelle est la probabilité qu'un client soit en situation de
    paiement à jour sachant que c'est une grande entreprise ?
\item Quelle est la probabilité qu'un client soit en situation de
  paiement en retard sachant que c'est une PME ?
\item Quelle est la probabilité qu'un client soit une grande
  entreprise sachant qu'il est en situation de paiement en retard ?
\item Quelle est la probabilité qu'un client soit une grande
  entreprise sachant qu'il est en situation de paiement à jour ?
\item Quelle est la probabilité qu'un client soit une PME sachant
  qu'il est en situation de paiement à jour ?
\end{enumerate}
\end{multicols}
\end{exercice}



\end{document}

%%% Local Variables:
%%% mode: LaTeX
%%% TeX-master: t
%%% End:
