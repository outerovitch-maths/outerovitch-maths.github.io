%% Font size %%
\documentclass[11pt]{article}

%% Load the custom package
\usepackage{Mathdoc}

%% Numéro de séquence %% Titre de la séquence %%
\renewcommand{\centerhead}{Chap. 4 : Fonctions polynôme de degré 2}

%% Spacing commands %%
\renewcommand{\baselinestretch}{1}
\setlength{\parindent}{0pt}

\begin{document}

\section{Fonctions affines}

\begin{exercice}[0][Exercice corrig�.]
\textbf{Soit $f: x \longmapsto 6x+4$. \\ Quelle est l'image de $1$ ?}
\\
Correction :\\
\textit{On remplace $x$ par $1$ dans l'expression de $f(x)$ : \\
$f(1)=6 \times 1 + 4$ \\
$f(1)=10$}
\end{exercice}

\begin{exercice}[1][Calculs d'images.]
\begin{multicols}{2}
\begin{enumerate}[itemsep=1.5em]
	\item \begin{minipage}[t]{\linewidth} Soit $f: x \longmapsto
            6x+4$. \\ Quelle est l'image de $2$ ?\\ \dtf \end{minipage}
	\item \begin{minipage}[t]{\linewidth} Soit $f: x \longmapsto 5x+2$. \\ Quelle est l'image de $2$ ?\\ \dtf \end{minipage}
	\item \begin{minipage}[t]{\linewidth} Soit $f: x \longmapsto 5x+2$. \\ Quelle est l'image de $4$ ?\\ \dtf \end{minipage}
	\item \begin{minipage}[t]{\linewidth} Soit $f: x \longmapsto 6x+4$. \\ Quelle est l'image de $6$ ?\\ \dtf \end{minipage}
	\item \begin{minipage}[t]{\linewidth} Soit $f: x \longmapsto 3x+2$. \\ Quelle est l'image de $5$ ?\\ \dtf \end{minipage}
	\item \begin{minipage}[t]{\linewidth} Soit $f: x \longmapsto 5x+2$. \\ Quelle est l'image de $-9$ ?\\ \dtf \end{minipage}
	\item \begin{minipage}[t]{\linewidth} Soit $f: x \longmapsto 6x+5$. \\ Quelle est l'image de $-9$ ?\\ \dtf \end{minipage}
	\item \begin{minipage}[t]{\linewidth} Soit $f: x \longmapsto 2x+5$. \\ Quelle est l'image de $-7$ ?\\ \dtf \end{minipage}
	\item \begin{minipage}[t]{\linewidth} Soit $f: x \longmapsto 5x+4$. \\ Quelle est l'image de $7$ ?\\ \dtf \end{minipage}
	\item \begin{minipage}[t]{\linewidth} Soit $f: x \longmapsto 3x+2$. \\ Quelle est l'image de $8$ ?\\ \dtf \end{minipage}
\end{enumerate}
\end{multicols}

%%% Local Variables:
%%% mode: LaTeX
%%% TeX-master: "../11.FA-01-S04-Evaluation-fonction"
%%% End:

\end{exercice}

\begin{exercice}[2][Calculs d'images.]
  \begin{enumerate}
    \begin{multicols}{2}
    \item \item Soit $f$ la fonction définie par $f(x)=\dfrac{5}{2}x-6$. \\
      Quelle est l'image de $14$ ?\\ \dtf
    \item Soit $f$ la fonction définie par $f(x)=\dfrac{5}{4}x-9$. \\
      Quelle est l'image de $28$ ?\\ \dtf
    \item Soit $f$ la fonction définie par $f(x)=\dfrac{3}{5}x+3$. \\
      Quelle est l'image de $20$ ?\\ \dtf
    \item Soit $f$ la fonction définie par $f(x)=\dfrac{2}{5}x+1$. \\
      Quelle est l'image de $10$ ?\\ \dtf
    \end{multicols}
  \end{enumerate}
\end{exercice}
\newpage
\section{Polynômes du second degré}

\begin{exercice}[0][Exercice corrigé.]
\textbf{Soit $f(x)=6x^2+4x-3$. \\ Quelle est l'image de $1$ ?}
\\
Correction :\\
\textit{On remplace $x$ par $1$ dans l'expression de $f(x)$ : \\
$f(1)=6 \times 1^2 + 4 \times 1 - 3$ \\
$f(1)=6+4-3$ \\
$f(1)=7$ }
\end{exercice}

\begin{exercice}[1][Calculs d'images.]
\begin{multicols}{2}
\begin{enumerate}[itemsep=1.5em]
	\item \begin{minipage}[t]{\linewidth} Soit $f$ la fonction qui
            à $x$ associe $6x^2+4x$. \\ Quelle est l'image de $2$ ?\\
            \dtf \end{minipage}
	\item \begin{minipage}[t]{\linewidth} Soit $f: x \longmapsto 5x^2+2x-7$. \\ Quelle est l'image de $7$ ?\\ \dtf \end{minipage}
	\item \begin{minipage}[t]{\linewidth} Soit $f: x \longmapsto x^2+6x+5$. \\ Quelle est l'image de $6$ ?\\ \dtf \end{minipage}
	\item \begin{minipage}[t]{\linewidth} Soit $f(x)=x^2-2x+5$. \\ Quelle est l'image de $-3$ ?\\ \dtf \end{minipage}
	\item \begin{minipage}[t]{\linewidth} Soit $f: x \longmapsto x^2+5x+2$. \\ Quelle est l'image de $-9$ ?\\ \dtf \end{minipage}
	\item \begin{minipage}[t]{\linewidth} Soit $f: x \longmapsto x^2+4x+5$. \\ Quelle est l'image de $-2$ ?\\ \dtf \end{minipage}
	\item \begin{minipage}[t]{\linewidth} Soit $f$ la fonction qui à $x$ associe $3x^2+2x$. \\ Quelle est l'image de $5$ ?\\ \dtf \end{minipage}
	\item \begin{minipage}[t]{\linewidth} Soit $f(x)=x^2-3x+2$. \\ Quelle est l'image de $4$ ?\\ \dtf \end{minipage}
	\item \begin{minipage}[t]{\linewidth} Soit $f$ la fonction qui à $x$ associe $3x^2+4x$. \\ Quelle est l'image de $-2$ ?\\ \dtf \end{minipage}
	\item \begin{minipage}[t]{\linewidth} Soit $f(x)=3x^2-5x-2$. \\ Quelle est l'image de $-7$ ?\\ \dtf \end{minipage}
\end{enumerate}
\end{multicols}



%%% Local Variables:
%%% mode: LaTeX
%%% TeX-master: "../11.FA-01-S04-Evaluation-fonction"
%%% End:

\end{exercice}

\begin{exercice}[2][Calculs d'images.]
  \begin{multicols}{2}
    \begin{enumerate}[itemsep=1.5em]
    \item Soit $f$ telle que $f(x)=\dfrac{-2x^2+2x}{x^2-2x}$. \\
      Quelle est l'image de $4$ ?\\ \dtf
    \item Soit $f$ telle que $f(x)=\dfrac{13x^2+4x}{x^2+13x}$. \\
      Quelle est l'image de $-9$ ?\\ \dtf
    \item Soit $f$ la fonction qui à $x$ associe $\dfrac{x}{x-3}$. \\
      Quelle est l'image de $5$ ?\\ \dtf
    \item Soit $f: x \longmapsto \dfrac{x-9}{x^2-18x+81}$. \\
      Quelle est l'image de $-5$ ?\\ \dtf
    \end{enumerate}
  \end{multicols}
\end{exercice}

\section{Reconnaitre un polynôme du second degré}
\begin{exercice}[1][Reconnaitre un polynôme.]
\begin{multicols}{2}
\begin{enumerate}[itemsep=1em]
	\item \begin{minipage}[t]{\linewidth} Soit $r$ la fonction définie  par :\\
              $r(x)=3x^3-3x^2-5x$. \\
              $r$ est une fonction polynôme du second degré.	$\square\;$ Vrai\qquad $\square\;$ Faux\qquad  \end{minipage}
	\item \begin{minipage}[t]{\linewidth} Soit $w$ la fonction définie  par :\\
            $w(x)=-2x^2+4$. \\
            $w$ est une fonction polynôme du second degré.	$\square\;$ Vrai\qquad $\square\;$ Faux\qquad  \end{minipage}
	\item \begin{minipage}[t]{\linewidth} Soit $v$ la fonction définie  par :\\
              $v(x)=-7-9x^3$. \\
              $v$ est une fonction polynôme du second degré.	$\square\;$ Vrai\qquad $\square\;$ Faux\qquad  \end{minipage}
	\item \begin{minipage}[t]{\linewidth} Soit $h$ la fonction définie  par :\\
              $h(x)=-4+5x^3$. \\
              $h$ est une fonction polynôme du second degré.	$\square\;$ Vrai\qquad $\square\;$ Faux\qquad  \end{minipage}
	\item \begin{minipage}[t]{\linewidth} Soit $u$ la fonction définie  par :\\
                  $u(x)=4(x-6)^2$. \\
                  $u$ est une fonction polynôme du second degré.	$\square\;$ Vrai\qquad $\square\;$ Faux\qquad  \end{minipage}
	\item \begin{minipage}[t]{\linewidth} Soit $r$ la fonction définie  par :\\
            $r(x)=3x^2-2$. \\
            $r$ est une fonction polynôme du second degré.	$\square\;$ Vrai\qquad $\square\;$ Faux\qquad  \end{minipage}
	\item \begin{minipage}[t]{\linewidth} Soit $w$ la fonction définie  par :\\
            $w(x)=-6x-x^2-6$. \\
            $w$ est une fonction polynôme du second degré.	$\square\;$ Vrai\qquad $\square\;$ Faux\qquad  \end{minipage}
	\item \begin{minipage}[t]{\linewidth} Soit $u$ la fonction définie  par :\\
            $u(x)=5x+9-3x^2$. \\
            $u$ est une fonction polynôme du second degré.	$\square\;$ Vrai\qquad $\square\;$ Faux\qquad  \end{minipage}
	\item \begin{minipage}[t]{\linewidth} Soit $w$ la fonction définie  par :\\
                    $w(x)=-3x(x-3)(x+5)$. \\          
                    $w$ est une fonction polynôme du second degré.	$\square\;$ Vrai\qquad $\square\;$ Faux\qquad  \end{minipage}
	\item \begin{minipage}[t]{\linewidth} Soit $u$ la fonction définie  par :\\
                  $u(x)=2(x+7)^2-9$. \\         
                  $u$ est une fonction polynôme du second degré.	$\square\;$ Vrai\qquad $\square\;$ Faux\qquad  \end{minipage}
\end{enumerate}
\end{multicols}

\end{exercice}

\section{Identifier les coefficients}

\begin{exercice}[0][Exercice Corrigé.]
Identifier les valeurs de $a$, $b$ et
$c$.
\begin{enumerate}[itemsep=1em,label={}]
\item $A(x) = 2x^2+3x+4$. \\ \\
\textit{$A(x)$ est une fonction polynôme du second degré écrite sous forme
développée $ax^2+bx+c$, \\avec : \\}
$a = 2$ \\ 
$b = 3$ \\ 
$c = 4$   
\end{enumerate}
\end{exercice}


\begin{exercice}[1][Identifier les coefficients.]
Dans chacun des cas suivant, identifier les valeurs de $a$, $b$ et
$c$.
\begin{enumerate}[itemsep=1em,label={}]
  \begin{multicols}{2}
  \item $A(x) = x^2+5x$. \\ a = \dotfill \\ b = \dotfill \\ c =
    \dotfill
  \item $B(x) = 4x^2+3x$. \\ a = \dotfill \\ b = \dotfill \\ c =
    \dotfill
  \item $C(x) = x^2+3$. \\ a = \dotfill \\ b = \dotfill \\ c =
    \dotfill
  \item $D(x) = 3x^2+2x+5$. \\ a = \dotfill \\ b = \dotfill \\ c =
    \dotfill
  \item $E(x) = 3x^2+2x+2$. \\ a = \dotfill \\ b = \dotfill \\ c =
    \dotfill
  \item $F(x) = 2x+2-7x^2$. \\ a = \dotfill \\ b = \dotfill \\ c =
    \dotfill
  \end{multicols}
\end{enumerate}
\end{exercice}


\begin{exercice}[2][Identifier les coefficients.]
Dans chacun des cas suivant, identifier les valeurs de $a$, $b$ et
$c$.
\begin{enumerate}[itemsep=1em,label={}]
  \begin{multicols}{2}
        \item $A(x) = 4x^2-4x+2$. \\ a = \dotfill \\ b = \dotfill \\ c =
    \dotfill
	\item $B(x) = x^2-5x-4$. \\ a = \dotfill \\ b = \dotfill \\ c =
    \dotfill
	\item $C(x) = 2x-5x^2$. \\ a = \dotfill \\ b = \dotfill \\ c =
    \dotfill
	\item $D(x) = -x-1+5x^2$. \\ a = \dotfill \\ b = \dotfill \\ c =
    \dotfill
	\item $E(x) = -5-2x^2-3x$. \\ a = \dotfill \\ b = \dotfill \\ c =
    \dotfill
	\item $F(x) = -5x^2-2-x$. \\ a = \dotfill \\ b = \dotfill \\ c =
    \dotfill
  \end{multicols}
\end{enumerate}
\end{exercice}


\begin{exercice}[3][Identifier les coefficients.]
Dans chacun des cas suivant, identifier les valeurs de $a$, $b$ et
$c$.
\begin{enumerate}[itemsep=1em,label={}]
  \begin{multicols}{2}
        \item $A(x) = -4x-2x^2$. \\ a = \dotfill \\ b = \dotfill \\ c =
    \dotfill
	\item $B(x) = -153x^2-\frac{3}{4}$. \\ a = \dotfill \\ b = \dotfill \\ c =
    \dotfill
  \end{multicols}
\end{enumerate}
\end{exercice}

\newpage

\section{Coordonnées du sommet}

\begin{exercice}[0][Exercice Corrigé.]
 Soit $u$ la fonction définie sur $\mathbb{R}$ par :\\
 $u(x)=-x^2+4x+7$. \\
Quelles sont les coordonnées du sommet de la parabole représentant $u$ ? \\
Correction :\\
\textit{ $u$ est une fonction polynôme du second degré écrite sous
forme développée $ax^2+bx+c$.\\
Avec $a=-1$ ; $b=+4$ ; $c=+7$. \\
          Le sommet de la parabole a pour abscisse $\alpha =-\dfrac{b}{2a}$.\\
          $\alpha =-\dfrac{4}{2\times(-1) }= 2$ \\
          L'ordonnée du sommet est donnée par $f(\alpha)$, soit \\
          $f(2)=-1\times 2^2+4\times 2+7=11$.}
\end{exercice}


\begin{exercice}[3][Sur le cahiers, déterminer les coordonnées du sommet.]
\begin{enumerate}[itemsep=1em]
	\item Soit $u$ la fonction définie sur $\mathbb{R}$ par :\\          $u(x)=-x^2+4x+7$. \\          Quelles sont les coordonnées du sommet de la parabole représentant $u$ ?
	\item Soit $g$ la fonction définie sur $\mathbb{R}$ par :\\          $g(x)=3x^2+5$. \\          Quelles sont les coordonnées du sommet de la parabole représentant $g$ ?
	\item Soit $h$ la fonction définie sur $\mathbb{R}$ par :\\      $h(x)=-3x^2+4x+6$. \\
      Quelle est l'abscisse du sommet de la parabole représentant $h$ ?
	\item Soit $h$ la fonction définie sur $\mathbb{R}$ par :\\          $h(x)=-2x^2+8x+1$. \\          Quelles sont les coordonnées du sommet de la parabole représentant $h$ ?
	\item Soit $g$ la fonction définie sur $\mathbb{R}$ par :\\          $g(x)=3x^2+6x-2$. \\          Quelles sont les coordonnées du sommet de la parabole représentant $g$ ?
	\item Soit $v$ la fonction définie sur $\mathbb{R}$ par :\\          $v(x)=x^2-4x-3$. \\          Quelles sont les coordonnées du sommet de la parabole représentant $v$ ?
	\item Soit $h$ la fonction définie sur $\mathbb{R}$ par :\\          $h(x)=-3x^2-6x-7$. \\          Quelles sont les coordonnées du sommet de la parabole représentant $h$ ?
\end{enumerate}

%%% Local Variables:
%%% mode: LaTeX
%%% TeX-master: "../11.FA-01-S04-Evaluation-fonction"
%%% End:

\end{exercice}


\end{document}
