%% Font size %%
\documentclass[11pt]{article}

%% Load the custom package
\usepackage{Mathdoc}

%% Numéro de séquence %% Titre de la séquence %%
\renewcommand{\centerhead}{Calculs sur les puissances}

%% Spacing commands %%
\renewcommand{\baselinestretch}{1} \setlength{\parindent}{0pt}

\begin{document}

\begin{exercice}[0][Exercice corrigé]
Étudier les variations de la fonction $f(x) = 8 \times (3)^x$ \\
\textit{Correction. \\
$8>0$ et $3>1$ ; \\
Donc la fonction $x \mapsto 8$ est croissante et la fonction  $x \mapsto
(3)^x$ est croissante. \\
$f(x)$ est le produit de deux fonctions croissantes donc $f$ est croissante.}
\end{exercice}

\begin{exercice}
Étudier les variations des fonctions suivantes sur $\R$.
\begin{multicols}{2}
\begin{enumerate}
\item \( f(x) = -5 \times (2)^x \)
\item \( f(x) = 4 \times (0.7)^x \)
\item \( f(x) = -2 \times (1.5)^x \)
\item \( f(x) = 6 \times (0.4)^x \)
\end{enumerate}
\end{multicols}
\end{exercice}


\end{document}

% Local Variables:
% gptel-model: deepseek-chat
% gptel--backend-name: "DeepSeek"
% gptel--bounds: ((662 . 663) (688 . 714) (715 . 720) (758 . 759) (765 . 794) (800 . 829) (836 . 866) (873 . 903))
% End:
