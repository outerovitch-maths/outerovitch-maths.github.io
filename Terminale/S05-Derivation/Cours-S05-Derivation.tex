\documentclass[12pt]{article}

\usepackage{mathdoc}


%% Numéro de séquence %% Titre de la séquence %%
\renewcommand{\centerhead}{Chap. 5 : Dérivation}

\begin{document}

\setlength{\parindent}{0cm} 


\section{Taux d'accroissement} 

\begin{definition}
   Le \underline{taux d'accroissement} de la fonction $f$ entre $a$ et $x$ est le quotient :
$$\dfrac{f(x)-f(a)}{x-a}$$
Avec $x=a+h$, ce quotient s'écrit aussi : $$\dfrac{f(a+h)-f(a)}{h}$$
\end{definition}

\begin{exemple}
   Pour $f$ définie sur $\R$ par $f(x)=x^2$, le taux d'accroissement
   de $f$ entre $a$ et $a+h$ est : \\ \\
   \begin{tabular}{p{0.3cm}l!{=}l}
      &  $\dfrac{f(a+h)-f(a)}{h}$ & $\dfrac{(a+h)^2-a^2}{h}$ \\
      & & $\dfrac{a^2+2ah+h^2-a^2}{h}$ \\
      &   $\dfrac{f(a+h)-f(a)}{h}$ & $2a+h$ \\
   \end{tabular}
\end{exemple}

\section{Nombre dérivé} 

\begin{definition}
   On dit que $f$ est \underline{dérivable} en $a$ et on note cette dérivée $f'(a)$ si la limite suivante existe :
$$f'(a)=\lim_{h\to0}\frac{f(a+h)-f(a)}{h}$$
\end{definition}

\begin{remarque}
   On note aussi la dérivée $f'(a)$ comme la limite suivante :
$$f'(a)=\lim_{x\to a}\frac{f(x)-f(a)}{x-a}$$
\end{remarque}

\begin{propriete}[Interprétation géométrique]
  Si une fonction $f$ est dérivable en $a$, alors $f'(a)$ est la pente de la
  tangente à la courbe de $f$ en $(a, f(a))$.
\end{propriete}

\begin{exemple}
  Pour $f$ définie sur $\R$ par $f(x)=x^2$, calculer le nombre dérivé de $f$ en $3$ puis en $-1$ :
   \begin{enumerate}
      \item $\dfrac{f(a+h)-f(a)}{h}=2a+h$ d'après l'exemple $1$
      \item $f'(a)=\lim_{h\to0}2a+h=2a $
      \item $f'(2)=2\times 3=6$
      \item $f'(-1)=2\times (-1)=-2$
   \end{enumerate}
\end{exemple}

\begin{exemple}
On considère la fonction $f$ définie sur $\R$ par $f(x)=x^2+1$.\\
Son taux d'accroissement en $a=1$ est donné par le calcul suivant :
\begin{center}
  \begin{tabular}{p{0.3cm}l!{=}l}
    &  $\dfrac{f(x)-f(a)}{x-a}$ & $\dfrac{(x^2+1)-(1^2+1)}{x-1}$ \\
    & & $\dfrac{(x+1)(x-1)}{x-1}$ \\
    &   $\dfrac{f(x)-f(a)}{x-a}$ & $x+1$ \\
  \end{tabular}
\end{center}
Or, $\lim\limits_{x\to1}x+1=2$\\
Donc $f$ est dérivable en $1$ et $f'(1)=2$
\end{exemple}

\section{Equation de la tangente}

\begin{propriete}
   Soit $f$ une fonction numérique définie sur un intervalle $I$ et dérivable en $a\in I$\\
La \underline{tangente} $T_a$ en à la courbe $C_f$ en $a$ a pour équation :
$$T_a : y=f'(a)(x-a)+f(a)$$
\end{propriete}

\begin{demonstration}
\begin{center}
  \begin{tabular}{p{0.3cm}l!{=}l}
    &  $f'(x)$ & $\dfrac{f(x)-f(a)}{x-a}$ \\
    & $f'(x)+\dfrac{f(a)}{x-a}$& $\dfrac{f(x)}{x-a}$ \\
    &   $f(x)$ & $f'(x)(x-a)+\dfrac{f(a)(x-a)}{x-a}$ \\
    &   $f(x)$ & $f'(x)(x-a)+f(a)$ \\
  \end{tabular}
\end{center}
\end{demonstration}

\begin{exemple}
   Soit $f(x)=x^2+2$. Déterminer l'équation de la tangente en $0$ et en $-1$
   \begin{enumerate}
      \item $f'(0)=0$ donc $T_0 : y=0\times(x-0)+f(0)=2$
      \item $f'(-1)=-2$ donc $T_{-1} : y=-2\times(x+1)+f(-1)=-2x+1$
   \end{enumerate}
\end{exemple}

\end{document}
