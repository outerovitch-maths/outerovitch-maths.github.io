%% Font size %%
\documentclass[11pt]{article}

%% Load the custom package
\usepackage{Mathdoc}

%% Numéro de séquence %% Titre de la séquence %%
\renewcommand{\centerhead}{Chap. 5 : Dérivation - Dérivée en un point}

%% Spacing commands %%
\renewcommand{\baselinestretch}{1} \setlength{\parindent}{0pt}

\begin{document}

\begin{exercice}[0][Exercice Corrigé]
Soit la fonction définie sur $\R$ par $f(x)=3x+6$
\begin{enumerate}
\item Soit $h$ un réel non nul. \\
Calculer $f(1+h)-f(1)$ : \\
$f(1+h) = 3(1+h)+6 = 3+3h+6$ \\
$f(1)=3 \times 1 + 6 = 9$ \\
$f(1+h)-f(1)=3+3h+6-9=3h$
\item Donner la valeur du nombre dérivé de $f$ en $0$ : \\
$f'(0)=\lim\limits_{h\to0}\dfrac{f(1+h)-f(1)}{h}=\lim\limits_{h\to0}\dfrac{3h}{h}$ \\
$f'(0)=\lim\limits_{h\to0}3=3$
\end{enumerate}
\end{exercice}

\begin{exercice}[1][Dérivée d'une fonction constante]
Soit la fonction définie sur $\R$ par $f(x)=15$
\begin{enumerate}
\item Soit $h$ un réel non nul. \\
Calculer $f(0+h)-f(0)$ ;
\item Donner la valeur du nombre dérivé de $f$ en $0$.
\end{enumerate}
\end{exercice}

\begin{exercice}[2][Dérivée d'une fonction linéaire I]
Soit la fonction définie sur $\R$ par $f(x)=2x$
\begin{enumerate}
\item Soit $h$ un réel non nul. \\
Exprimer $f(5+h)-f(5)$ en fonction de h ;
\item Donner la valeur du nombre dérivé de $f$ en $5$.
\end{enumerate}
\end{exercice}

\begin{exercice}[2][Dérivée d'une fonction linéaire II]
Soit la fonction définie sur $\R$ par $f(x)=-7x$
\begin{enumerate}
\item Soit $h$ un réel non nul. \\
Exprimer $f(3+h)-f(3)$ en fonction de h ;
\item Donner la valeur du nombre dérivé de $f$ en $3$.
\end{enumerate}
\end{exercice}

\begin{exercice}[2][Dérivée d'une fonction affine I]
Soit la fonction définie sur $\R$ par $f(x)=5x-2$
\begin{enumerate}
\item Soit $h$ un réel non nul. \\
Exprimer $f(-1+h)-f(-1)$ en fonction de h ;
\item Donner la valeur du nombre dérivé de $f$ en $-1$.
\end{enumerate}
\end{exercice}

\begin{exercice}[2][Dérivée d'une fonction affine II]
Soit la fonction définie sur $\R$ par $f(x)=-6x+3$
\begin{enumerate}
\item Soit $h$ un réel non nul. \\
Exprimer $f(2+h)-f(2)$ en fonction de h ;
\item Donner la valeur du nombre dérivé de $f$ en $2$.
\end{enumerate}
\end{exercice}

\begin{exercice}[3][Dérivée d'une fonction du second degré I]
Soit la fonction définie sur $\R$ par $f(x)=x^2$
\begin{enumerate}
\item Soit $h$ un réel non nul. \\
Exprimer $f(1+h)-f(1)$ en fonction de h ;
\item Donner la valeur du nombre dérivé de $f$ en $1$.
\end{enumerate}
\end{exercice}

\begin{exercice}[4][Dérivée d'une fonction du second degré II]
Soit la fonction définie sur $\R$ par $f(x)=2x^2+4x$
\begin{enumerate}
\item Soit $h$ un réel non nul. \\
Exprimer $f(0+h)-f(0)$ en fonction de h ;
\item Donner la valeur du nombre dérivé de $f$ en $0$.
\item Soit $h$ un réel non nul. \\
Exprimer $f(-2+h)-f(-2)$ en fonction de h ;
\item Donner la valeur du nombre dérivé de $f$ en $-2$.
\end{enumerate}
\end{exercice}

\begin{comment}
\begin{exercice}[5][Dérivée de la fonction inverse]
Soit la fonction définie sur $\R^*$ par $f(x)=\dfrac{1}{x}$
\begin{enumerate}
\item Soit $h$ un réel non nul. \\
Montrer que $f(1+h)-f(1)=-\dfrac{h}{1+h}$ ;
\item Donner la valeur du nombre dérivé de $f$ en $1$.
\end{enumerate}
\end{exercice}

\end{comment}
\end{document}
