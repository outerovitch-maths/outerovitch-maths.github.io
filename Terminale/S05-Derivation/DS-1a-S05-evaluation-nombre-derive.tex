
%% Font size %%
\documentclass[11pt]{article}

%% Load the custom package
\usepackage{Mathdoc}

%% Numéro de séquence %% Titre de la séquence %%
\renewcommand{\centerhead}{Chap. 5 : Dérivation - Évaluation 1 : Calcul de nombre dérivé}

%% Spacing commands %%
\renewcommand{\baselinestretch}{1} \setlength{\parindent}{0pt}

\begin{document}
\phantom{0}
\vspace{-1cm}
\begin{center}
\duree{55 minutes} 
\total{10 points}
\coefficient{0,5}
\calculatrice{1}
\copieseparee{1}
\end{center}

\begin{exercice}[0][Exercice corrigé]
Soit la fonction définie sur $\R$ par $f(x)=3x+6$
\begin{enumerate}
\item Soit $h$ un réel non nul. \\
Calculer $f(1+h)-f(1)$ : \\
$f(1+h) = 3(1+h)+6 = 3+3h+6$ \\
$f(1)=3 \times 1 + 6 = 9$ \\
$f(1+h)-f(1)=3+3h+6-9=3h$
\item Donner la valeur du nombre dérivé de $f$ en $0$ : \\
$f'(0)=\lim\limits_{h\to0}\dfrac{f(1+h)-f(1)}{h}=\lim\limits_{h\to0}\dfrac{3h}{h}$ \\
$f'(0)=\lim\limits_{h\to0}3=3$
\end{enumerate}
\end{exercice}
\begin{multicols}{2}
\begin{exercicedevoir}[2]
Soit la fonction définie sur $\R$ par $f(x)=2x$
\begin{enumerate}
\item Soit $h$ un réel non nul. \\
Exprimer $f(5+h)-f(5)$ en fonction de h ;
\item Determiner $f'(5)$.
\end{enumerate}
\end{exercicedevoir}
\begin{exercicedevoir}[2]
Soit la fonction définie sur $\R$ par $f(x)=-7x$
\begin{enumerate}
\item Soit $h$ un réel non nul. \\
Exprimer $f(3+h)-f(3)$ en fonction de h ;
\item Determiner $f'(3)$.
\end{enumerate}
\end{exercicedevoir}
\begin{exercicedevoir}[2]
Soit la fonction définie sur $\R$ par $f(x)=5x-2$
\begin{enumerate}
\item Soit $h$ un réel non nul. \\
Exprimer $f(-1+h)-f(-1)$ en fonction de h
\item Determiner $f'(-1)$.
\end{enumerate}
\end{exercicedevoir}
\begin{exercicedevoir}[2]
Soit la fonction définie sur $\R$ par $f(x)=-6x+3$
\begin{enumerate}
\item Soit $h$ un réel non nul. \\
Exprimer $f(2+h)-f(2)$ en fonction de h ;
\item Donner la valeur du nombre dérivé de $f$ en $2$.
\end{enumerate}
\end{exercicedevoir}
\end{multicols}
\begin{exercicedevoir}[3]
Soit la fonction définie sur $\R$ par $f(x)=x^2$
\begin{enumerate}
\item Soit $h$ un réel non nul. \\
Exprimer $f(1+h)-f(1)$ en fonction de h ;
\item Donner la valeur du nombre dérivé de $f$ en $1$.
\end{enumerate}
\end{exercicedevoir}
\nonewpage
\end{document}

%%% Local Variables:
%%% mode: LaTeX
%%% TeX-master: t
%%% TeX-master: t
%%% End:

