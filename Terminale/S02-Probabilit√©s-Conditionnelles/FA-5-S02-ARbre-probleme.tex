%% Font size %%
\documentclass[11pt]{article}

%% Load the custom package
\usepackage{Mathdoc}

%% Numéro de séquence %% Titre de la séquence %%
\renewcommand{\centerhead}{}

%% Spacing commands %%
\renewcommand{\baselinestretch}{1}
\setlength{\parindent}{0pt}

\begin{document}

\section{Partie 1}

 \begin{exercice}[3][]
 \medskip

Pierre joue à un jeu dont une partie est constituée d'un lancer d'une fléchette sur une cible suivi d'un tirage au sort dans deux urnes contenant des tickets marqués ''gagnant'' ou
''perdant'' indiscernables.

\setlength\parindent{8mm}
\begin{itemize}
\item[$\bullet~~$] S'il tire un ticket marqué " gagnant ", il pourra recommencer une partie.
\item[$\bullet~~$] S'il atteint le centre de la cible, Pierre tire un ticket dans l'urne $U_1$ contenant exactement neuf tickets marqués " gagnant" et un ticket marqué " perdant ".
\item[$\bullet~~$] S'il n'atteint pas le centre de la cible (donc même s'il n'atteint pas la cible), Pierre tire un ticket dans l'urne $U_2$ contenant exactement quatre tickets marqués " gagnant" et six tickets marqués " perdant ".
\end{itemize}
\setlength\parindent{0mm}

\smallskip

Pierre atteint le centre de la cible avec une probabilité de 0,3.

\smallskip
On note les évènements suivants:

\qquad $C$ : " Pierre atteint le centre de la cible" ;

\qquad $G$ : " Pierre tire un ticket lui offrant une autre partie ".

\medskip

\begin{enumerate}
\item Recopier l'arbre pondéré ci-dessous et justifier la valeur $0,9$.



\item Compléter sur la copie l'arbre pondéré en traduisant les données de l'exercice.
\item Calculer la probabilité de l'évènement $\overline{C} \cap G$.
\item Montrer que la probabilité qu'à l'issue d'une partie Pierre en gagne une nouvelle est égale à $0,55$.
\item Sachant que Pierre a gagné une nouvelle partie, quelle est la probabilité qu'il ait atteint le centre de la cible ? Arrondir le résultat à $10^{-3}$.
\end{enumerate}
 \end{exercice}
 \begin{correction} 
 \medskip

Pierre joue à un jeu dont une partie est constituée d'un lancer d'une fléchette sur une cible suivi d'un tirage au sort dans deux urnes contenant des tickets marqués " gagnant" ou
" perdant" indiscernables.

\setlength\parindent{8mm}
\begin{itemize}
\item[$\bullet~~$] S'il tire un ticket marqué " gagnant ", il pourra recommencer une partie.
\item[$\bullet~~$] S'il atteint le centre de la cible, Pierre tire un ticket dans l'urne $U_1$ contenant exactement neuf tickets marqués " gagnant" et un ticket marqué " perdant ".
\item[$\bullet~~$] S'il n'atteint pas le centre de la cible (donc même s'il n'atteint pas la cible), Pierre tire un ticket dans l'urne $U_2$ contenant exactement quatre tickets marqués " gagnant" et six tickets marqués " perdant ".
\end{itemize}
\setlength\parindent{0mm}

\smallskip

Pierre atteint le centre de la cible avec une probabilité de 0,3.

\smallskip
On note les évènements suivants:

\qquad $C$ : " Pierre atteint le centre de la cible" ;

\qquad $G$ : " Pierre tire un ticket lui offrant une autre partie ".

\medskip

\begin{enumerate}
\item Recopier l'arbre pondéré ci-dessous et justifier la valeur $0,9$.

\item Compléter sur la copie l'arbre pondéré en traduisant les données de l'exercice.
\item Calculer la probabilité de l'évènement $\overline{C} \cap G$.
\item Montrer que la probabilité qu'à l'issue d'une partie Pierre en gagne une nouvelle est égale à $0,55$.
\item Sachant que Pierre a gagné une nouvelle partie, quelle est la probabilité qu'il ait atteint le centre de la cible ? Arrondir le résultat à $10^{-3}$.
\end{enumerate} 
 \end{correction}
\end{document}
