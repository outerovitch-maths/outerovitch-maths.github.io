%% Font size %%
\documentclass[11pt]{article}

%% Load the custom package
\usepackage{Mathdoc}

%% Numéro de séquence %% Titre de la séquence %%
\renewcommand{\centerhead}{Chap. 4 : Trigonométrie - Premiers exercices}

%% Spacing commands %%
\renewcommand{\baselinestretch}{1}
\setlength{\parindent}{0pt}

\begin{document}
% @see : https://coopmaths.fr/alea?uuid=4e684&id=1AN40&n=10&d=10&s=4&s2=-2\%2C2&alea=3kzl&cd=1&cols=2

\begin{exercice}[1][Déterminer la valeur exacte.]

\begin{multicols}{2}
\begin{enumerate}[itemsep=1em]
	\item \begin{minipage}[t]{\linewidth} $\cos\left(0\right)$ \end{minipage}
	\item \begin{minipage}[t]{\linewidth} $\cos\left(\dfrac{\pi}{2}\right)$ \end{minipage}
	\item \begin{minipage}[t]{\linewidth} $\sin\left(\dfrac{9\pi}{4}\right)$ \end{minipage}
	\item \begin{minipage}[t]{\linewidth} $\sin\left(-2\pi\right)$ \end{minipage}
	\item \begin{minipage}[t]{\linewidth} $\sin\left(-\dfrac{\pi}{2}\right)$ \end{minipage}
	\item \begin{minipage}[t]{\linewidth} $\cos\left(\dfrac{-7\pi}{4}\right)$ \end{minipage}
	\item \begin{minipage}[t]{\linewidth} $\sin\left(\dfrac{\pi}{2}\right)$ \end{minipage}
	\item \begin{minipage}[t]{\linewidth} $\sin\left(\dfrac{-11\pi}{6}\right)$ \end{minipage}
	\item \begin{minipage}[t]{\linewidth} $\cos\left(\dfrac{5\pi}{2}\right)$ \end{minipage}
	\item \begin{minipage}[t]{\linewidth} $\cos\left(\dfrac{9\pi}{4}\right)$ \end{minipage}
\end{enumerate}
\end{multicols}
\end{exercice}

% @see : https://coopmaths.fr/alea?uuid=a720c&id=1AN41&n=10&d=10&alea=4W1Y&cd=1&cols=2
\begin{exercice}[2][Déterminer la mesure principale de l'angle $\epsilon$.]
\begin{multicols}{2}
\begin{enumerate}[itemsep=1em]
	\item \begin{minipage}[t]{\linewidth} $\epsilon=\dfrac{-39\pi}{5}$ \end{minipage}
	\item \begin{minipage}[t]{\linewidth} $\epsilon=\dfrac{-51\pi}{7}$ \end{minipage}
	\item \begin{minipage}[t]{\linewidth} $\epsilon=\dfrac{32\pi}{3}$ \end{minipage}
	\item \begin{minipage}[t]{\linewidth} $\epsilon=\dfrac{-94\pi}{9}$ \end{minipage}
	\item \begin{minipage}[t]{\linewidth} $\epsilon=\dfrac{-50\pi}{11}$ \end{minipage}
	\item \begin{minipage}[t]{\linewidth} $\epsilon=\dfrac{-41\pi}{4}$ \end{minipage}
	\item \begin{minipage}[t]{\linewidth} $\epsilon=\dfrac{71\pi}{10}$ \end{minipage}
	\item \begin{minipage}[t]{\linewidth} $\epsilon=\dfrac{-128\pi}{13}$ \end{minipage}
	\item \begin{minipage}[t]{\linewidth} $\epsilon=\dfrac{19\pi}{6}$ \end{minipage}
	\item \begin{minipage}[t]{\linewidth} $\epsilon=\dfrac{59\pi}{6}$ \end{minipage}
\end{enumerate}
\end{multicols}
\end{exercice}

% @see : https://coopmaths.fr/alea?uuid=b9e6a&id=1AN42&n=10&d=10&s=1&alea=8rcn&cd=1&cols=2
\begin{exercice}[3][Déterminer une écriture en fonction de $\cos(x)$ ou $\sin(x)$.]

\begin{multicols}{2}
\begin{enumerate}[itemsep=1em]
	\item \begin{minipage}[t]{\linewidth} $\sin\big(x+5\pi\big)=$$\ldots$ \end{minipage}
	\item \begin{minipage}[t]{\linewidth} $\cos\left(x+\dfrac{\pi}{2}\right)=$$\ldots$ \end{minipage}
	\item \begin{minipage}[t]{\linewidth} $\sin\big(x-2\pi\big)=$$\ldots$ \end{minipage}
	\item \begin{minipage}[t]{\linewidth} $\cos\left(\dfrac{\pi}{2}-x\right)=$$\ldots$ \end{minipage}
	\item \begin{minipage}[t]{\linewidth} $\sin\big(x-\pi\big)=$$\ldots$ \end{minipage}
	\item \begin{minipage}[t]{\linewidth} $\cos\big(x-\pi\big)=$$\ldots$ \end{minipage}
	\item \begin{minipage}[t]{\linewidth} $\sin\big(-x\big)=$$\ldots$ \end{minipage}
	\item \begin{minipage}[t]{\linewidth} $\sin\big(x-3\pi\big)=$$\ldots$ \end{minipage}
	\item \begin{minipage}[t]{\linewidth} $\sin\big(x-4\pi\big)=$$\ldots$ \end{minipage}
	\item \begin{minipage}[t]{\linewidth} $\sin\big(x+2\pi\big)=$$\ldots$ \end{minipage}
\end{enumerate}
\end{multicols}
\end{exercice}
\end{document}
