%% Font size %%
\documentclass[11pt]{article}

%% Load the custom package
\usepackage{Mathdoc}

%% Numéro de séquence %% Titre de la séquence %%
\renewcommand{\centerhead}{}

%% Spacing commands %%
\renewcommand{\baselinestretch}{1}
\setlength{\parindent}{0pt}

\begin{document}

\section{Rappels de cours}

\begin{definition}
Le \underline{taux d'accroissement} de la fonction $f$ entre $a$ et $x$ est le quotient :
$$\dfrac{f(x)-f(a)}{x-a}$$
Avec $x=a+h$, ce quotient s'écrit aussi : $$\dfrac{f(a+h)-f(a)}{h}$$
\end{definition}

\begin{exercice}[0][Exercice corrigé.]
   Soit $f$ une fonction définie sur $\R$ par $f(x)=7x-3$, calculer le taux
   d'accroissement de $f$ entre $x$ et $a$. \\
\textbf{Correction.}\\ \\
   \begin{tabular}{p{0.3cm}l!{=}l}
      &  $\dfrac{f(x)-f(a)}{x-a}$ & $\dfrac{(7x-3)-(7a-3)}{x-a}$ \\
      & & $\dfrac{7x-7a}{x-a}$ \\
      & & $\dfrac{7(x-a)}{x-a}$ \\
      & & $7$ \\
   \end{tabular} \\
Donc le taux d'accroissement de la fonction $f$ entre $x$ et $a$ est
égal à $7$.
\end{exercice}

\begin{exercice}[0][Exercice corrigé.]
   Soit $f$ une fonction définie sur $\R$ par $f(x)=3x^2+5$, calculer le taux
   d'accroissement de $f$ entre $a$ et $a+h$. \\
\textbf{Correction.}\\ \\
   \begin{tabular}{p{0.3cm}l!{=}l}
      &  $\dfrac{f(a+h)-f(a)}{h}$ & $\dfrac{3(a+h)^2+5-(3a^2+5)}{h}$ \\
      & & $\dfrac{3(a^2+2ah+h^2)-3a^2-5}{h}$ \\
      & & $\dfrac{3a^2+6ah+3h^2-3a^2-5}{h}$ \\
      & & $6h+3h^2-5$ \\
   \end{tabular}\\
Donc le taux d'accroissement de la fonction $f$ entre $x$ et $a+h$ est
égal à $6h+3h^2-5$.
\end{exercice}

\end{document}
