%% Font size %%
\documentclass[11pt]{article}

%% Load the custom package
\usepackage{Mathdoc}

%% Numéro de séquence %% Titre de la séquence %%
\renewcommand{\centerhead}{Chap. 2 : Dérivation I (Taux d'accroissement)}

%% Spacing commands %%
\renewcommand{\baselinestretch}{1}
\setlength{\parindent}{0pt}

\begin{document}

\section{Pente entre deux points}

\begin{exercice}[0][Exercice corrigé.]
Soit \( f \) une fonction définie sur \( \mathbb{R} \) par \( f(x) = 25x + 36 \). Calculer la pente de la droite reliant les points d'abscisses \( a = 4 \) et \( b = -5 \).\\
\textbf{Correction.}\\ \\
La pente \( m \) entre deux points \( a \) et \( b \) est donnée par la formule :
\[
m = \frac{f(b) - f(a)}{b - a}
\]
Soit \( f(x) = 25x + 36 \),  calculons  \( f(a) \) et \( f(b) \).

\[
f(4) = 25 \times 4 + 36 = 100 + 36 = 136
\]
\[
f(-5) = 25 \times (-5) + 36 = -125 + 36 = -89
\]

Ensuite, nous appliquons la formule de la pente :
\[
m = \frac{f(-5) - f(4)}{-5 - 4} = \frac{-89 - 136}{-5 - 4} = \frac{-225}{-9} = 25
\]

Ainsi, la pente de la droite reliant les points d'abscisses \( a = 4 \) et \( b = -5 \) est \( 25 \).
\end{exercice}

\begin{exercice}
\begin{enumerate}
    \item Soit \( f(x) = 3x + 2 \). Calculer la pente de la droite reliant les points d'abscisses \( a = 1 \) et \( b = 4 \).
    
    \item Soit \( f(x) = -2x + 5 \). Calculer la pente de la droite reliant les points d'abscisses \( a = -3 \) et \( b = 2 \).
    
    \item Soit \( f(x) = x^2 - x \). Calculer la pente de la droite reliant les points d'abscisses \( a = 0 \) et \( b = 3 \).
    
    \item Soit \( f(x) = 2x^3 - 4x \). Calculer la pente de la droite reliant les points d'abscisses \( a = -1 \) et \( b = 2 \).
    
    \item Soit \( f(x) = \frac{1}{x} \). Calculer la pente de la droite reliant les points d'abscisses \( a = 1 \) et \( b = 4 \).
    
    \item Soit \( f(x) = \sqrt{x+1} \). Calculer la pente de la droite reliant les points d'abscisses \( a = 0 \) et \( b = 3 \).
\end{enumerate}

\end{exercice}

\section{Rappels de cours}

\begin{definition}
Le \underline{taux d'accroissement} de la fonction $f$ entre $a$ et $x$ est le quotient :
$$\dfrac{f(x)-f(a)}{x-a}$$
Avec $x=a+h$, ce quotient s'écrit aussi : $$\dfrac{f(a+h)-f(a)}{h}$$
\end{definition}

\section{Taux d'accroissement}

\begin{exercice}[0][Exercice corrigé.]
   Soit $f$ une fonction définie sur $\R$ par $f(x)=3x^2+5$, calculer le taux
   d'accroissement de $f$ entre $a$ et $a+h$. \\
\textbf{Correction.}\\ \\
   \begin{tabular}{p{0.3cm}l!{=}l}
      &  $\dfrac{f(a+h)-f(a)}{h}$ & $\dfrac{3(a+h)^2+5-(3a^2+5)}{h}$ \\
      & & $\dfrac{3(a^2+2ah+h^2)-3a^2-5}{h}$ \\
      & & $\dfrac{3a^2+6ah+3h^2-3a^2-5}{h}$ \\
      & & $6h+3h^2-5$ \\
   \end{tabular}\\
Donc le taux d'accroissement de la fonction $f$ entre $x$ et $a+h$ est
égal à $6h+3h^2-5$.
\end{exercice}

\begin{exercice}
\textbf{Dans chacun des cas, calculer le taux d'accroissement entre
$a$ et $a+h$. Exprimer le résultat en fonction de $a$ et de $h$.}
\begin{enumerate}
    \item Soit \( f(x) = 2x + 3 \). Calculer le taux d'accroissement de \( f \) entre \( a \) et \( a + h \).
    
    \item Soit \( f(x) = -x + 5 \). Calculer le taux d'accroissement de \( f \) entre \( a \) et \( a + h \).
    
    \item Soit \( f(x) = x^2 - 4x + 1 \). Calculer le taux d'accroissement de \( f \) entre \( a \) et \( a + h \).
    
    \item Soit \( f(x) = 3x^2 + 19 \). Calculer le taux d'accroissement de \( f \) entre \( a \) et \( a + h \).
    
    \item Soit \( f(x) = 4x^3 - 2x \). Calculer le taux d'accroissement de \( f \) entre \( a \) et \( a + h \).
    
    \item Soit \( f(x) = \frac{1}{x+5} \). Calculer le taux d'accroissement de \( f \) entre \( a \) et \( a + h \).
    
    \item Soit \( f(x) = \sqrt{x+2} \). Calculer le taux d'accroissement de \( f \) entre \( a \) et \( a + h \).
    
    \item Soit \( f(x) = 5x^3 + x^2 - 7 \). Calculer le taux d'accroissement de \( f \) entre \( a \) et \( a + h \).
    
    \item Soit \( f(x) = x^4 - 3x + 1 \). Calculer le taux d'accroissement de \( f \) entre \( a \) et \( a + h \).
    
    \item Soit \( f(x) = \frac{2}{x+1} \). Calculer le taux d'accroissement de \( f \) entre \( a \) et \( a + h \).
\end{enumerate}
\end{exercice}

\end{document}
