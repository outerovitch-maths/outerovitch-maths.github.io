%% Font size %%
\documentclass[11pt]{article}

%% Load the custom package
\usepackage{Mathdoc}

%% Numéro de séquence %% Titre de la séquence %%
\renewcommand{\centerhead}{Chap. 5 : Suites - Fiche de révision}

%% Spacing commands %%
\renewcommand{\baselinestretch}{1} \setlength{\parindent}{0pt}

\begin{document}

\begin{multicols}{2}

\section{Sens de variation}

\begin{definition}[Suite croissante]
$(u_n)$ croissante $\iff$ $u_{n+1}
\ge u_n$, $\forall n \ge n_0$
\end{definition}

\begin{definition}[Suite décroissante]
$(u_n)$ décroissante $\iff$ $u_{n+1}
\le u_n$, $\forall n \ge n_0$
\end{definition}

\begin{definition}[Suite constante]
$(u_n)$ constante $\iff$ $u_{n+1}
= u_n$, $\forall n \ge n_0$
\end{definition}

\section{Suites arithmétiques}

\begin{propriete}
Soit $(u_n)$ une suite arithmétique de raison $r$ ;
\begin{itemize}
\item Si $r > 0$ la suite est strictement croissante ;
\item Si $r < 0$ la suite est strictement décroissante ;
\item Si $r = 0$ la suite est constante.
\end{itemize}
\end{propriete}

\begin{theoreme}[Terme général.]
Soit une suite arithmétique de raison $r$ définie à partir d'un
certain rang $n_0$.\\
$\forall n \ge 0$ :
\[ u_n = u_{p}+(n-p)r\]
En particulier si $p=0$, on a : 
\[ u_n = u_0 + n \times r\]
\end{theoreme}

\section{Suites géométriques}

\begin{propriete}
Soit $(u_n)$ une suite arithmétique de raison $q>0$ ;
\begin{itemize}
\item Si $q > 1$ la suite est strictement croissante ;
\item Si $0 < q < 1$ la suite est strictement décroissante ;
\item Si $q = 1$ la suite est constante.
\end{itemize}
\end{propriete}

\begin{theoreme}[Terme général.]
Soit une suite géométrique de raison $q$ définie à partir d'un
certain rang $n_0$.\\
$\forall n \ge 0$ :
\[ u_n = u_{p} \times q^{(n-p)}\]
En particulier si $p=0$, on a : 
\[ u_n = u_0 \times q^n\]
\end{theoreme}

\newcolumn

\section{Déterminer la raison}

\begin{propriete}
Soit $(u_n)$ une suite arithmétique, étant donné $n_1$ et $n_2$ deux
rangs de la suite tels que $u_{n_1}$ et $u_{n_2}$, on  à :
\[ r = \dfrac{u_{n_2}-u_{n_1}}{n_2-n_1}\]
\end{propriete}

\begin{propriete}
Soit $(u_n)$ une suite géométrique, étant donné $n_1$ et $n_2$ deux
termes de la suite tels que $u_{n_1}=a$ et $u_{n_2}=b$, on  à:
\[ q = \sqrt[k]{\dfrac{u_{n_2}}{u_{n_1}}} = \left(
 \dfrac{u_{n_2}}{u_{n_1}} \right)^{\dfrac{1}{k}}, \text{ Avec : }
k = n_{2}-n_{1}\]
\end{propriete}

\section{Séries}

\begin{theoreme}[Suite arithmétique.]
Soit $(u_n)$ une suite arithmétique :
\[ S_n = u_0 + u_1 + u_2 + \ldots + u_n = \sum_{k=0}^{n} u_k = \dfrac{(n+1)(u_0+u_n)}{2} \]
Plus généralement, pour tout entier naturel $p<n$ :
\[S_n = u_p + u_{p+1} + \ldots + u_n = \sum_{k=p}^{n} u_k = \dfrac{(n-p+1)(u_p+u_n)}{2} \]
\end{theoreme}

\begin{theoreme}[Suites géométrique.]
Soit $(u_n)$ une suite géométrique :
\[ S_n = u_0 + u_1 + u_2 + \ldots + u_n = \sum_{k=0}^{n} u_k = u_0
\times \dfrac{1-q^{n+1}}{1-q} \]
Plus généralement, pour tout entier naturel $p<n$ :
\[ S_n = u_p + u_{p+1} + \ldots + u_n = \sum_{k=p}^{n} u_k = u_p
\times \dfrac{1-q^{n-p+1}}{1-q} \]
\end{theoreme}
\end{multicols}

\end{document}
