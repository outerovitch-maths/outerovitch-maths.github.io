%% Font size %%
\documentclass[11pt]{article}

%% Load the custom package
\usepackage{Mathdoc}

%% Numéro de séquence %% Titre de la séquence %%
\renewcommand{\centerhead}{Devoir Bilan : Suites et Séries}

%% Spacing commands %%
\renewcommand{\baselinestretch}{1} \setlength{\parindent}{0pt}


\begin{document}

\begin{center}
\duree{2 heures} 
\total{30 points}
\coefficient{1}
\calculatrice{1}
\brouillon
\recherche 
\copieseparee{1}
\end{center}

\begin{exercicedevoir}[4]
\begin{enumerate}
\item Soit $(w_n)$ une suite définie par $w_{6}=10$ et  
$w_{n+1}=-15\times w_n$ pour tout entier naturel $n$.\\
Donner l'expression de $w_n$ en fonction de $n$.
\item Soit $(v_n)$ une suite arithmétique de raison $r=5$ telle que $v_{1}=7$.\\
Donner l'expression de $v_n$ en fonction de $n$.
\item Soit $(t_n)$ une suite définie par $t_{8}=6{,}3$ et  
$t_{n+1}=t_n+9$ pour tout entier naturel $n$.\\
Donner l'expression de $t_n$ en fonction de $n$.
\item Soit $(v_n)$ une suite géométrique de raison $q=11$ telle que $v_{6}=-4$.\\
Donner l'expression de $v_n$ en fonction de $n$.
\end{enumerate}
\end{exercicedevoir}

\begin{exercicedevoir}[4]
\begin{enumerate}
\item Soit $(v_n)_{n \in \N}$ une suite arithmétique telle que $v_4=76$ et $v_{17}=32$, en détaillant toutes les étapes de votre raisonnement, calculer $v_{100}$.
\item Soit $(w_n)_{n \in \N}$ une suite géométrique telle que $v_{32}=0,14$ et $v_{15}=0,99$, en détaillant toutes les étapes de votre raisonnement, calculer $v_{8}$.
\end{enumerate}
\end{exercicedevoir}

\begin{exercicedevoir}[4]
Soit $(u_n)$ la suite définie pour tout entier naturel $n$ par $u_{n+1}=-1{,}8u_n +22{,}4$ et $u_0=-1{,}2$.
\begin{enumerate}
\item On pose $v_n=u_n -8$ pour tout entier naturel $n$.\\
Montrer que  $(v_n)$ est une suite géométrique.\\
Donner sa raison et son premier terme.
\item Exprimer $v_n$ en fonction de $n$.
\item En déduire l'expression du terme général de $(u_n)$ en fonction de $n$.
\end{enumerate}
\end{exercicedevoir}

\begin{exercicedevoir}[6]
\begin{enumerate}
\item Soit $u$ la suite géométrique de premier terme $u_0 = 6$ et de raison $0{,}4$.\\Calculer $\displaystyle S = u_0 + u_1 + ... + u_{12} =\sum_{k=0}^{12}u_k$ et donner un arrondi au millième près.
\item Soit $w$ la suite géométrique de premier terme $w_1 = 4$ et de
raison $0{,}5$.\\Calculer
$\displaystyle S = w_1 + w_2 + ... + w_{15} =\sum_{k=1}^{15}w_k$ et
donner un arrondi au millième près.
\item Soit $u$ la suite terme générale $u_n = \dfrac{3}{4^n}$\\
Calculer $\displaystyle S = u_5 + u_2 + ... + u_{15} =\sum_{k=5}^{15}u_k$ et
donner un arrondi au millième près.
\end{enumerate}
\end{exercicedevoir}

\begin{exercicedevoir}[6]
Pour placer un capital de 5~000 euros, une banque propose un placement à taux fixe de 5\,\%
par an. Avec ce placement, le capital augmente de 5\,\% chaque année par rapport à l'année
précédente. Pour bénéficier de ce taux avantageux, il ne faut effectuer aucun retrait
d'argent durant les quinze premières années.

On modélise l'évolution du capital disponible par une suite $\left(u_n\right)$. On note $u_n$ le capital
disponible après $n$ années de placement.

On dépose 5000 euros le 1er janvier 2020. Ainsi $u_0 = 5~000 $.

\medskip

\begin{enumerate}
\item Montrer que $u_2 = 5~512,5$. Interpréter ce résultat dans le contexte de l'exercice.
\item Exprimer $u_{n+1}$ en fonction de $u_n$.
\item Quelle est la nature de la suite $\left(u_n\right)$ ? Préciser son premier terme et sa raison.
\item Exprimer $u_n$ en fonction de $n$.
\item Justifier que le capital aura doublé après $15$ années de placement.
\end{enumerate}
\end{exercicedevoir}

\begin{exercicedevoir}[6]
En $2024$, le nombre d'abonnés à une page de réseau social d'un musicien était de $12\,480$.\\
On suppose que chaque année, il obtient $1040$ abonnés supplémentaires.\\
On désigne par $w_n$ le nombre d'abonnés en $2024 + n$ pour tout
entier naturel $n$.\\
\begin{enumerate}
\item Calculer le nombre d'abonnés en $2025$ et $2026$.
\item Exprimer $w_{n+1}$ en fonction de $w_n$, pour tout entier naturel $n$ et en déduire la nature de
la suite $(w_n)$. \\
Préciser sa raison et son premier terme.
\item Exprimer, pour tout entier $n$, $w_n$ en fonction de $n$.
\item En quelle année le nombre d'abonnés aura triplé par rapport à l'année $2024$ ?
\end{enumerate}
\end{exercicedevoir}

\nonewpage

\end{document}

%%% Local Variables:
%%% mode: LaTeX
%%% TeX-master: t
%%% TeX-master: t
%%% End:

