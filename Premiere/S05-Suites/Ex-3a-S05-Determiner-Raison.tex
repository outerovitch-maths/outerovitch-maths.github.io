%% Font size %%
\documentclass[11pt]{article}

%% Load the custom package
\usepackage{Mathdoc}

%% Numéro de séquence %% Titre de la séquence %%
\renewcommand{\centerhead}{Chap. 5 : Suites - Calcul de raisons}

%% Spacing commands %%
\renewcommand{\baselinestretch}{1} \setlength{\parindent}{0pt}

\begin{document}

\begin{multicols}{2}
\begin{exercice}[1]
\begin{enumerate}
\item Soit $(v_n)$ une suite arithmétique telle que :\\
$v_{6}=-3$ et $v_{7}=-15$.\\Donner la raison $r$ de cette suite.
\item Soit $(u_n)$ une suite arithmétique telle que :\\
$u_{5}=-14$ et $u_{6}=-28$.\\Donner la raison $r$ de cette suite.
\item Soit $(u_n)$ une suite arithmétique telle que :\\
$u_{6}=6$ et $u_{7}=17$.\\Donner la raison $r$ de cette suite.
\item Soit $(w_n)$ une suite arithmétique telle que :\\
$w_{10}=2$ et $w_{11}=-2$.\\Donner la raison $r$ de cette suite.
\end{enumerate}
\end{exercice}

\begin{exercice}[2]
\begin{enumerate}
\item Soit $(w_n)$ une suite arithmétique telle que $w_{8}=2{,}7$ et $w_{13}=-14{,}3$.\\
Quelle est la valeur de la raison $r$ de cette suite ?
\item Soit $(w_n)$ une suite arithmétique telle que $w_{3}=6{,}2$ et $w_{8}=29{,}7$.\\
Quelle est la valeur de la raison $r$ de cette suite ?
\item Soit $(t_n)$ une suite arithmétique telle que $t_{8}=8{,}6$ et $t_{14}=-41{,}2$.\\
Quelle est la valeur de la raison $r$ de cette suite ?
\item Soit $(t_n)$ une suite arithmétique telle que $t_{0}=7{,}4$ et $t_{8}=25$.\\
Quelle est la valeur de la raison $r$ de cette suite ?
\end{enumerate}
\end{exercice}
\end{multicols}

\begin{multicols}{2}
\begin{exercice}[1]
Donner la raison $q$ de ces suites.
\begin{enumerate}
\item Soit $(u_n)$ une suite géométrique  telle que :\\
$u_{3}=-8$ et $u_{4}=24$.
\item Soit $(u_n)$ une suite géométrique  telle que :\\
$u_{4}=8$ et $u_{5}=80$.
\item Soit $(u_n)$ une suite géométrique  telle que :\\
$u_{0}=-7$ et $u_{1}=-70$.
\item Soit $(w_n)$ une suite géométrique  telle que :\\
$w_{8}=-8$ et $w_{9}=40$.
\end{enumerate}
\end{exercice}

\begin{exercice}[2]
Donner la raison $q$ de ces suites.
\begin{enumerate}
\item Soit $(w_n)$ une suite géométrique de raison $q$ strictement négative telle que $w_{4}=2{,}7$ et $w_{6}=213{,}867$.
\item Soit $(t_n)$ une suite géométrique de raison $q$ strictement positive telle que $t_{1}=6$ et $t_{3}=31{,}74$.
\item Soit $(w_n)$ une suite géométrique de raison $q$ strictement positive telle que $w_{9}=2{,}8$ et $w_{11}=107{,}632$.
\item Soit $(u_n)$ une suite géométrique de raison $q$ strictement positive telle que $u_{9}=0{,}9$ et $u_{11}=31{,}329$.
\end{enumerate}
\end{exercice}
\end{multicols}

\begin{exercice}[3][Problème]
Soit une suite arithmétique $(a_n)$ définie par son terme général
$a_n=a_0+n\times r$. On sait que :
\[ a_3 + a_8 = 14 \text{ et } a_{12} - a_2 = -20 \]
\begin{multicols}{2}
\begin{enumerate}
\item Déterminer la raison $r$ de la suite ;
\item Calculer le premier terme $a_0$ ;
\item Vérifier que $a_5= -6$ ;
\item Déterminer $n$ tel que $a_n = -40$.
\end{enumerate}
\end{multicols}
\end{exercice}

\end{document}
