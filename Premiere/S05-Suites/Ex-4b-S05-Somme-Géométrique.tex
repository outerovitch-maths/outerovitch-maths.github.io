%% Font size %%
\documentclass[11pt]{article}

%% Load the custom package
\usepackage{Mathdoc}

%% Numéro de séquence %% Titre de la séquence %%
\renewcommand{\centerhead}{Chap. 5 : Suites - Somme des termes d'une
  suite géométrique}

%% Spacing commands %%
\renewcommand{\baselinestretch}{1} \setlength{\parindent}{0pt}

\begin{document}

\begin{exercice}[1][Par définition]
\begin{multicols}{2}
\begin{enumerate}
\item Soit $v$ la suite géométrique de premier terme $v_0 = 3$ et de
raison $0{,}5$.\\Calculer
$\displaystyle S = v_0 + v_1 + ... + v_{15} =\sum_{k=0}^{15}v_k$ et
donner un arrondi au millième près.
\item Soit $v$ la suite géométrique de premier terme $v_0 = 6$ et de
raison $0{,}2$.\\Calculer
$\displaystyle S = v_0 + v_1 + ... + v_{15} =\sum_{k=0}^{15}v_k$.
\item Soit $v$ la suite géométrique de premier terme $v_0 = 3$ et de
raison $1{,}5$.\\Calculer
$\displaystyle S = v_0 + v_1 + ... + v_{10} =\sum_{k=0}^{10}v_k$ et
donner un arrondi au millième près.
\item Soit $w$ la suite géométrique de premier terme $w_1 = 5$ et de
raison $1{,}1$.\\Calculer
$\displaystyle S = w_1 + w_2 + ... + w_{12} =\sum_{k=1}^{12}w_k$ et
donner un arrondi au millième près.
\end{enumerate}
\end{multicols}
\end{exercice}

\begin{exercice}[2][Calcul de raison, puis somme des termes]
\begin{enumerate}
\item Soit $v$ la suite géométrique telle que $v_6 = 3$ et $v_8 =
0,75$.\\
Calculer $\displaystyle S = v_4 + v_5 + ... + v_{15} =\sum_{k=4}^{15}v_k$ et donner un arrondi au millième près.
\item Soit $v$ la suite géométrique telle que $v_{19} = \dfrac{116}{2}$ et $v_{35} = \dfrac{1~657}{13}$.\\
Calculer $\displaystyle S = v_{25} + v_{26} + ... + v_{40}
=\sum_{k=25}^{40}v_k$ et donner un arrondi au millième près.
\end{enumerate}
\end{exercice}

\begin{exercice}[2][Par extension]
Somme des termes d'une suite géométrique .
\begin{enumerate}
    \item Calculer : \( S = 4 + 4^2 + 4^3 + \dots + 4^7 \)
    \item Calculer : \( S = \frac{1}{4} + \frac{1}{8} + \frac{1}{16} + \dots + \frac{1}{1~048~576} \)
    \item Calculer : \( S = \frac{1}{3} - \frac{1}{9} + \frac{1}{27} - \dots - \frac{1}{6~561} \)
    \item Calculer : \( S = 1 + \frac{1}{10} + \frac{1}{100} + \dots + \frac{1}{10^7} \)
\end{enumerate}
\end{exercice}

\begin{exercice}[3][Problème]
\( (u_n) \) est une suite géométrique croissante dont les termes sont négatifs. Son premier terme est \( u_1 \).
    \begin{enumerate}
        \item Que peut-on dire de sa raison ?
        \item On sait que \( u_1 \times u_3 = \frac{4}{9} \) et \( u_1
        + u_2 + u_3 = -\frac{19}{9} \). \\
        Calculer \( u_1 \), \( u_2 \) et \( u_3 \).
        \item Calculer \( u_n \) en fonction de \( n \).
        \end{enumerate}
\end{exercice}

\end{document}
