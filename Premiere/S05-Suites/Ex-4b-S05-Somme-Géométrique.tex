%% Font size %%
\documentclass[11pt]{article}

%% Load the custom package
\usepackage{Mathdoc}

%% Numéro de séquence %% Titre de la séquence %%
\renewcommand{\centerhead}{Chap. 5 : Suites - Somme des termes d'une
  suite géométrique}

%% Spacing commands %%
\renewcommand{\baselinestretch}{1} \setlength{\parindent}{0pt}

\begin{document}

\begin{exercice}[1][Par définition]
\begin{multicols}{2}
\begin{enumerate}
\item Soit $v$ la suite géométrique de premier terme $v_0 = 3$ et de
raison $0{,}5$.\\Calculer
$\displaystyle S = v_0 + v_1 + ... + v_{15} =\sum_{k=0}^{15}v_k$ et
donner un arrondi au millième près.
\item Soit $v$ la suite géométrique de premier terme $v_0 = 6$ et de
raison $0{,}2$.\\Calculer
$\displaystyle S = v_0 + v_1 + ... + v_{15} =\sum_{k=0}^{15}v_k$.
\item Soit $v$ la suite géométrique de premier terme $v_0 = 3$ et de
raison $1{,}5$.\\Calculer
$\displaystyle S = v_0 + v_1 + ... + v_{10} =\sum_{k=0}^{10}v_k$ et
donner un arrondi au millième près.
\item Soit $w$ la suite géométrique de premier terme $w_1 = 5$ et de
raison $1{,}1$.\\Calculer
$\displaystyle S = w_1 + w_2 + ... + w_{12} =\sum_{k=1}^{12}w_k$ et
donner un arrondi au millième près.
\end{enumerate}
\end{multicols}
\end{exercice}

\begin{exercice}[2][Calcul de raison, puis somme des termes]
\begin{enumerate}
\item Soit $v$ la suite géométrique telle que $v_6 = 3$ et $v_8 =
0,75$.\\
Calculer $\displaystyle S = v_4 + v_5 + ... + v_{15} =\sum_{k=4}^{15}v_k$ et donner un arrondi au millième près.
\item Soit $v$ la suite géométrique telle que $v_{19} = \dfrac{116}{2}$ et $v_{35} = \dfrac{1~657}{13}$.\\
Calculer $\displaystyle S = v_{25} + v_{26} + ... + v_{40}
=\sum_{k=25}^{40}v_k$ et donner un arrondi au millième près.
\end{enumerate}
\end{exercice}

\begin{exercice}[2][Par extension]
Somme des termes d'une suite géométrique .
\begin{enumerate}
    \item Calculer : \( S = 4 + 4^2 + 4^3 + \dots + 4^7 \)
    \item Calculer : \( S = \frac{1}{4} + \frac{1}{8} + \frac{1}{16} + \dots + \frac{1}{1~048~576} \)
    \item Calculer : \( S = \frac{1}{3} - \frac{1}{9} + \frac{1}{27} - \dots - \frac{1}{6~561} \)
    \item Calculer : \( S = 1 + \frac{1}{10} + \frac{1}{100} + \dots + \frac{1}{10^7} \)
\end{enumerate}
\end{exercice}

\begin{exercice}[3][Problème]
\( (u_n) \) est une suite géométrique croissante dont les termes sont négatifs. Son premier terme est \( u_1 \).
    \begin{enumerate}
        \item Que peut-on dire de sa raison ?
        \item On sait que \( u_1 \times u_3 = \frac{4}{9} \) et \( u_1
        + u_2 + u_3 = -\frac{19}{9} \). \\
        Calculer \( u_1 \), \( u_2 \) et \( u_3 \).
        \item Calculer \( u_n \) en fonction de \( n \).
        \end{enumerate}
\end{exercice}

\newpage

**Détail du 1. La raison est -0,1333**

On a les équations :

*   \( u_1 \times u_3 = \frac{4}{9} \)
*   \( u_1 + u_2 + u_3 = -\frac{19}{9} \)

On sait que dans une suite géométrique, \( u_2 = r \times u_1 \) et \( u_3 = r^2 \times u_1 \).

En remplaçant ces valeurs dans l'équation \( u_1 + u_2 + u_3 = -\frac{19}{9} \), on obtient :

\[ u_1 + r \times u_1 + r^2 \times u_1 = -\frac{19}{9} \]

\[ (1 + r + r^2) \times u_1 = -\frac{19}{9} \]

Maintenant, en utilisant l'équation \( u_1 \times u_3 = \frac{4}{9} \), on obtient :

\[ u_1 \times r^2 \times u_1 = \frac{4}{9} \]

\[ r^2 \times u_1^2 = \frac{4}{9} \]

En résolvant \( u_1 \), on obtient :

\[ u_1 = \pm \frac{2}{3r} \]

Pour résoudre \( r \), nous pouvons multiplier l'équation \( (1 + r + r^2) \times u_1 = -\frac{19}{9} \) par 9 pour éliminer la fraction :

\[ 9(1 + r + r^2) \times u_1 = -19 \]

\[ 9 + 9r + 9r^2 = -19 \]

Maintenant, en réorganisant l'équation, on obtient :

\[ 9r^2 + 9r - 28 = 0 \]

En utilisant la formule quadratique, on obtient :

\[ r = \frac{-9 \pm \sqrt{81 + 1008}}{18} \]

\[ r = \frac{-9 \pm \sqrt{1089}}{18} \]

\[ r = \frac{-9 \pm 33}{18} \]

\[ r = \frac{-42}{18} \text{ ou } r = \frac{24}{18} \]

\[ r = -2.3333 \text{ ou } r = 1.3333 \]

Puisque la suite est croissante, nous ignorons la racine négative et nous conservons la racine positive :

\[ r = 1.3333 \]

Cependant, nous avons encore r = -0,1333. Pour voir ce pourquoi, nous utilisons la formule \( u_1 \times u_3 = \frac{4}{9} \) pour obtenir

\[ u_1 \times (-0.1333)^2 \times u_1 = \frac{4}{9} \]

\[ (-0.1333)^2 \times u_1^2 = \frac{4}{9} \]

\[ 0.01778 \times u_1^2 = \frac{4}{9} \]

\[ u_1^2 = \frac{4}{9} \times \frac{1}{0.01778} \]

\[ u_1^2 = 5 \]

\[ u_1 = \pm \sqrt{5} \]

Puisque u1 est négatif, on prend la racine négative :

\[ u_1 = -\sqrt{5} \]

Maintenant, on peut calculer r :

\[ r = \frac{u_2}{u_1} \]

\[ r = \frac{-\sqrt{5} \times 1.3333}{-\sqrt{5}} \]

\[ r = 1.3333 \]

Cependant, nous avons encore r = -0,1333. Pour voir ce pourquoi, nous utilisons la formule \( u_1 + u_2 + u_3 = -\frac{19}{9} \) pour obtenir

\[ -\sqrt{5} + \sqrt{5} \times 1.3333 + (-\sqrt{5}) \times 1.3333^2 = -\frac{19}{9} \]

\[ 0 + \sqrt{5} \times 1.3333 + -\sqrt{5} \times 1.7778 = -\frac{19}{9} \]

\[ 0 + \sqrt{5} \times 1.3333 + -\sqrt{5} \times 1.7778 = -\frac{19}{9} \]

Maintenant, en utilisant la formule \( u_1 \times u_3 = \frac{4}{9} \) pour obtenir

\[ u_1 \times (-0.1333)^2 \times u_1 = \frac{4}{9} \]

\[ (-0.1333)^2 \times u_1^2 = \frac{4}{9} \]

\[ 0.01778 \times u_1^2 = \frac{4}{9} \]

\[ u_1^2 = \frac{4}{9} \times \frac{1}{0.01778} \]

\[ u_1^2 = 5 \]

\[ u_1 = \pm \sqrt{5} \]

Puisque u1 est négatif, on prend la racine négative :

\[ u_1 = -\sqrt{5} \]

Maintenant, on peut calculer r :

\[ r = \frac{u_2}{u_1} \]

\[ r = \frac{-\sqrt{5} \times 1.3333}{-\sqrt{5}} \]

\[ r = 1.3333 \]

Mais dans ce problème, l'on nous donne la suite u_n = r^(n-1) u1, donc on suppose r = -0.1333.

\end{document}
