
%% Font size %%
\documentclass[11pt]{article}

%% Load the custom package
\usepackage{Mathdoc}

%% Numéro de séquence %% Titre de la séquence %%
\renewcommand{\centerhead}{}

%% Spacing commands %%
\renewcommand{\baselinestretch}{1} \setlength{\parindent}{0pt}

\begin{document}

\section{Étude globale}

\subsection{Définitions}

\begin{definition}[Suite numérique]
Une suite numérique est une fonction de dans $\N$ dans $\R$.
\end{definition}

\begin{exemple}
La fonction définie pour tout entier naturel $n$ par $u(n)=2n+1$ est une suite.
\end{exemple}

\begin{notation}
\vspace{-0.5cm}
\begin{itemize}
\item On peut désigner la suite $u$ avec la notation $(u_n)$ (entre
parenthèses) ;
\item L'écriture $u_n$ (sans parenthèses) désigne le terme de rang $n$
de la suite $u$, c'est à dire $u(n)$ ;
\end{itemize}
\end{notation}

\begin{remarque}
Une suite $u$ peut être définie à partir d'un certain rang $u_0$, on
notera alors $(u_n)_{n \ge n_0}$ pour désigner la suite $u$. 
\end{remarque}

\begin{definition}[Modes de génération]
Il existe trois façon de définir une suite :
\begin{enumerate}
\item \textbf{Définition explicite :} \\
La suite $(u_n)$ est définie
directement par son terme général :
$$ u_n=f(n) $$
Avec $f$ une fonction dépendant de
$n$ définie sur $\N$ ; 
\item \textbf{Définition par récurrence :} \\
Soit $f$ une fonction définie sur
$\R$ et $a \in \R$, une suite $u_n$
peut être définie par récurrence par :
$$\begin{cases}
u_0 = a \\
u_{n+1} = f(u_n)
\end{cases}$$
\item \textbf{Définition implicite :} \\
La suite est définie par une
propriété géométrique, économique \dots au sein d'un
problème.
\end{enumerate}
\end{definition}

\begin{remarque}
Quel que soit le mode de définition
d'une suite, il se peut que celle-ci
ne soit définie qu'à partir d'un
certain rang $n_0 > 0$.
\end{remarque}

\begin{remarque}
On peut faire les analogies suivantes
entre les suites et les fonctions :

\begin{center}
\begin{tabular}{|c|c|c|}
\hline
Fonctions&&Suites \\ \hline
f&$\leftrightarrow$&u \\ \hline
x&$\leftrightarrow$&n \\ \hline
antécédent&$\leftrightarrow$&rang \\ \hline
image&$\leftrightarrow$&terme \\ \hline
f&$\leftrightarrow$&u \\ \hline
\end{tabular}
\end{center}
\end{remarque}

\subsection{Sens de variation}

\begin{remarque}
Dans la suite, on considère $(u_n)$
une suite définie sur $\N$ pour tout
$n \ge n_0$, avec $n_0 \ge 0$.
\end{remarque}

\begin{definition}[Suite croissante]
$(u_n)$ croissante $\iff$ $u_{n+1}
\ge u_n$, $\forall n \ge n_0$
\end{definition}

\begin{exemple}
Considérons $(u_n)$ définie par :
\[ \begin{cases}
u_0 = 12 \\
u_{n+1} = (u_n)^2 + u_n, \forall n
\in \N
\end{cases} \]
\end{exemple}

\begin{definition}[Suite strictement
croissante]
$(u_n)$ strictement croissante $\iff$ $u_{n+1}
> u_n$, $\forall n \ge n_0$
\end{definition}

\begin{exemple}

\end{exemple}

\begin{definition}[Suite décroissante]
$(u_n)$ décroissante $\iff$ $u_{n+1}
\le u_n$, $\forall n \ge n_0$
\end{definition}

\begin{exemple}

\end{exemple}

\begin{definition}[Suite strictement
décroissante]
$(u_n)$ strictement décroissante $\iff$ $u_{n+1}
< u_n$, $\forall n \ge n_0$
\end{definition}

\begin{exemple}

\end{exemple}

\begin{definition}[Suite monotone]
La suite $(u_n)$ est monotone si et
seulement si elle uniquement est croissante ou décroissante (sans
changer de sens de variation).
\end{definition}

\begin{exemple}

\end{exemple}

\begin{definition}[Suite constante]
$(u_n)$ constante $\iff$ $u_{n+1}
= u_n$, $\forall n \ge n_0$
\end{definition}

\begin{exemple}

\end{exemple}

\begin{comment}
Méthode : Calculer les premiers termes d'une suite Méthode :
Représenter graphiquement une suite dé�nie de manière explicite
Méthode : Représenter graphiquement une suite dé�nie par récurrence
Méthode : Montrer qu'une suite est bornée Méthode : Montrer qu'une
suite est arithmétique et donner sa forme explicite Méthode : Montrer
qu'une suite est géométrique et donner sa forme explicite Méthode :
Calculer une somme de termes consécutifs d'une suite Exercice :
Calculer les premiers termes d'une suite dé�nie de manière explicite
Exercice : Calculer les premiers termes d'une suite dé�nie par
récurrence Exercice : Montrer qu'une suite est bornée Exercice :
Déterminer si une suite est arithmétique Exercice : Calculer les
termes d'une suite arithmétique Exercice : Déterminer le premier terme
et la raison d'une suite arithmétique Exercice : Calculer la somme des
termes consécutifs d'une suite arithmétique Exercice : Déterminer si
une suite est géométrique Exercice : Calculer les termes d'une suite
géométrique Exercice : Déterminer le premier terme et la raison d'une
suite géométrique Exercice : Calculer la somme des termes consécutifs
d'une suite géométrique Exercice : Calculer une somme Problème :
Raison et premier terme d'une suite arithmétique à partir d'un système
Problème : Etudier deux suites imbriquées Problème : Utilisation d'une
suite géométrique dans une situation réelle Problème : Etudier une
suite géométrique et un taux d'intérêt
\end{comment}
\end{document}
