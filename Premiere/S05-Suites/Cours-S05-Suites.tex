
%% Font size %%
\documentclass[11pt]{article}

%% Load the custom package
\usepackage{Mathdoc}

%% Numéro de séquence %% Titre de la séquence %%
\renewcommand{\centerhead}{}

%% Spacing commands %%
\renewcommand{\baselinestretch}{1} \setlength{\parindent}{0pt}

\begin{document}

\section{Étude globale}

\subsection{Définitions}

\begin{definition}[Suite numérique]
Une suite numérique est une fonction de dans $\N$ dans $\R$.
\end{definition}

\begin{exemple}
La fonction définie pour tout entier naturel $n$ par $u(n)=2n+1$ est une suite.
\end{exemple}

\begin{notation}
\vspace{-0.5cm}
\begin{itemize}
\item On peut désigner la suite $u$ avec la notation $(u_n)$ (entre
parenthèses) ;
\item L'écriture $u_n$ (sans parenthèses) désigne le terme de rang $n$
de la suite $u$, c'est à dire $u(n)$ ;
\end{itemize}
\end{notation}

\begin{remarque}
Une suite $u$ peut être définie à partir d'un certain rang $u_0$, on
notera alors $(u_n)_{n \ge n_0}$ pour désigner la suite $u$. 
\end{remarque}

\begin{definition}[Modes de génération]
Il existe trois façon de définir une suite :
\begin{enumerate}
\item \textbf{Définition explicite :} \\
La suite $(u_n)$ est définie
directement par son terme général :
$$ u_n=f(n) $$
Avec $f$ une fonction dépendant de
$n$ définie sur $\N$ ; 
\item \textbf{Définition par récurrence :} \\
Soit $f$ une fonction définie sur
$\R$ et $a \in \R$, une suite $u_n$
peut être définie par récurrence par :
$$\begin{cases}
u_0 = a \\
u_{n+1} = f(u_n)
\end{cases}$$
\item \textbf{Définition implicite :} \\
La suite est définie par une
propriété géométrique, économique \dots au sein d'un
problème.
\end{enumerate}
\end{definition}

\begin{remarque}
Quel que soit le mode de définition
d'une suite, il se peut que celle-ci
ne soit définie qu'à partir d'un
certain rang $n_0 > 0$.
\end{remarque}

\begin{remarque}
On peut faire les analogies suivantes
entre les suites et les fonctions :

\begin{center}
\begin{tabular}{|c|c|c|}
\hline
Fonctions&&Suites \\ \hline
f&$\leftrightarrow$&u \\ \hline
x&$\leftrightarrow$&n \\ \hline
antécédent&$\leftrightarrow$&rang \\ \hline
image&$\leftrightarrow$&terme \\ \hline
f&$\leftrightarrow$&u \\ \hline
\end{tabular}
\end{center}
\end{remarque}

\subsection{Sens de variation}

\begin{remarque}
Dans la suite, on considère $(u_n)$
une suite définie sur $\N$ pour tout
$n \ge n_0$, avec $n_0 \ge 0$.
\end{remarque}

\begin{definition}[Suite croissante]
$(u_n)$ croissante $\iff$ $u_{n+1}
\ge u_n$, $\forall n \ge n_0$
\end{definition}

\begin{exemple}
Considérons $(u_n)$ définie par :
\[ \begin{cases}
u_0 = 12 \\
u_{n+1} = (u_n)^2 + u_n, \forall n
\in \N
\end{cases} \]
\end{exemple}


\begin{definition}[Suite décroissante]
$(u_n)$ décroissante $\iff$ $u_{n+1}
\le u_n$, $\forall n \ge n_0$
\end{definition}

\begin{definition}[Suite monotone]
La suite $(u_n)$ est monotone si et
seulement si elle uniquement est croissante ou décroissante (sans
changer de sens de variation).
\end{definition}

\begin{definition}[Suite constante]
$(u_n)$ constante $\iff$ $u_{n+1}
= u_n$, $\forall n \ge n_0$
\end{definition}

\subsection{Représentation graphique}

\begin{definition}
Dans un repère orthonormé direct du plan, la représentation graphique
d'une suite $u$ est l'ensemble des points ayant pour coordonnées
$(n;u_n)$ avec $n \in \N, n \leq n_0$ 
\end{definition}

\begin{exemple}
On considère la suite $(u_n)_{n\in\N}$ par : $u_n = n^2 - 1$  
Les premiers termes de la suite sont donnés dans le tableau suivant :
\begin{center}
\begin{tabular}{|c|c|c|c|c|c|c|}
\hline
\textbf{n} & 0 & 1 & 2 & 3 & 4 & 5 \\ \hline
\textbf{$u_n$} & -1 & 0 & 3 & 8 & 15 & 24 \\ \hline
\end{tabular}
\end{center}
On obtient la représentation graphique des premiers points de la suite
:
\begin{center}
\begin{tikzpicture}
    % Grille
    \draw[step=1,gray,opacity=0.5] (-2,-2) grid (6,9);

    % Axes
    \draw[->] (-2,0) -- (6.5,0) node[below] {$n$};
    \draw[->] (0,-2) -- (0,9) node[left] {$u_n$};

    % Points de la suite
    \foreach \n/\u in {0/-1, 1/0, 2/3, 3/8} {
        \draw[red,thick] (\n,\u) node[cross out, draw, thick, scale=0.6] {};
        \node[above] at (\n,\u) {$u_{\n}$};
    }

    % Noms des graduations
    \foreach \x in {-2,-1,0,1,2,3,4,5,6} {
        \draw (\x,0) -- (\x,-0.5) node[below] {\small $\x$};
    }
    \foreach \y in {-2,0,2,4,6,8} {
        \draw (0,\y) -- (-0.2,\y) node[left] {\small $\y$};
    }
\end{tikzpicture}
\end{center}
\end{exemple}

\subsection{Suites arithmétiques}

\begin{definition}
Une suite $(u_n)$ est une suite arithmétique s'il existe un nombre $r$
tel que :
\[ u_{n+1} = u_n + r \]
Le nombre $r$ est appelé raison de la suite.
\end{definition}

\begin{exemple}
Considérons $(u_n)$ définie par :
\[ \begin{cases}
u_0 = 1 \\
u_{n+1} = u_n - 2, \forall n
\in \N
\end{cases} \]
$(u_n)$ est une suite arithmétique de raison $r=-2$
\end{exemple}

\begin{propriete}
Soit $(u_n)$ une suite arithmétique de raison $r$ ;
\begin{itemize}
\item Si $r > 0$ la suite est strictement croissante ;
\item Si $r < 0$ la suite est strictement décroissante ;
\item Si $r = 0$ la suite est constante.
\end{itemize}
\end{propriete}


\begin{theoreme}[Terme général d'une suite arithmétique]
Soit une suite arithmétique de raison $r$ définie à partir d'un
certain rang $n_0$.\\
Pour tout entier supérieur ou égal à $n_0$ son terme général est égal à :
\[ u_n = u_{n_0}+(n-n_0)r\]
En particulier si $n_0=0$, on a : 
\[ u_n = u_0 + n \times r\]
\end{theoreme}

\begin{exemple}
On considère $(u_n)$  une suite arithmétique de raison $r=2$ et de
premier terme $u_5=18$.
\[ \forall n \ge 5, u_n=18+(n-5)2 = 2n + 8\]
\end{exemple}

\begin{propriete}
Soit $(u_n)$ une suite arithmétique. Les points de sa représentation graphique sont alignés.
\end{propriete}

\subsection{Suites géométriques}

\begin{definition}
Une suite $(u_n)$ est une suite géométrique s'il existe un nombre $q$
tel que :
\[ u_{n+1} = u_n \times q \]
Le nombre $q$ est appelé raison de la suite.
\end{definition}

\begin{exemple}
Considérons $(u_n)$ définie par :
\[ \begin{cases}
u_0 = 1 \\
u_{n+1} = u_n \times 2, \forall n
\in \N
\end{cases} \]
$(u_n)$ est une suite arithmétique de raison $r=2$
\end{exemple}

\begin{propriete}
Soit $(u_n)$ une suite arithmétique de raison $q>0$ ;
\begin{itemize}
\item Si $q > 1$ la suite est strictement croissante ;
\item Si $0 < q < 1$ la suite est strictement décroissante ;
\item Si $q = 1$ la suite est constante.
\end{itemize}
\end{propriete}


\begin{theoreme}[Terme général d'une suite géométrique]
Soit une suite géométrique de raison $q$ définie à partir d'un
certain rang $n_0$.\\
Pour tout entier supérieur ou égal à $n_0$ son terme général est égal à :
\[ u_n = u_{n_0} \times q^{(n-n_0)}\]
En particulier si $n_0=0$, on a : 
\[ u_n = u_0 \times q^n\]
\end{theoreme}

\begin{exemple}
On considère $(u_n)$ une suite géométrique de raison $q=5$ et de
premier terme $u_3=10$.
\[ \forall n \ge 5, u_n=10 \times 5^{(n-3)} = \dfrac{10}{125} \times 5^n\]
\end{exemple}

\begin{propriete}
Soit $(u_n)$ une suite géométrique de raison $q \neq 1$. Les points de
sa représentation graphique ne sont pas alignés.
\end{propriete}


\subsection{Suites  arithmético-géométriques}

\begin{definition}
Une suite $(u_n)$ est une suite arithmético-géométrique s'il existe
deux nombres $a$ et $b$ tels que :
\[ u_{n+1} = a \times u_n + b \]
\end{definition}

\begin{theoreme}[Terme général d'une suite arithmético-géométrique]
Soit $(u_n)$ une suite arithmético-géométrique avec $a \neq 1$ définie à
partir d'un certain rang $n_0$.\\
En posant : $r = \dfrac{b}{1-a}$
Pour tout entier supérieur ou égal à $n_0$ son terme général est égal
à :
\[ u_n = a^{n-n_0}(u_{n_0}-r)+r\]
En particulier si $n_0=0$, on a : 
\[ u_n = a^n(u_0-r)+r\]
\end{theoreme}


\subsection{Séries}

\begin{definition}
Étant donné une suite de terme général $u_n$, étudier la série de
terme général $u_n $ c'est étudier la suite de terme général $S_n$
définie par :
\[ S_n = u_0 + u_1 + u_2 + \ldots + u_n = \sum_{k=0}^{n} u_k \]
\end{definition}

\begin{exemple}
Soit $(u_n)$ la suite arithmétique de terme général $u_n=n$
(\textit{i.e. $r=1$ ; $u_0=0$}), on a donc : 
\[ S_n = 1 + 2 + 3 + \ldots + n = \sum_{k=0}^{n} k \]
\end{exemple}

\begin{theoreme}[Somme des termes d'une suite arithmétique.]
Soit $(u_n)$ une suite arithmétique :
\[ S_n = u_0 + u_1 + u_2 + \ldots + u_n = \sum_{k=0}^{n} u_k = \dfrac{(n+1)(u_0+u_n)}{2} \]
\end{theoreme}

\begin{theoreme}[Somme des termes d'une suites géométrique.]
Soit $(u_n)$ une suite géométrique :
\[ S_n = u_0 + u_1 + u_2 + \ldots + u_n = \sum_{k=0}^{n} u_k = u_0
\times \dfrac{1-q^{n+1}}{1-q} \]
Plus généralement, pour tout entier naturel $k<n$ :
\[ S_n = u_0 + u_1 + u_2 + \ldots + u_n = \sum_{k=0}^{n} u_k = u_k
\times \dfrac{1-q^{n-k+1}}{1-q} \]
\end{theoreme}


\end{document}
