%% Font size %%
\documentclass[11pt]{article}

%% Load the custom package
\usepackage{Mathdoc}

%% Numéro de séquence %% Titre de la séquence %%
\renewcommand{\centerhead}{}

%% Spacing commands %%
\renewcommand{\baselinestretch}{1} \setlength{\parindent}{0pt}

\newcommand{\vecteur}[3]{
$\vec{#1} = \begin{pmatrix} #2 \\ #3 \end{pmatrix}$}

\begin{document}

\begin{exercice}[1][Calculer les produits scalaires suivants]
\underline{Rappels} : Soit $ \vec{u} = \begin{pmatrix} x \\ y \end{pmatrix} $ et $\vec{v} = \begin{pmatrix} x' \\ y' \end{pmatrix}$ alors $\vec{u}
. \vec{v} = xx' + yy'$

\begin{multicols}{2}
\begin{enumerate}
\item $\vec{a} = \begin{pmatrix} 3 \\ 7 \end{pmatrix}$ et
$\vec{b} = \begin{pmatrix} 11 \\ 3 \end{pmatrix}$
\item \vecteur{c}{4}{9} et \vecteur{d}{-1}{2}
\item \vecteur{d}{-2}{8} et \vecteur{d}{-3}{6}
\item \vecteur{a}{3}{5} et \vecteur{b}{2}{-4}  
\item \vecteur{u}{-1}{7} et \vecteur{v}{6}{3}  
\item \vecteur{x}{0}{-2} et \vecteur{y}{5}{4}
\end{enumerate}
\end{multicols}
\end{exercice}

\begin{exercice}
Soit $A=(0;0)$, $B=(1;2)$, $C=(-3;9)$ et $D=(-1;-8)$ \\
Calculer $\vec{AB}.\vec{AC}$, $\vec{AB}.\vec{DC}$ et $\vec{BC}.\vec{AD}$
\end{exercice}

\begin{exercice}
Soit $IJK$ un triangle rectangle-isocèle en I tel que $IK=5cm$
\begin{enumerate}
\item En utilisant le théorème de Pythagore déterminer
$\|\vec{JK}\|$
\item on se place dans le repère orthonormé direct de
$(I,\vec{IJ},\vec{IK})$, calculer les coordonnées des vecteurs
$\vec{IJ}$, $\vec{IK}$ et $\vec{IK}$
\item Calculer $\vec{IJ}.\vec{JK}$ et $\vec{IJ}.\vec{IK}$, que remarque-t-on ?
\end{enumerate}
\end{exercice}

\end{document}

% Local Variables:
% gptel-model: deepseek-chat
% gptel--backend-name: "DeepSeek"
% gptel--bounds: ((290 . 339) (451 . 498) (505 . 553) (640 . 686) (692 . 739) (833 . 975))
% End:
