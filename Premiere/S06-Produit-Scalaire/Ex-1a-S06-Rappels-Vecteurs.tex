%% Font size %%
\documentclass[11pt]{article}

%% Load the custom package
\usepackage{Mathdoc}

%% Numéro de séquence %% Titre de la séquence %%
\renewcommand{\centerhead}{Chap. 6 : Produit Scalaire - Rappels sur
  les vecteurs}

%% Spacing commands %%
\renewcommand{\baselinestretch}{1} \setlength{\parindent}{0pt}

\begin{document}

\section{Coordonnées vectorielles}

\begin{exercice}[1]

\begin{enumerate}[itemsep=1em]
\item Dans un repère orthonormé $\big(O\,;\,\vec \imath,\,\vec \jmath\big)$, on donne les points suivants : $E\left(0\,;\,3\right)$ et $F\left(-4\,;\,5\right)$.\\Déterminer les coordonnées du vecteur $\overrightarrow{EF}$.
\item Dans un repère orthonormé $\big(O\,;\,\vec \imath,\,\vec \jmath\big)$, on donne les points suivants : $U\left(3\,;\,0\right)$ et $V\left(3\,;\,-2\right)$.\\Déterminer les coordonnées du vecteur $\overrightarrow{UV}$.
\end{enumerate}
\end{exercice}

\begin{exercice}[1]

\begin{enumerate}[itemsep=1em]
\item Dans un repère orthonormé $\big(O\,;\,\vec \imath,\,\vec \jmath\big)$, on donne les points suivants : $B\left(\dfrac{3}{5}\,;\,-\dfrac{1}{2}\right)$ et $C\left(\dfrac{8}{5}\,;\,2\right)$.\\Déterminer les coordonnées du vecteur $\overrightarrow{BC}$.
\item Dans un repère orthonormé $\big(O\,;\,\vec \imath,\,\vec \jmath\big)$, on donne les points suivants : $K\left(\dfrac{4}{5}\,;\,-\dfrac{10}{3}\right)$ et $L\left(-\dfrac{1}{5}\,;\,-2\right)$.\\Déterminer les coordonnées du vecteur $\overrightarrow{KL}$.
\end{enumerate}
\end{exercice}

\section{Opérations sur les vecteurs}

\begin{exercice}[1]

\begin{enumerate}[itemsep=1em]
\item Dans un repère orthonormé $\big(O ; \vec \imath,\vec \jmath\big)$, on donne les vecteurs suivants : $\vec{u}\begin{pmatrix}-3\\2\end{pmatrix}$ et $\vec{v}\begin{pmatrix}1\\5\end{pmatrix}$.\\Déterminer les coordonnées du vecteur $\overrightarrow{w}=\overrightarrow{u}+\overrightarrow{v}$.
\item Dans un repère orthonormé $\big(O ; \vec \imath,\vec \jmath\big)$, on donne les vecteurs suivants : $\vec{u}\begin{pmatrix}1\\-3\end{pmatrix}$ et $\vec{v}\begin{pmatrix}-9\\1\end{pmatrix}$.\\Déterminer les coordonnées du vecteur $\overrightarrow{w}=\overrightarrow{u}+\overrightarrow{v}$.
\end{enumerate}
\end{exercice}

\begin{exercice}[1]

\begin{enumerate}[itemsep=1em]
\item Dans un repère orthonormé $\big(O ; \vec \imath,\vec \jmath\big)$, on donne les vecteurs suivants : $\vec{u}\begin{pmatrix}\dfrac{2}{5}\\[0.7em]\dfrac{5}{6}\end{pmatrix}$ et $\vec{v}\begin{pmatrix}\dfrac{3}{2}\\[0.7em]-1\end{pmatrix}$.\\Déterminer les coordonnées du vecteur $\overrightarrow{w}=\overrightarrow{u}+\overrightarrow{v}$.
\item Dans un repère orthonormé $\big(O ; \vec \imath,\vec \jmath\big)$, on donne les vecteurs suivants : $\vec{u}\begin{pmatrix}\dfrac{5}{3}\\[0.7em]\dfrac{4}{5}\end{pmatrix}$ et $\vec{v}\begin{pmatrix}\dfrac{1}{6}\\[0.7em]1\end{pmatrix}$.\\Déterminer les coordonnées du vecteur $\overrightarrow{w}=\overrightarrow{u}+\overrightarrow{v}$.
\end{enumerate}
\end{exercice}

\begin{exercice}[1]
\begin{enumerate}[itemsep=1em]
\item Dans un repère orthonormé $\big(O ; \vec \imath,\vec \jmath\big)$, on donne les points suivants : $A\left(5\,;\,7\right)$, $B\left(-8\,;\,-9\right)$, $C\left(-5\,;\,-6\right)$ et $D\left(-4\,;\,1\right)$.\\Déterminer les coordonnées du vecteur $\overrightarrow{w}=\overrightarrow{AB}+\overrightarrow{CD}$.
\item Dans un repère orthonormé $\big(O ; \vec \imath,\vec \jmath\big)$, on donne les points suivants : $A\left(-6\,;\,-9\right)$, $B\left(1\,;\,2\right)$, $C\left(5\,;\,7\right)$ et $D\left(7\,;\,0\right)$.\\Déterminer les coordonnées du vecteur $\overrightarrow{w}=\overrightarrow{AB}+\overrightarrow{CD}$.
\end{enumerate}
\end{exercice}

\begin{exercice}[1]

\begin{enumerate}[itemsep=1em]
\item Dans un repère orthonormé $\big(O ; \vec \imath,\vec \jmath\big)$, on donne les vecteurs suivants : $\vec{u}\begin{pmatrix}-7\\7\end{pmatrix}$ et $\vec{v}\begin{pmatrix}3\\-7\end{pmatrix}$.\\Déterminer les coordonnées du vecteur $\overrightarrow{w}=\overrightarrow{u}-\overrightarrow{v}$.
\item Dans un repère orthonormé $\big(O ; \vec \imath,\vec \jmath\big)$, on donne les vecteurs suivants : $\vec{u}\begin{pmatrix}-3\\-3\end{pmatrix}$ et $\vec{v}\begin{pmatrix}4\\4\end{pmatrix}$.\\Déterminer les coordonnées du vecteur $\overrightarrow{w}=\overrightarrow{u}-\overrightarrow{v}$.
\end{enumerate}
\end{exercice}

\begin{comment}
\begin{exercice}[1]

\begin{enumerate}[itemsep=1em]
\item Dans un repère orthonormé $\big(O ; \vec \imath,\vec \jmath\big)$, on donne les vecteurs suivants : $\vec{u}\begin{pmatrix}\dfrac{1}{4}\\[0.7em]\dfrac{5}{2}\end{pmatrix}$ et $\vec{v}\begin{pmatrix}\dfrac{1}{2}\\[0.7em]-3\end{pmatrix}$.\\Déterminer les coordonnées du vecteur $\overrightarrow{w}=\overrightarrow{u}-\overrightarrow{v}$.
\item Dans un repère orthonormé $\big(O ; \vec \imath,\vec \jmath\big)$, on donne les vecteurs suivants : $\vec{u}\begin{pmatrix}\dfrac{4}{3}\\[0.7em]\dfrac{4}{5}\end{pmatrix}$ et $\vec{v}\begin{pmatrix}\dfrac{5}{6}\\[0.7em]-7\end{pmatrix}$.\\Déterminer les coordonnées du vecteur $\overrightarrow{w}=\overrightarrow{u}-\overrightarrow{v}$.
\end{enumerate}
\end{exercice}
\end{comment}
\begin{exercice}[1]

\begin{enumerate}[itemsep=1em]
\item Dans un repère orthonormé $\big(O ; \vec \imath,\vec \jmath\big)$, on donne les points suivants : $A\left(-2\,;\,1\right)$, $B\left(9\,;\,-4\right)$, $C\left(-2\,;\,1\right)$ et $D\left(2\,;\,6\right)$.\\Déterminer les coordonnées du vecteur $\overrightarrow{w}=\overrightarrow{AB}-\overrightarrow{CD}$.
\item Dans un repère orthonormé $\big(O ; \vec \imath,\vec \jmath\big)$, on donne les points suivants : $A\left(-2\,;\,-4\right)$, $B\left(1\,;\,-5\right)$, $C\left(-8\,;\,-2\right)$ et $D\left(9\,;\,6\right)$.\\Déterminer les coordonnées du vecteur $\overrightarrow{w}=\overrightarrow{AB}-\overrightarrow{CD}$.
\end{enumerate}
\end{exercice}

\begin{exercice}[1]

\begin{enumerate}[itemsep=1em]
\item Dans un repère orthonormé $\big(O ; \vec \imath,\vec \jmath\big)$, on donne les vecteurs suivants : $\vec{u}\begin{pmatrix}8\\0\end{pmatrix}$ et $\vec{v}\begin{pmatrix}2\\-1\end{pmatrix}$.\\Déterminer les coordonnées du vecteur $\overrightarrow{w}=\overrightarrow{u}+3\overrightarrow{v}$.
\item Dans un repère orthonormé $\big(O ; \vec \imath,\vec \jmath\big)$, on donne les vecteurs suivants : $\vec{u}\begin{pmatrix}3\\-1\end{pmatrix}$ et $\vec{v}\begin{pmatrix}-1\\-7\end{pmatrix}$.\\Déterminer les coordonnées du vecteur $\overrightarrow{w}=\overrightarrow{u}-5\overrightarrow{v}$.
\end{enumerate}
\end{exercice}

\begin{exercice}[1]

\begin{enumerate}[itemsep=1em]
\item Dans un repère orthonormé $\big(O ; \vec \imath,\vec \jmath\big)$, on donne les vecteurs suivants : $\vec{u}\begin{pmatrix}9\\[0.7em]-7\end{pmatrix}$ et $\vec{v}\begin{pmatrix}\dfrac{3}{2}\\[0.7em]-6\end{pmatrix}$.\\Déterminer les coordonnées du vecteur $\overrightarrow{w}=\overrightarrow{u}+\dfrac{1}{6}\overrightarrow{v}$.
\item Dans un repère orthonormé $\big(O ; \vec \imath,\vec \jmath\big)$, on donne les vecteurs suivants : $\vec{u}\begin{pmatrix}6\\[0.7em]7\end{pmatrix}$ et $\vec{v}\begin{pmatrix}\dfrac{1}{5}\\[0.7em]4\end{pmatrix}$.\\Déterminer les coordonnées du vecteur $\overrightarrow{w}=\overrightarrow{u}-\dfrac{5}{3}\overrightarrow{v}$.
\end{enumerate}
\end{exercice}

\begin{exercice}[1]

\begin{enumerate}[itemsep=1em]
\item Dans un repère orthonormé $\big(O ; \vec \imath,\vec \jmath\big)$, on donne les points suivants : $A\left(0\,;\,7\right)$, $B\left(1\,;\,-6\right)$, $C\left(5\,;\,-6\right)$ et $D\left(-3\,;\,6\right)$.\\Déterminer les coordonnées du vecteur $\overrightarrow{w}=\overrightarrow{AB}-6\overrightarrow{CD}$.
\item Dans un repère orthonormé $\big(O ; \vec \imath,\vec \jmath\big)$, on donne les points suivants : $A\left(6\,;\,8\right)$, $B\left(1\,;\,5\right)$, $C\left(9\,;\,-5\right)$ et $D\left(8\,;\,1\right)$.\\Déterminer les coordonnées du vecteur $\overrightarrow{w}=\overrightarrow{AB}-4\overrightarrow{CD}$.
\end{enumerate}
\end{exercice}

\section{Manipulations de vecteurs}

\begin{exercice}[1]

\begin{enumerate}[itemsep=1em]
\item Dans un repère orthonormé $\big(O ; \vec \imath,\vec \jmath\big)$, on donne le point $A(-6\,;\,-4)$ et le vecteur $\vec{u}\begin{pmatrix}-1\\6\end{pmatrix}$.\\Déterminer les coordonnées du point $B$ tel que $\overrightarrow{u}=\overrightarrow{AB}$.
\item Dans un repère orthonormé $\big(O ; \vec \imath,\vec \jmath\big)$, on donne le point $A(-2\,;\,-7)$ et le vecteur $\vec{u}\begin{pmatrix}5\\-8\end{pmatrix}$.\\Déterminer les coordonnées du point $B$ tel que $\overrightarrow{u}=\overrightarrow{BA}$.
\end{enumerate}
\end{exercice}

\begin{comment}
\begin{exercice}[1]

\begin{enumerate}[itemsep=1em]
\item Dans un repère orthonormé $\big(O ; \vec \imath,\vec \jmath\big)$, on donne les points $A(8\,;\,3)$, $C(-1\,;\,-8)$, $D(-3\,;\,7)$ et le vecteur $\vec{u}\begin{pmatrix}3\\3\end{pmatrix}$.\\Déterminer les coordonnées du point $B$ tel que $\overrightarrow{u}=\overrightarrow{AB}+\overrightarrow{CD}$.
\item Dans un repère orthonormé $\big(O ; \vec \imath,\vec \jmath\big)$, on donne les points $A(-5\,;\,-4)$, $C(-6\,;\,-7)$, $D(4\,;\,1)$ et le vecteur $\vec{u}\begin{pmatrix}-8\\-6\end{pmatrix}$.\\Déterminer les coordonnées du point $B$ tel que $\overrightarrow{u}=\overrightarrow{AB}+\overrightarrow{CD}$.
\end{enumerate}
\end{exercice}
\end{comment}
\begin{exercice}[1]

\begin{enumerate}[itemsep=1em]
\item Dans un repère orthonormé $\big(O ; \vec \imath,\vec \jmath\big)$, on donne les points suivants : $A(1;-3)$, $C(-8;-5)$ et $D(2;2)$.\\Déterminer les coordonnées du point $B$ tel que $\overrightarrow{CD}=4\overrightarrow{AB}$.
\item Dans un repère orthonormé $\big(O ; \vec \imath,\vec \jmath\big)$, on donne les points suivants : $A(6;1)$, $C(-2;3)$ et $D(5;7)$.\\Déterminer les coordonnées du point $B$ tel que $\overrightarrow{CD}=9\overrightarrow{AB}$.
\end{enumerate}
\end{exercice}

\begin{exercice}[1]

\begin{enumerate}[itemsep=1em]
\item Dans un repère $(O\;;\;\vec{i}\;;\;\vec{j})$, on considère les points $H(-7\;;\;10)$, $I(7\;;\;-9)$ et $J(-3\;;\;1)$.\\Déterminer les coordonnées du point $K$ tel que $HIJK$ soit un parallélogramme.
\item Dans un repère $(O\;;\;\vec{i}\;;\;\vec{j})$, on considère les points $F(10\;;\;4)$, $G(-3\;;\;0)$ et $H(-5\;;\;-4)$.\\Déterminer les coordonnées du point $I$ tel que $FGHI$ soit un parallélogramme.
\end{enumerate}
\end{exercice}

\begin{exercice}[1]

\begin{enumerate}[itemsep=1em]
\item Dans un repère $(O\;;\;\vec{i}\;;\;\vec{j})$, on considère les points $A(5\;;\;8)$, $B(-4\;;\;-1)$ et $C(2\;;\;-10)$.\\Déterminer les coordonnées du point $D$ tel que $ABCD$ soit un parallélogramme.
\item Dans un repère $(O\;;\;\vec{i}\;;\;\vec{j})$, on considère les points $J(-10\;;\;-5)$, $K(-2\;;\;-3)$ et $L(6\;;\;4)$.\\Déterminer les coordonnées du point $M$ tel que $JKLM$ soit un parallélogramme.
\end{enumerate}
\end{exercice}

\section{Problèmes}



\end{document}
