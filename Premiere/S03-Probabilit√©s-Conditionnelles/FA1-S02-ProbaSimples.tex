%% Font size %%
\documentclass[11pt]{article}

%% Load the custom package
\usepackage{Mathdoc}

%% Numéro de séquence %% Titre de la séquence %%
\renewcommand{\centerhead}{Probabilités conditionnellses, tableau}

%% Spacing commands %%
\renewcommand{\baselinestretch}{1}
\setlength{\parindent}{0pt}

\begin{document}

\begin{exercice}
  Une banque propose deux types de placements : un \textbf{placement à
    court terme} et un \textbf{placement à long terme}. Parmi ses
  clients, certains réalisent un bénéfice, d'autres subissent une
  perte. Le tableau ci-dessous présente la répartition des clients
  selon le type de placement et le résultat obtenu.

  \[
    \begin{array}{|c|c|c|c|}
      \hline
      & \textbf{Bénéfice} & \textbf{Perte} & \textbf{Total} \\
      \hline
      \textbf{Placement court terme}  & 60 & 40 & 100 \\ \hline
      \textbf{Placement long terme}   & 90 & 10 & 100 \\ \hline
      \textbf{Total}                  & 150 & 50 & 200 \\
      \hline
    \end{array}
  \]

  \begin{multicols}{2}
    \begin{enumerate}
    \item Quelle est la probabilité qu'un client choisi au hasard ait
      réalisé un bénéfice ?
    \item Quelle est la probabilité qu'un client choisi au hasard ait
      réalisé une perte ?
    \item Quelle est la probabilité qu'un client ait choisi un
      placement long terme ?
    \item Quelle est la probabilité qu'un client ait réalisé un
      bénéfice sachant qu'il a opté pour un placement à court terme ?
    \item Quelle est la probabilité qu'un client ait subi une perte
      sachant qu'il a opté pour un placement long terme ?
    \item Quelle est la probabilité qu'un client ait opté pour un
      placement à court terme sachant qu'il a réalisé une perte ?
    \item Quelle est la probabilité qu'un client ait opté pour un
      placement long terme sachant qu'il a réalisé un bénéfice ?
    \item Quelle est la probabilité qu'un client ait subi une perte
      sachant qu'il a opté pour un placement court terme ?
    \end{enumerate}
  \end{multicols}
\end{exercice}

\begin{exercice}
  Une entreprise réalise des ventes à deux types de clients : des \textbf{grandes entreprises} et des \textbf{PME}. À la fin de l'année, elle constate que certains clients sont en situation de paiement à jour, tandis que d'autres sont en retard de paiement. Le tableau ci-dessous présente la répartition des clients selon leur type et leur situation de paiement.

\[
\begin{array}{|c|c|c|c|}
\hline
                          & \textbf{Paiement à jour} & \textbf{Paiement en retard} & \textbf{Total} \\
\hline
\textbf{Grandes entreprises}  & 600 & 150 & 750 \\ \hline
\textbf{PME}                  & 850 & 400 & 1250 \\
\hline
\textbf{Total}                & 1450 & 550 & 2000 \\
\hline
\end{array}
\]

\begin{multicols}{2}
  \begin{enumerate}
  \item Quelle est la probabilité qu'un client choisi au hasard soit
    en situation de paiement à jour ?
  \item Quelle est la probabilité qu'un client choisi au hasard soit
    en situation de paiement en retard ?
  \item Quelle est la probabilité qu'un client choisi soit une grande
    entreprise ?
  \item Quelle est la probabilité qu'un client soit en situation de
    paiement à jour sachant que c'est une grande entreprise ?
\item Quelle est la probabilité qu'un client soit en situation de
  paiement en retard sachant que c'est une PME ?
\item Quelle est la probabilité qu'un client soit une grande
  entreprise sachant qu'il est en situation de paiement en retard ?
\item Quelle est la probabilité qu'un client soit une grande
  entreprise sachant qu'il est en situation de paiement à jour ?
\item Quelle est la probabilité qu'un client soit une PME sachant
  qu'il est en situation de paiement à jour ?
\end{enumerate}
\end{multicols}
\end{exercice}

\end{document}

%%% Local Variables:
%%% mode: LaTeX
%%% TeX-master: t
%%% End:
