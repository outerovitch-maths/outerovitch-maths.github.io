%% Font size %%
\documentclass[11pt]{article}

%% Load the custom package
\usepackage{Mathdoc}

%% Numéro de séquence %% Titre de la séquence %%
\renewcommand{\centerhead}{Chap. 2 : Dérivation (Pentes et limites)}

%% Spacing commands %%
\renewcommand{\baselinestretch}{1}
\setlength{\parindent}{0pt}

\begin{document}
\phantom{0}
\vspace{-1.5cm}
\section{Pentes et fonctions affines}

\begin{multicols}{2}
\begin{exercice}
\textbf{Soit $\big(O ; \vec \imath,\vec \jmath\big)$ un repère orthogonal.  Déterminer, s'il existe et en l'expliquant, le coefficient directeur de la droite $(AB)$.}
\begin{enumerate}[itemsep=1em]
	\item \begin{minipage}[t]{\linewidth} Avec $A(0;-3)$ et $B(4;-3)$.  \end{minipage}
	\item \begin{minipage}[t]{\linewidth} Avec $A(4;4)$ et $B(4;2)$.  \end{minipage}
	\item \begin{minipage}[t]{\linewidth} Avec $A(-3;4)$ et $B(-1;-3)$.  \end{minipage}
	\item \begin{minipage}[t]{\linewidth} Avec $A(-3;-1)$ et $B(-1;-4)$.  \end{minipage}
	\item \begin{minipage}[t]{\linewidth} Avec $A(-3;-3)$ et $B(5;5)$.  \end{minipage}
\end{enumerate}
\end{exercice}

\begin{exercice}
\begin{enumerate}[itemsep=1em]
	\item Soit $f$ une fonction affine telle que $f(x)=ax+3$ et $f(1)=1$.\\    Donner la valeur de $a$.
    
	\item Soit $f$ une fonction affine telle que $f(x)=ax+1$ et $f(3)=-8$.\\    Donner la valeur de $a$.
    
	\item Soit $f$ une fonction affine telle que $f(x)=ax-2$ et $f(5)=-12$.\\      Donner la valeur de $a$.
    
	\item Soit $f$ une fonction affine telle que $f(x)=ax+4$ et $f(5)=12$.\\    Donner la valeur de $a$.
    
	\item Soit $f$ une fonction affine telle que $f(x)=ax-6$ et $f(5)=-13$.\\    Donner la valeur de $a$.
\end{enumerate}
\end{exercice}
\end{multicols}

\vspace{-1cm}
\section{Calculs de limites}

\begin{multicols}{2}
  \begin{exercice}
    \textbf{Calculer les limites en $0$ suivantes.}
    \begin{enumerate}
    \item \(\lim\limits_{x \to 0} (2x + 3)\)
    \item \(\lim\limits_{x \to 0} (x^2 + 5)\)
    \item \(\lim\limits_{x \to 0} \frac{1}{x + 1}\)
    \item \(\lim\limits_{x \to 0} (4x^3 - 2x)\)
    \item \(\lim\limits_{x \to 0} \sin(x)\)
    \item \(\lim\limits_{x \to 0} \frac{3x + 2}{x + 4}\)
    \item \(\lim\limits_{x \to 0} \cos(x)\)
    \item \(\lim\limits_{x \to 0} \frac{x}{x + 2}\)
    \item \(\lim\limits_{x \to 0} \frac{2x + 1}{x^2 + 1}\)
    \item \(\lim\limits_{x \to 0} \frac{5x^2}{x + 3}\)
    \end{enumerate}
  \end{exercice}

\begin{exercice}
  \textbf{Calculer les limites suivantes.}
  \begin{enumerate}
  \item \(\lim\limits_{x \to 2} (3x + 1)\)
  \item \(\lim\limits_{x \to -1} (x^2 + 4)\)
  \item \(\lim\limits_{x \to 3} \frac{2x + 5}{x + 1}\)
  \item \(\lim\limits_{x \to 4} (x^3 - 3x)\)
  \item \(\lim\limits_{x \to -2} \sin(x)\)
  \item \(\lim\limits_{x \to 1} \frac{4x^2 + 2x}{x + 3}\)
  \item \(\lim\limits_{x \to 0} \frac{x^2}{x + 2}\)
  \item \(\lim\limits_{x \to 5} \frac{3x - 2}{2x + 1}\)
  \item \(\lim\limits_{x \to -3} \frac{2x^2 + x}{x + 4}\)
  \item \(\lim\limits_{x \to 6} \frac{x^2 - 1}{x - 2}\)
  \end{enumerate}
\end{exercice}
\end{multicols}
\end{document}
