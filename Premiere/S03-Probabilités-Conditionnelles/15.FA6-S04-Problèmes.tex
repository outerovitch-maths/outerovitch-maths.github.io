%% Font size %%
\documentclass[11pt]{article}

%% Load the custom package
\usepackage{Mathdoc}

%% Numéro de séquence %% Titre de la séquence %%
\renewcommand{\centerhead}{Probabilités conditionnelles : Problèmes}

%% Spacing commands %%
\renewcommand{\baselinestretch}{1}
\setlength{\parindent}{0pt}

\begin{document}

\begin{exercice}
Dans une usine, on utilise conjointement deux machines $M_1$ et $M_2$ pour
fabriquer des pièces cylindriques en série. Pour une période donnée,
leurs probabilités de tomber en panne sont respectivement $0,01$ et
$0,008$. De plus la probabilité de l’événement "la machine $M_2$ est en
panne sachant que $M_1$ est en panne" est égale à $0,4$.
\begin{enumerate}
\item Quelle est la probabilité d’avoir les deux machines en panne
  au même moment ? 
\item Quelle est la probabilité d’avoir au moins
  une machine qui fonctionne ?
\end{enumerate}
\end{exercice}

\begin{exercice}
Au CSMH, 40\% de garçons et 15\% des filles mesurent plus de
$1,80m$. De plus, 60\% des élèves sont des filles. \\
Sachant qu’un élève, choisi au hasard, mesure plus de $1,80m$, quelle
est la probabilité que ce soit une fille ? 
\end{exercice}

\begin{exercice}
Dans une population $\Omega$, deux maladies $M_1$ et $M_2$ sont
présentes respectivement chez 10\% et 20\%. \\
On suppose que le nombre de ceux qui souffrent des deux maladies est
négligeable. On entreprend un dépistage systématique des maladies
$M_1$ et $M_2$.\\
 Pour cela, on applique un test qui réagit sur 90\% des malades de
 $M_1$, sur 70\% des malades $M_2$, et sur 10\% des individus qui
 n’ont aucune de ces deux affections.
 \begin{enumerate}
 \item Quand on choisit au hasard un individu $\omega$ dans $\Omega$, quelle est
   la probabilité pour que le test réagisse ? 
 \item Sachant que pour un
   individu $\omega$, le test a réagi, donner les probabitités : 
   \begin{itemize}
   \item pour que le test ait réagi à cause de la maladie $M_1$ ;
   \item pour que le test ait réagi à cause de la maladie $M_2$ ; 
   \item pour que le test ait réagi alors que l’individu n’est infecté par
     qu’aucune des deux maladies $M_1$ et $M_2$.
   \end{itemize}
 \end{enumerate}
\end{exercice}

\end{document}

%%% Local Variables:
%%% mode: LaTeX
%%% TeX-master: t
%%% End:
