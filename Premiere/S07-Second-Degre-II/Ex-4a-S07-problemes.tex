%% Font size %%
\documentclass[11pt]{article}

%% Load the custom package
\usepackage{Mathdoc}

%% Numéro de séquence %% Titre de la séquence %%
\renewcommand{\centerhead}{}

%% Spacing commands %%
\renewcommand{\baselinestretch}{1} \setlength{\parindent}{0pt}

\begin{document}


\begin{exercice}
Le sponsor d'une célèbre équipe de football vend des maillots aux couleurs de l'équipe. Il achète ces maillots à son fournisseur au prix de 10\,€ l'unité. Il en vend 200 unités par jour à 50\,€ chacune. Au vu des bons résultats de l'équipe au cours de la saison, il souhaite augmenter le prix de ces maillots. Il réalise une étude de marché qui montre que toute augmentation de 1\,€ fera baisser la vente de deux maillots par jour. On admet que le nombre de maillots achetés au fournisseur est le même que le nombre de maillots vendus, quel que soit le prix.

Déterminer le nouveau prix que devra fixer le sponsor pour réaliser un bénéfice maximum.
\end{exercice}

\begin{exercice}
Grégoire, 10 ans, veut délimiter dans son jardin un enclos rectangulaire pour son lapin nain. Son père lui donne 18 m de grillage.

1. Déterminer les dimensions de cet enclos rectangulaire qui donnent une aire maximale.

2. Quelle est alors la valeur de cette aire ?
\end{exercice}



\begin{exercice}
Un propriétaire paie pour son appartement une taxe foncière de 750 € en 2016. En 2017, cette taxe foncière augmente de $t\%$. L'année suivante, elle augmente de $(t+5)\%$. Le propriétaire paie à présent 834,30 €.

Calculer la valeur de $t$.
\end{exercice}


\begin{exercice}
\section*{Équation bicarrée}
On cherche à résoudre l'équation (E) suivante.
\[ x^4 - 11x^2 + 18 = 0 \]

1. On pose \( X = x^2 \).
Montrer que \( x \) est solution de (E) si et seulement si \( X \) est solution de l'équation (E') du second degré :
\[ X^2 - 11X + 18 = 0 \]

2. Résoudre (E').

3. En déduire les solutions de (E).
\end{exercice}

\end{document}

% Local Variables:
% gptel-model: deepseek-chat
% gptel--backend-name: "DeepSeek"
% gptel--bounds: ((response (281 962) (964 1262) (1264 1538) (1541 1900)))
% End:
