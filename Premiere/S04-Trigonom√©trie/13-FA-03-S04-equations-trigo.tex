%% Font size %%
\documentclass[11pt]{article}

%% Load the custom package
\usepackage{Mathdoc}

%% Numéro de séquence %% Titre de la séquence %%
\renewcommand{\centerhead}{Chap. 4 : Trigonométrie - Résoudre une équation trigonométrique }

%% Spacing commands %%
\renewcommand{\baselinestretch}{1} \setlength{\parindent}{0pt}


\begin{document}

\section{Exercice résolu}

\begin{exercice}[0][Résolution d'équation trigonométrique]
\underline{Énoncé :} \\

\textit{Résoudre l’équation \(\sin(x) = -\frac{1}{2}\) :  
\begin{enumerate}
    \item Sur l’intervalle \([-\pi ; \pi[\).
    \item En écrivant l’ensemble des solutions dans \(\mathbb{R}\).
\end{enumerate}
}

\underline{Correction :} \\

\textbf{1. Solutions dans \([-\pi ; \pi[\)}

Nous cherchons les angles \(x \in [-\pi ; \pi[\) tels que \(\sin(x) = -\frac{1}{2}\).

On sait que $sin(\dfrac{\pi}{6})=\dfrac{1}{2}$

Or la fonction sinus est impaire donc
$sin(-\dfrac{\pi}{6})=-\dfrac{1}{2}$

Et on sait que $sin(x)=sin(\pi+x)$

Donc, $sin(\dfrac{-\pi}{6})=sin(\pi+\dfrac{-\pi}{6})=sin(-\dfrac{5\pi}{6})$

\textbf{Conclusion} : Les solutions dans \([-\pi ; \pi[\) sont :  
\[
x \in \left\{-\frac{5\pi}{6} ; -\frac{\pi}{6}\right\}.
\]

\textbf{2. Ensemble solutions dans \(\mathbb{R}\)}

La fonction \(\sin(x)\) est \(2\pi\)-périodique.  
\begin{itemize}
    \item Si \(x_1 = -\frac{5\pi}{6}\) est une solution, alors toutes les solutions associées sont données par :  
    \[
    x = -\frac{5\pi}{6} + 2k\pi, \quad k \in \mathbb{Z}.
    \]
    \item Si \(x_2 = -\frac{\pi}{6}\) est une solution, alors toutes les solutions associées sont données par :  
    \[
    x = -\frac{\pi}{6} + 2k\pi, \quad k \in \mathbb{Z}.
    \]
\end{itemize}

\textbf{Conclusion} : L’ensemble des solutions dans \(\mathbb{R}\) est :  
\[
x \in \left\{-\frac{5\pi}{6} + 2k\pi ; -\frac{\pi}{6} + 2k\pi \mid k \in \mathbb{Z} \right\}.
\]

\end{exercice}

\newpage

\section{Exercices d'application}

\begin{exercice}[2][Résolution d'équation trigonométrique]
Résoudre les équations suivante sur l’intervalle \([-\pi ; \pi[\).
\begin{multicols}{2}
\begin{enumerate}
\item $\cos(x) = -\frac{1}{2}$
\item $\sin(x) = \frac{\sqrt{3}}{2}$
\item $\sin(x) = -\frac{\sqrt{2}}{2}$
\item $\cos(x) = \frac{\sqrt{3}}{2}$
\end{enumerate}
\end{multicols}
\end{exercice}

\begin{exercice}[2][Résolution d'équation trigonométrique]
Résoudre les équations trigonométriques suivantes sur l'intervalle
\([-\pi ; \pi[\) puis en déduire les solutions générales sur $\R$.
\begin{multicols}{2}
\begin{enumerate}
    \item \(\sin(x) = \frac{\sqrt{2}}{2}\),
    \item \(\cos(x) = -\frac{\sqrt{3}}{2}\),
    \item \(\sin(x) = -\frac{1}{2}\),
    \item \(\cos(x) = \frac{1}{2}\).
\end{enumerate}
\end{multicols}
\end{exercice}

\begin{exercice}[3][Résolution d'équation trigonométrique]
Résoudre les équations trigonométriques suivantes sur l'intervalle
\([-\pi ; \pi[\) puis en déduire les solutions générales sur $\R$.
\begin{multicols}{2}
\begin{enumerate}
    \item \(2\sin(x) = 1\),
    \item \(2\cos(x) = -\sqrt{3}\),
    \item \(\sqrt{2}\sin(x) = -2\),
    \item \(10\cos(x) = 5\).
\end{enumerate}
\end{multicols}
\end{exercice}

\begin{exercice}[4][Résolution d'équation trigonométrique]
Résoudre les équations trigonométriques suivantes sur l'intervalle
\([-\pi ; \pi[\) puis en déduire les solutions générales sur $\R$.
\begin{multicols}{2}
\begin{enumerate}
    \item \(\cos(2x) = \frac{1}{2}\),
    \item \(\sin(4x) = -\frac{\sqrt{2}}{2}\),
    \item \(\cos(-3x) = -\frac{\sqrt{3}}{2}\),
    \item \(\sin(10x) = \frac{1}{2}\).
\end{enumerate}
\end{multicols}
\end{exercice}

\begin{exercice}[4][Résolution d'équation trigonométrique]
Résoudre les équations trigonométriques suivantes sur l'intervalle
\([-\pi ; \pi[\) puis en déduire les solutions générales sur $\R$.
\begin{multicols}{2}
\begin{enumerate}
    \item $2 \sin \left( x + \dfrac{\pi}{4} \right) - 1 = 0$
    \item $1 - \sqrt{2} \cos \left(  \dfrac{\pi}{3}-x \right) = 0$
    \item $\sin\left( 2x-\dfrac{\pi}{4}\right) = \dfrac{1}{2}$
    \item $\cos\left( 2x-\dfrac{\pi}{3}\right)  = \dfrac{1}{2}$
\end{enumerate}
\end{multicols}
\end{exercice}


\end{document}
