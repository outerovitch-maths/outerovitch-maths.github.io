%% Font size %%
\documentclass[11pt]{article}

%% Load the custom package
\usepackage{Mathdoc}

%% Numéro de séquence %% Titre de la séquence %%
\renewcommand{\centerhead}{}

%% Spacing commands %%
\renewcommand{\baselinestretch}{1}
\setlength{\parindent}{0pt}

\begin{document}


\begin{correction}[Exercice 1.1]
Calculer le nombre dérivé de $f(x)=x^2+x$ en $a=1$.\\
   \begin{tabular}{p{0.3cm}l!{=}l}
      &  $f'(a)$ & $\lim_{h\to0}\dfrac{f(a+h)-f(a)}{h}$ \\
   \end{tabular} \\

Or, $a=1$ donc :

   \begin{tabular}{p{0.3cm}l!{=}l}
      &  $f'(1)$ & $\lim\limits_{h\to0}\dfrac{f(1+h)-f(1)}{h}$ \\
   \end{tabular} \\

Calculons :\\
\begin{tabular}{p{0.3cm}l!{=}l}
      &  $f(1+h)$ & $(1+h)^2+1+h$\\
      &  & $1^2+2h+h^2+1+h$\\
      &  & $2+3h+h^2$\\
\end{tabular} \\


Calculons :\\
\begin{tabular}{p{0.3cm}l!{=}l}
      &  $f(1)$ & $(1)^2+1$\\
      &  & $2$\\
\end{tabular} \\

Donc :

   \begin{tabular}{p{0.3cm}l!{=}l}
      &  $f'(1)$ & $\lim\limits_{h\to0}\dfrac{2+3h+h^2-2}{h}$ \\
      &   & $\lim\limits_{h\to0}\dfrac{3h+h^2}{h}$ \\
      &   & $\lim\limits_{h\to0}3+h$ \\
      &  $f'(1)$ & $3$ \\
   \end{tabular} \\

\end{correction}

\newpage

\begin{exercice}
  
\end{exercice}

\end{document}
