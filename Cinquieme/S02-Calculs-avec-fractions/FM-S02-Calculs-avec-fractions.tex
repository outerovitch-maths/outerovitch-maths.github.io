%% Font size %%
\documentclass[11pt]{article}

%% Load the custom package
\usepackage{Mathdoc}

%% Numéro de séquence %% Titre de la séquence %%
\renewcommand{\centerhead}{Séquence 2 : Fractions}


%% Spacing commands %%
\renewcommand{\baselinestretch}{1}
\setlength{\parindent}{0pt}

\begin{document}

\section{Notions de fractions}

\subsection{Rappels}

\begin{vocabulaire}
Dans une fraction $\dfrac{n}{d}$, on appelle numérateur le nombre du
haut et dénominateur le nombre du bas.
\end{vocabulaire}

\begin{exemple}
  \begin{enu}
  \item $\dfrac{10}{5}$ est une fraction ;
  \item $\dfrac{354}{542}$ \textit{idem} ;
  \item $\dfrac{0,45}{6,2}$ n'est pas une fraction.
  \end{enu}
\end{exemple}

\subsection{Forme décimale}

\begin{propriete}
Certaines fractions possèdent une forme décimale.
\end{propriete}

\begin{exemple}
$\dfrac{5}{4}=1 + \dfrac{1}{4} = 1,25$
\end{exemple}

\begin{propriete}
D'autres fractions ne possèdent pas de formes décimales.
\end{propriete}

\begin{exemple}
$\dfrac{1}{3}=0,333333\ldots$ \\
$\dfrac{1}{3} \approx 0,33$
\end{exemple}

\begin{propriete}
Par contre, tous nombres décimal peut s'écrire sous la forme d'une
fraction. 
\end{propriete}

\begin{exemple}
  \begin{enu}
  \item $2,8=\dfrac{28}{10}$
  \item $0,03=\dfrac{3}{100}$
  \item $34,43=\dfrac{3~443}{100}$
  \end{enu}
\end{exemple}

\begin{remarque}
Sous forme décimale, certaines fractions sont des entiers.
\end{remarque}

\begin{exemple}
  \begin{enu}
  \item $\dfrac{9}{9}=1$
  \item $\dfrac{10}{2}=5$
  \item $\dfrac{19}{1}=19$
  \end{enu}
\end{exemple}

\section{Fractions égales}

\subsection{Multiplication}
\begin{propriete}
Dans une fraction, on peut multiplier le numérateur et
le dénominateur par un même nombre, la fraction reste égale.
\end{propriete}

\begin{exemple}
$\dfrac{5}{9}=\dfrac{5 \times 10}{9 \times 10}=\dfrac{50}{90}$
\end{exemple}

\subsection{Division}
\begin{propriete}
Dans une fraction, on peut diviser le numérateur et
le dénominateur par un même nombre, la fraction reste égale.
\end{propriete}

\begin{exemple}
$\dfrac{18}{42}=\dfrac{18 \div 6}{42 \div 6}=\dfrac{3}{7}$
\end{exemple}

\subsection{Simplification}
\begin{definition}
Simplifier une fraction c'est trouver un nombre par lequel diviser le
numérateur et le dénominateur pour qu'ils soient le plus petit possible.
\end{definition}

\begin{exemple}
$\dfrac{25}{35}=\dfrac{5 \times 5}{7 \div 5}=\dfrac{5}{7}$ \\ \\
\textit{(On dit que l'on simplifie par 5, on ''enlève'' un facteur 5
au numérateur et au dénominateur.)}
\end{exemple}

\section{Comparaisons et encadrements}

\subsection{Comparer}
\begin{definition}
\textbf{Comparer} deux nombres (entiers ou décimaux), c'est dire s'ils sont supérieurs, inferieurs ou égaux.
\end{definition}

\begin{remarque}
On utilise les symboles : $>$ ; $<$ et $=$
\end{remarque}
\newpage
\begin{exercice}[0][Comparer deux nombres décimaux]
\textbf{Comparer 14,59 et 14,521}
\begin{enu}
\item Comparer les parties entières : 14=14
\item Comparer le chiffre des dixièmes : 5=5
\item Comparer le chiffre des centièmes : 9>2
\item Conclure : 9>2 donc 14,59>14,521
\end{enu}
\end{exercice}

\subsection{Ranger}
\begin{definition}
  Ranger des nombres :
  \begin{enu}
  \item Dans l'ordre \textbf{croissant}, c'est les ranger du plus
    petit au plus grand ;
  \item Dans l'ordre \textbf{décroissant}, c'est les ranger du plus
    grand au plus petit.
  \end{enu}
\end{definition}

\begin{exemple}
Ranger les nombres suivants dans l'ordre croissant :\\
  \hfill $85\,349$ \hfill  ;  \hfill $85\,943$ \hfill ;  \hfill $8\,767$  \hfill ;  \hfill $83\,549$  \hfill ; \hfill  $85\,967$ \hfill  ;  \hfill $166\,083$ \hfill \\
  Les nombres rangés dans l'ordre {\bfseries croissant} : $8\,767$ $<$
  $83\,549$ $<$ $85\,349$ $<$ $85\,943$ $<$ $85\,967$ $<$ $166\,083$
\end{exemple}

\subsection{Encadrer}

\begin{definition}
\textbf{Encadrer} un nombre (entier ou décimal), c'est trouver un
  nombre inférieur et un nombre supérieur à ce nombre.
\end{definition}


\begin{exercice}[0][Encadrer un nombre décimal]
  \begin{enumerate}
  \item \textit{Donner un encadrement à l'unité de 72,46}
    \begin{enu}
    \item 72 est inférieur à 72,46
    \item 73 supérieur à 72,46
    \item Un encadrement à l'unité est donc 72<72,46<73
    \end{enu}
    
  \item \textit{Donner un encadrement au dixième ($\frac{1}{10}$) de
    103,7}
    \begin{enu}
    \item 103,8 est supérieur à 103,7
    \item 103,6 est inférieur à 103,7
    \item Un encadrement au dixième ($\frac{1}{10}$) est donc
      103,6<103,7<103,8
    \end{enu}
  \end{enumerate}
\end{exercice}
\newpage
\subsection{Intercaler}
\begin{definition}
  \textbf{Intercaler} un nombre entre deux nombres décimaux donnés,
  c'est trouver un nombre compris entre ces deux nombres.
\end{definition}

\begin{propriete}
On peut toujours intercaler un nombre décimal entre deux nombres décimaux.
\end{propriete}

\begin{exemple}[Compléter avec un nombre décimal.]
  \begin{enu}
  \item $19{,}99<19{,}995<20$
  \item $9{,}201<9{,}3<9{,}7$
  \end{enu}
\end{exemple}

\end{document}
