%% Font size %%
\documentclass[11pt]{article}

%% Load the custom package
\usepackage{Mathdoc}

%% Numéro de séquence %% Titre de la séquence %%
\renewcommand{\centerhead}{Séquence 2 : Fractions}


%% Spacing commands %%
\renewcommand{\baselinestretch}{1}
\setlength{\parindent}{0pt}

\begin{document}

\section{Notions de fractions}

\subsection{Rappels}

\begin{vocabulaire}
Dans une fraction $\dfrac{n}{d}$, on appelle numérateur le nombre du
haut et dénominateur le nombre du bas.
\end{vocabulaire}

\begin{exemple}
  \begin{enu}
  \item $\dfrac{10}{5}$ est une fraction ;
  \item $\dfrac{354}{542}$ \textit{idem} ;
  \item $\dfrac{0,45}{6,2}$ n'est pas une fraction.
  \end{enu}
\end{exemple}

\subsection{Forme décimale}

\begin{propriete}
Certaines fractions possèdent une forme décimale.
\end{propriete}

\begin{exemple}
$\dfrac{5}{4}=1 + \dfrac{1}{4} = 1,25$
\end{exemple}

\begin{propriete}
D'autres fractions ne possèdent pas de formes décimales.
\end{propriete}

\begin{exemple}
$\dfrac{1}{3}=0,333333\ldots$ \\
$\dfrac{1}{3} \approx 0,33$
\end{exemple}

\begin{propriete}
Par contre, tous nombres décimal peut s'écrire sous la forme d'une
fraction. 
\end{propriete}

\begin{exemple}
  \begin{enu}
  \item $2,8=\dfrac{28}{10}$
  \item $0,03=\dfrac{3}{100}$
  \item $34,43=\dfrac{3~443}{100}$
  \end{enu}
\end{exemple}

\begin{remarque}
Sous forme décimale, certaines fractions sont des entiers.
\end{remarque}

\begin{exemple}
  \begin{enu}
  \item $\dfrac{9}{9}=1$
  \item $\dfrac{10}{2}=5$
  \item $\dfrac{19}{1}=19$
  \end{enu}
\end{exemple}
\end{document}
