%% Font size %%
\documentclass[11pt]{article}

%% Load the custom package
\usepackage{Mathdoc}

%% Numéro de séquence %% Titre de la séquence %%
\renewcommand{\centerhead}{}

%% Spacing commands %%
\renewcommand{\baselinestretch}{1} \setlength{\parindent}{0pt}


\begin{document}

\section{Définition et vocabulaire}

\begin{definition}
On appelle \emph{expression littérale} une suite d'opérations dans laquelle
figurent des lettres, représentant des nombres inconnus.
\end{definition}

\begin{exemple}
Le périmètre d'un rectangle de longueur $L$ et de largeur $l$ est
donnée par :
\[ L \times 2 + l \times 2 \quad \text{ou} \quad (L + l) \times 2 \]
\end{exemple}

\begin{remarque}
Pour alléger les écritures, les mathématiciens ont décidé de ne plus
écrire les signes opératoires $\times$ des expressions littérales : devant
et derrière une parenthèse, devant et derrière une lettre.
\end{remarque}

\begin{exemple}
\begin{align*}
3 \times x - 7 &= 3x - 7 \\
(2 - 9 \times y) \times t &= (2 - 9y)t \\
1 \times x &= 1x = x \\
2 \times x \times y \times 7 &= 2 \times 7 \times x \times y = 14xy \\
3 \times a \times a - 7 \times b &= 3a - 7b
\end{align*}
\end{exemple}

\section{Distributivité}
\subsection{Simple distributivité}

\begin{propriete}
Soient $a$, $b$, $k$ trois nombres relatifs. On a alors :
\[ k \times (a + b) = k \times a + k \times b \]
\end{propriete}

\begin{exemple}
\begin{align*}
3 \times (x - 9) &= 3 \times x - 3 \times 9 = 3x - 27 \\
(-2y) \times (4 - 7x) &= (-2y) \times 4 - (-2y) \times 7x \\
&= -8y - (-14xy) = -8y + 14xy
\end{align*}
\end{exemple}

\begin{remarque}
Dans les exemples précédents, on a développé des expressions entre
parenthèses. Le procédé inverse s'appelle \emph{factoriser}.
\end{remarque}

\begin{exemple}
\begin{align*}
5x + 35 &= 5 \times x + 5 \times 7 = 5 \times (x + 7) \\
18 - 6x &= 6 \times 3 - 6 \times x = 6 \times (3 - x)
\end{align*}
\end{exemple}

\subsection{Double distributivité}

\begin{propriete}
Soient $a$, $b$, $c$, $d$ quatre nombres relatifs. On a alors :
\[ (a + b)(c + d) = a \times c + a \times d + b \times c + b \times d = ac + ad + bc +
bd \]
\end{propriete}

\begin{exemple}
\begin{align*}
(3 + x)(2x - 7) &= (3 + x)(2x + (-7)) \\
&= 3 \times 2x + 3 \times (-7) + x \times 2x + x \times (-7) \\
&= 6x - 21 + 2x^2 - 7x \\
&= 2x^2 - x - 21
\end{align*}
\end{exemple}

\end{document}

% Local Variables:
% gptel-model: deepseek-chat
% gptel--backend-name: "DeepSeek"
% gptel--bounds: nil
% End:
