%% Font size %%
\documentclass[11pt]{article}

%% Load the custom package
\usepackage{Mathdoc}

%% Numéro de séquence %% Titre de la séquence %%
\renewcommand{\centerhead}{Chap. 8 - Les proportionnalités : Devoir Bilan}

%% Spacing commands %%
\renewcommand{\baselinestretch}{1} \setlength{\parindent}{0pt}

\begin{document}

\phantom{0}
\vspace{-1.5cm}

\begin{center}
\duree{1 heures} 
\total{25 points}
\coefficient{1}
\calculatrice{1}
\brouillon
\recherche 
\copieseparee{1}
\end{center}

\begin{exercicedevoir}[4][Reconnaître une situation]
Dire si les tableaux suivants sont de tableaux de proportionnalité. Justifier.
\begin{multicols}{2}
\begin{enumerate}[itemsep=2em]
\item $\renewcommand{\arraystretch}{1.5}
\begin{array}{|c|c|c|}
\hline
\cellcolor{white} \phantom{000}7\phantom{000} & \cellcolor{white} \phantom{000}8\phantom{000} & \cellcolor{white} \phantom{000}9\phantom{000}\\
\hline
12 & 13 & 14\\
\hline
\end{array}
\renewcommand{\arraystretch}{1}$

\item $\renewcommand{\arraystretch}{1.5}
\begin{array}{|c|c|c|}
\hline
\cellcolor{white} \phantom{000}16\phantom{000} & \cellcolor{white} \phantom{000}10\phantom{000} & \cellcolor{white} \phantom{000}14\phantom{000}\\
\hline
8 & 5 & 7\\
\hline
\end{array}
\renewcommand{\arraystretch}{1.5}$

\item $\renewcommand{\arraystretch}{1.5}
\begin{array}{|c|c|c|}
\hline
\cellcolor{white} \phantom{000}9\phantom{000} & \cellcolor{white} \phantom{000}8\phantom{000} & \cellcolor{white} \phantom{000}6\phantom{000}\\
\hline
7 & 6 & 4\\
\hline
\end{array}
\renewcommand{\arraystretch}{1.5}$

\item $\renewcommand{\arraystretch}{1.5}
\begin{array}{|c|c|c|}
\hline
\cellcolor{white} \phantom{000}5{,}5\phantom{000} & \cellcolor{white} \phantom{000}6\phantom{000} & \cellcolor{white} \phantom{000}1\phantom{000}\\
\hline
38{,}5 & 42 & 7\\
\hline
\end{array}
\renewcommand{\arraystretch}{1.5}$
\end{enumerate}
\end{multicols}

\end{exercicedevoir}

\begin{exercicedevoir}[2][Reconnaître une situation]
\begin{enumerate}[itemsep=1.4em]
\item Teresa achète dans un magasin de bricolage des outils.\\Elle  repart avec 5 outils pour $12$\,\,€.\\ Arthur achète quant à lui, au même endroit 15 outils pour $36$\,\,€.\\Le prix des outils est-il proportionnel à la quantité achetée  ?
\item Yasmine vient d'avoir 9 ans cette année. Son père David vient de fêter  son 42ème anniversaire.\\L'âge de son père est-il proportionnel à l'âge de Yasmine ?
\end{enumerate}
\end{exercicedevoir}

\begin{exercicedevoir}[4][Résoudre un problème simple]
\begin{enumerate}[itemsep=1em]
\item \textbf {a.}  Dalila a repéré, dans un magasin de bricolage, des ampoules qui l'intéressent. Elle lit que 3 ampoules coûtent $11$\,\,€. Elle veut en acheter 6.\\ Combien va-t-elle dépenser\,? \\\textbf {b.}  Bernard veut lui aussi acheter ces ampoules. Il dispose de $44$\,\,€.\\ Combien peut-il en acheter\,?\\
\item \textbf {a.}  Sur une carte sur laquelle 3 cm représente 8\,km dans la réalité,
Nadia mesure son trajet et elle trouve une distance de 6 cm. \\À quelle distance cela correspond dans la réalité\,? \\ \textbf {b.}  Deux villes sont distantes de 40\,km. Quelle distance va-t-on mesurer sur la carte entre ces deux villes\,?
\end{enumerate}
\end{exercicedevoir}

\begin{exercicedevoir}[6][Compléter un tableau]
Avec la méthode de votre choix, déterminer la valeur manquante dans les tableaux de
proportionnalité suivants. Justifier.
\begin{multicols}{2}
\begin{enumerate}
\item 
$\renewcommand{\arraystretch}{1.5}
\begin{array}{|c|c|c|c|} \hline
\text{Quantité A} & \phantom{000}12\phantom{000} & \phantom{000}\ldots\phantom{000} & \phantom{000}9\phantom{000} \\
\hline
\text{Quantité B} & 8 & 6 & 6 \\ 
\hline
\end{array}$

\item 
$\renewcommand{\arraystretch}{1.5}
\begin{array}{|c|c|c|c|}
\hline
\text{Quantité A} & \phantom{000}15\phantom{000} & \phantom{000}10\phantom{000} & \ldots \\ 
\hline
\text{Quantité B} & 3 & 2 & \phantom{000}5\phantom{000} \\ 
\hline
\end{array}$

\item 
$\renewcommand{\arraystretch}{1.5}
\begin{array}{|c|c|c|c|}
\hline
\text{Quantité A} & \ldots & 14 & 21 \\ 
\hline
\text{Quantité B} & \phantom{000}5\phantom{000} & \phantom{000}10\phantom{000} & \phantom{000}15\phantom{000} \\ 
\hline
\end{array}$

\item 
$\renewcommand{\arraystretch}{1.5}
\begin{array}{|c|c|c|c|}
\hline
\text{Quantité A} & \phantom{000}18\phantom{000} & \phantom{000}12\phantom{000} & \phantom{000}6\phantom{000} \\ 
\hline
\text{Quantité B} & 9 & \ldots & 3 \\ 
\hline
\end{array}$

\item 
$\renewcommand{\arraystretch}{1.5}
\begin{array}{|c|c|c|c|}
\hline
\text{Quantité A} & \phantom{000}25\phantom{000} & \phantom{000}15\phantom{000} & \phantom{000}5\phantom{000} \\ 
\hline
\text{Quantité B} & \ldots & 6 & 2 \\ 
\hline
\end{array}$

\item 
$\renewcommand{\arraystretch}{1.5}
\begin{array}{|c|c|c|c|}
\hline
\text{Quantité A} & \phantom{000}16\phantom{000} & \phantom{000}8\phantom{000} & \phantom{000}4\phantom{000} \\ 
\hline
\text{Quantité B} & 12 & 6 & \ldots \\ 
\hline
\end{array}$
\end{enumerate}
\end{multicols}



\end{exercicedevoir}

\begin{exercicedevoir}[2][Résoudre un problème complexe]
\begin{enumerate}
\item Nadia doit acheter du carrelage. \\Sur la notice, il est indiqué de prévoir 100 carreaux pour 4 m$^2$. \\ Combien doit-elle en acheter pour une surface de 3 m$^2$ ?
\item Un cycliste parcourt en moyenne $129,6$ km en 6 heures.
\\ Quelle distance va-t-il parcourir, à la même vitesse, en 13 heures ?
\end{enumerate}
\end{exercicedevoir}

\begin{exercicedevoir}[3][Résoudre un problème de vitesse]
\begin{enumerate} 
\item Elsa met 1\,h\,45\,min pour aller jusqu'à sa location de vacances qui est à une distance de 157{,}5 km. Déterminer sa vitesse moyenne.
\item Si Benjamin roule à 110 km/h, combien de temps lui faudra-t-il  pour aller dans la maison de ses parents qui est à une distance de 170{,}5 km ?
\end{enumerate}
\end{exercicedevoir}

\begin{exercicedevoir}[4][Problème à prise d'initiative]
Voici la recette du far breton pour 6 personnes :
\begin{itemize}
    \item Dans un grand saladier, mélanger 4 œufs avec 120g de sucre jusqu'à blanchiment ;
    \item Ajouter 180g de farine petit à petit en évitant les grumeaux ;
    \item Incorporer 1L lait progressivement puis laisser reposer 1h;
    \item Ajouter 16 pruneaux ;
    \item Verser dans un moule beurré et enfourner à 180°C pendant 45 minutes.
\end{itemize}
Marie souhaite adapter cette recette pour 10 personnes, calculer les
quantités nécessaires.
\end{exercicedevoir}

\nonewpage
\end{document}

%%% Local Variables:
%%% mode: LaTeX
%%% TeX-master: t
%%% TeX-master: t
%%% gptel-model: deepseek-chat
%%% gptel--backend-name: "DeepSeek"
%%% gptel--bounds: ((response (3342 3351) (3384 3526) (3559 3703) (3736 3882) (3915 4059) (4092 4236) (4269 4403) (5343 5381) (5390 5720)))
%%% End:
