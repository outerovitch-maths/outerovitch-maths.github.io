\input{~/Documents/Cours/.parametre/parametre-fm/parametre-fm.tex}

\begin{document}

\entete{11}{1}{Proportionnalité}

\colonnesep{0}

\section{Multiplications à trous et division}
\propriete{Compléter une multiplication à trou revient à calculer le résultat d'une division.}

\exemple{Si je dois calculer $45 \div 5$, il est plus facile de chercher : $5 \times ?=45$. Ici, la réponse est 45.}

\exemple{$60 \div 10 = 6$ car $6 \times 10 = 60$}


\section{Situations de proportionnalité}
\definition{Un tableau est un tableau de proportionnalité lorsque l'on passe d'une ligne à l'autre en multipliant toujours par le même nombre.}

\vocabulaire{Ce nombre est appelé le coefficient de proportionnalité.}

\vocabulaire{On dira que les deux grandeurs, correspondant à chaque ligne, sont proportionnelles.}

\exemple{
Une station-essence vend du sans-plomb 98 à 2 € le litre. \\
La quantité d’essence et le prix sont donc proportionnels. \\
On a le tableau de proportionnalité : \\

\begin{tabular}{|l|c|c|c|c|}
\hline
Quantité (en L) & 1 &	5 & 8 &	10  \\ \hline
Prix (en €) & 2 & 10 & 16 &	20  \\ \hline
\end{tabular}}

\methode{ 
\textbf{(Vérifier si un tableau est de proportionalité)} \\
Dire si les tableaux suivants sont bien des tableaux de proportionalités

\mc{
\cntr{\begin{tabular}{|l|c|c|c|c|}
\hline
Quantité A & 7 & 2 & 4 & 3  \\ \hline
Quantité B & 49 & 14 & 28 & 21 \\ \hline
\end{tabular}}

Calculons :

$49 \div 7 = 7$ \hfill 
$14 \div 2 = 7$ \hfill \hfill  \\
$28 \div 4 = 7$ \hfill 
$21 \div 3 = 7$ \hfill \hfill \\

Les coefficients sont égaux, les quantités sont bien proportionnelles.

\cntr{\begin{tabular}{|l|c|c|c|c|}
\hline
Quantité A & 6 & 4 & 10 & 7  \\ \hline
Quantité B & 18 & 8 & 20 & 21 \\ \hline
\end{tabular}}

Calculons :

$18 \div 6 = 3$ \hfill 
$8 \div 4 = 2$ \hfill \hfill  \\
$20 \div 10 = 2$ \hfill 
$21 \div 7 = 3$ \hfill \hfill  \\

Les coefficients ne sont pas égaux, les quantités ne sont pas proportionnelles.}}

\newpage

\section{Quatrième proportionnelle, passage à l'unité}

On suppose nos deux quantités sont proportionnelles. \\
Pour compléter le tableau suivant on va passer par le calcul d'une unité de notre quantité A.

\mc{
\begin{tabular}{|l|c|c|c|}
\hline
Quantité A & 7 & 2 & 1 \\ \hline
Quantité B & 35 & $x$ & \phantom{000} \\ \hline
\end{tabular}

\columnbreak

$35 \div 7 = 5$ \\
Donc 1 unité correspond à 5 \\
$2 \times 5 = x$ 
c.a.d. $x = 10$ }

On à donc : 
\cntr{
\begin{tabular}{|l|c|c|c|}
\hline
Quantité A & 7 & 2 & 1 \\ \hline
Quantité B & 35 & 10 & 5  \\ \hline
\end{tabular}
}

\section{Quatrième proportionnelle, linéarité}

On suppose nos deux quantités sont proportionnelles. \\
Pour compléter le tableau suivant on va utiliser les propiétés de linéarité du taleau de proportionalité.

\mc{
\cntr{
\begin{tabular}{|l|c|c|c|c|}
\hline
Quantité A & 7 & 2 & 9 & 14 \\ \hline
Quantité B & 63 & 18 & $x$ & $y$  \\ \hline
\end{tabular}
}
\columnbreak

$7 + 2 = 9$  et $7 \times 2 = 14$\\
De plus, $63 + 18 = 81$ et $61 \times 2 = 126$ \\
Donc $63 + 18 = x$ et $61 \times 2 = y $ \\
c.a.d. $x = 81$  et $y = 126$ }

On à donc : 
\cntr{
\begin{tabular}{|l|c|c|c|c|}
\hline
Quantité A & 7 & 2 & 9 & 14 \\ \hline
Quantité B & 63 & 18 & 81 & 126  \\ \hline
\end{tabular}
}

\section{Résolution de problèmes}

\textit{Énoncé :}
\enu{
\item Léa lit sur sa recette de mousse au chocolat pour 9 personnes qu'il faut 270 g de chocolat. \\
Elle veut adapter sa recette pour 11 personnes.\\
Quelle masse de chocolat doit-elle prévoir ?
\item Elsa a repéré, à l'épicerie, des melons qui l'intéressent.\\
Elle lit que $9$ melons coûtent $27$ €. Elle veut en acheter $11$.\\
Combien va-t-elle dépenser ?
\item Un piéton parcourt en moyenne $21$ km en 7 heures.
\\
Quelle distance va-t-il parcourir, à la même vitesse, en 11 heures ?
}


\end{document}
