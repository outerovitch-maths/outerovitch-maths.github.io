%%CLASS%%
\documentclass[french,12pt]{article}
%%PACKAGES%%
\usepackage[T1]{fontenc}
\usepackage[none]{hyphenat}
\usepackage[utf8]{inputenc}
\usepackage{multicol, caption}
\usepackage{tabularx}
\usepackage{graphicx}
\renewcommand{\arraystretch}{1.5}
%\setlength{\arrayrulewidth}{0,5mm}
%\usepackage[french]{babel}
\usepackage[left=2cm,right=2cm,top=2cm,bottom=2cm]{geometry}
\usepackage{mathtools, bm}
\usepackage{enumitem}
\usepackage{amsmath}
\usepackage{amssymb}
\usepackage{pstricks}
\usepackage{titlesec}
\usepackage{xlop}
\usepackage[pdftex=true,colorlinks=true,linkcolor=black,citecolor=black,filecolor=black,urlcolor=black,bookmarks=true,bookmarksopen=false,bookmarksnumbered=false,bookmarksopenlevel=0,pdfstartview=FitH,pdftitle={},pdfauthor={ColinOUTEROVITCH,bookmarks=false}]{hyperref}
\newenvironment{Figure}
  {\par\medskip\noindent\minipage{\linewidth}}
  {\endminipage\par\medskip}
%usepackage{biblatex}
%\addbibresource{biblio.bib}
\usepackage{fancyhdr}
\usepackage{pgfplots}
\pgfplotsset{compat=1.15}
\usepackage{mathrsfs}
\usetikzlibrary{arrows}

%%PAGE STYLE%%
\pagestyle{fancy}
\usepackage{tikz}
\fancyhf{}

%%NEW COMMAND%%



\newcommand*\circled[1]{\tikz[baseline=(char.base)]{
            \node[shape=circle,draw,inner sep=2pt] (char) {#1};}}
            
\renewcommand{\baselinestretch}{1}
\titleformat{\section}
{\normalfont\Large\bfseries}{\thesection}{1em}{-~~}

\newcommand{\mc}[1]{\begin{multicols}{2}#1\end{multicols}}

\newcommand{\enu}[1]{\begin{enumerate}[label=(\alph*)]#1\end{enumerate}}

\newcommand{\itmz}[1]{\begin{itemize}[label=\textbullet]#1\end{itemize}}

\newcommand{\cntr}[1]{\begin{center}#1\end{center}}

\newcommand{\dtf}{\makebox[\linewidth]{\dotfill}}

\newcommand{\exercice}{\section{}}

\newcommand{\mkb}[2]{\makebox[#1]{#2}}

\newcommand{\mybox}[1]{\begin{tabular}{|l|}
\hline
#1 \\
\hline
\end{tabular}}

\newcommand{\mkbdtf}[1]{\makebox[#1cm]{\dotfill}}

\newcommand{\mkblw}[1]{\makebox[\linewidth]{#1}}

\newcommand{\myrule}[1]{\rule[2mm]{#1cm}{.1pt}}

\newcommand{\myfigure}[1]{\begin{Figure}
\centering
\includegraphics[width=\linewidth]{images/#1}
\end{Figure}}

\newcommand{\entete}[1]{\lhead{Collège La Vallée - \textit{Mathématiques}}   \rhead{2021/2022}
\renewcommand{\headrulewidth}{0.5pt}
\cfoot{\circled{\thepage}}
\chead{\textbf{#1}}}

\newcommand{\titre}[2]{\cntr{
\begin{LARGE}
\myrule{#2} \textsc{#1} \myrule{#2}
\end{LARGE}}}

\newcommand{\colonnesep}[1]{\setlength{\columnseprule}{#1pt}}

\titleformat{\section}
{\normalfont\Large\bfseries}{Exercice~\thesection}{1em}{}

\setlength{\columnseprule}{1pt}
\setlength{\parindent}{0pt}

\newcommand{\myfig}[2]{\begin{Figure}
\centering
\includegraphics[width=#1]{images/#2}
\end{Figure}}


\begin{document}


\entete{11}{8}{Proportionnalité, Linéarité (1)}

\colonnesep{0}

\exercice \corrige \\
\textbf{Compléter le tableau suivant par linéarité.}
\mc{
\begin{tabular}{|c|c|c|c|}
\hline
Quantité A & 7 & 6 & 13 \\ \hline
Quantité B & 21 & 18 & \kern1cm \\ \hline
\end{tabular}

\columnbreak

\phantom{0} \\
\circled{1} On remarque que $7 + 6 = 13$  \\
\circled{2} On ajoute donc $21 + 18 = 39$
}

\exercice \diff[1] \\
\textbf{Compléter le tableau suivant par linéarité.}
\mc{
\begin{tabular}{|c|c|c|c|}
\hline
Quantité A & 9 & 8 & 17 \\ \hline
Quantité B & 36 & 32 & \kern1cm \\ \hline
\end{tabular}

\columnbreak

\phantom{0} \\
\circled{1} \dotfill  \\
\circled{2} \dotfill
}

\exercice \diff[1] \\
\textbf{Compléter le tableau suivant par linéarité.}
\mc{
\begin{tabular}{|c|c|c|c|}
\hline
Quantité A & 7 & 18 & 11 \\ \hline
Quantité B & 28 & \kern1cm & 44 \\ \hline
\end{tabular}

\columnbreak

\phantom{0} \\
\circled{1} \dotfill  \\
\circled{2} \dotfill 
}


\exercice \diff[1] \\
\textbf{Compléter le tableau suivant par linéarité.}
\mc{
\begin{tabular}{|c|c|c|c|}
\hline
Quantité A & 16 & 9 & 7 \\ \hline
Quantité B & \kern1cm & 27 & 21 \\ \hline
\end{tabular}

\columnbreak

\phantom{0} \\
\circled{1} \dotfill  \\
\circled{2} \dotfill 
}

\exercice \diff[1] \\
\textbf{Compléter le tableau suivant par linéarité.}
\mc{
\begin{tabular}{|c|c|c|c|}
\hline
Quantité A & 21 & 11 & 10 \\ \hline
Quantité B & \kern1cm & 22 & 20 \\ \hline
\end{tabular}

\columnbreak


\circled{1} \dotfill  \\
\circled{2} \dotfill 
}

\newpage%%%%%%%%%%%%%%%%%%%%%%%%%%%%%%%%%%%%%%%%%%%%%%%%%%%%%%%%%%
\exercice \diff[2]  \\
\textbf{Compléter le tableau suivant par linéarité.}
\mc{
\begin{tabular}{|c|c|c|c|}
\hline
Quantité A & 21 & 11 & 10 \\ \hline
Quantité B & \kern1cm & 99 & 90 \\ \hline
\end{tabular}


\columnbreak

\phantom{0} \\
\circled{1} \dotfill  \\
\circled{2} \dotfill 
}

\exercice \diff[2] \\
\textbf{Compléter le tableau suivant par linéarité.}
\mc{
\begin{tabular}{|c|c|c|c|}
\hline
Quantité A & 7 & 18 & 11 \\ \hline
Quantité B & 28 & \kern1cm & 44 \\ \hline
\end{tabular}



\columnbreak

\phantom{0} \\
\circled{1} \dotfill  \\
\circled{2} \dotfill
}

\exercice \diff[2] \\
\textbf{Compléter le tableau suivant par linéarité.}
\mc{
\begin{tabular}{|c|c|c|c|}
\hline
Quantité A & 12 & 8 & 4 \\ \hline
Quantité B & 24 & \kern1cm & 8 \\ \hline
\end{tabular}


\columnbreak

\phantom{0} \\
\circled{1} \dotfill  \\
\circled{2} \dotfill 
}


\exercice \diff[2] \\
\textbf{Compléter le tableau suivant par linéarité.}
\mc{
\begin{tabular}{|c|c|c|c|}
\hline
Quantité A & 11 & 10 & 21 \\ \hline
Quantité B & \kern1cm & 110 & 231 \\ \hline
\end{tabular}



\columnbreak

\phantom{0} \\
\circled{1} \dotfill  \\
\circled{2} \dotfill 
}

\exercice \diff[2] \\
\textbf{Compléter le tableau suivant par linéarité.}
\mc{
\begin{tabular}{|c|c|c|c|}
\hline
Quantité A & 15 & 9 & 6 \\ \hline
Quantité B & 75 & \kern1cm & 30 \\ \hline
\end{tabular}



\columnbreak


\circled{1} \dotfill  \\
\circled{2} \dotfill 
}




\end{document}
