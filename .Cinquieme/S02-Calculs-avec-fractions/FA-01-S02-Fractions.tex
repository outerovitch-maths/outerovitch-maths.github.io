%% Font size %%
\documentclass[9pt]{article}

%% Load the custom package
\usepackage{Mathdoc}

%% Numéro de séquence %% Titre de la séquence %%
\renewcommand{\centerhead}{Chapitre 2 : Écritures fractionnaires}

%% Spacing commands %%
\renewcommand{\baselinestretch}{1}
\setlength{\parindent}{0pt}

\begin{document}

\begin{multicols}{2}
\begin{exercice}[0][Passer d'une écriture à une autre.]
\begin{multicols}{2}
\begin{enumerate}[itemsep=2em]
	\item \begin{minipage}[t]{\linewidth} Écrire $3{,}5$ sous la forme d'une fraction. \end{minipage}
	\item \begin{minipage}[t]{\linewidth} Écrire $0{,}3$ sous la forme d'une fraction. \end{minipage}
	\item \begin{minipage}[t]{\linewidth} Écrire $0{,}03$ sous la forme d'une fraction. \end{minipage}
	\item \begin{minipage}[t]{\linewidth} Écrire $1{,}4$ sous la forme d'une fraction. \end{minipage}
	\item \begin{minipage}[t]{\linewidth} Écrire $0{,}09$ sous la forme d'une fraction. \end{minipage}
	\item \begin{minipage}[t]{\linewidth} Écrire $0{,}6$ sous la forme d'une fraction. \end{minipage}
	\item \begin{minipage}[t]{\linewidth} Écrire $3{,}17$ sous la forme d'une fraction. \end{minipage}
	\item \begin{minipage}[t]{\linewidth} Écrire $1{,}2$ sous la forme d'une fraction. \end{minipage}
	\item \begin{minipage}[t]{\linewidth} Écrire $2{,}5$ sous la forme d'une fraction. \end{minipage}
	\item \begin{minipage}[t]{\linewidth} Écrire $0{,}273$ sous la forme d'une fraction. \end{minipage}
\end{enumerate}
\end{multicols}
\end{exercice}

\begin{exercice}[0][Passer d'une écriture à une autre.]
\begin{multicols}{2}
\begin{enumerate}[itemsep=2em]
	\item \begin{minipage}[t]{\linewidth} Donner l'écriture décimale de $\dfrac{557}{1000}$. \end{minipage}
	\item \begin{minipage}[t]{\linewidth} Donner l'écriture décimale de $\dfrac{9}{100}$. \end{minipage}
	\item \begin{minipage}[t]{\linewidth} Donner l'écriture décimale de $\dfrac{239}{10}$. \end{minipage}
	\item \begin{minipage}[t]{\linewidth} Donner l'écriture décimale de $\dfrac{31}{10}$. \end{minipage}
	\item \begin{minipage}[t]{\linewidth} Donner l'écriture décimale de $\dfrac{27}{1000}$. \end{minipage}
	\item \begin{minipage}[t]{\linewidth} Donner l'écriture décimale de $\dfrac{37}{10}$. \end{minipage}
	\item \begin{minipage}[t]{\linewidth} Donner l'écriture décimale de $\dfrac{7}{1000}$. \end{minipage}
	\item \begin{minipage}[t]{\linewidth} Donner l'écriture décimale de $\dfrac{7}{100}$. \end{minipage}
	\item \begin{minipage}[t]{\linewidth} Donner l'écriture décimale de $\dfrac{577}{100}$. \end{minipage}
	\item \begin{minipage}[t]{\linewidth} Donner l'écriture décimale de $\dfrac{51}{1000}$. \end{minipage}
\end{enumerate}
\end{multicols}
\end{exercice}

\begin{exercice}[0][Mettre les fractions au même dénominateur.]
\begin{enumerate}
	\item $\dfrac{1}{3} = \dfrac{\phantom{00000000000000}}{\phantom{00000000000000}} = $$\dfrac{\phantom{0000}}{15}$
	\item $\dfrac{5}{9} = \dfrac{\phantom{00000000000000}}{\phantom{00000000000000}} = $$\dfrac{\phantom{0000}}{72}$
	\item $5 = \dfrac{\phantom{00000000000000}}{\phantom{00000000000000}} = \dfrac{\phantom{0000}}{7}$
	\item $\dfrac{7}{10} = \dfrac{\phantom{00000000000000}}{\phantom{00000000000000}} = $$\dfrac{\phantom{0000}}{30}$
	\item $\dfrac{6}{7} = \dfrac{\phantom{00000000000000}}{\phantom{00000000000000}} = $$\dfrac{\phantom{0000}}{42}$
	\item $\dfrac{5}{8} = \dfrac{\phantom{00000000000000}}{\phantom{00000000000000}} = $$\dfrac{\phantom{0000}}{80}$
	\item $1 = \dfrac{\phantom{00000000000000}}{\phantom{00000000000000}} = \dfrac{\phantom{0000}}{7}$
	\item $\dfrac{1}{5} = \dfrac{\phantom{00000000000000}}{\phantom{00000000000000}} = $$\dfrac{\phantom{0000}}{20}$
	\item $\dfrac{7}{9} = \dfrac{\phantom{00000000000000}}{\phantom{00000000000000}} = $$\dfrac{\phantom{0000}}{18}$
	\item $\dfrac{1}{8} = \dfrac{\phantom{00000000000000}}{\phantom{00000000000000}} = $$\dfrac{\phantom{0000}}{48}$
\end{enumerate}
\end{exercice}

\begin{exercice}[0][Mettre les fractions au même dénominateur.]
\begin{enumerate}
	\item $\dfrac{1}{5} = \dfrac{\phantom{00000000000000}}{\phantom{00000000000000}} = $$\dfrac{4}{\phantom{0000}}$
	\item $\dfrac{3}{7} = \dfrac{\phantom{00000000000000}}{\phantom{00000000000000}} = $$\dfrac{24}{\phantom{0000}}$
	\item $8 = \dfrac{\phantom{00000000000000}}{\phantom{00000000000000}} = $$\dfrac{64}{\phantom{0000}}$
	\item $\dfrac{3}{5} = \dfrac{\phantom{00000000000000}}{\phantom{00000000000000}} = $$\dfrac{18}{\phantom{0000}}$
	\item $\dfrac{2}{7} = \dfrac{\phantom{00000000000000}}{\phantom{00000000000000}} = $$\dfrac{22}{\phantom{0000}}$
	\item $\dfrac{1}{10} = \dfrac{\phantom{00000000000000}}{\phantom{00000000000000}} = $$\dfrac{3}{\phantom{0000}}$
	\item $\dfrac{3}{8} = \dfrac{\phantom{00000000000000}}{\phantom{00000000000000}} = $$\dfrac{21}{\phantom{0000}}$
	\item $\dfrac{7}{10} = \dfrac{\phantom{00000000000000}}{\phantom{00000000000000}} = $$\dfrac{63}{\phantom{0000}}$
	\item $\dfrac{5}{7} = \dfrac{\phantom{00000000000000}}{\phantom{00000000000000}} = $$\dfrac{10}{\phantom{0000}}$
	\item $8 = \dfrac{\phantom{00000000000000}}{\phantom{00000000000000}} = $$\dfrac{56}{\phantom{0000}}$
\end{enumerate}
\end{exercice}

\begin{exercice}[0][Simplifier les fractions suivantes.]
\begin{enumerate}
	\item $ \dfrac{40}{64} = \dfrac{\phantom{00000000000000}}{} = \dfrac{\phantom{0000}}{} $
	\item $ \dfrac{5}{30} = \dfrac{\phantom{00000000000000}}{} = \dfrac{\phantom{0000}}{} $
	\item $ \dfrac{12}{42} = \dfrac{\phantom{00000000000000}}{} = \dfrac{\phantom{0000}}{} $
	\item $ \dfrac{2}{20} = \dfrac{\phantom{00000000000000}}{} = \dfrac{\phantom{0000}}{} $
	\item $ \dfrac{42}{48} = \dfrac{\phantom{00000000000000}}{} = \dfrac{\phantom{0000}}{} $
	\item $ \dfrac{56}{80} = \dfrac{\phantom{00000000000000}}{} = \dfrac{\phantom{0000}}{} $
	\item $ \dfrac{9}{18} = \dfrac{\phantom{00000000000000}}{} = \dfrac{\phantom{0000}}{} $
	\item $ \dfrac{20}{24} = \dfrac{\phantom{00000000000000}}{} = \dfrac{\phantom{0000}}{} $
	\item $ \dfrac{90}{100} = \dfrac{\phantom{00000000000000}}{} = \dfrac{\phantom{0000}}{} $
	\item $ \dfrac{18}{30} = \dfrac{\phantom{00000000000000}}{} = \dfrac{\phantom{0000}}{} $
\end{enumerate}
\end{exercice}

\begin{exercice}[0][Décomposer les fractions suivantes comme la somme
  d'un entiers et d''une fraction décimale.]
\begin{enumerate}
	\item $ \dfrac{31}{10} = \ldots\ldots + \dfrac{\ldots\ldots}{\ldots\ldots} $
	\item $ \dfrac{3}{2} = \ldots\ldots + \dfrac{\ldots\ldots}{\ldots\ldots} $
	\item $ \dfrac{6}{5} = \ldots\ldots + \dfrac{\ldots\ldots}{\ldots\ldots} $
	\item $ \dfrac{5}{4} = \ldots\ldots + \dfrac{\ldots\ldots}{\ldots\ldots} $
	\item $ \dfrac{7}{4} = \ldots\ldots + \dfrac{\ldots\ldots}{\ldots\ldots} $
	\item $ \dfrac{9}{8} = \ldots\ldots + \dfrac{\ldots\ldots}{\ldots\ldots} $
	\item $ \dfrac{11}{10} = \ldots\ldots + \dfrac{\ldots\ldots}{\ldots\ldots} $
	\item $ \dfrac{18}{5} = \ldots\ldots + \dfrac{\ldots\ldots}{\ldots\ldots} $
	\item $ \dfrac{5}{4} = \ldots\ldots + \dfrac{\ldots\ldots}{\ldots\ldots} $
	\item $ \dfrac{13}{5} = \ldots\ldots + \dfrac{\ldots\ldots}{\ldots\ldots} $
\end{enumerate}
\end{exercice}

\begin{exercice}[0][Comparer les fractions suivantes.]
\begin{enumerate}
	\item $\dfrac{37}{77}\quad\quad\ldots\quad\quad\dfrac{3}{7}$
	\item $\dfrac{1}{2}\quad\quad\ldots\quad\quad\dfrac{10}{22}$
	\item $\dfrac{1}{5}\quad\quad\ldots\quad\quad\dfrac{10}{40}$
	\item $\dfrac{3}{4}\quad\quad\ldots\quad\quad\dfrac{7}{8}$
	\item $\dfrac{9}{99}\quad\quad\ldots\quad\quad\dfrac{1}{9}$
	\item $\dfrac{4}{7}\quad\quad\ldots\quad\quad\dfrac{11}{14}$
	\item $\dfrac{5}{7}\quad\quad\ldots\quad\quad\dfrac{37}{56}$
	\item $\dfrac{11}{18}\quad\quad\ldots\quad\quad\dfrac{5}{6}$
	\item $\dfrac{12}{64}\quad\quad\ldots\quad\quad\dfrac{1}{8}$
	\item $\dfrac{3}{5}\quad\quad\ldots\quad\quad\dfrac{20}{35}$
\end{enumerate}
\end{exercice}

\newpage

\begin{exercice}[0][Ranger les fractions suivantes dans l'ordre \textbf{croissant}.]
\begin{enumerate}
	\item $\dfrac{9}{2}\quad\text{;}\quad$$3\quad\text{;}\quad$$\dfrac{10}{18}\quad\text{;}\quad$$\dfrac{5}{3}\quad\text{;}\quad$$\dfrac{3}{9}$
	\item $\dfrac{11}{2}\quad\text{;}\quad$$\dfrac{1}{3}\quad\text{;}\quad$$\dfrac{3}{30}\quad\text{;}\quad$$\dfrac{2}{6}\quad\text{;}\quad$$2$
	\item $\dfrac{4}{24}\quad\text{;}\quad$$\dfrac{11}{4}\quad\text{;}\quad$$3\quad\text{;}\quad$$\dfrac{4}{3}\quad\text{;}\quad$$\dfrac{5}{12}$
	\item $\dfrac{9}{4}\quad\text{;}\quad$$\dfrac{2}{10}\quad\text{;}\quad$$3\quad\text{;}\quad$$\dfrac{4}{5}\quad\text{;}\quad$$\dfrac{9}{20}$
	\item $2\quad\text{;}\quad$$\dfrac{3}{2}\quad\text{;}\quad$$\dfrac{5}{4}\quad\text{;}\quad$$\dfrac{3}{10}\quad\text{;}\quad$$\dfrac{10}{20}$
\end{enumerate}
\end{exercice}

\begin{exercice}[0][Ranger les fractions suivantes dans l'ordre \textbf{décroissant}.]
\begin{enumerate}
	\item $\dfrac{7}{2}\quad\text{;}\quad$$3\quad\text{;}\quad$$\dfrac{1}{18}\quad\text{;}\quad$$\dfrac{4}{3}\quad\text{;}\quad$$\dfrac{4}{9}$
	\item $\dfrac{10}{8}\quad\text{;}\quad$$\dfrac{3}{4}\quad\text{;}\quad$$\dfrac{10}{16}\quad\text{;}\quad$$\dfrac{7}{2}\quad\text{;}\quad$$1$
	\item $\dfrac{7}{3}\quad\text{;}\quad$$\dfrac{7}{24}\quad\text{;}\quad$$3\quad\text{;}\quad$$\dfrac{11}{2}\quad\text{;}\quad$$\dfrac{6}{4}$
	\item $\dfrac{7}{8}\quad\text{;}\quad$$\dfrac{11}{4}\quad\text{;}\quad$$\dfrac{6}{16}\quad\text{;}\quad$$\dfrac{9}{8}\quad\text{;}\quad$$3$
	\item $2\quad\text{;}\quad$$\dfrac{1}{3}\quad\text{;}\quad$$\dfrac{7}{12}\quad\text{;}\quad$$\dfrac{11}{6}\quad\text{;}\quad$$\dfrac{11}{4}$
\end{enumerate}
\end{exercice}

\begin{exercice}[0][Encadrer les fractions suivantes entre deux entiers
  consécutifs.]
\begin{enumerate}
	\item $\ldots~~ < ~~\dfrac{9\,732}{1\,000}~~ < ~~\ldots$
	\item $\ldots~~ < ~~\dfrac{15}{10}~~ < ~~\ldots$
	\item $\ldots~~ < ~~\dfrac{226}{100}~~ < ~~\ldots$
	\item $\ldots~~ < ~~\dfrac{306}{1\,000}~~ < ~~\ldots$
	\item $\ldots~~ < ~~\dfrac{53}{10}~~ < ~~\ldots$
	\item $\ldots~~ < ~~\dfrac{167}{100}~~ < ~~\ldots$
	\item $\ldots~~ < ~~\dfrac{19}{10}~~ < ~~\ldots$
	\item $\ldots~~ < ~~\dfrac{717}{100}~~ < ~~\ldots$
	\item $\ldots~~ < ~~\dfrac{8\,814}{1\,000}~~ < ~~\ldots$
	\item $\ldots~~ < ~~\dfrac{9\,020}{1\,000}~~ < ~~\ldots$
\end{enumerate}
\end{exercice}

\begin{exercice}[0][Encadrer les fractions suivantes entre deux entiers
  consécutifs.]
\begin{multicols}{2}
\begin{enumerate}
	\item \begin{minipage}[t]{\linewidth} $\ldots < \dfrac{7}{5} < \ldots$ \end{minipage}
	\item \begin{minipage}[t]{\linewidth} $\ldots < \dfrac{11}{10} < \ldots$ \end{minipage}
	\item \begin{minipage}[t]{\linewidth} $\ldots < \dfrac{2}{3} < \ldots$ \end{minipage}
	\item \begin{minipage}[t]{\linewidth} $\ldots < \dfrac{3}{2} < \ldots$ \end{minipage}
	\item \begin{minipage}[t]{\linewidth} $\ldots < \dfrac{6}{4} < \ldots$ \end{minipage}
	\item \begin{minipage}[t]{\linewidth} $\ldots < \dfrac{8}{3} < \ldots$ \end{minipage}
	\item \begin{minipage}[t]{\linewidth} $\ldots < \dfrac{1}{2} < \ldots$ \end{minipage}
	\item \begin{minipage}[t]{\linewidth} $\ldots < \dfrac{2}{10} < \ldots$ \end{minipage}
	\item \begin{minipage}[t]{\linewidth} $\ldots < \dfrac{16}{5} < \ldots$ \end{minipage}
	\item \begin{minipage}[t]{\linewidth} $\ldots < \dfrac{3}{4} < \ldots$ \end{minipage}
\end{enumerate}
\end{multicols}
\end{exercice}
\end{multicols}

\end{document}
