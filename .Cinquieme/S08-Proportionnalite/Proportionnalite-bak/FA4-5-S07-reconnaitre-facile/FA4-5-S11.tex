%%CLASS%%
\documentclass[french,12pt]{article}
%%PACKAGES%%
\usepackage[T1]{fontenc}
\usepackage[none]{hyphenat}
\usepackage[utf8]{inputenc}
\usepackage{multicol, caption}
\usepackage{tabularx}
\usepackage{graphicx}
\renewcommand{\arraystretch}{1.5}
%\setlength{\arrayrulewidth}{0,5mm}
%\usepackage[french]{babel}
\usepackage[left=2cm,right=2cm,top=2cm,bottom=2cm]{geometry}
\usepackage{mathtools, bm}
\usepackage{enumitem}
\usepackage{amsmath}
\usepackage{amssymb}
\usepackage{pstricks}
\usepackage{titlesec}
\usepackage{xlop}
\usepackage[pdftex=true,colorlinks=true,linkcolor=black,citecolor=black,filecolor=black,urlcolor=black,bookmarks=true,bookmarksopen=false,bookmarksnumbered=false,bookmarksopenlevel=0,pdfstartview=FitH,pdftitle={},pdfauthor={ColinOUTEROVITCH,bookmarks=false}]{hyperref}
\newenvironment{Figure}
  {\par\medskip\noindent\minipage{\linewidth}}
  {\endminipage\par\medskip}
%usepackage{biblatex}
%\addbibresource{biblio.bib}
\usepackage{fancyhdr}
\usepackage{pgfplots}
\pgfplotsset{compat=1.15}
\usepackage{mathrsfs}
\usetikzlibrary{arrows}

%%PAGE STYLE%%
\pagestyle{fancy}
\usepackage{tikz}
\fancyhf{}

%%NEW COMMAND%%



\newcommand*\circled[1]{\tikz[baseline=(char.base)]{
            \node[shape=circle,draw,inner sep=2pt] (char) {#1};}}
            
\renewcommand{\baselinestretch}{1}
\titleformat{\section}
{\normalfont\Large\bfseries}{\thesection}{1em}{-~~}

\newcommand{\mc}[1]{\begin{multicols}{2}#1\end{multicols}}

\newcommand{\enu}[1]{\begin{enumerate}[label=(\alph*)]#1\end{enumerate}}

\newcommand{\itmz}[1]{\begin{itemize}[label=\textbullet]#1\end{itemize}}

\newcommand{\cntr}[1]{\begin{center}#1\end{center}}

\newcommand{\dtf}{\makebox[\linewidth]{\dotfill}}

\newcommand{\exercice}{\section{}}

\newcommand{\mkb}[2]{\makebox[#1]{#2}}

\newcommand{\mybox}[1]{\begin{tabular}{|l|}
\hline
#1 \\
\hline
\end{tabular}}

\newcommand{\mkbdtf}[1]{\makebox[#1cm]{\dotfill}}

\newcommand{\mkblw}[1]{\makebox[\linewidth]{#1}}

\newcommand{\myrule}[1]{\rule[2mm]{#1cm}{.1pt}}

\newcommand{\myfigure}[1]{\begin{Figure}
\centering
\includegraphics[width=\linewidth]{images/#1}
\end{Figure}}

\newcommand{\entete}[1]{\lhead{Collège La Vallée - \textit{Mathématiques}}   \rhead{2021/2022}
\renewcommand{\headrulewidth}{0.5pt}
\cfoot{\circled{\thepage}}
\chead{\textbf{#1}}}

\newcommand{\titre}[2]{\cntr{
\begin{LARGE}
\myrule{#2} \textsc{#1} \myrule{#2}
\end{LARGE}}}

\newcommand{\colonnesep}[1]{\setlength{\columnseprule}{#1pt}}

\titleformat{\section}
{\normalfont\Large\bfseries}{Exercice~\thesection}{1em}{}

\setlength{\columnseprule}{1pt}
\setlength{\parindent}{0pt}

\newcommand{\myfig}[2]{\begin{Figure}
\centering
\includegraphics[width=#1]{images/#2}
\end{Figure}}


\begin{document}

\entete{11}{4}{Reconnaitre une situation de proportionnalité}

\colonnesep{0}

\phantom{0} \\
\exercice (\textit{Exercice Corrigé 1}) \\
Dire si le tableau suivant est un tableau de proportionnalité. Justifier.
\mc{
\cntr{
\begin{tabular}{|l|c|c|c|}
\hline
Quantité A & 3 & 4 & 2  \\ \hline
Quantité B & 12 & 16 & 8 \\ \hline
\end{tabular} }

\columnbreak

\circled{1} On calcule, $12 \div 3 = 4$ \\
\circled{2} Puis, $16 \div 4 = 4 $ \\
\circled{3} Enfin, $8 \div 2 = 4$ \\
}
\circled{4} En conclusion, les coefficients sont égaux donc \underline{oui} c'est un tableau de proportionnalité.

\exercice (\textit{Exercice Corrigé 2}) \\
Dire si le tableau suivant est un tableau de proportionnalité. Justifier.
\mc{
\cntr{
\begin{tabular}{|l|c|c|c|}
\hline
Quantité A & 1 & 3 & 4  \\ \hline
Quantité B & 2 & 6 & 16 \\ \hline
\end{tabular}}

\columnbreak

\circled{1} On calcule, $2 \div 1 = 2$ \\
\circled{2} Puis, $6 \div 3 = 2 $ \\
\circled{3} Enfin, $15 \div 5 = 3$ \\
}
\circled{4} En conclusion, les coefficients ne sont pas égaux donc \underline{non} ce n'est pas un tableau de proportionnalité.

\exercice \diff[1] \\
Dire si le tableau suivant est un tableau de proportionnalité. Justifier.
\mc{
\cntr{
\begin{tabular}{|l|c|c|c|}
\hline
Quantité A & 1 & 3 & 4  \\ \hline
Quantité B & 3 & 9 & 12 \\ \hline
\end{tabular} }

\columnbreak

\circled{1} On calcule, $3 \div 1 = $ \ldots\ldots\ldots \\
\circled{2} Puis, $9 \div 3 = $ \ldots\ldots\ldots \\
\circled{3} Enfin, $12 \div 4 = $ \ldots\ldots\ldots \\
}
\circled{4} En conclusion, \dotfill \\ \dtf

\newpage \phantom{0} \\
\exercice \diff[1] \\
Dire si le tableau suivant est un tableau de proportionnalité. Justifier.
\mc{
\cntr{
\begin{tabular}{|l|c|c|c|}
\hline
Quantité A & 5 & 2 &  3 \\ \hline
Quantité B & 10 & 4 & 6 \\  \hline
\end{tabular}  
}

\columnbreak

\circled{1} \dotfill \\
\circled{2} \dotfill \\
\circled{3} \dotfill \\
}
\circled{4} En conclusion, \dotfill \\ \dtf

\exercice \diff[1] \\
Dire si le tableau suivant est un tableau de proportionnalité. Justifier.
\mc{
\cntr{
\begin{tabular}{|l|c|c|c|}
\hline
Quantité A & 4 & 5 &  1 \\ \hline
Quantité B & 16 & 15 & 6 \\  \hline
\end{tabular}  }

\columnbreak

\circled{1} \dotfill \\
\circled{2} \dotfill \\
\circled{3} \dotfill \\
}
\circled{4} En conclusion, \dotfill \\ \dtf

\exercice \diff[2] \\
Dire si le tableau suivant est un tableau de proportionnalité. Justifier.
\mc{
\cntr{
\begin{tabular}{|l|c|c|c|}
\hline
Quantité A & 5 & 2 & 6  \\ \hline
Quantité B & 25 & 10 & 30 \\ \hline
\end{tabular}  }

\columnbreak

\circled{1} \dotfill \\
\circled{2} \dotfill \\
\circled{3} \dotfill \\
}
\circled{4} En conclusion, \dotfill \\ \dtf

\newpage \phantom{0} \\
\exercice \diff[2] \\
Dire si le tableau suivant est un tableau de proportionnalité. Justifier.
\mc{
\cntr{
\begin{tabular}{|l|c|c|c|}
\hline
Quantité A & 4 & 5 &  6 \\ \hline
Quantité B & 24 & 25 & 18 \\  \hline
\end{tabular}  
}

\columnbreak

\circled{1} \dotfill \\
\circled{2} \dotfill \\
\circled{3} \dotfill \\
}
\circled{4} En conclusion, \dotfill \\ \dtf

\exercice \diff[2] \\
Dire si le tableau suivant est un tableau de proportionnalité. Justifier.
\mc{
\cntr{
\begin{tabular}{|l|c|c|c|}
\hline
Quantité A & 5 & 4 &  7 \\ \hline
Quantité B & 20 & 16 & 28 \\  \hline
\end{tabular}  }

\columnbreak

\circled{1} \dotfill \\
\circled{2} \dotfill \\
\circled{3} \dotfill \\
}
\circled{4} En conclusion, \dotfill \\ \dtf

\exercice \diff[2] \\
Dire si le tableau suivant est un tableau de proportionnalité. Justifier.
\mc{
\cntr{
\begin{tabular}{|l|c|c|c|}
\hline
Quantité A & 5 & 6 &  4 \\ \hline
Quantité B & 30 & 24 & 20 \\  \hline
\end{tabular}  }

\columnbreak

\circled{1} \dotfill \\
\circled{2} \dotfill \\
\circled{3} \dotfill \\
}
\circled{4} En conclusion, \dotfill \\ \dtf




\end{document}
