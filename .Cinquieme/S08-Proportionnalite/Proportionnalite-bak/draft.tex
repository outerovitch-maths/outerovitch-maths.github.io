\item Joachim relève les prix des livres sur un catalogue par correspondance en fonction de la quantité saisie dans le panier. Il note les prix dans le tableau suivant :
$\def\arraystretch{1.5}\begin{array}{|c|c|c|c|c|c|}\hline  \text{livres}&2&3&5&9\\
\hline \text{Prix (en $\,\euro{}$})&6&9&14&27\\
\hline\end{array}$
Le prix des livres est-il proportionnel à la quantité achetée ?\\

\item Joachim relève les prix des maquettes sur un catalogue par correspondance en fonction de la quantité saisie dans le panier. Il note les prix dans le tableau suivant :
$\def\arraystretch{1.5}\begin{array}{|c|c|c|c|c|c|}\hline  \text{maquettes}&3&4&7&12\\
\hline \text{Prix (en $\,\euro{}$})&24&32&56&96\\
\hline\end{array}$
Le prix des maquettes est-il proportionnel à la quantité achetée ?\\

\item Fernando relève les prix des livres sur un catalogue par correspondance en fonction de la quantité saisie dans le panier. Il note les prix dans le tableau suivant :
$\def\arraystretch{1.5}\begin{array}{|c|c|c|c|c|c|}\hline  \text{livres}&4&5&9&15\\
\hline \text{Prix (en $\,\euro{}$})&23&30&54&90\\
\hline\end{array}$
Le prix des livres est-il proportionnel à la quantité achetée ?\\

\item Guillaume relève les prix des roches sur un catalogue par correspondance en fonction de la quantité saisie dans le panier. Il note les prix dans le tableau suivant :
$\def\arraystretch{1.5}\begin{array}{|c|c|c|c|c|c|}\hline  \text{roches}&2&3&5&9\\
\hline \text{Prix (en $\,\euro{}$})&18&27&45&80\\
\hline\end{array}$
Le prix des roches est-il proportionnel à la quantité achetée ?\\

\item Benjamin relève les prix des puzzles sur un catalogue par correspondance en fonction de la quantité saisie dans le panier. Il note les prix dans le tableau suivant :
$\def\arraystretch{1.5}\begin{array}{|c|c|c|c|c|c|}\hline  \text{puzzles}&4&5&9&15\\
\hline \text{Prix (en $\,\euro{}$})&54&67{,}50&120{,}50&202{,}50\\
\hline\end{array}$
Le prix des puzzles est-il proportionnel à la quantité achetée ?\\
\end{spacing}
\end{enumerate}
