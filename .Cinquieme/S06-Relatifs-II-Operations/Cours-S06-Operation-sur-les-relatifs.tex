%% Font size %%
\documentclass[11pt]{article}

%% Load the custom package
\usepackage{Mathdoc}

%% Numéro de séquence %% Titre de la séquence %%
\renewcommand{\centerhead}{Chap. 6 : Opérations sur les nombres
  relatifs}

%% Spacing commands %%
\renewcommand{\baselinestretch}{1} \setlength{\parindent}{0pt}


\begin{document}

\section{Rappels}

\begin{propriete}
Un nombre relatif est formé d’un signe (+ ou -) et d’un nombre (sa distance à zéro).
\end{propriete}

\begin{definition}
Lorsque deux nombres relatifs ont la même distance à zéro mais ont des
signes différents, on dit que ces nombres sont opposés.
\end{definition}

\begin{exemple}
$(+5)$ est un nombre positif \\
$(-14)$ est un nombre négatif \\
$18$ est un nombre positif \\
L'opposé de $(-4)$ est $(+4)$
\end{exemple}


\section{Additions de relatifs}

\subsection{Même signe}

\subsection{Signe contraire}

\section{Soustraction de relatifs}

\section{Multiplication de relatifs}



\end{document}
