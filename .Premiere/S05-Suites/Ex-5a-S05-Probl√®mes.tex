%% Font size %%
\documentclass[11pt]{article}

%% Load the custom package
\usepackage{Mathdoc}

%% Numéro de séquence %% Titre de la séquence %%
\renewcommand{\centerhead}{}

%% Spacing commands %%
\renewcommand{\baselinestretch}{1} \setlength{\parindent}{0pt}

\begin{document}

\begin{exercice}[6][Carrés successifs]
$n$ carrés sont disposés comme l’indique la figure ci-dessous (réalisé avec 5 carrés). Le côté d’un carré vaut la moitié du précédent.

Le premier carré a pour côté $c_0 = 5$ cm et pour aire $a_0$.

On pose $\ell_n = c_0 + c_1 + \dots + c_n$ et $s_n = a_0 + a_1 + \dots + a_n$.

\begin{center}
\begin{tikzpicture}
    % Premier carré
    \draw[thick] (0,0) rectangle ++(5,5);
    \node at (0.5*5, 0.5*5) {\small $a_0$};
    \node[above] at (0.5*5, 5) {\small $c_0$};
    % Deuxième carré
    \draw[thick] (5,0) rectangle ++(5/2,5/2);
    \node at (5+0.5*5/2,0.5*5/2) {\small $a_1$};
    \node[above] at (5+0.5*5/2,5/2) {\small $c_1$};
    % Troisième carré
    \draw[thick] (5+5/2,0) rectangle ++(5/4,5/4);
    \node at (7.5+0.5*5/4,0.5*5/4) {\small $a_2$};
    \node[above] at (7.5+0.5*5/4,5/4) {\small $c_2$};
    % Quatrième carré
    \draw[thick] (5+5/2+5/4,0) rectangle ++(5/8,5/8);
    \node at (5+5/2+5/4+0.5*5/8,0.5*5/8) {\small $a_3$};
    \node[above] at (5+5/2+5/4+0.5*5/8,5/8) {\small $c_3$};


    % Axes et labels
    \draw[] (0,0) -- (5/2+5/4+5/8+5/16,0) node[below] {$\ell_3$};
    \draw[<->] (-0.5,0) -- (-0.5,5) node[above] {5 cm};
\end{tikzpicture}

\end{center}

\begin{enumerate}
    \item Calculer les cinq premiers termes des suites $(\ell_n)$ et $(s_n)$. On pourra s’aider éventuellement d’un algorithme.
    \item 
    \begin{enumerate}
        \item Exprimer $\ell_n$ et $s_n$ en fonction de $n$.
        \item Existe-t-il un entier $p$ tel que $\ell_p \geq 10$ ?
        \item Donner la limite (éventuelle) de chacune des suites $(\ell_n)$ et $(s_n)$.
    \end{enumerate}
\end{enumerate}
\end{exercice}

\begin{exercice}[6][Négociation]
Pierre essaie de vendre sa vieille voiture 1000 € à Paul. Paul trouve
ce prix trop cher et lui propose 500 €. Pierre décide de couper la
poire en deux et lui propose alors 750 €. Paul tient alors le même raisonnement et lui propose 625 €. \\
Et ainsi de suite... Vont-il finir par se mettre d’accord ? \\
On pose $u_0 = 1000$ la 1re proposition de Pierre et $u_1 = 500$ la
1re proposition de Paul.
\begin{enumerate}
\item Exprimer la proposition $u_{n+2}$ en fonction des 2 propositions précédentes $u_{n+1}$ et $u_n$.
\item Vers quel prix Pierre et Paul vont-il tomber d’accord ?
\end{enumerate}
\end{exercice}

\newpage

\begin{exercice}[6][Évolution de la population française]
On souhaite modéliser l'évolution de la population française sur 10
ans à l'aide d'une suite arithmético-géométrique.

Les données suivantes sont issues de l'INSEE :
\begin{itemize}
\item Population de la France au 1er janvier 2025 : $68{,}606$
millions d'habitants.
\item Nombre de naissances en 2024 : $663~000$, en baisse de $2{,}2\%$
par rapport à 2023.
\item Nombre de décès en 2024 : $667~000$, en hausse de $1{,}2\%$ par
rapport à 2023.
\item Solde migratoire estimé en 2024 : $+100~000$ personnes.
\end{itemize}

\begin{enumerate}
\item Calculer le solde naturel de la population en 2024 et en déduire
le taux d'accroissement naturel $t$.
\item Exprimer la population $P_{n+1}$ en fonction de $P_n$ sous la
forme d'une suite arithmético-géométrique.
\item Déterminer la nature de la suite et proposer une méthode pour
obtenir son expression explicite.
\item Calculer la population estimée en 2035 ($P_{10}$) en supposant
que les tendances actuelles restent constantes.
\end{enumerate}
\end{exercice}
\end{document}
