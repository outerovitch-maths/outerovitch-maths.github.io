%% Font size %%
\documentclass[11pt]{article}

%% Load the custom package
\usepackage{Mathdoc}

%% Numéro de séquence %% Titre de la séquence %%
\renewcommand{\centerhead}{Chap. 5 : Suites - Formes explicites}

%% Spacing commands %%
\renewcommand{\baselinestretch}{1} \setlength{\parindent}{0pt}

\begin{document}

\begin{exercice}[1][Déterminer les valeurs d''un suite I]
\begin{enumerate}
	\item Soit $(u_n)$ une suite définie pour tout entier $n\in\mathbb{N}$ par $u_n =-8n-10$. \\Calculer $u_{12}$.
	\item Soit $(u_n)$ une suite définie pour tout entier $n\in\mathbb{N}$ par $u_n =-8n+9$. \\Calculer $u_{20}$.
	\item Soit $(u_n)$ une suite définie pour tout entier $n\in\mathbb{N}$ par $u_n =-3n+2$. \\Calculer $u_{9}$.
	\item Soit $(u_n)$ une suite définie pour tout entier $n\in\mathbb{N}$ par $u_n =-2n-10$. \\Calculer $u_{7}$.
\end{enumerate}
\end{exercice}

\begin{exercice}[1][Déterminer les valeurs d'une suite II]
\begin{enumerate}
	\item Soit $(u_n)$ une suite définie pour tout entier $n\in\mathbb{N}$ par 
        $u_n = -3n^2-9n+6$  .\\Calculer $u_{4}$.
	\item Soit $(u_n)$ une suite définie pour tout entier $n\in\mathbb{N}$ par 
        $u_n = 3n^2-9n-2$  .\\Calculer $u_{7}$.
	\item Soit $(u_n)$ une suite définie pour tout entier $n\in\mathbb{N}$ par 
        $u_n = -2n^2+n-1$  .\\Calculer $u_{2}$.
	\item Soit $(u_n)$ une suite définie pour tout entier $n\in\mathbb{N}$ par 
        $u_n = -n^2+8n-7$  .\\Calculer $u_{9}$.
\end{enumerate}
\end{exercice}

\begin{exercice}[2][Déterminer les valeurs d'une suite III]
\begin{enumerate}
	\item Soit $(u_n)$ une suite définie pour tout entier $n\in\mathbb{N}$ par 
          $u_n =\dfrac{-4n-5}{3n+5} $.\\Calculer $u_{4}$. \\
          Donner le résultat sous la forme d'une fraction irréductible ou d'un nombre entier.
	\item Soit $(u_n)$ une suite définie pour tout entier $n\in\mathbb{N}$ par 
          $u_n =\dfrac{5n-5}{2n+2} $.\\Calculer $u_{6}$. \\
          Donner le résultat sous la forme d'une fraction irréductible ou d'un nombre entier.
	\item Soit $(u_n)$ une suite définie pour tout entier $n\in\mathbb{N}$ par 
          $u_n =\dfrac{-3n+1}{2n+3} $.\\Calculer $u_{3}$. \\
          Donner le résultat sous la forme d'une fraction irréductible ou d'un nombre entier.
	\item Soit $(u_n)$ une suite définie pour tout entier $n\in\mathbb{N}$ par 
          $u_n =\dfrac{-n+5}{4n+4} $.\\Calculer $u_{8}$. \\
          Donner le résultat sous la forme d'une fraction irréductible ou d'un nombre entier.
\end{enumerate}
\end{exercice}

\end{document}
