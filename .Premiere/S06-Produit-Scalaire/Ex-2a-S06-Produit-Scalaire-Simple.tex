%% Font size %%
\documentclass[11pt]{article}

%% Load the custom package
\usepackage{Mathdoc}

%% Numéro de séquence %% Titre de la séquence %%
\renewcommand{\centerhead}{}

%% Spacing commands %%
\renewcommand{\baselinestretch}{1} \setlength{\parindent}{0pt}

\newcommand{\vecteur}[3]{
$\vec{#1} = \begin{pmatrix} #2 \\ #3 \end{pmatrix}$}

\begin{document}

\begin{exercice}[1][Calculer les produits scalaires suivants]
\underline{Rappels} : Soit $ \vec{u} = \begin{pmatrix} x \\ y \end{pmatrix} $ et $\vec{v} = \begin{pmatrix} x' \\ y' \end{pmatrix}$ alors $\vec{u}
. \vec{v} = xx' + yy'$

\begin{multicols}{2}
\begin{enumerate}
\item $\vec{a} = \begin{pmatrix} 3 \\ 7 \end{pmatrix}$ et
$\vec{b} = \begin{pmatrix} 11 \\ 3 \end{pmatrix}$
\item \vecteur{c}{4}{9} et \vecteur{d}{-1}{2}
\item \vecteur{d}{-2}{8} et \vecteur{d}{-3}{6}
\item \vecteur{a}{3}{5} et \vecteur{b}{2}{-4}  
\item \vecteur{u}{-1}{7} et \vecteur{v}{6}{3}  
\item \vecteur{x}{0}{-2} et \vecteur{y}{5}{4}
\end{enumerate}
\end{multicols}
\end{exercice}

\begin{exercice}
Soit $A=(0;0)$, $B=(1;2)$, $C=(-3;9)$ et $D=(-1;-8)$ \\
Calculer $\vec{AB}.\vec{AC}$, $\vec{AB}.\vec{DC}$ et $\vec{BC}.\vec{AD}$
\end{exercice}

\begin{exercice}
Soit $IJK$ un triangle rectangle-isocèle en I tel que $IK=5cm$
\begin{enumerate}
\item En utilisant le théorème de Pythagore déterminer
$\|\vec{JK}\|$
\item on se place dans le repère orthonormé direct de
$(I,\vec{IJ},\vec{IK})$, calculer les coordonnées des vecteurs
$\vec{IJ}$, $\vec{IK}$ et $\vec{IK}$
\item Calculer $\vec{IJ}.\vec{JK}$ et $\vec{IJ}.\vec{IK}$, que remarque-t-on ?
\end{enumerate}
\end{exercice}

\newpage


\begin{exercice}[1][Produit scalaire dans le plan]
Soit un rectangle $ABCD$ tel que \ldots\ (préciser les conditions sur les côtés) et soit un point $M$ tel que \ldots. 
\begin{enumerate}
  \item Calculer $\vec{AB}\cdot\vec{AD}$.
  \item En déduire \ldots.
  \item Calculer \ldots.
  \item Calculer \ldots.
  \item En déduire la mesure de l'angle $\theta$.
\end{enumerate}
% Correction Exercice 1
\end{exercice}

\begin{exercice}[2]
Dans un repère orthonormé, on considère les points $A$, $B$ et $C$. 
\begin{enumerate}
  \item Déterminer une équation cartésienne de la droite $(AB)$, puis les coordonnées d’un vecteur normal à cette droite.
  \item Déterminer une équation cartésienne de la droite perpendiculaire à $(AB)$ passant par le point $C$.
  \item Quelles sont les coordonnées du point symétrique de $C$ par rapport à la droite $(AB)$ ?
\end{enumerate}
% Correction Exercice 2
\end{exercice}

\begin{exercice}[3]
On considère un triangle isocèle $ABC$ en $A$ tel que $AB=AC$. On définit :
\begin{itemize}
  \item $I$ le milieu du segment $[BC]$,
  \item $H$ le projeté orthogonal de $A$ sur la droite $(BC)$,
  \item $J$ le milieu du segment $[AH]$.
\end{itemize}
En choisissant un repère orthonormé adapté, démontrer que les droites $(IJ)$ et $(AC)$ sont perpendiculaires.
% Correction Exercice 3
\end{exercice}

\begin{exercice}[4]
Dans un repère orthonormé, on considère les points $A$ et $B$.
\begin{enumerate}
  \item Déterminer une équation cartésienne de la droite $(AB)$ ainsi que celle de la médiatrice du segment $[AB]$.
  \item Soit $M$ un point de la droite $(AB)$ tel que \ldots\ (préciser la condition géométrique). Déterminer une mesure de l’angle $\theta$ formé par \ldots.
  \item[a)] Quelles sont les coordonnées du point $N$ de la droite $(AB)$ tel que le triangle $(A,B,N)$ soit rectangle direct (les points $A$, $B$, $N$ se lisent dans cet ordre dans le sens trigonométrique) ?
  \item[b)] Déterminer alors une mesure de l’angle $\theta'$.
  \item Quelles sont les coordonnées du point $P$ de la droite $(AB)$ tel que le triangle $(A,B,P)$ soit équilatéral direct ?
\end{enumerate}
% Correction Exercice 4
\end{exercice}

\begin{exercice}[5]
On considère un segment $[AB]$ mesurant \ldots\ cm et le cercle de centre $A$ et de rayon \ldots\ cm.
\begin{enumerate}
  \item On définit $M$ comme le point d’intersection de \ldots\ et $N$ comme un point du cercle tel que \ldots\ cm, $N$ étant situé de l’autre côté de \ldots\ et vérifiant \ldots\ cm.
  \item Démontrer que les points $A$, $M$ et $N$ sont alignés.
  \item Calculer la mesure de l’angle $\theta$ au millième de radian près.
\end{enumerate}
% Correction Exercice 5
\end{exercice}



\end{document}

% Local Variables:
% gptel-model: deepseek-chat
% gptel--backend-name: "DeepSeek"
% gptel--bounds: ((290 . 339) (451 . 498) (505 . 553) (640 . 686) (692 . 739) (833 . 975))
% End:
