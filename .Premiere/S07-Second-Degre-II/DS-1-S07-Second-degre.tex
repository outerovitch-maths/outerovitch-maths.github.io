
%% Font size %%
\documentclass[11pt]{article}

%% Load the custom package
\usepackage{Mathdoc}

%% Numéro de séquence %% Titre de la séquence %%
\renewcommand{\centerhead}{Second degré II - Résolution d'équations}

%% Spacing commands %%
\renewcommand{\baselinestretch}{1} \setlength{\parindent}{0pt}

\begin{document}

\phantom{0}
\vspace{-1.5cm}

\begin{center}
\duree{1 heure} 
\total{15 points}
\coefficient{1}
\calculatrice{1}
\recherche 
\copieseparee{1}
\end{center}

\begin{exercicedevoir}[5][Résoudre les équations du second degré suivantes]
\begin{multicols}{2}
\begin{enumerate}
\item $3x^2+5x-7=0$
\item $-25-x^2+10x=0$
\end{enumerate}
\end{multicols}
\end{exercicedevoir}

\begin{exercicedevoir}[5][Résoudre les inéquations du second degré suivantes]
\begin{multicols}{2}
\begin{enumerate}
\item $x^2 \leq -3x-14$
\item $9x < 3x^2-3$
\end{enumerate}
\end{multicols}
\end{exercicedevoir}

\begin{exercicedevoir}[5][Comparaison de prix entre deux formules]

Une entreprise propose deux options de paiement pour un service :

\begin{itemize}
    \item \textbf{Formule A} : $P_A(x) = 5x + 20$
    \item \textbf{Formule B} : $P_B(x) = x^2 + 10$
\end{itemize}

avec $x$ représentant le nombre d'unités achetées et $P(x)$ le prix total en euros.

\begin{enumerate}
    \item Pour combien d'unités les deux formules donnent-elles le même prix ?
    \item Déterminer pour quelles valeurs de $x$ la formule B est plus avantageuse que la formule A.
\end{enumerate}
\end{exercicedevoir}

\end{document}

%%% Local Variables:
%%% mode: LaTeX
%%% TeX-master: t
%%% TeX-master: t
%%% gptel-model: deepseek-chat
%%% gptel--backend-name: "DeepSeek"
%%% gptel--bounds: ((response (790 1386)))
%%% End:

